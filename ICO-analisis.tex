% Options for packages loaded elsewhere
\PassOptionsToPackage{unicode}{hyperref}
\PassOptionsToPackage{hyphens}{url}
%
\documentclass[
]{article}
\usepackage{amsmath,amssymb}
\usepackage{iftex}
\ifPDFTeX
  \usepackage[T1]{fontenc}
  \usepackage[utf8]{inputenc}
  \usepackage{textcomp} % provide euro and other symbols
\else % if luatex or xetex
  \usepackage{unicode-math} % this also loads fontspec
  \defaultfontfeatures{Scale=MatchLowercase}
  \defaultfontfeatures[\rmfamily]{Ligatures=TeX,Scale=1}
\fi
\usepackage{lmodern}
\ifPDFTeX\else
  % xetex/luatex font selection
\fi
% Use upquote if available, for straight quotes in verbatim environments
\IfFileExists{upquote.sty}{\usepackage{upquote}}{}
\IfFileExists{microtype.sty}{% use microtype if available
  \usepackage[]{microtype}
  \UseMicrotypeSet[protrusion]{basicmath} % disable protrusion for tt fonts
}{}
\makeatletter
\@ifundefined{KOMAClassName}{% if non-KOMA class
  \IfFileExists{parskip.sty}{%
    \usepackage{parskip}
  }{% else
    \setlength{\parindent}{0pt}
    \setlength{\parskip}{6pt plus 2pt minus 1pt}}
}{% if KOMA class
  \KOMAoptions{parskip=half}}
\makeatother
\usepackage{xcolor}
\usepackage[margin=1in]{geometry}
\usepackage{color}
\usepackage{fancyvrb}
\newcommand{\VerbBar}{|}
\newcommand{\VERB}{\Verb[commandchars=\\\{\}]}
\DefineVerbatimEnvironment{Highlighting}{Verbatim}{commandchars=\\\{\}}
% Add ',fontsize=\small' for more characters per line
\usepackage{framed}
\definecolor{shadecolor}{RGB}{248,248,248}
\newenvironment{Shaded}{\begin{snugshade}}{\end{snugshade}}
\newcommand{\AlertTok}[1]{\textcolor[rgb]{0.94,0.16,0.16}{#1}}
\newcommand{\AnnotationTok}[1]{\textcolor[rgb]{0.56,0.35,0.01}{\textbf{\textit{#1}}}}
\newcommand{\AttributeTok}[1]{\textcolor[rgb]{0.13,0.29,0.53}{#1}}
\newcommand{\BaseNTok}[1]{\textcolor[rgb]{0.00,0.00,0.81}{#1}}
\newcommand{\BuiltInTok}[1]{#1}
\newcommand{\CharTok}[1]{\textcolor[rgb]{0.31,0.60,0.02}{#1}}
\newcommand{\CommentTok}[1]{\textcolor[rgb]{0.56,0.35,0.01}{\textit{#1}}}
\newcommand{\CommentVarTok}[1]{\textcolor[rgb]{0.56,0.35,0.01}{\textbf{\textit{#1}}}}
\newcommand{\ConstantTok}[1]{\textcolor[rgb]{0.56,0.35,0.01}{#1}}
\newcommand{\ControlFlowTok}[1]{\textcolor[rgb]{0.13,0.29,0.53}{\textbf{#1}}}
\newcommand{\DataTypeTok}[1]{\textcolor[rgb]{0.13,0.29,0.53}{#1}}
\newcommand{\DecValTok}[1]{\textcolor[rgb]{0.00,0.00,0.81}{#1}}
\newcommand{\DocumentationTok}[1]{\textcolor[rgb]{0.56,0.35,0.01}{\textbf{\textit{#1}}}}
\newcommand{\ErrorTok}[1]{\textcolor[rgb]{0.64,0.00,0.00}{\textbf{#1}}}
\newcommand{\ExtensionTok}[1]{#1}
\newcommand{\FloatTok}[1]{\textcolor[rgb]{0.00,0.00,0.81}{#1}}
\newcommand{\FunctionTok}[1]{\textcolor[rgb]{0.13,0.29,0.53}{\textbf{#1}}}
\newcommand{\ImportTok}[1]{#1}
\newcommand{\InformationTok}[1]{\textcolor[rgb]{0.56,0.35,0.01}{\textbf{\textit{#1}}}}
\newcommand{\KeywordTok}[1]{\textcolor[rgb]{0.13,0.29,0.53}{\textbf{#1}}}
\newcommand{\NormalTok}[1]{#1}
\newcommand{\OperatorTok}[1]{\textcolor[rgb]{0.81,0.36,0.00}{\textbf{#1}}}
\newcommand{\OtherTok}[1]{\textcolor[rgb]{0.56,0.35,0.01}{#1}}
\newcommand{\PreprocessorTok}[1]{\textcolor[rgb]{0.56,0.35,0.01}{\textit{#1}}}
\newcommand{\RegionMarkerTok}[1]{#1}
\newcommand{\SpecialCharTok}[1]{\textcolor[rgb]{0.81,0.36,0.00}{\textbf{#1}}}
\newcommand{\SpecialStringTok}[1]{\textcolor[rgb]{0.31,0.60,0.02}{#1}}
\newcommand{\StringTok}[1]{\textcolor[rgb]{0.31,0.60,0.02}{#1}}
\newcommand{\VariableTok}[1]{\textcolor[rgb]{0.00,0.00,0.00}{#1}}
\newcommand{\VerbatimStringTok}[1]{\textcolor[rgb]{0.31,0.60,0.02}{#1}}
\newcommand{\WarningTok}[1]{\textcolor[rgb]{0.56,0.35,0.01}{\textbf{\textit{#1}}}}
\usepackage{graphicx}
\makeatletter
\def\maxwidth{\ifdim\Gin@nat@width>\linewidth\linewidth\else\Gin@nat@width\fi}
\def\maxheight{\ifdim\Gin@nat@height>\textheight\textheight\else\Gin@nat@height\fi}
\makeatother
% Scale images if necessary, so that they will not overflow the page
% margins by default, and it is still possible to overwrite the defaults
% using explicit options in \includegraphics[width, height, ...]{}
\setkeys{Gin}{width=\maxwidth,height=\maxheight,keepaspectratio}
% Set default figure placement to htbp
\makeatletter
\def\fps@figure{htbp}
\makeatother
\setlength{\emergencystretch}{3em} % prevent overfull lines
\providecommand{\tightlist}{%
  \setlength{\itemsep}{0pt}\setlength{\parskip}{0pt}}
\setcounter{secnumdepth}{-\maxdimen} % remove section numbering
\usepackage{multirow}
\usepackage{multicol}
\usepackage{colortbl}
\usepackage{hhline}
\newlength\Oldarrayrulewidth
\newlength\Oldtabcolsep
\usepackage{longtable}
\usepackage{array}
\usepackage{hyperref}
\usepackage{float}
\usepackage{wrapfig}
\ifLuaTeX
  \usepackage{selnolig}  % disable illegal ligatures
\fi
\IfFileExists{bookmark.sty}{\usepackage{bookmark}}{\usepackage{hyperref}}
\IfFileExists{xurl.sty}{\usepackage{xurl}}{} % add URL line breaks if available
\urlstyle{same}
\hypersetup{
  pdftitle={Análisis de Mercado: Barra de Ensaladas},
  pdfauthor={Noel Lirón Martínez, Raquel Mus Maquieira, Carles Sahuquillo Soriano},
  hidelinks,
  pdfcreator={LaTeX via pandoc}}

\title{Análisis de Mercado: Barra de Ensaladas}
\usepackage{etoolbox}
\makeatletter
\providecommand{\subtitle}[1]{% add subtitle to \maketitle
  \apptocmd{\@title}{\par {\large #1 \par}}{}{}
}
\makeatother
\subtitle{Asignatura: Investigación Comercial}
\author{Noel Lirón Martínez, Raquel Mus Maquieira, Carles Sahuquillo
Soriano}
\date{2025-12-13}

\begin{document}
\maketitle

{
\setcounter{tocdepth}{2}
\tableofcontents
}
\hypertarget{preparaciuxf3n-del-entorno}{%
\section{Preparación del Entorno}\label{preparaciuxf3n-del-entorno}}

\begin{Shaded}
\begin{Highlighting}[]
\CommentTok{\# Carga de librerías y configuraciones}
\ControlFlowTok{if}\NormalTok{ (}\SpecialCharTok{!}\FunctionTok{require}\NormalTok{(}\StringTok{"pacman"}\NormalTok{)) }\FunctionTok{install.packages}\NormalTok{(}\StringTok{"pacman"}\NormalTok{)}

\NormalTok{pacman}\SpecialCharTok{::}\FunctionTok{p\_load}\NormalTok{(}
\NormalTok{  readxl, splitstackshape, descr, rstatix, }
\NormalTok{  tibble, dplyr, tidyr, ggplot2, }
\NormalTok{  xtable, likert, ggpubr, flextable, }
\NormalTok{  nortest, DT, RColorBrewer, graphics}
\NormalTok{)}

\FunctionTok{options}\NormalTok{(}\AttributeTok{digits =} \DecValTok{3}\NormalTok{)}

\FunctionTok{theme\_set}\NormalTok{(}\FunctionTok{theme\_minimal}\NormalTok{() }\SpecialCharTok{+} 
            \FunctionTok{theme}\NormalTok{(}\AttributeTok{plot.title =} \FunctionTok{element\_text}\NormalTok{(}\AttributeTok{face =} \StringTok{"bold"}\NormalTok{, }\AttributeTok{hjust =} \FloatTok{0.5}\NormalTok{),}
                  \AttributeTok{legend.position =} \StringTok{"bottom"}\NormalTok{))}
\end{Highlighting}
\end{Shaded}

\begin{center}\rule{0.5\linewidth}{0.5pt}\end{center}

\hypertarget{ingesta-y-limpieza-de-datos}{%
\section{Ingesta y Limpieza de
Datos}\label{ingesta-y-limpieza-de-datos}}

En esta sección procesamos el archivo crudo
\texttt{respuestasEncuesta.xlsx}, renombrando las columnas para
facilitar su manipulación y convirtiendo las variables categóricas.

\hypertarget{carga-y-renombrado}{%
\subsection{Carga y Renombrado}\label{carga-y-renombrado}}

\begin{Shaded}
\begin{Highlighting}[]
\CommentTok{\# Carga de datos y renombramiento de columnas}

\NormalTok{datos }\OtherTok{\textless{}{-}} \FunctionTok{read\_excel}\NormalTok{(}\StringTok{\textquotesingle{}respuestasEncuesta.xlsx\textquotesingle{}}\NormalTok{)}

\NormalTok{nuevos\_nombres }\OtherTok{\textless{}{-}} \FunctionTok{c}\NormalTok{(}
  \StringTok{\textquotesingle{}Marca.Temporal\textquotesingle{}}\NormalTok{,              }\CommentTok{\# Marca temporal}
  \StringTok{\textquotesingle{}Motivo\textquotesingle{}}\NormalTok{,                      }\CommentTok{\# ¿Cuál es el principal motivo por el que suele comer fuera de casa?}
  \StringTok{\textquotesingle{}Frecuencia\textquotesingle{}}\NormalTok{,                  }\CommentTok{\# ¿Con qué frecuencia suele comer fuera de casa?}
  \StringTok{\textquotesingle{}Lugar.Habitual\textquotesingle{}}\NormalTok{,              }\CommentTok{\# Cuando come fuera ¿dónde suele comer?}
  \StringTok{\textquotesingle{}Tipo.Comida\textquotesingle{}}\NormalTok{,                 }\CommentTok{\# Cuando come fuera, ¿qué tipo de comida suele elegir?}
  \StringTok{\textquotesingle{}Frecuencia.Comida.Preparada\textquotesingle{}}\NormalTok{, }\CommentTok{\# ¿Con qué frecuencia compra comida preparada en un supermercado?}
  \StringTok{\textquotesingle{}Supermercado.Habitual\textquotesingle{}}\NormalTok{,       }\CommentTok{\# ¿En qué supermercados suele comprar comida?}
  \StringTok{\textquotesingle{}Importancia.Saludable\textquotesingle{}}\NormalTok{,       }\CommentTok{\# ¿Qué importancia tiene para usted comer de forma saludable cuando está fuera de casa?}
  \StringTok{\textquotesingle{}Experiencia.Barra.Ensaladas\textquotesingle{}}\NormalTok{, }\CommentTok{\# ¿Ha probado alguna vez una barra de ensaladas?}
  \StringTok{\textquotesingle{}Mala.Experiencia.Descripcion\textquotesingle{}}\NormalTok{,}\CommentTok{\# ¿Puede describir alguna mala experiencia que haya tenido con comida preparada o autoservicio...?}
  \StringTok{\textquotesingle{}Situaciones.Uso\textquotesingle{}}\NormalTok{,             }\CommentTok{\# ¿En qué situaciones utilizaría la barra de ensaladas?}
  \StringTok{\textquotesingle{}Formato.Preferido\textquotesingle{}}\NormalTok{,           }\CommentTok{\# ¿Qué formato preferiría?}
  \StringTok{\textquotesingle{}Imp.Sostenibilidad\textquotesingle{}}\NormalTok{,          }\CommentTok{\# Valore importancia: [Sostenibilidad de los envases]}
  \StringTok{\textquotesingle{}Imp.Origen.Local\textquotesingle{}}\NormalTok{,            }\CommentTok{\# Valore importancia: [Origen local de los productos]}
  \StringTok{\textquotesingle{}Imp.Variedad\textquotesingle{}}\NormalTok{,                }\CommentTok{\# Valore importancia: [Variedad de ingredientes]}
  \StringTok{\textquotesingle{}Imp.Sabor\textquotesingle{}}\NormalTok{,                   }\CommentTok{\# Valore importancia: [Sabor]}
  \StringTok{\textquotesingle{}Imp.Precio\textquotesingle{}}\NormalTok{,                  }\CommentTok{\# Valore importancia: [Precio]}
  \StringTok{\textquotesingle{}Imp.Higiene\textquotesingle{}}\NormalTok{,                 }\CommentTok{\# Valore importancia: [Higiene]}
  \StringTok{\textquotesingle{}Imp.Rapidez\textquotesingle{}}\NormalTok{,                 }\CommentTok{\# Valore importancia: [Rapidez del servicio]}
  \StringTok{\textquotesingle{}Preferencia.Base\textquotesingle{}}\NormalTok{,            }\CommentTok{\# ¿Qué tipo de base prefiere para sus ensaladas? (Hasta 2 opciones)}
  \StringTok{\textquotesingle{}Preferencia.Proteina\textquotesingle{}}\NormalTok{,        }\CommentTok{\# De las siguientes proteínas ¿cuáles prefiere? (Hasta 2 opciones)}
  \StringTok{\textquotesingle{}Preferencia.Complementos\textquotesingle{}}\NormalTok{,    }\CommentTok{\# De los siguientes complementos ¿cuáles prefiere? (Hasta 2 opciones)}
  \StringTok{\textquotesingle{}Disposicion.Pago\textquotesingle{}}\NormalTok{,            }\CommentTok{\# ¿Cuánto está dispuesto/a a pagar por una ensalada completa de tamaño mediano (500g)?}
  \StringTok{\textquotesingle{}Aptitud.Social\textquotesingle{}}\NormalTok{,              }\CommentTok{\# ¿Qué tan apropiadas considera las ensaladas para comidas en contextos sociales?}
  \StringTok{\textquotesingle{}Preferencia.Dispensacion\textquotesingle{}}\NormalTok{,    }\CommentTok{\# ¿Qué formato de dispensación preferiría? (Comodidad, rapidez e higiene)}
  \StringTok{\textquotesingle{}Intencion.Uso\textquotesingle{}}\NormalTok{,               }\CommentTok{\# Si su supermercado habitual ofreciera nuestra barra de ensaladas, ¿la usaría?}
  \StringTok{\textquotesingle{}Edad\textquotesingle{}}\NormalTok{,                        }\CommentTok{\# ¿En qué rango de edades se encuentra?}
  \StringTok{\textquotesingle{}Genero\textquotesingle{}}\NormalTok{,                      }\CommentTok{\# ¿Con qué género se identifica?}
  \StringTok{\textquotesingle{}Nivel.Educativo\textquotesingle{}}\NormalTok{,             }\CommentTok{\# Indique su nivel educativo máximo alcanzado}
  \StringTok{\textquotesingle{}Situacion.Laboral\textquotesingle{}}            \CommentTok{\# ¿En qué situación laboral se encuentra?}
\NormalTok{)}

\FunctionTok{colnames}\NormalTok{(datos) }\OtherTok{\textless{}{-}}\NormalTok{ nuevos\_nombres}
\end{Highlighting}
\end{Shaded}

\hypertarget{transformaciuxf3n-de-variables}{%
\subsection{Transformación de
Variables}\label{transformaciuxf3n-de-variables}}

Definimos las variables como categóricas.

\begin{Shaded}
\begin{Highlighting}[]
\NormalTok{datos}\SpecialCharTok{$}\NormalTok{f\_Motivo }\OtherTok{\textless{}{-}} \FunctionTok{factor}\NormalTok{(datos}\SpecialCharTok{$}\NormalTok{Motivo,}
  \AttributeTok{levels =} \FunctionTok{c}\NormalTok{(}\StringTok{\textquotesingle{}Conveniencia (falta de tiempo, por trabajo, estudios, etc.)\textquotesingle{}}\NormalTok{, }
             \StringTok{\textquotesingle{}Reuniones sociales (familia, amigos, compañeros de trabajo, etc.)\textquotesingle{}}\NormalTok{, }
             \StringTok{\textquotesingle{}Placer o disfrute personal (por gusto, por probar cosas nuevas, por cambiar de ambiente)\textquotesingle{}}\NormalTok{, }
             \StringTok{\textquotesingle{}Circunstancias extraordinarias (viajes, eventos...)\textquotesingle{}}\NormalTok{, }
             \StringTok{\textquotesingle{}Nunca como fuera de casa\textquotesingle{}}\NormalTok{),}
  \AttributeTok{labels =} \FunctionTok{c}\NormalTok{(}\StringTok{\textquotesingle{}Conveniencia\textquotesingle{}}\NormalTok{, }\StringTok{\textquotesingle{}Social\textquotesingle{}}\NormalTok{, }\StringTok{\textquotesingle{}Placer\textquotesingle{}}\NormalTok{, }\StringTok{\textquotesingle{}Extraordinario\textquotesingle{}}\NormalTok{, }\StringTok{\textquotesingle{}Nunca\textquotesingle{}}\NormalTok{))}

\NormalTok{datos}\SpecialCharTok{$}\NormalTok{f\_Lugar.Habitual }\OtherTok{\textless{}{-}} \FunctionTok{factor}\NormalTok{(datos}\SpecialCharTok{$}\NormalTok{Lugar.Habitual,}
  \AttributeTok{levels =} \FunctionTok{c}\NormalTok{(}\StringTok{\textquotesingle{}En un bar o restaurante\textquotesingle{}}\NormalTok{, }
             \StringTok{\textquotesingle{}En una oficina o centro educativo (cantina, comedor, aula...)\textquotesingle{}}\NormalTok{, }
             \StringTok{\textquotesingle{}En la vía pública (parques, calle, transporte, etc.)\textquotesingle{}}\NormalTok{),}
  \AttributeTok{labels =} \FunctionTok{c}\NormalTok{(}\StringTok{\textquotesingle{}Restaurante/Bar\textquotesingle{}}\NormalTok{, }\StringTok{\textquotesingle{}Oficina/Centro\textquotesingle{}}\NormalTok{, }\StringTok{\textquotesingle{}Vía Pública\textquotesingle{}}\NormalTok{))}

\NormalTok{datos}\SpecialCharTok{$}\NormalTok{f\_Formato.Preferido }\OtherTok{\textless{}{-}} \FunctionTok{factor}\NormalTok{(datos}\SpecialCharTok{$}\NormalTok{Formato.Preferido,}
  \AttributeTok{levels =} \FunctionTok{c}\NormalTok{(}\StringTok{\textquotesingle{}Tamaños de recipiente fijos con un precio establecido por tamaño\textquotesingle{}}\NormalTok{, }\StringTok{\textquotesingle{}Pago por peso\textquotesingle{}}\NormalTok{),}
  \AttributeTok{labels =} \FunctionTok{c}\NormalTok{(}\StringTok{\textquotesingle{}Precio Fijo\textquotesingle{}}\NormalTok{, }\StringTok{\textquotesingle{}Pago Peso\textquotesingle{}}\NormalTok{))}

\NormalTok{datos}\SpecialCharTok{$}\NormalTok{f\_Preferencia.Dispensacion }\OtherTok{\textless{}{-}} \FunctionTok{factor}\NormalTok{(datos}\SpecialCharTok{$}\NormalTok{Preferencia.Dispensacion,}
  \AttributeTok{levels =} \FunctionTok{c}\NormalTok{(}\StringTok{\textquotesingle{}Ingredientes expuestos tras una mampara de cristal, donde el cliente abre la protección y utiliza pinzas para servirse en su propio recipiente (similar a un autoservicio tipo buffet)\textquotesingle{}}\NormalTok{, }
             \StringTok{\textquotesingle{}Ingredientes protegidos por una barrera de cristal fija, sin acceso directo. Se utilizan palas largas para dispensarlos desde un lateral, evitando cualquier contacto del cliente con el producto, aunque con algo menos de comodidad (similar al sistema de algunas panaderías)\textquotesingle{}}\NormalTok{, }
             \StringTok{\textquotesingle{}Ingredientes servidos por personal de atención.\textquotesingle{}}\NormalTok{),}
  \AttributeTok{labels =} \FunctionTok{c}\NormalTok{(}\StringTok{\textquotesingle{}Autoservicio\textquotesingle{}}\NormalTok{, }\StringTok{\textquotesingle{}Protección Fija\textquotesingle{}}\NormalTok{, }\StringTok{\textquotesingle{}Personal\textquotesingle{}}\NormalTok{))}

\NormalTok{datos}\SpecialCharTok{$}\NormalTok{f\_Genero }\OtherTok{\textless{}{-}} \FunctionTok{factor}\NormalTok{(datos}\SpecialCharTok{$}\NormalTok{Genero,}
  \AttributeTok{levels =} \FunctionTok{c}\NormalTok{(}\StringTok{\textquotesingle{}Mujer\textquotesingle{}}\NormalTok{, }\StringTok{\textquotesingle{}Hombre\textquotesingle{}}\NormalTok{, }\StringTok{\textquotesingle{}Prefiero no decirlo\textquotesingle{}}\NormalTok{, }\StringTok{\textquotesingle{}Otro:\textquotesingle{}}\NormalTok{),}
  \AttributeTok{labels =} \FunctionTok{c}\NormalTok{(}\StringTok{\textquotesingle{}Mujer\textquotesingle{}}\NormalTok{, }\StringTok{\textquotesingle{}Hombre\textquotesingle{}}\NormalTok{, }\StringTok{\textquotesingle{}NS/NC\textquotesingle{}}\NormalTok{, }\StringTok{\textquotesingle{}Otro\textquotesingle{}}\NormalTok{))}

\NormalTok{datos}\SpecialCharTok{$}\NormalTok{f\_Situacion.Laboral }\OtherTok{\textless{}{-}} \FunctionTok{factor}\NormalTok{(datos}\SpecialCharTok{$}\NormalTok{Situacion.Laboral,}
  \AttributeTok{levels =} \FunctionTok{c}\NormalTok{(}\StringTok{\textquotesingle{}Estudiante\textquotesingle{}}\NormalTok{, }\StringTok{\textquotesingle{}Trabajador/a por cuenta ajena\textquotesingle{}}\NormalTok{, }\StringTok{\textquotesingle{}Autónomo/a\textquotesingle{}}\NormalTok{, }\StringTok{\textquotesingle{}Desempleado/a\textquotesingle{}}\NormalTok{, }\StringTok{\textquotesingle{}Jubilado/a\textquotesingle{}}\NormalTok{, }\StringTok{\textquotesingle{}Prefiero no decirlo\textquotesingle{}}\NormalTok{),}
  \AttributeTok{labels =} \FunctionTok{c}\NormalTok{(}\StringTok{\textquotesingle{}Estudiante\textquotesingle{}}\NormalTok{, }\StringTok{\textquotesingle{}Cuenta Ajena\textquotesingle{}}\NormalTok{, }\StringTok{\textquotesingle{}Autónomo\textquotesingle{}}\NormalTok{, }\StringTok{\textquotesingle{}Desempleado\textquotesingle{}}\NormalTok{, }\StringTok{\textquotesingle{}Jubilado\textquotesingle{}}\NormalTok{, }\StringTok{\textquotesingle{}NS/NC\textquotesingle{}}\NormalTok{))}

\NormalTok{datos}\SpecialCharTok{$}\NormalTok{f\_Experiencia.Barra }\OtherTok{\textless{}{-}} \FunctionTok{factor}\NormalTok{(datos}\SpecialCharTok{$}\NormalTok{Experiencia.Barra.Ensaladas, }\AttributeTok{levels =} \FunctionTok{c}\NormalTok{(}\StringTok{\textquotesingle{}Sí\textquotesingle{}}\NormalTok{, }\StringTok{\textquotesingle{}No\textquotesingle{}}\NormalTok{))}
\NormalTok{datos}\SpecialCharTok{$}\NormalTok{f\_Intencion.Uso }\OtherTok{\textless{}{-}} \FunctionTok{factor}\NormalTok{(datos}\SpecialCharTok{$}\NormalTok{Intencion.Uso, }\AttributeTok{levels =} \FunctionTok{c}\NormalTok{(}\StringTok{\textquotesingle{}Sí\textquotesingle{}}\NormalTok{, }\StringTok{\textquotesingle{}No\textquotesingle{}}\NormalTok{))}

\NormalTok{datos}\SpecialCharTok{$}\NormalTok{f\_Frecuencia }\OtherTok{\textless{}{-}} \FunctionTok{factor}\NormalTok{(datos}\SpecialCharTok{$}\NormalTok{Frecuencia,}
  \AttributeTok{levels =} \FunctionTok{c}\NormalTok{(}\StringTok{\textquotesingle{}Rara vez / Nunca\textquotesingle{}}\NormalTok{, }\StringTok{\textquotesingle{}Ocasionalmente\textquotesingle{}}\NormalTok{, }\StringTok{\textquotesingle{}1 o 2 veces por semana\textquotesingle{}}\NormalTok{, }\StringTok{\textquotesingle{}3 o 4 veces por semana\textquotesingle{}}\NormalTok{, }\StringTok{\textquotesingle{}5 o más veces por semana\textquotesingle{}}\NormalTok{),}
  \AttributeTok{labels =} \FunctionTok{c}\NormalTok{(}\StringTok{\textquotesingle{}Nunca/Rara\textquotesingle{}}\NormalTok{, }\StringTok{\textquotesingle{}Ocasional\textquotesingle{}}\NormalTok{, }\StringTok{\textquotesingle{}1{-}2/sem\textquotesingle{}}\NormalTok{, }\StringTok{\textquotesingle{}3{-}4/sem\textquotesingle{}}\NormalTok{, }\StringTok{\textquotesingle{}5+/sem\textquotesingle{}}\NormalTok{))}

\NormalTok{datos}\SpecialCharTok{$}\NormalTok{f\_Frecuencia.Comida.Preparada }\OtherTok{\textless{}{-}} \FunctionTok{factor}\NormalTok{(datos}\SpecialCharTok{$}\NormalTok{Frecuencia.Comida.Preparada,}
  \AttributeTok{levels =} \FunctionTok{c}\NormalTok{(}\StringTok{\textquotesingle{}Rara vez / Nunca\textquotesingle{}}\NormalTok{, }\StringTok{\textquotesingle{}Ocasionalmente\textquotesingle{}}\NormalTok{, }\StringTok{\textquotesingle{}1 o 2 veces por semana\textquotesingle{}}\NormalTok{, }\StringTok{\textquotesingle{}3 o 4 veces por semana\textquotesingle{}}\NormalTok{, }\StringTok{\textquotesingle{}5 o más veces por semana\textquotesingle{}}\NormalTok{),}
  \AttributeTok{labels =} \FunctionTok{c}\NormalTok{(}\StringTok{\textquotesingle{}Nunca/Rara\textquotesingle{}}\NormalTok{, }\StringTok{\textquotesingle{}Ocasional\textquotesingle{}}\NormalTok{, }\StringTok{\textquotesingle{}1{-}2/sem\textquotesingle{}}\NormalTok{, }\StringTok{\textquotesingle{}3{-}4/sem\textquotesingle{}}\NormalTok{, }\StringTok{\textquotesingle{}5+/sem\textquotesingle{}}\NormalTok{))}

\CommentTok{\# Escalas Likert (1{-}5)}
\NormalTok{niveles\_likert }\OtherTok{\textless{}{-}} \FunctionTok{c}\NormalTok{(}\DecValTok{1}\SpecialCharTok{:}\DecValTok{5}\NormalTok{)}
\NormalTok{etiquetas\_likert }\OtherTok{\textless{}{-}} \FunctionTok{c}\NormalTok{(}\StringTok{\textquotesingle{}1\textquotesingle{}}\NormalTok{, }\StringTok{\textquotesingle{}2\textquotesingle{}}\NormalTok{, }\StringTok{\textquotesingle{}3\textquotesingle{}}\NormalTok{, }\StringTok{\textquotesingle{}4\textquotesingle{}}\NormalTok{, }\StringTok{\textquotesingle{}5\textquotesingle{}}\NormalTok{)}

\NormalTok{datos}\SpecialCharTok{$}\NormalTok{f\_Importancia.Saludable }\OtherTok{\textless{}{-}} \FunctionTok{factor}\NormalTok{(datos}\SpecialCharTok{$}\NormalTok{Importancia.Saludable, }\AttributeTok{levels =}\NormalTok{ niveles\_likert, }\AttributeTok{labels =}\NormalTok{ etiquetas\_likert)}
\NormalTok{datos}\SpecialCharTok{$}\NormalTok{f\_Imp.Sostenibilidad    }\OtherTok{\textless{}{-}} \FunctionTok{factor}\NormalTok{(datos}\SpecialCharTok{$}\NormalTok{Imp.Sostenibilidad, }\AttributeTok{levels =}\NormalTok{ niveles\_likert, }\AttributeTok{labels =}\NormalTok{ etiquetas\_likert)}
\NormalTok{datos}\SpecialCharTok{$}\NormalTok{f\_Imp.Origen.Local      }\OtherTok{\textless{}{-}} \FunctionTok{factor}\NormalTok{(datos}\SpecialCharTok{$}\NormalTok{Imp.Origen.Local, }\AttributeTok{levels =}\NormalTok{ niveles\_likert, }\AttributeTok{labels =}\NormalTok{ etiquetas\_likert)}
\NormalTok{datos}\SpecialCharTok{$}\NormalTok{f\_Imp.Variedad          }\OtherTok{\textless{}{-}} \FunctionTok{factor}\NormalTok{(datos}\SpecialCharTok{$}\NormalTok{Imp.Variedad, }\AttributeTok{levels =}\NormalTok{ niveles\_likert, }\AttributeTok{labels =}\NormalTok{ etiquetas\_likert)}
\NormalTok{datos}\SpecialCharTok{$}\NormalTok{f\_Imp.Sabor             }\OtherTok{\textless{}{-}} \FunctionTok{factor}\NormalTok{(datos}\SpecialCharTok{$}\NormalTok{Imp.Sabor, }\AttributeTok{levels =}\NormalTok{ niveles\_likert, }\AttributeTok{labels =}\NormalTok{ etiquetas\_likert)}
\NormalTok{datos}\SpecialCharTok{$}\NormalTok{f\_Imp.Precio            }\OtherTok{\textless{}{-}} \FunctionTok{factor}\NormalTok{(datos}\SpecialCharTok{$}\NormalTok{Imp.Precio, }\AttributeTok{levels =}\NormalTok{ niveles\_likert, }\AttributeTok{labels =}\NormalTok{ etiquetas\_likert)}
\NormalTok{datos}\SpecialCharTok{$}\NormalTok{f\_Imp.Higiene           }\OtherTok{\textless{}{-}} \FunctionTok{factor}\NormalTok{(datos}\SpecialCharTok{$}\NormalTok{Imp.Higiene, }\AttributeTok{levels =}\NormalTok{ niveles\_likert, }\AttributeTok{labels =}\NormalTok{ etiquetas\_likert)}
\NormalTok{datos}\SpecialCharTok{$}\NormalTok{f\_Imp.Rapidez           }\OtherTok{\textless{}{-}} \FunctionTok{factor}\NormalTok{(datos}\SpecialCharTok{$}\NormalTok{Imp.Rapidez, }\AttributeTok{levels =}\NormalTok{ niveles\_likert, }\AttributeTok{labels =}\NormalTok{ etiquetas\_likert)}
\NormalTok{datos}\SpecialCharTok{$}\NormalTok{f\_Aptitud.Social        }\OtherTok{\textless{}{-}} \FunctionTok{factor}\NormalTok{(datos}\SpecialCharTok{$}\NormalTok{Aptitud.Social, }\AttributeTok{levels =}\NormalTok{ niveles\_likert, }\AttributeTok{labels =}\NormalTok{ etiquetas\_likert)}

\NormalTok{datos}\SpecialCharTok{$}\NormalTok{f\_Disposicion.Pago }\OtherTok{\textless{}{-}} \FunctionTok{factor}\NormalTok{(datos}\SpecialCharTok{$}\NormalTok{Disposicion.Pago,}
  \AttributeTok{levels =} \FunctionTok{c}\NormalTok{(}\StringTok{\textquotesingle{}Menos de 4€\textquotesingle{}}\NormalTok{, }\StringTok{\textquotesingle{}Entre 4 y 5,49€\textquotesingle{}}\NormalTok{, }\StringTok{\textquotesingle{}Entre 5,50 y 7€\textquotesingle{}}\NormalTok{, }\StringTok{\textquotesingle{}Entre 7 y 9,50€\textquotesingle{}}\NormalTok{, }\StringTok{\textquotesingle{}Más de 9,50€\textquotesingle{}}\NormalTok{),}
  \AttributeTok{labels =} \FunctionTok{c}\NormalTok{(}\StringTok{\textquotesingle{}\textless{}4€\textquotesingle{}}\NormalTok{, }\StringTok{\textquotesingle{}4{-}5.5€\textquotesingle{}}\NormalTok{, }\StringTok{\textquotesingle{}5.5{-}7€\textquotesingle{}}\NormalTok{, }\StringTok{\textquotesingle{}7{-}9.5€\textquotesingle{}}\NormalTok{, }\StringTok{\textquotesingle{}\textgreater{}9.5€\textquotesingle{}}\NormalTok{))}

\NormalTok{datos}\SpecialCharTok{$}\NormalTok{f\_Edad }\OtherTok{\textless{}{-}} \FunctionTok{factor}\NormalTok{(datos}\SpecialCharTok{$}\NormalTok{Edad,}
  \AttributeTok{levels =} \FunctionTok{c}\NormalTok{(}\StringTok{\textquotesingle{}Menor de 18\textquotesingle{}}\NormalTok{, }\StringTok{\textquotesingle{}18 {-} 24\textquotesingle{}}\NormalTok{, }\StringTok{\textquotesingle{}25 {-} 44\textquotesingle{}}\NormalTok{, }\StringTok{\textquotesingle{}45 {-} 64\textquotesingle{}}\NormalTok{, }\StringTok{\textquotesingle{}65+\textquotesingle{}}\NormalTok{))}

\NormalTok{datos}\SpecialCharTok{$}\NormalTok{f\_Nivel.Educativo }\OtherTok{\textless{}{-}} \FunctionTok{factor}\NormalTok{(datos}\SpecialCharTok{$}\NormalTok{Nivel.Educativo,}
  \AttributeTok{levels =} \FunctionTok{c}\NormalTok{(}\StringTok{\textquotesingle{}ESO (antiguo BUP)\textquotesingle{}}\NormalTok{, }\StringTok{\textquotesingle{}FP medio o Bachillerato (antiguo COU)\textquotesingle{}}\NormalTok{, }\StringTok{\textquotesingle{}FP superior\textquotesingle{}}\NormalTok{, }\StringTok{\textquotesingle{}Universitario/a\textquotesingle{}}\NormalTok{),}
  \AttributeTok{labels =} \FunctionTok{c}\NormalTok{(}\StringTok{\textquotesingle{}ESO\textquotesingle{}}\NormalTok{, }\StringTok{\textquotesingle{}Bachiller/FPm\textquotesingle{}}\NormalTok{, }\StringTok{\textquotesingle{}FPs\textquotesingle{}}\NormalTok{, }\StringTok{\textquotesingle{}Univ\textquotesingle{}}\NormalTok{))}

\CommentTok{\# {-}{-}{-}{-}{-} PREGUNTAS CON MÁS DE UNA RESPUESTA {-}{-}{-}}

\CommentTok{\# Lista de variables multirespuesta}
\NormalTok{multi\_res }\OtherTok{\textless{}{-}} \FunctionTok{c}\NormalTok{(}\StringTok{"Tipo.Comida"}\NormalTok{, }\StringTok{"Supermercado.Habitual"}\NormalTok{, }\StringTok{"Situaciones.Uso"}\NormalTok{, }
                \StringTok{"Preferencia.Base"}\NormalTok{, }\StringTok{"Preferencia.Proteina"}\NormalTok{, }\StringTok{"Preferencia.Complementos"}\NormalTok{)}

\CommentTok{\# Aplicamos cSplit\_e a cada una (separa por ", " y rellena con 0)}
\ControlFlowTok{for}\NormalTok{ (var }\ControlFlowTok{in}\NormalTok{ multi\_res) \{}
\NormalTok{  datos }\OtherTok{\textless{}{-}} \FunctionTok{cSplit\_e}\NormalTok{(datos, var, }\AttributeTok{sep =} \StringTok{\textquotesingle{}, \textquotesingle{}}\NormalTok{, }\AttributeTok{type =} \StringTok{\textquotesingle{}character\textquotesingle{}}\NormalTok{, }\AttributeTok{fill =} \DecValTok{0}\NormalTok{)}
\NormalTok{\}}


\CommentTok{\# !!!!!!!!!!!!!!!!!!!!! Si alguno da error es porque nadie lo ha marcado en la encuesta y la columna no existe, sería comentar esa línea}

\CommentTok{\# {-}{-}{-} P4: Tipo de Comida {-}{-}{-}}
\NormalTok{datos}\SpecialCharTok{$}\NormalTok{f\_Tipo.Comida.Rapida }\OtherTok{\textless{}{-}} \FunctionTok{factor}\NormalTok{(datos}\SpecialCharTok{$}\StringTok{\textasciigrave{}}\AttributeTok{Tipo.Comida\_Comida rápida}\StringTok{\textasciigrave{}}\NormalTok{, }
                                     \AttributeTok{levels =} \FunctionTok{c}\NormalTok{(}\DecValTok{0}\NormalTok{, }\DecValTok{1}\NormalTok{), }\AttributeTok{labels =} \FunctionTok{c}\NormalTok{(}\StringTok{\textquotesingle{}No\textquotesingle{}}\NormalTok{, }\StringTok{\textquotesingle{}Sí\textquotesingle{}}\NormalTok{))}
\NormalTok{datos}\SpecialCharTok{$}\NormalTok{f\_Tipo.Comida.Restaurante }\OtherTok{\textless{}{-}} \FunctionTok{factor}\NormalTok{(datos}\SpecialCharTok{$}\StringTok{\textasciigrave{}}\AttributeTok{Tipo.Comida\_Comida preparada en un bar o restaurante (Excluyendo comida rápida)}\StringTok{\textasciigrave{}}\NormalTok{, }
                                          \AttributeTok{levels =} \FunctionTok{c}\NormalTok{(}\DecValTok{0}\NormalTok{, }\DecValTok{1}\NormalTok{), }\AttributeTok{labels =} \FunctionTok{c}\NormalTok{(}\StringTok{\textquotesingle{}No\textquotesingle{}}\NormalTok{, }\StringTok{\textquotesingle{}Sí\textquotesingle{}}\NormalTok{))}
\NormalTok{datos}\SpecialCharTok{$}\NormalTok{f\_Tipo.Comida.Super }\OtherTok{\textless{}{-}} \FunctionTok{factor}\NormalTok{(datos}\SpecialCharTok{$}\StringTok{\textasciigrave{}}\AttributeTok{Tipo.Comida\_Platos preparados de supermercado}\StringTok{\textasciigrave{}}\NormalTok{, }
                                    \AttributeTok{levels =} \FunctionTok{c}\NormalTok{(}\DecValTok{0}\NormalTok{, }\DecValTok{1}\NormalTok{), }\AttributeTok{labels =} \FunctionTok{c}\NormalTok{(}\StringTok{\textquotesingle{}No\textquotesingle{}}\NormalTok{, }\StringTok{\textquotesingle{}Sí\textquotesingle{}}\NormalTok{))}
\NormalTok{datos}\SpecialCharTok{$}\NormalTok{f\_Tipo.Comida.Propia }\OtherTok{\textless{}{-}} \FunctionTok{factor}\NormalTok{(datos}\SpecialCharTok{$}\StringTok{\textasciigrave{}}\AttributeTok{Tipo.Comida\_De elaboración propia}\StringTok{\textasciigrave{}}\NormalTok{, }
                                     \AttributeTok{levels =} \FunctionTok{c}\NormalTok{(}\DecValTok{0}\NormalTok{, }\DecValTok{1}\NormalTok{), }\AttributeTok{labels =} \FunctionTok{c}\NormalTok{(}\StringTok{\textquotesingle{}No\textquotesingle{}}\NormalTok{, }\StringTok{\textquotesingle{}Sí\textquotesingle{}}\NormalTok{))}

\CommentTok{\# {-}{-}{-} P6: Supermercados (Principales) {-}{-}{-}}

\NormalTok{datos}\SpecialCharTok{$}\NormalTok{f\_Super.Mercadona }\OtherTok{\textless{}{-}} \FunctionTok{factor}\NormalTok{(datos}\SpecialCharTok{$}\NormalTok{Supermercado.Habitual\_Mercadona, }\AttributeTok{levels =} \FunctionTok{c}\NormalTok{(}\DecValTok{0}\NormalTok{, }\DecValTok{1}\NormalTok{), }\AttributeTok{labels =} \FunctionTok{c}\NormalTok{(}\StringTok{\textquotesingle{}No\textquotesingle{}}\NormalTok{, }\StringTok{\textquotesingle{}Sí\textquotesingle{}}\NormalTok{))}
\NormalTok{datos}\SpecialCharTok{$}\NormalTok{f\_Super.Consum    }\OtherTok{\textless{}{-}} \FunctionTok{factor}\NormalTok{(datos}\SpecialCharTok{$}\NormalTok{Supermercado.Habitual\_Consum, }\AttributeTok{levels =} \FunctionTok{c}\NormalTok{(}\DecValTok{0}\NormalTok{, }\DecValTok{1}\NormalTok{), }\AttributeTok{labels =} \FunctionTok{c}\NormalTok{(}\StringTok{\textquotesingle{}No\textquotesingle{}}\NormalTok{, }\StringTok{\textquotesingle{}Sí\textquotesingle{}}\NormalTok{))}
\NormalTok{datos}\SpecialCharTok{$}\NormalTok{f\_Super.Carrefour }\OtherTok{\textless{}{-}} \FunctionTok{factor}\NormalTok{(datos}\SpecialCharTok{$}\NormalTok{Supermercado.Habitual\_Carrefour, }\AttributeTok{levels =} \FunctionTok{c}\NormalTok{(}\DecValTok{0}\NormalTok{, }\DecValTok{1}\NormalTok{), }\AttributeTok{labels =} \FunctionTok{c}\NormalTok{(}\StringTok{\textquotesingle{}No\textquotesingle{}}\NormalTok{, }\StringTok{\textquotesingle{}Sí\textquotesingle{}}\NormalTok{))}
\NormalTok{datos}\SpecialCharTok{$}\NormalTok{f\_Super.Lidl      }\OtherTok{\textless{}{-}} \FunctionTok{factor}\NormalTok{(datos}\SpecialCharTok{$}\NormalTok{Supermercado.Habitual\_Lidl, }\AttributeTok{levels =} \FunctionTok{c}\NormalTok{(}\DecValTok{0}\NormalTok{, }\DecValTok{1}\NormalTok{), }\AttributeTok{labels =} \FunctionTok{c}\NormalTok{(}\StringTok{\textquotesingle{}No\textquotesingle{}}\NormalTok{, }\StringTok{\textquotesingle{}Sí\textquotesingle{}}\NormalTok{))}
\NormalTok{datos}\SpecialCharTok{$}\NormalTok{f\_Super.Aldi      }\OtherTok{\textless{}{-}} \FunctionTok{factor}\NormalTok{(datos}\SpecialCharTok{$}\NormalTok{Supermercado.Habitual\_Aldi, }\AttributeTok{levels =} \FunctionTok{c}\NormalTok{(}\DecValTok{0}\NormalTok{, }\DecValTok{1}\NormalTok{), }\AttributeTok{labels =} \FunctionTok{c}\NormalTok{(}\StringTok{\textquotesingle{}No\textquotesingle{}}\NormalTok{, }\StringTok{\textquotesingle{}Sí\textquotesingle{}}\NormalTok{))}
\NormalTok{datos}\SpecialCharTok{$}\NormalTok{f\_Super.Masymas   }\OtherTok{\textless{}{-}} \FunctionTok{factor}\NormalTok{(datos}\SpecialCharTok{$}\NormalTok{Supermercado.Habitual\_Masymas, }\AttributeTok{levels =} \FunctionTok{c}\NormalTok{(}\DecValTok{0}\NormalTok{, }\DecValTok{1}\NormalTok{), }\AttributeTok{labels =} \FunctionTok{c}\NormalTok{(}\StringTok{\textquotesingle{}No\textquotesingle{}}\NormalTok{, }\StringTok{\textquotesingle{}Sí\textquotesingle{}}\NormalTok{))}
\NormalTok{datos}\SpecialCharTok{$}\NormalTok{f\_Super.Alcampo   }\OtherTok{\textless{}{-}} \FunctionTok{factor}\NormalTok{(datos}\SpecialCharTok{$}\NormalTok{Supermercado.Habitual\_Alcampo, }\AttributeTok{levels =} \FunctionTok{c}\NormalTok{(}\DecValTok{0}\NormalTok{, }\DecValTok{1}\NormalTok{), }\AttributeTok{labels =} \FunctionTok{c}\NormalTok{(}\StringTok{\textquotesingle{}No\textquotesingle{}}\NormalTok{, }\StringTok{\textquotesingle{}Sí\textquotesingle{}}\NormalTok{))}
\NormalTok{datos}\SpecialCharTok{$}\NormalTok{f\_Super.Gadis     }\OtherTok{\textless{}{-}} \FunctionTok{factor}\NormalTok{(datos}\SpecialCharTok{$}\NormalTok{Supermercado.Habitual\_Gadis, }\AttributeTok{levels =} \FunctionTok{c}\NormalTok{(}\DecValTok{0}\NormalTok{, }\DecValTok{1}\NormalTok{), }\AttributeTok{labels =} \FunctionTok{c}\NormalTok{(}\StringTok{\textquotesingle{}No\textquotesingle{}}\NormalTok{, }\StringTok{\textquotesingle{}Sí\textquotesingle{}}\NormalTok{))}
\NormalTok{datos}\SpecialCharTok{$}\NormalTok{f\_Super.Fnac      }\OtherTok{\textless{}{-}} \FunctionTok{factor}\NormalTok{(datos}\SpecialCharTok{$}\NormalTok{Supermercado.Habitual\_Fnac, }\AttributeTok{levels =} \FunctionTok{c}\NormalTok{(}\DecValTok{0}\NormalTok{, }\DecValTok{1}\NormalTok{), }\AttributeTok{labels =} \FunctionTok{c}\NormalTok{(}\StringTok{\textquotesingle{}No\textquotesingle{}}\NormalTok{, }\StringTok{\textquotesingle{}Sí\textquotesingle{}}\NormalTok{))}
\NormalTok{datos}\SpecialCharTok{$}\NormalTok{f\_Super.DIA       }\OtherTok{\textless{}{-}} \FunctionTok{factor}\NormalTok{(datos}\SpecialCharTok{$}\NormalTok{Supermercado.Habitual\_DIA, }\AttributeTok{levels =} \FunctionTok{c}\NormalTok{(}\DecValTok{0}\NormalTok{, }\DecValTok{1}\NormalTok{), }\AttributeTok{labels =} \FunctionTok{c}\NormalTok{(}\StringTok{\textquotesingle{}No\textquotesingle{}}\NormalTok{, }\StringTok{\textquotesingle{}Sí\textquotesingle{}}\NormalTok{))}
\NormalTok{datos}\SpecialCharTok{$}\NormalTok{f\_Super.Dialprix  }\OtherTok{\textless{}{-}} \FunctionTok{factor}\NormalTok{(datos}\SpecialCharTok{$}\NormalTok{Supermercado.Habitual\_Dialprix, }\AttributeTok{levels =} \FunctionTok{c}\NormalTok{(}\DecValTok{0}\NormalTok{, }\DecValTok{1}\NormalTok{), }\AttributeTok{labels =} \FunctionTok{c}\NormalTok{(}\StringTok{\textquotesingle{}No\textquotesingle{}}\NormalTok{, }\StringTok{\textquotesingle{}Sí\textquotesingle{}}\NormalTok{))}
\NormalTok{datos}\SpecialCharTok{$}\NormalTok{f\_Super.Lefties   }\OtherTok{\textless{}{-}} \FunctionTok{factor}\NormalTok{(datos}\SpecialCharTok{$}\NormalTok{Supermercado.Habitual\_Lefties, }\AttributeTok{levels =} \FunctionTok{c}\NormalTok{(}\DecValTok{0}\NormalTok{, }\DecValTok{1}\NormalTok{), }\AttributeTok{labels =} \FunctionTok{c}\NormalTok{(}\StringTok{\textquotesingle{}No\textquotesingle{}}\NormalTok{, }\StringTok{\textquotesingle{}Sí\textquotesingle{}}\NormalTok{))}

\CommentTok{\# {-}{-}{-} P10: Situaciones de Uso {-}{-}{-}}
\NormalTok{datos}\SpecialCharTok{$}\NormalTok{f\_Uso.Momento }\OtherTok{\textless{}{-}} \FunctionTok{factor}\NormalTok{(datos}\SpecialCharTok{$}\StringTok{\textasciigrave{}}\AttributeTok{Situaciones.Uso\_Para comer en el momento}\StringTok{\textasciigrave{}}\NormalTok{, }
                              \AttributeTok{levels =} \FunctionTok{c}\NormalTok{(}\DecValTok{0}\NormalTok{, }\DecValTok{1}\NormalTok{), }\AttributeTok{labels =} \FunctionTok{c}\NormalTok{(}\StringTok{\textquotesingle{}No\textquotesingle{}}\NormalTok{, }\StringTok{\textquotesingle{}Sí\textquotesingle{}}\NormalTok{))}
\NormalTok{datos}\SpecialCharTok{$}\NormalTok{f\_Uso.Llevar  }\OtherTok{\textless{}{-}} \FunctionTok{factor}\NormalTok{(datos}\SpecialCharTok{$}\StringTok{\textasciigrave{}}\AttributeTok{Situaciones.Uso\_Para llevar al trabajo/universidad}\StringTok{\textasciigrave{}}\NormalTok{, }
                              \AttributeTok{levels =} \FunctionTok{c}\NormalTok{(}\DecValTok{0}\NormalTok{, }\DecValTok{1}\NormalTok{), }\AttributeTok{labels =} \FunctionTok{c}\NormalTok{(}\StringTok{\textquotesingle{}No\textquotesingle{}}\NormalTok{, }\StringTok{\textquotesingle{}Sí\textquotesingle{}}\NormalTok{))}
\NormalTok{datos}\SpecialCharTok{$}\NormalTok{f\_Uso.Cena    }\OtherTok{\textless{}{-}} \FunctionTok{factor}\NormalTok{(datos}\SpecialCharTok{$}\StringTok{\textasciigrave{}}\AttributeTok{Situaciones.Uso\_Como cena ligera}\StringTok{\textasciigrave{}}\NormalTok{, }
                              \AttributeTok{levels =} \FunctionTok{c}\NormalTok{(}\DecValTok{0}\NormalTok{, }\DecValTok{1}\NormalTok{), }\AttributeTok{labels =} \FunctionTok{c}\NormalTok{(}\StringTok{\textquotesingle{}No\textquotesingle{}}\NormalTok{, }\StringTok{\textquotesingle{}Sí\textquotesingle{}}\NormalTok{))}
\NormalTok{datos}\SpecialCharTok{$}\NormalTok{f\_Uso.Complem }\OtherTok{\textless{}{-}} \FunctionTok{factor}\NormalTok{(datos}\SpecialCharTok{$}\StringTok{\textasciigrave{}}\AttributeTok{Situaciones.Uso\_Como complemento de otra comida}\StringTok{\textasciigrave{}}\NormalTok{, }
                              \AttributeTok{levels =} \FunctionTok{c}\NormalTok{(}\DecValTok{0}\NormalTok{, }\DecValTok{1}\NormalTok{), }\AttributeTok{labels =} \FunctionTok{c}\NormalTok{(}\StringTok{\textquotesingle{}No\textquotesingle{}}\NormalTok{, }\StringTok{\textquotesingle{}Sí\textquotesingle{}}\NormalTok{))}
\NormalTok{datos}\SpecialCharTok{$}\NormalTok{f\_Uso.Nunca   }\OtherTok{\textless{}{-}} \FunctionTok{factor}\NormalTok{(datos}\SpecialCharTok{$}\StringTok{\textasciigrave{}}\AttributeTok{Situaciones.Uso\_Nunca}\StringTok{\textasciigrave{}}\NormalTok{, }
                              \AttributeTok{levels =} \FunctionTok{c}\NormalTok{(}\DecValTok{0}\NormalTok{, }\DecValTok{1}\NormalTok{), }\AttributeTok{labels =} \FunctionTok{c}\NormalTok{(}\StringTok{\textquotesingle{}No\textquotesingle{}}\NormalTok{, }\StringTok{\textquotesingle{}Sí\textquotesingle{}}\NormalTok{))}

\CommentTok{\# {-}{-}{-} P13: Preferencia Base {-}{-}{-}}
\NormalTok{datos}\SpecialCharTok{$}\NormalTok{f\_Base.Lechuga }\OtherTok{\textless{}{-}} \FunctionTok{factor}\NormalTok{(datos}\SpecialCharTok{$}\StringTok{\textasciigrave{}}\AttributeTok{Preferencia.Base\_Lechuga iceberg o romana}\StringTok{\textasciigrave{}}\NormalTok{, }
                               \AttributeTok{levels =} \FunctionTok{c}\NormalTok{(}\DecValTok{0}\NormalTok{, }\DecValTok{1}\NormalTok{), }\AttributeTok{labels =} \FunctionTok{c}\NormalTok{(}\StringTok{\textquotesingle{}No\textquotesingle{}}\NormalTok{, }\StringTok{\textquotesingle{}Sí\textquotesingle{}}\NormalTok{))}
\CommentTok{\# Cuidado con esta línea, el nombre es muy largo y tiene símbolos:}
\NormalTok{datos}\SpecialCharTok{$}\NormalTok{f\_Base.Verde   }\OtherTok{\textless{}{-}} \FunctionTok{factor}\NormalTok{(datos}\SpecialCharTok{$}\StringTok{\textasciigrave{}}\AttributeTok{Preferencia.Base\_Mezcla verde: rúcula/canónigos/espinacas...}\StringTok{\textasciigrave{}}\NormalTok{,}
                               \AttributeTok{levels =} \FunctionTok{c}\NormalTok{(}\DecValTok{0}\NormalTok{, }\DecValTok{1}\NormalTok{), }\AttributeTok{labels =} \FunctionTok{c}\NormalTok{(}\StringTok{\textquotesingle{}No\textquotesingle{}}\NormalTok{, }\StringTok{\textquotesingle{}Sí\textquotesingle{}}\NormalTok{))}
\NormalTok{datos}\SpecialCharTok{$}\NormalTok{f\_Base.Pasta   }\OtherTok{\textless{}{-}} \FunctionTok{factor}\NormalTok{(datos}\SpecialCharTok{$}\NormalTok{Preferencia.Base\_Pasta, }
                               \AttributeTok{levels =} \FunctionTok{c}\NormalTok{(}\DecValTok{0}\NormalTok{, }\DecValTok{1}\NormalTok{), }\AttributeTok{labels =} \FunctionTok{c}\NormalTok{(}\StringTok{\textquotesingle{}No\textquotesingle{}}\NormalTok{, }\StringTok{\textquotesingle{}Sí\textquotesingle{}}\NormalTok{))}
\NormalTok{datos}\SpecialCharTok{$}\NormalTok{f\_Base.Quinoa  }\OtherTok{\textless{}{-}} \FunctionTok{factor}\NormalTok{(datos}\SpecialCharTok{$}\NormalTok{Preferencia.Base\_Quinoa, }
                               \AttributeTok{levels =} \FunctionTok{c}\NormalTok{(}\DecValTok{0}\NormalTok{, }\DecValTok{1}\NormalTok{), }\AttributeTok{labels =} \FunctionTok{c}\NormalTok{(}\StringTok{\textquotesingle{}No\textquotesingle{}}\NormalTok{, }\StringTok{\textquotesingle{}Sí\textquotesingle{}}\NormalTok{))}
\NormalTok{datos}\SpecialCharTok{$}\NormalTok{f\_Base.Cuscus  }\OtherTok{\textless{}{-}} \FunctionTok{factor}\NormalTok{(datos}\SpecialCharTok{$}\StringTok{\textasciigrave{}}\AttributeTok{Preferencia.Base\_Cuscús}\StringTok{\textasciigrave{}}\NormalTok{, }
                               \AttributeTok{levels =} \FunctionTok{c}\NormalTok{(}\DecValTok{0}\NormalTok{, }\DecValTok{1}\NormalTok{), }\AttributeTok{labels =} \FunctionTok{c}\NormalTok{(}\StringTok{\textquotesingle{}No\textquotesingle{}}\NormalTok{, }\StringTok{\textquotesingle{}Sí\textquotesingle{}}\NormalTok{))}
\NormalTok{datos}\SpecialCharTok{$}\NormalTok{f\_Base.Legum   }\OtherTok{\textless{}{-}} \FunctionTok{factor}\NormalTok{(datos}\SpecialCharTok{$}\NormalTok{Preferencia.Base\_Legumbres, }
                               \AttributeTok{levels =} \FunctionTok{c}\NormalTok{(}\DecValTok{0}\NormalTok{, }\DecValTok{1}\NormalTok{), }\AttributeTok{labels =} \FunctionTok{c}\NormalTok{(}\StringTok{\textquotesingle{}No\textquotesingle{}}\NormalTok{, }\StringTok{\textquotesingle{}Sí\textquotesingle{}}\NormalTok{))}

\CommentTok{\# {-}{-}{-} P14: Proteínas {-}{-}{-}}
\NormalTok{datos}\SpecialCharTok{$}\NormalTok{f\_Prot.Pollo   }\OtherTok{\textless{}{-}} \FunctionTok{factor}\NormalTok{(datos}\SpecialCharTok{$}\NormalTok{Preferencia.Proteina\_Pollo, }\AttributeTok{levels =} \FunctionTok{c}\NormalTok{(}\DecValTok{0}\NormalTok{, }\DecValTok{1}\NormalTok{), }\AttributeTok{labels =} \FunctionTok{c}\NormalTok{(}\StringTok{\textquotesingle{}No\textquotesingle{}}\NormalTok{, }\StringTok{\textquotesingle{}Sí\textquotesingle{}}\NormalTok{))}
\NormalTok{datos}\SpecialCharTok{$}\NormalTok{f\_Prot.Pescado }\OtherTok{\textless{}{-}} \FunctionTok{factor}\NormalTok{(datos}\SpecialCharTok{$}\NormalTok{Preferencia.Proteina\_Pescado, }\AttributeTok{levels =} \FunctionTok{c}\NormalTok{(}\DecValTok{0}\NormalTok{, }\DecValTok{1}\NormalTok{), }\AttributeTok{labels =} \FunctionTok{c}\NormalTok{(}\StringTok{\textquotesingle{}No\textquotesingle{}}\NormalTok{, }\StringTok{\textquotesingle{}Sí\textquotesingle{}}\NormalTok{))}
\NormalTok{datos}\SpecialCharTok{$}\NormalTok{f\_Prot.Huevo   }\OtherTok{\textless{}{-}} \FunctionTok{factor}\NormalTok{(datos}\SpecialCharTok{$}\NormalTok{Preferencia.Proteina\_Huevo, }\AttributeTok{levels =} \FunctionTok{c}\NormalTok{(}\DecValTok{0}\NormalTok{, }\DecValTok{1}\NormalTok{), }\AttributeTok{labels =} \FunctionTok{c}\NormalTok{(}\StringTok{\textquotesingle{}No\textquotesingle{}}\NormalTok{, }\StringTok{\textquotesingle{}Sí\textquotesingle{}}\NormalTok{))}
\NormalTok{datos}\SpecialCharTok{$}\NormalTok{f\_Prot.Queso   }\OtherTok{\textless{}{-}} \FunctionTok{factor}\NormalTok{(datos}\SpecialCharTok{$}\NormalTok{Preferencia.Proteina\_Quesos, }\AttributeTok{levels =} \FunctionTok{c}\NormalTok{(}\DecValTok{0}\NormalTok{, }\DecValTok{1}\NormalTok{), }\AttributeTok{labels =} \FunctionTok{c}\NormalTok{(}\StringTok{\textquotesingle{}No\textquotesingle{}}\NormalTok{, }\StringTok{\textquotesingle{}Sí\textquotesingle{}}\NormalTok{))}
\NormalTok{datos}\SpecialCharTok{$}\NormalTok{f\_Prot.Tofu    }\OtherTok{\textless{}{-}} \FunctionTok{factor}\NormalTok{(datos}\SpecialCharTok{$}\NormalTok{Preferencia.Proteina\_Tofu, }\AttributeTok{levels =} \FunctionTok{c}\NormalTok{(}\DecValTok{0}\NormalTok{, }\DecValTok{1}\NormalTok{), }\AttributeTok{labels =} \FunctionTok{c}\NormalTok{(}\StringTok{\textquotesingle{}No\textquotesingle{}}\NormalTok{, }\StringTok{\textquotesingle{}Sí\textquotesingle{}}\NormalTok{))}
\NormalTok{datos}\SpecialCharTok{$}\NormalTok{f\_Prot.Marisco }\OtherTok{\textless{}{-}} \FunctionTok{factor}\NormalTok{(datos}\SpecialCharTok{$}\NormalTok{Preferencia.Proteina\_Marisco, }\AttributeTok{levels =} \FunctionTok{c}\NormalTok{(}\DecValTok{0}\NormalTok{, }\DecValTok{1}\NormalTok{), }\AttributeTok{labels =} \FunctionTok{c}\NormalTok{(}\StringTok{\textquotesingle{}No\textquotesingle{}}\NormalTok{, }\StringTok{\textquotesingle{}Sí\textquotesingle{}}\NormalTok{))}
\NormalTok{datos}\SpecialCharTok{$}\NormalTok{f\_Prot.Carne   }\OtherTok{\textless{}{-}} \FunctionTok{factor}\NormalTok{(datos}\SpecialCharTok{$}\NormalTok{Preferencia.Proteina\_Carne, }\AttributeTok{levels =} \FunctionTok{c}\NormalTok{(}\DecValTok{0}\NormalTok{, }\DecValTok{1}\NormalTok{), }\AttributeTok{labels =} \FunctionTok{c}\NormalTok{(}\StringTok{\textquotesingle{}No\textquotesingle{}}\NormalTok{, }\StringTok{\textquotesingle{}Sí\textquotesingle{}}\NormalTok{))}

\CommentTok{\# {-}{-}{-} P15: Complementos {-}{-}{-}}
\NormalTok{datos}\SpecialCharTok{$}\NormalTok{f\_Complem.Frutas   }\OtherTok{\textless{}{-}} \FunctionTok{factor}\NormalTok{(datos}\SpecialCharTok{$}\StringTok{\textasciigrave{}}\AttributeTok{Preferencia.Complementos\_Frutas varias}\StringTok{\textasciigrave{}}\NormalTok{, }\AttributeTok{levels =} \FunctionTok{c}\NormalTok{(}\DecValTok{0}\NormalTok{, }\DecValTok{1}\NormalTok{), }\AttributeTok{labels =} \FunctionTok{c}\NormalTok{(}\StringTok{\textquotesingle{}No\textquotesingle{}}\NormalTok{, }\StringTok{\textquotesingle{}Sí\textquotesingle{}}\NormalTok{))}
\NormalTok{datos}\SpecialCharTok{$}\NormalTok{f\_Complem.Secos    }\OtherTok{\textless{}{-}} \FunctionTok{factor}\NormalTok{(datos}\SpecialCharTok{$}\StringTok{\textasciigrave{}}\AttributeTok{Preferencia.Complementos\_Frutos secos/semillas}\StringTok{\textasciigrave{}}\NormalTok{, }\AttributeTok{levels =} \FunctionTok{c}\NormalTok{(}\DecValTok{0}\NormalTok{, }\DecValTok{1}\NormalTok{), }\AttributeTok{labels =} \FunctionTok{c}\NormalTok{(}\StringTok{\textquotesingle{}No\textquotesingle{}}\NormalTok{, }\StringTok{\textquotesingle{}Sí\textquotesingle{}}\NormalTok{))}
\NormalTok{datos}\SpecialCharTok{$}\NormalTok{f\_Complem.Verduras }\OtherTok{\textless{}{-}} \FunctionTok{factor}\NormalTok{(datos}\SpecialCharTok{$}\StringTok{\textasciigrave{}}\AttributeTok{Preferencia.Complementos\_Verduras cocinadas}\StringTok{\textasciigrave{}}\NormalTok{, }\AttributeTok{levels =} \FunctionTok{c}\NormalTok{(}\DecValTok{0}\NormalTok{, }\DecValTok{1}\NormalTok{), }\AttributeTok{labels =} \FunctionTok{c}\NormalTok{(}\StringTok{\textquotesingle{}No\textquotesingle{}}\NormalTok{, }\StringTok{\textquotesingle{}Sí\textquotesingle{}}\NormalTok{))}
\NormalTok{datos}\SpecialCharTok{$}\NormalTok{f\_Complem.Salsas   }\OtherTok{\textless{}{-}} \FunctionTok{factor}\NormalTok{(datos}\SpecialCharTok{$}\StringTok{\textasciigrave{}}\AttributeTok{Preferencia.Complementos\_Salsas varias}\StringTok{\textasciigrave{}}\NormalTok{, }\AttributeTok{levels =} \FunctionTok{c}\NormalTok{(}\DecValTok{0}\NormalTok{, }\DecValTok{1}\NormalTok{), }\AttributeTok{labels =} \FunctionTok{c}\NormalTok{(}\StringTok{\textquotesingle{}No\textquotesingle{}}\NormalTok{, }\StringTok{\textquotesingle{}Sí\textquotesingle{}}\NormalTok{))}
\NormalTok{datos}\SpecialCharTok{$}\NormalTok{f\_Complem.Especias    }\OtherTok{\textless{}{-}} \FunctionTok{factor}\NormalTok{(datos}\SpecialCharTok{$}\NormalTok{Preferencia.Complementos\_Especias, }\AttributeTok{levels =} \FunctionTok{c}\NormalTok{(}\DecValTok{0}\NormalTok{, }\DecValTok{1}\NormalTok{), }\AttributeTok{labels =} \FunctionTok{c}\NormalTok{(}\StringTok{\textquotesingle{}No\textquotesingle{}}\NormalTok{, }\StringTok{\textquotesingle{}Sí\textquotesingle{}}\NormalTok{))}

\CommentTok{\# Verificación final}
\FunctionTok{colnames}\NormalTok{(datos)}
\end{Highlighting}
\end{Shaded}

\begin{verbatim}
##   [1] "Marca.Temporal"                                                                 
##   [2] "Motivo"                                                                         
##   [3] "Frecuencia"                                                                     
##   [4] "Lugar.Habitual"                                                                 
##   [5] "Tipo.Comida"                                                                    
##   [6] "Frecuencia.Comida.Preparada"                                                    
##   [7] "Supermercado.Habitual"                                                          
##   [8] "Importancia.Saludable"                                                          
##   [9] "Experiencia.Barra.Ensaladas"                                                    
##  [10] "Mala.Experiencia.Descripcion"                                                   
##  [11] "Situaciones.Uso"                                                                
##  [12] "Formato.Preferido"                                                              
##  [13] "Imp.Sostenibilidad"                                                             
##  [14] "Imp.Origen.Local"                                                               
##  [15] "Imp.Variedad"                                                                   
##  [16] "Imp.Sabor"                                                                      
##  [17] "Imp.Precio"                                                                     
##  [18] "Imp.Higiene"                                                                    
##  [19] "Imp.Rapidez"                                                                    
##  [20] "Preferencia.Base"                                                               
##  [21] "Preferencia.Proteina"                                                           
##  [22] "Preferencia.Complementos"                                                       
##  [23] "Disposicion.Pago"                                                               
##  [24] "Aptitud.Social"                                                                 
##  [25] "Preferencia.Dispensacion"                                                       
##  [26] "Intencion.Uso"                                                                  
##  [27] "Edad"                                                                           
##  [28] "Genero"                                                                         
##  [29] "Nivel.Educativo"                                                                
##  [30] "Situacion.Laboral"                                                              
##  [31] "f_Motivo"                                                                       
##  [32] "f_Lugar.Habitual"                                                               
##  [33] "f_Formato.Preferido"                                                            
##  [34] "f_Preferencia.Dispensacion"                                                     
##  [35] "f_Genero"                                                                       
##  [36] "f_Situacion.Laboral"                                                            
##  [37] "f_Experiencia.Barra"                                                            
##  [38] "f_Intencion.Uso"                                                                
##  [39] "f_Frecuencia"                                                                   
##  [40] "f_Frecuencia.Comida.Preparada"                                                  
##  [41] "f_Importancia.Saludable"                                                        
##  [42] "f_Imp.Sostenibilidad"                                                           
##  [43] "f_Imp.Origen.Local"                                                             
##  [44] "f_Imp.Variedad"                                                                 
##  [45] "f_Imp.Sabor"                                                                    
##  [46] "f_Imp.Precio"                                                                   
##  [47] "f_Imp.Higiene"                                                                  
##  [48] "f_Imp.Rapidez"                                                                  
##  [49] "f_Aptitud.Social"                                                               
##  [50] "f_Disposicion.Pago"                                                             
##  [51] "f_Edad"                                                                         
##  [52] "f_Nivel.Educativo"                                                              
##  [53] "Tipo.Comida_Comida preparada en un bar o restaurante (Excluyendo comida rápida)"
##  [54] "Tipo.Comida_Comida rápida"                                                      
##  [55] "Tipo.Comida_De elaboración propia"                                              
##  [56] "Tipo.Comida_Platos preparados de supermercado"                                  
##  [57] "Supermercado.Habitual_Alcampo"                                                  
##  [58] "Supermercado.Habitual_Aldi"                                                     
##  [59] "Supermercado.Habitual_Carrefour"                                                
##  [60] "Supermercado.Habitual_Consum"                                                   
##  [61] "Supermercado.Habitual_DIA"                                                      
##  [62] "Supermercado.Habitual_Dialprix"                                                 
##  [63] "Supermercado.Habitual_Fnac"                                                     
##  [64] "Supermercado.Habitual_Gadis"                                                    
##  [65] "Supermercado.Habitual_Lefties"                                                  
##  [66] "Supermercado.Habitual_Lidl"                                                     
##  [67] "Supermercado.Habitual_Masymas"                                                  
##  [68] "Supermercado.Habitual_Mercadona"                                                
##  [69] "Situaciones.Uso_Como cena ligera"                                               
##  [70] "Situaciones.Uso_Como complemento de otra comida"                                
##  [71] "Situaciones.Uso_Nunca"                                                          
##  [72] "Situaciones.Uso_Para comer en el momento"                                       
##  [73] "Situaciones.Uso_Para llevar al trabajo/universidad"                             
##  [74] "Preferencia.Base_Cuscús"                                                        
##  [75] "Preferencia.Base_Lechuga iceberg o romana"                                      
##  [76] "Preferencia.Base_Legumbres"                                                     
##  [77] "Preferencia.Base_Mezcla verde: rúcula/canónigos/espinacas..."                   
##  [78] "Preferencia.Base_Pasta"                                                         
##  [79] "Preferencia.Base_Quinoa"                                                        
##  [80] "Preferencia.Proteina_Carne"                                                     
##  [81] "Preferencia.Proteina_Huevo"                                                     
##  [82] "Preferencia.Proteina_Marisco"                                                   
##  [83] "Preferencia.Proteina_Pescado"                                                   
##  [84] "Preferencia.Proteina_Pollo"                                                     
##  [85] "Preferencia.Proteina_Quesos"                                                    
##  [86] "Preferencia.Proteina_Tofu"                                                      
##  [87] "Preferencia.Complementos_Especias"                                              
##  [88] "Preferencia.Complementos_Frutas varias"                                         
##  [89] "Preferencia.Complementos_Frutos secos/semillas"                                 
##  [90] "Preferencia.Complementos_Salsas varias"                                         
##  [91] "Preferencia.Complementos_Verduras cocinadas"                                    
##  [92] "f_Tipo.Comida.Rapida"                                                           
##  [93] "f_Tipo.Comida.Restaurante"                                                      
##  [94] "f_Tipo.Comida.Super"                                                            
##  [95] "f_Tipo.Comida.Propia"                                                           
##  [96] "f_Super.Mercadona"                                                              
##  [97] "f_Super.Consum"                                                                 
##  [98] "f_Super.Carrefour"                                                              
##  [99] "f_Super.Lidl"                                                                   
## [100] "f_Super.Aldi"                                                                   
## [101] "f_Super.Masymas"                                                                
## [102] "f_Super.Alcampo"                                                                
## [103] "f_Super.Gadis"                                                                  
## [104] "f_Super.Fnac"                                                                   
## [105] "f_Super.DIA"                                                                    
## [106] "f_Super.Dialprix"                                                               
## [107] "f_Super.Lefties"                                                                
## [108] "f_Uso.Momento"                                                                  
## [109] "f_Uso.Llevar"                                                                   
## [110] "f_Uso.Cena"                                                                     
## [111] "f_Uso.Complem"                                                                  
## [112] "f_Uso.Nunca"                                                                    
## [113] "f_Base.Lechuga"                                                                 
## [114] "f_Base.Verde"                                                                   
## [115] "f_Base.Pasta"                                                                   
## [116] "f_Base.Quinoa"                                                                  
## [117] "f_Base.Cuscus"                                                                  
## [118] "f_Base.Legum"                                                                   
## [119] "f_Prot.Pollo"                                                                   
## [120] "f_Prot.Pescado"                                                                 
## [121] "f_Prot.Huevo"                                                                   
## [122] "f_Prot.Queso"                                                                   
## [123] "f_Prot.Tofu"                                                                    
## [124] "f_Prot.Marisco"                                                                 
## [125] "f_Prot.Carne"                                                                   
## [126] "f_Complem.Frutas"                                                               
## [127] "f_Complem.Secos"                                                                
## [128] "f_Complem.Verduras"                                                             
## [129] "f_Complem.Salsas"                                                               
## [130] "f_Complem.Especias"
\end{verbatim}

\hypertarget{control-de-consistencia}{%
\subsection{Control de consistencia}\label{control-de-consistencia}}

Hemos introducido un control de consistencia en la pregunta sobre el
supermercado en el que el encuestado compra comida habitualmente. Dos de
las posibles respuestas son \texttt{Fnac} y \texttt{Lefties}, que no son
supermercados. Por tanto, tendremos que descartar aquellos cuestionarios
que tengan marcada alguna de estas respuestas, ya que es muy probable
que hayan sido respondidos al azar o de forma poca seria.

\begin{Shaded}
\begin{Highlighting}[]
\NormalTok{n.filas.bruto }\OtherTok{\textless{}{-}} \FunctionTok{dim}\NormalTok{(datos)[}\DecValTok{1}\NormalTok{]}

\NormalTok{datos }\OtherTok{\textless{}{-}} \FunctionTok{subset}\NormalTok{(datos, }\SpecialCharTok{!}\FunctionTok{grepl}\NormalTok{(}\StringTok{\textquotesingle{}Fnac|Lefties\textquotesingle{}}\NormalTok{, Supermercado.Habitual))}

\NormalTok{n.filas }\OtherTok{\textless{}{-}} \FunctionTok{dim}\NormalTok{(datos)[}\DecValTok{1}\NormalTok{]}
\end{Highlighting}
\end{Shaded}

En el trabajo de campo recogimos inicialmente 65 cuestionarios. Tras
realizar el control de consistencia, nuestra muestra efectiva es de 63
cuestionarios válidos.

\hypertarget{estaduxedsticos-descriptivos-anuxe1lisis-univariante}{%
\section{Estadísticos descriptivos (Análisis
Univariante)}\label{estaduxedsticos-descriptivos-anuxe1lisis-univariante}}

Vamos a distinguir tres tipos de variables: categóricas, numéricas y
cualitativas. Cada una tiene un tratamiento descriptivo diferente.

\hypertarget{variables-catuxe9goricas}{%
\subsection{Variables catégoricas}\label{variables-catuxe9goricas}}

\hypertarget{caso-de-variables-con-categoruxedas-carentes-de-orden}{%
\subsubsection{Caso de variables con categorías carentes de
orden}\label{caso-de-variables-con-categoruxedas-carentes-de-orden}}

\hypertarget{f_motivo}{%
\paragraph{\texorpdfstring{\texttt{f\_Motivo}}{f\_Motivo}}\label{f_motivo}}

Las categorías de \texttt{f\_Motivo} son los motivos por los que los
encuestados comen fuera de casa. Se trata de una variable cuyas
categorías no guardan una relación de orden entre sí.

\begin{Shaded}
\begin{Highlighting}[]
\CommentTok{\# Crear el data frame de frecuencias y porcentajes usando dplyr}
\NormalTok{datos\_grafico }\OtherTok{\textless{}{-}}\NormalTok{ datos }\SpecialCharTok{\%\textgreater{}\%}
  \CommentTok{\# Contar la frecuencia de cada nivel de la variable f\_Motivo}
  \FunctionTok{count}\NormalTok{(f\_Motivo, }\AttributeTok{name =} \StringTok{"Frecuencia"}\NormalTok{) }\SpecialCharTok{\%\textgreater{}\%}
  \CommentTok{\# Calcular el porcentaje y el texto de la etiqueta}
  \FunctionTok{mutate}\NormalTok{(}
    \AttributeTok{Porcentaje =}\NormalTok{ Frecuencia }\SpecialCharTok{/} \FunctionTok{sum}\NormalTok{(Frecuencia) }\SpecialCharTok{*} \DecValTok{100}\NormalTok{,}
    \AttributeTok{Etiquetas =} \FunctionTok{paste0}\NormalTok{(f\_Motivo, }\StringTok{"}\SpecialCharTok{\textbackslash{}n}\StringTok{("}\NormalTok{, }\FunctionTok{round}\NormalTok{(Porcentaje, }\DecValTok{1}\NormalTok{), }\StringTok{"\%)"}\NormalTok{) }\CommentTok{\# Etiqueta para el gráfico}
\NormalTok{  ) }\SpecialCharTok{\%\textgreater{}\%}
  \CommentTok{\# Eliminar filas con 0\% si existen, y categorías no deseadas}
  \FunctionTok{filter}\NormalTok{(Porcentaje }\SpecialCharTok{\textgreater{}} \DecValTok{0} \SpecialCharTok{\&}\NormalTok{ f\_Motivo }\SpecialCharTok{!=} \StringTok{"Total"}\NormalTok{) }\CommentTok{\# Se filtra \textquotesingle{}Total\textquotesingle{} aunque dplyr no lo añade}

\CommentTok{\# Usamos el vector de Porcentaje para el tamaño de las porciones}
\FunctionTok{pie}\NormalTok{(datos\_grafico}\SpecialCharTok{$}\NormalTok{Porcentaje,}
    \CommentTok{\# Usamos las Etiquetas que combinan nombre y porcentaje}
    \AttributeTok{labels =}\NormalTok{ datos\_grafico}\SpecialCharTok{$}\NormalTok{Etiquetas,}
    \CommentTok{\# Título}
    \AttributeTok{main =} \StringTok{\textquotesingle{}Figura 1. Motivos de compra.\textquotesingle{}}\NormalTok{,}
    \CommentTok{\# Asignar un color diferente a cada porción}
    \AttributeTok{col =}\NormalTok{ RColorBrewer}\SpecialCharTok{::}\FunctionTok{brewer.pal}\NormalTok{(}\AttributeTok{n =} \FunctionTok{nrow}\NormalTok{(datos\_grafico), }\AttributeTok{name =} \StringTok{"Set3"}\NormalTok{))}
\end{Highlighting}
\end{Shaded}

\begin{center}\includegraphics{ICO-analisis_files/figure-latex/P1-tarta-1} \end{center}

Se observa que \texttt{Conveniencia} es el principal motivo por el que
los encuestados comen fuera de casa, representando el 38.1\% de la
muestra. A continuación presentamos los resultados en formato de tabla:

\begin{Shaded}
\begin{Highlighting}[]
\CommentTok{\# Preparación de datos}
\NormalTok{df\_tabla\_motivo }\OtherTok{\textless{}{-}}\NormalTok{ datos }\SpecialCharTok{\%\textgreater{}\%}
  \CommentTok{\# Contar la frecuencia de cada nivel}
\NormalTok{  dplyr}\SpecialCharTok{::}\FunctionTok{count}\NormalTok{(f\_Motivo, }\AttributeTok{name =} \StringTok{"Frecuencia"}\NormalTok{) }\SpecialCharTok{\%\textgreater{}\%}
  \CommentTok{\# Calcular porcentajes}
\NormalTok{  dplyr}\SpecialCharTok{::}\FunctionTok{mutate}\NormalTok{(}
    \AttributeTok{Porcentaje =}\NormalTok{ Frecuencia }\SpecialCharTok{/} \FunctionTok{sum}\NormalTok{(Frecuencia) }\SpecialCharTok{*} \DecValTok{100}
\NormalTok{  ) }\SpecialCharTok{\%\textgreater{}\%}
  \CommentTok{\# Redondear y formatear los porcentajes}
\NormalTok{  dplyr}\SpecialCharTok{::}\FunctionTok{mutate}\NormalTok{(}
    \AttributeTok{Porcentaje =} \FunctionTok{paste0}\NormalTok{(}\FunctionTok{round}\NormalTok{(Porcentaje, }\DecValTok{1}\NormalTok{), }\StringTok{"\%"}\NormalTok{)}
\NormalTok{  )}

\CommentTok{\# Calcular y añadir la fila total}
\NormalTok{df\_total }\OtherTok{\textless{}{-}} \FunctionTok{data.frame}\NormalTok{(}
  \AttributeTok{f\_Motivo =} \StringTok{"Total"}\NormalTok{,}
  \AttributeTok{Frecuencia =} \FunctionTok{sum}\NormalTok{(df\_tabla\_motivo}\SpecialCharTok{$}\NormalTok{Frecuencia),}
  \AttributeTok{Porcentaje =} \StringTok{"100.0\%"}
\NormalTok{)}

\CommentTok{\# Unir el data frame de frecuencias con la fila total}
\NormalTok{df\_tabla\_final }\OtherTok{\textless{}{-}} \FunctionTok{bind\_rows}\NormalTok{(df\_tabla\_motivo, df\_total)}

\CommentTok{\# Generación de la tabla (flextable)}
\NormalTok{ft }\OtherTok{\textless{}{-}}\NormalTok{ flextable}\SpecialCharTok{::}\FunctionTok{flextable}\NormalTok{(df\_tabla\_final) }\SpecialCharTok{\%\textgreater{}\%}

\NormalTok{flextable}\SpecialCharTok{::}\FunctionTok{set\_caption}\NormalTok{(}\AttributeTok{caption =} \StringTok{"Tabla 1. Distribución de frecuencias para Motivo de Comer Fuera."}\NormalTok{) }\SpecialCharTok{\%\textgreater{}\%}
  
  \CommentTok{\# Renombrar las cabeceras}
\NormalTok{  flextable}\SpecialCharTok{::}\FunctionTok{set\_header\_labels}\NormalTok{(}
    \AttributeTok{f\_Motivo =} \StringTok{"Motivo de Comer Fuera"}\NormalTok{,}
    \AttributeTok{Frecuencia =} \StringTok{"Frecuencia (n)"}\NormalTok{,}
    \AttributeTok{Porcentaje =} \StringTok{"Porcentaje"}
\NormalTok{  ) }\SpecialCharTok{\%\textgreater{}\%}
  
  \CommentTok{\# Aplicar negrita y bordes a la fila "Total"}
\NormalTok{  flextable}\SpecialCharTok{::}\FunctionTok{bold}\NormalTok{(}\AttributeTok{i =} \FunctionTok{nrow}\NormalTok{(df\_tabla\_final), }\AttributeTok{part =} \StringTok{"body"}\NormalTok{) }\SpecialCharTok{\%\textgreater{}\%} \CommentTok{\# Negrita a la última fila (Total)}
\NormalTok{  flextable}\SpecialCharTok{::}\FunctionTok{border\_remove}\NormalTok{() }\SpecialCharTok{\%\textgreater{}\%} \CommentTok{\# Quitar bordes predeterminados}
\NormalTok{  flextable}\SpecialCharTok{::}\FunctionTok{theme\_booktabs}\NormalTok{() }\SpecialCharTok{\%\textgreater{}\%} \CommentTok{\# Aplicar un tema con líneas horizontales}
  
  \CommentTok{\# Formato de alineación y cabecera}
\NormalTok{  flextable}\SpecialCharTok{::}\FunctionTok{align}\NormalTok{(}\AttributeTok{j =} \DecValTok{1}\NormalTok{, }\AttributeTok{align =} \StringTok{"left"}\NormalTok{, }\AttributeTok{part =} \StringTok{"body"}\NormalTok{) }\SpecialCharTok{\%\textgreater{}\%}
\NormalTok{  flextable}\SpecialCharTok{::}\FunctionTok{align}\NormalTok{(}\AttributeTok{j =} \DecValTok{2}\SpecialCharTok{:}\DecValTok{3}\NormalTok{, }\AttributeTok{align =} \StringTok{"center"}\NormalTok{, }\AttributeTok{part =} \StringTok{"all"}\NormalTok{) }\SpecialCharTok{\%\textgreater{}\%} \CommentTok{\# Columnas 2 y 3 (Datos) CENTRADAS}
\NormalTok{  flextable}\SpecialCharTok{::}\FunctionTok{align}\NormalTok{(}\AttributeTok{align =} \StringTok{"center"}\NormalTok{, }\AttributeTok{part =} \StringTok{"header"}\NormalTok{) }\SpecialCharTok{\%\textgreater{}\%}        \CommentTok{\# Encabezados CENTRADOS}
  
  \CommentTok{\# Añadir una línea superior a la fila "Total" para separarla}
\NormalTok{  flextable}\SpecialCharTok{::}\FunctionTok{hline}\NormalTok{(}\AttributeTok{i =} \FunctionTok{nrow}\NormalTok{(df\_tabla\_final) }\SpecialCharTok{{-}} \DecValTok{1}\NormalTok{, }\AttributeTok{border =}\NormalTok{ officer}\SpecialCharTok{::}\FunctionTok{fp\_border}\NormalTok{(}\AttributeTok{width =} \FloatTok{1.5}\NormalTok{, }\AttributeTok{color =} \StringTok{"black"}\NormalTok{)) }\SpecialCharTok{\%\textgreater{}\%}
  
  \CommentTok{\# Ajustar el ancho de las columnas}
\NormalTok{  flextable}\SpecialCharTok{::}\FunctionTok{autofit}\NormalTok{()}

\CommentTok{\# Mostrar la tabla}
\NormalTok{ft}
\end{Highlighting}
\end{Shaded}

\global\setlength{\Oldarrayrulewidth}{\arrayrulewidth}

\global\setlength{\Oldtabcolsep}{\tabcolsep}

\setlength{\tabcolsep}{2pt}

\renewcommand*{\arraystretch}{1.5}



\providecommand{\ascline}[3]{\noalign{\global\arrayrulewidth #1}\arrayrulecolor[HTML]{#2}\cline{#3}}

\begin{longtable}[c]{|p{1.89in}|p{1.27in}|p{1.02in}}

\caption{Tabla\ 1.\ Distribución\ de\ frecuencias\ para\ Motivo\ de\ Comer\ Fuera.}\\

\ascline{1.5pt}{666666}{1-3}

\multicolumn{1}{>{\centering}m{\dimexpr 1.89in+0\tabcolsep}}{\textcolor[HTML]{000000}{\fontsize{11}{11}\selectfont{Motivo\ de\ Comer\ Fuera}}} & \multicolumn{1}{>{\centering}m{\dimexpr 1.27in+0\tabcolsep}}{\textcolor[HTML]{000000}{\fontsize{11}{11}\selectfont{Frecuencia\ (n)}}} & \multicolumn{1}{>{\centering}m{\dimexpr 1.02in+0\tabcolsep}}{\textcolor[HTML]{000000}{\fontsize{11}{11}\selectfont{Porcentaje}}} \\

\ascline{1.5pt}{666666}{1-3}\endfirsthead \caption[]{Tabla\ 1.\ Distribución\ de\ frecuencias\ para\ Motivo\ de\ Comer\ Fuera.}\\

\ascline{1.5pt}{666666}{1-3}

\multicolumn{1}{>{\centering}m{\dimexpr 1.89in+0\tabcolsep}}{\textcolor[HTML]{000000}{\fontsize{11}{11}\selectfont{Motivo\ de\ Comer\ Fuera}}} & \multicolumn{1}{>{\centering}m{\dimexpr 1.27in+0\tabcolsep}}{\textcolor[HTML]{000000}{\fontsize{11}{11}\selectfont{Frecuencia\ (n)}}} & \multicolumn{1}{>{\centering}m{\dimexpr 1.02in+0\tabcolsep}}{\textcolor[HTML]{000000}{\fontsize{11}{11}\selectfont{Porcentaje}}} \\

\ascline{1.5pt}{666666}{1-3}\endhead



\multicolumn{1}{>{\raggedright}m{\dimexpr 1.89in+0\tabcolsep}}{\textcolor[HTML]{000000}{\fontsize{11}{11}\selectfont{Conveniencia}}} & \multicolumn{1}{>{\centering}m{\dimexpr 1.27in+0\tabcolsep}}{\textcolor[HTML]{000000}{\fontsize{11}{11}\selectfont{24}}} & \multicolumn{1}{>{\centering}m{\dimexpr 1.02in+0\tabcolsep}}{\textcolor[HTML]{000000}{\fontsize{11}{11}\selectfont{38.1\%}}} \\





\multicolumn{1}{>{\raggedright}m{\dimexpr 1.89in+0\tabcolsep}}{\textcolor[HTML]{000000}{\fontsize{11}{11}\selectfont{Social}}} & \multicolumn{1}{>{\centering}m{\dimexpr 1.27in+0\tabcolsep}}{\textcolor[HTML]{000000}{\fontsize{11}{11}\selectfont{22}}} & \multicolumn{1}{>{\centering}m{\dimexpr 1.02in+0\tabcolsep}}{\textcolor[HTML]{000000}{\fontsize{11}{11}\selectfont{34.9\%}}} \\





\multicolumn{1}{>{\raggedright}m{\dimexpr 1.89in+0\tabcolsep}}{\textcolor[HTML]{000000}{\fontsize{11}{11}\selectfont{Placer}}} & \multicolumn{1}{>{\centering}m{\dimexpr 1.27in+0\tabcolsep}}{\textcolor[HTML]{000000}{\fontsize{11}{11}\selectfont{13}}} & \multicolumn{1}{>{\centering}m{\dimexpr 1.02in+0\tabcolsep}}{\textcolor[HTML]{000000}{\fontsize{11}{11}\selectfont{20.6\%}}} \\





\multicolumn{1}{>{\raggedright}m{\dimexpr 1.89in+0\tabcolsep}}{\textcolor[HTML]{000000}{\fontsize{11}{11}\selectfont{Extraordinario}}} & \multicolumn{1}{>{\centering}m{\dimexpr 1.27in+0\tabcolsep}}{\textcolor[HTML]{000000}{\fontsize{11}{11}\selectfont{4}}} & \multicolumn{1}{>{\centering}m{\dimexpr 1.02in+0\tabcolsep}}{\textcolor[HTML]{000000}{\fontsize{11}{11}\selectfont{6.3\%}}} \\

\ascline{1.5pt}{000000}{1-3}



\multicolumn{1}{>{\raggedright}m{\dimexpr 1.89in+0\tabcolsep}}{\textcolor[HTML]{000000}{\fontsize{11}{11}\selectfont{\textbf{Total}}}} & \multicolumn{1}{>{\centering}m{\dimexpr 1.27in+0\tabcolsep}}{\textcolor[HTML]{000000}{\fontsize{11}{11}\selectfont{\textbf{63}}}} & \multicolumn{1}{>{\centering}m{\dimexpr 1.02in+0\tabcolsep}}{\textcolor[HTML]{000000}{\fontsize{11}{11}\selectfont{\textbf{100.0\%}}}} \\

\ascline{1.5pt}{666666}{1-3}



\end{longtable}



\arrayrulecolor[HTML]{000000}

\global\setlength{\arrayrulewidth}{\Oldarrayrulewidth}

\global\setlength{\tabcolsep}{\Oldtabcolsep}

\renewcommand*{\arraystretch}{1}

\hypertarget{f_lugar.habitual}{%
\paragraph{\texorpdfstring{\texttt{f\_Lugar.Habitual}}{f\_Lugar.Habitual}}\label{f_lugar.habitual}}

Las categorías de \texttt{f\_Lugar.Habitual} son los lugares en los que
los encuestados comen fuera de casa. Se trata de una variable cuyas
categorías no guardan una relación de orden entre sí.

\begin{Shaded}
\begin{Highlighting}[]
\CommentTok{\# Crear el data frame de frecuencias y porcentajes usando dplyr}
\NormalTok{datos\_grafico }\OtherTok{\textless{}{-}}\NormalTok{ datos }\SpecialCharTok{\%\textgreater{}\%}
  \CommentTok{\# Contar la frecuencia de cada nivel de la variable f\_Lugar.Habitual}
  \FunctionTok{count}\NormalTok{(f\_Lugar.Habitual, }\AttributeTok{name =} \StringTok{"Frecuencia"}\NormalTok{) }\SpecialCharTok{\%\textgreater{}\%}
  \CommentTok{\# Calcular el porcentaje y el texto de la etiqueta}
  \FunctionTok{mutate}\NormalTok{(}
    \AttributeTok{Porcentaje =}\NormalTok{ Frecuencia }\SpecialCharTok{/} \FunctionTok{sum}\NormalTok{(Frecuencia) }\SpecialCharTok{*} \DecValTok{100}\NormalTok{,}
    \AttributeTok{Etiquetas =} \FunctionTok{paste0}\NormalTok{(f\_Lugar.Habitual, }\StringTok{"}\SpecialCharTok{\textbackslash{}n}\StringTok{("}\NormalTok{, }\FunctionTok{round}\NormalTok{(Porcentaje, }\DecValTok{1}\NormalTok{), }\StringTok{"\%)"}\NormalTok{) }\CommentTok{\# Etiqueta para el gráfico}
\NormalTok{  ) }\SpecialCharTok{\%\textgreater{}\%}
  \CommentTok{\# Eliminar filas con 0\% si existen, y categorías no deseadas}
  \FunctionTok{filter}\NormalTok{(Porcentaje }\SpecialCharTok{\textgreater{}} \DecValTok{0} \SpecialCharTok{\&}\NormalTok{ f\_Lugar.Habitual }\SpecialCharTok{!=} \StringTok{"Total"}\NormalTok{) }\CommentTok{\# Se filtra \textquotesingle{}Total\textquotesingle{} aunque dplyr no lo añade}

\CommentTok{\# Usamos el vector de Porcentaje para el tamaño de las porciones}
\FunctionTok{pie}\NormalTok{(datos\_grafico}\SpecialCharTok{$}\NormalTok{Porcentaje,}
    \CommentTok{\# Usamos las Etiquetas que combinan nombre y porcentaje}
    \AttributeTok{labels =}\NormalTok{ datos\_grafico}\SpecialCharTok{$}\NormalTok{Etiquetas,}
    \CommentTok{\# Título}
    \AttributeTok{main =} \StringTok{\textquotesingle{}Figura 2. Lugar habitual.\textquotesingle{}}\NormalTok{,}
    \CommentTok{\# Asignar un color diferente a cada porción}
    \AttributeTok{col =}\NormalTok{ RColorBrewer}\SpecialCharTok{::}\FunctionTok{brewer.pal}\NormalTok{(}\AttributeTok{n =} \FunctionTok{nrow}\NormalTok{(datos\_grafico), }\AttributeTok{name =} \StringTok{"Set3"}\NormalTok{))}
\end{Highlighting}
\end{Shaded}

\begin{center}\includegraphics{ICO-analisis_files/figure-latex/P3-tarta-1} \end{center}

Se observa que \texttt{Restaurante/Bar} es el principal lugar en el que
los encuestados comen fuera de casa, representando el 66.7\% de la
muestra. A continuación presentamos los resultados en formato de tabla:

\begin{Shaded}
\begin{Highlighting}[]
\CommentTok{\# Preparación de datos}
\NormalTok{df\_tabla\_lugar }\OtherTok{\textless{}{-}}\NormalTok{ datos }\SpecialCharTok{\%\textgreater{}\%}
  \CommentTok{\# Contar la frecuencia de cada nivel}
\NormalTok{  dplyr}\SpecialCharTok{::}\FunctionTok{count}\NormalTok{(f\_Lugar.Habitual, }\AttributeTok{name =} \StringTok{"Frecuencia"}\NormalTok{) }\SpecialCharTok{\%\textgreater{}\%}
  \CommentTok{\# Calcular porcentajes}
\NormalTok{  dplyr}\SpecialCharTok{::}\FunctionTok{mutate}\NormalTok{(}
    \AttributeTok{Porcentaje =}\NormalTok{ Frecuencia }\SpecialCharTok{/} \FunctionTok{sum}\NormalTok{(Frecuencia) }\SpecialCharTok{*} \DecValTok{100}
\NormalTok{  ) }\SpecialCharTok{\%\textgreater{}\%}
  \CommentTok{\# Redondear y formatear los porcentajes}
\NormalTok{  dplyr}\SpecialCharTok{::}\FunctionTok{mutate}\NormalTok{(}
    \AttributeTok{Porcentaje =} \FunctionTok{paste0}\NormalTok{(}\FunctionTok{round}\NormalTok{(Porcentaje, }\DecValTok{1}\NormalTok{), }\StringTok{"\%"}\NormalTok{)}
\NormalTok{  )}

\CommentTok{\# Calcular y añadir la fila total}
\NormalTok{df\_total }\OtherTok{\textless{}{-}} \FunctionTok{data.frame}\NormalTok{(}
  \AttributeTok{f\_Lugar.Habitual =} \StringTok{"Total"}\NormalTok{,}
  \AttributeTok{Frecuencia =} \FunctionTok{sum}\NormalTok{(df\_tabla\_lugar}\SpecialCharTok{$}\NormalTok{Frecuencia),}
  \AttributeTok{Porcentaje =} \StringTok{"100.0\%"}
\NormalTok{)}

\CommentTok{\# Unir el data frame de frecuencias con la fila total}
\NormalTok{df\_tabla\_final }\OtherTok{\textless{}{-}} \FunctionTok{bind\_rows}\NormalTok{(df\_tabla\_lugar, df\_total)}

\CommentTok{\# Generación de la tabla (flextable)}
\NormalTok{ft }\OtherTok{\textless{}{-}}\NormalTok{ flextable}\SpecialCharTok{::}\FunctionTok{flextable}\NormalTok{(df\_tabla\_final) }\SpecialCharTok{\%\textgreater{}\%}

\NormalTok{flextable}\SpecialCharTok{::}\FunctionTok{set\_caption}\NormalTok{(}\AttributeTok{caption =} \StringTok{"Tabla 2. Distribución de frecuencias para Lugar habitual en el que se come fuera de casa."}\NormalTok{) }\SpecialCharTok{\%\textgreater{}\%}
  
  \CommentTok{\# Renombrar las cabeceras}
\NormalTok{  flextable}\SpecialCharTok{::}\FunctionTok{set\_header\_labels}\NormalTok{(}
    \AttributeTok{f\_Lugar.Habitual =} \StringTok{"Lugar Habitual"}\NormalTok{,}
    \AttributeTok{Frecuencia =} \StringTok{"Frecuencia (n)"}\NormalTok{,}
    \AttributeTok{Porcentaje =} \StringTok{"Porcentaje"}
\NormalTok{  ) }\SpecialCharTok{\%\textgreater{}\%}
  
  \CommentTok{\# Aplicar negrita y bordes a la fila "Total"}
\NormalTok{  flextable}\SpecialCharTok{::}\FunctionTok{bold}\NormalTok{(}\AttributeTok{i =} \FunctionTok{nrow}\NormalTok{(df\_tabla\_final), }\AttributeTok{part =} \StringTok{"body"}\NormalTok{) }\SpecialCharTok{\%\textgreater{}\%} \CommentTok{\# Negrita a la última fila (Total)}
\NormalTok{  flextable}\SpecialCharTok{::}\FunctionTok{border\_remove}\NormalTok{() }\SpecialCharTok{\%\textgreater{}\%} \CommentTok{\# Quitar bordes predeterminados}
\NormalTok{  flextable}\SpecialCharTok{::}\FunctionTok{theme\_booktabs}\NormalTok{() }\SpecialCharTok{\%\textgreater{}\%} \CommentTok{\# Aplicar un tema con líneas horizontales}
  
  \CommentTok{\# Formato de alineación y cabecera}
\NormalTok{  flextable}\SpecialCharTok{::}\FunctionTok{align}\NormalTok{(}\AttributeTok{j =} \DecValTok{1}\NormalTok{, }\AttributeTok{align =} \StringTok{"left"}\NormalTok{, }\AttributeTok{part =} \StringTok{"body"}\NormalTok{) }\SpecialCharTok{\%\textgreater{}\%}
\NormalTok{  flextable}\SpecialCharTok{::}\FunctionTok{align}\NormalTok{(}\AttributeTok{j =} \DecValTok{2}\SpecialCharTok{:}\DecValTok{3}\NormalTok{, }\AttributeTok{align =} \StringTok{"center"}\NormalTok{, }\AttributeTok{part =} \StringTok{"all"}\NormalTok{) }\SpecialCharTok{\%\textgreater{}\%} \CommentTok{\# Columnas 2 y 3 (Datos) CENTRADAS}
\NormalTok{  flextable}\SpecialCharTok{::}\FunctionTok{align}\NormalTok{(}\AttributeTok{align =} \StringTok{"center"}\NormalTok{, }\AttributeTok{part =} \StringTok{"header"}\NormalTok{) }\SpecialCharTok{\%\textgreater{}\%}        \CommentTok{\# Encabezados CENTRADOS}
  
  \CommentTok{\# Añadir una línea superior a la fila "Total" para separarla}
\NormalTok{  flextable}\SpecialCharTok{::}\FunctionTok{hline}\NormalTok{(}\AttributeTok{i =} \FunctionTok{nrow}\NormalTok{(df\_tabla\_final) }\SpecialCharTok{{-}} \DecValTok{1}\NormalTok{, }\AttributeTok{border =}\NormalTok{ officer}\SpecialCharTok{::}\FunctionTok{fp\_border}\NormalTok{(}\AttributeTok{width =} \FloatTok{1.5}\NormalTok{, }\AttributeTok{color =} \StringTok{"black"}\NormalTok{)) }\SpecialCharTok{\%\textgreater{}\%}
  
  \CommentTok{\# Ajustar el ancho de las columnas}
\NormalTok{  flextable}\SpecialCharTok{::}\FunctionTok{autofit}\NormalTok{()}

\CommentTok{\# Mostrar la tabla}
\NormalTok{ft}
\end{Highlighting}
\end{Shaded}

\global\setlength{\Oldarrayrulewidth}{\arrayrulewidth}

\global\setlength{\Oldtabcolsep}{\tabcolsep}

\setlength{\tabcolsep}{2pt}

\renewcommand*{\arraystretch}{1.5}



\providecommand{\ascline}[3]{\noalign{\global\arrayrulewidth #1}\arrayrulecolor[HTML]{#2}\cline{#3}}

\begin{longtable}[c]{|p{1.40in}|p{1.27in}|p{1.02in}}

\caption{Tabla\ 2.\ Distribución\ de\ frecuencias\ para\ Lugar\ habitual\ en\ el\ que\ se\ come\ fuera\ de\ casa.}\\

\ascline{1.5pt}{666666}{1-3}

\multicolumn{1}{>{\centering}m{\dimexpr 1.4in+0\tabcolsep}}{\textcolor[HTML]{000000}{\fontsize{11}{11}\selectfont{Lugar\ Habitual}}} & \multicolumn{1}{>{\centering}m{\dimexpr 1.27in+0\tabcolsep}}{\textcolor[HTML]{000000}{\fontsize{11}{11}\selectfont{Frecuencia\ (n)}}} & \multicolumn{1}{>{\centering}m{\dimexpr 1.02in+0\tabcolsep}}{\textcolor[HTML]{000000}{\fontsize{11}{11}\selectfont{Porcentaje}}} \\

\ascline{1.5pt}{666666}{1-3}\endfirsthead \caption[]{Tabla\ 2.\ Distribución\ de\ frecuencias\ para\ Lugar\ habitual\ en\ el\ que\ se\ come\ fuera\ de\ casa.}\\

\ascline{1.5pt}{666666}{1-3}

\multicolumn{1}{>{\centering}m{\dimexpr 1.4in+0\tabcolsep}}{\textcolor[HTML]{000000}{\fontsize{11}{11}\selectfont{Lugar\ Habitual}}} & \multicolumn{1}{>{\centering}m{\dimexpr 1.27in+0\tabcolsep}}{\textcolor[HTML]{000000}{\fontsize{11}{11}\selectfont{Frecuencia\ (n)}}} & \multicolumn{1}{>{\centering}m{\dimexpr 1.02in+0\tabcolsep}}{\textcolor[HTML]{000000}{\fontsize{11}{11}\selectfont{Porcentaje}}} \\

\ascline{1.5pt}{666666}{1-3}\endhead



\multicolumn{1}{>{\raggedright}m{\dimexpr 1.4in+0\tabcolsep}}{\textcolor[HTML]{000000}{\fontsize{11}{11}\selectfont{Restaurante/Bar}}} & \multicolumn{1}{>{\centering}m{\dimexpr 1.27in+0\tabcolsep}}{\textcolor[HTML]{000000}{\fontsize{11}{11}\selectfont{42}}} & \multicolumn{1}{>{\centering}m{\dimexpr 1.02in+0\tabcolsep}}{\textcolor[HTML]{000000}{\fontsize{11}{11}\selectfont{66.7\%}}} \\





\multicolumn{1}{>{\raggedright}m{\dimexpr 1.4in+0\tabcolsep}}{\textcolor[HTML]{000000}{\fontsize{11}{11}\selectfont{Oficina/Centro}}} & \multicolumn{1}{>{\centering}m{\dimexpr 1.27in+0\tabcolsep}}{\textcolor[HTML]{000000}{\fontsize{11}{11}\selectfont{17}}} & \multicolumn{1}{>{\centering}m{\dimexpr 1.02in+0\tabcolsep}}{\textcolor[HTML]{000000}{\fontsize{11}{11}\selectfont{27\%}}} \\





\multicolumn{1}{>{\raggedright}m{\dimexpr 1.4in+0\tabcolsep}}{\textcolor[HTML]{000000}{\fontsize{11}{11}\selectfont{Vía\ Pública}}} & \multicolumn{1}{>{\centering}m{\dimexpr 1.27in+0\tabcolsep}}{\textcolor[HTML]{000000}{\fontsize{11}{11}\selectfont{4}}} & \multicolumn{1}{>{\centering}m{\dimexpr 1.02in+0\tabcolsep}}{\textcolor[HTML]{000000}{\fontsize{11}{11}\selectfont{6.3\%}}} \\

\ascline{1.5pt}{000000}{1-3}



\multicolumn{1}{>{\raggedright}m{\dimexpr 1.4in+0\tabcolsep}}{\textcolor[HTML]{000000}{\fontsize{11}{11}\selectfont{\textbf{Total}}}} & \multicolumn{1}{>{\centering}m{\dimexpr 1.27in+0\tabcolsep}}{\textcolor[HTML]{000000}{\fontsize{11}{11}\selectfont{\textbf{63}}}} & \multicolumn{1}{>{\centering}m{\dimexpr 1.02in+0\tabcolsep}}{\textcolor[HTML]{000000}{\fontsize{11}{11}\selectfont{\textbf{100.0\%}}}} \\

\ascline{1.5pt}{666666}{1-3}



\end{longtable}



\arrayrulecolor[HTML]{000000}

\global\setlength{\arrayrulewidth}{\Oldarrayrulewidth}

\global\setlength{\tabcolsep}{\Oldtabcolsep}

\renewcommand*{\arraystretch}{1}

\hypertarget{f_tipo.comida}{%
\paragraph{\texorpdfstring{\texttt{f\_Tipo.Comida}}{f\_Tipo.Comida}}\label{f_tipo.comida}}

Las categorías de \texttt{f\_Tipo.Comida} hacen referencia al tipo de
comida que los encuestados suelen comer cuando comen fuera de casa. Es
importante destacar que esta variable proviene de una pregunta con
respuesta múltiple por lo que optamos por un gráfico de barras múltiple
basándonos en si ha se ha seleccionado cada respuesta o no.

\begin{Shaded}
\begin{Highlighting}[]
\CommentTok{\# Definir las variables a analizar (P4: Tipo de Comida)}
\NormalTok{multi\_vars }\OtherTok{\textless{}{-}} \FunctionTok{c}\NormalTok{(}\StringTok{"f\_Tipo.Comida.Rapida"}\NormalTok{, }\StringTok{"f\_Tipo.Comida.Restaurante"}\NormalTok{, }\StringTok{"f\_Tipo.Comida.Super"}\NormalTok{, }
                \StringTok{"f\_Tipo.Comida.Propia"}\NormalTok{)}

\CommentTok{\# Transformar a formato largo y contar frecuencias de SÍ/NO}
\NormalTok{df\_barras\_dobles }\OtherTok{\textless{}{-}}\NormalTok{ datos }\SpecialCharTok{\%\textgreater{}\%}
\NormalTok{  dplyr}\SpecialCharTok{::}\FunctionTok{select}\NormalTok{(}\FunctionTok{all\_of}\NormalTok{(multi\_vars)) }\SpecialCharTok{\%\textgreater{}\%}
\NormalTok{  tidyr}\SpecialCharTok{::}\FunctionTok{pivot\_longer}\NormalTok{(}
    \AttributeTok{cols =} \FunctionTok{everything}\NormalTok{(), }
    \AttributeTok{names\_to =} \StringTok{"Tipo\_de\_Comida"}\NormalTok{, }
    \AttributeTok{values\_to =} \StringTok{"Respuesta"}
\NormalTok{  ) }\SpecialCharTok{\%\textgreater{}\%}
\NormalTok{  dplyr}\SpecialCharTok{::}\FunctionTok{count}\NormalTok{(Tipo\_de\_Comida, Respuesta, }\AttributeTok{name =} \StringTok{"Frecuencia"}\NormalTok{) }\SpecialCharTok{\%\textgreater{}\%}
\NormalTok{  dplyr}\SpecialCharTok{::}\FunctionTok{mutate}\NormalTok{(}
    \AttributeTok{Tipo\_de\_Comida =} \FunctionTok{gsub}\NormalTok{(}\StringTok{"f\_Tipo.Comida."}\NormalTok{, }\StringTok{""}\NormalTok{, Tipo\_de\_Comida, }\AttributeTok{fixed =} \ConstantTok{TRUE}\NormalTok{),}
    \AttributeTok{Tipo\_de\_Comida =}\NormalTok{ tools}\SpecialCharTok{::}\FunctionTok{toTitleCase}\NormalTok{(Tipo\_de\_Comida),}
    \AttributeTok{Respuesta =} \FunctionTok{factor}\NormalTok{(Respuesta, }\AttributeTok{levels =} \FunctionTok{c}\NormalTok{(}\StringTok{"No"}\NormalTok{, }\StringTok{"Sí"}\NormalTok{)) }
\NormalTok{  )}

\CommentTok{\# Generación del Gráfico de Barras Agrupadas (Horizontal)}
\FunctionTok{ggplot}\NormalTok{(df\_barras\_dobles, }\FunctionTok{aes}\NormalTok{(}\AttributeTok{x =}\NormalTok{ Tipo\_de\_Comida, }\AttributeTok{y =}\NormalTok{ Frecuencia, }\AttributeTok{fill =}\NormalTok{ Respuesta)) }\SpecialCharTok{+}
  
  \CommentTok{\# Barras agrupadas}
  \FunctionTok{geom\_bar}\NormalTok{(}\AttributeTok{stat =} \StringTok{"identity"}\NormalTok{, }\AttributeTok{position =} \FunctionTok{position\_dodge}\NormalTok{(}\AttributeTok{width =} \FloatTok{0.9}\NormalTok{)) }\SpecialCharTok{+} 
  
  \CommentTok{\# Etiquetas de Frecuencia}
  \FunctionTok{geom\_text}\NormalTok{(}
    \FunctionTok{aes}\NormalTok{(}\AttributeTok{label =}\NormalTok{ Frecuencia),}
    \AttributeTok{position =} \FunctionTok{position\_dodge}\NormalTok{(}\AttributeTok{width =} \FloatTok{0.9}\NormalTok{),}
    \AttributeTok{hjust =} \SpecialCharTok{{-}}\FloatTok{0.2}\NormalTok{, }\CommentTok{\# Mueve la etiqueta {-}0.2 unidades a lo largo del eje X (separa de la barra)}
    \AttributeTok{size =} \FloatTok{3.5}
\NormalTok{  ) }\SpecialCharTok{+}
  
  \CommentTok{\# Títulos y Ejes}
  \FunctionTok{labs}\NormalTok{(}
    \AttributeTok{title =} \StringTok{"Figura 3. Gráfico de barras (Sí/No) por Tipo de Comida"}\NormalTok{,}
    \AttributeTok{x =} \StringTok{"Tipo de Comida Consumida"}\NormalTok{,}
    \AttributeTok{y =} \StringTok{"Número de Respuestas"}\NormalTok{,}
    \AttributeTok{fill =} \StringTok{"Consumo"}
\NormalTok{  ) }\SpecialCharTok{+}
  
  \CommentTok{\# Colores}
  \FunctionTok{scale\_fill\_manual}\NormalTok{(}\AttributeTok{values =} \FunctionTok{c}\NormalTok{(}\StringTok{"No"} \OtherTok{=} \StringTok{"\#a6cee3"}\NormalTok{, }\StringTok{"Sí"} \OtherTok{=} \StringTok{"\#1f78b4"}\NormalTok{)) }\SpecialCharTok{+} 
  
  \CommentTok{\# Invertir coordenadas para hacerlo horizontal}
  \FunctionTok{coord\_flip}\NormalTok{() }\SpecialCharTok{+} 
  
  \CommentTok{\# Ajuste del eje para que no se corten las etiquetas (Aumentamos el espacio)}
  \FunctionTok{scale\_y\_continuous}\NormalTok{(}\AttributeTok{expand =} \FunctionTok{expansion}\NormalTok{(}\AttributeTok{mult =} \FunctionTok{c}\NormalTok{(}\DecValTok{0}\NormalTok{, }\FloatTok{0.20}\NormalTok{))) }\SpecialCharTok{+} 
  
  \CommentTok{\# Ajuste de Tema}
  \FunctionTok{theme}\NormalTok{(}\AttributeTok{legend.position =} \StringTok{"bottom"}\NormalTok{)}
\end{Highlighting}
\end{Shaded}

\begin{center}\includegraphics{ICO-analisis_files/figure-latex/P4-barras-1} \end{center}

Se observa una clara dominancia del consumo de comida de restaurante
(con 43 respuestas afirmativas), seguido de comida de elaboración propia
(con 19 respuestas afirmativas). Estos resultados sugieren que los
consumidores están acostumbrados a un formato de comida que prioriza una
comida preparada y de calidad. A continuación presentamos los resultados
en formato de tabla:

\begin{Shaded}
\begin{Highlighting}[]
\CommentTok{\# Transformar a formato largo, contar frecuencias de SÍ/NO y calcular porcentajes}
\NormalTok{df\_tabla\_temp }\OtherTok{\textless{}{-}}\NormalTok{ datos }\SpecialCharTok{\%\textgreater{}\%}
\NormalTok{  dplyr}\SpecialCharTok{::}\FunctionTok{select}\NormalTok{(}\FunctionTok{all\_of}\NormalTok{(multi\_vars)) }\SpecialCharTok{\%\textgreater{}\%}
\NormalTok{  tidyr}\SpecialCharTok{::}\FunctionTok{pivot\_longer}\NormalTok{(}
    \AttributeTok{cols =} \FunctionTok{everything}\NormalTok{(), }
    \AttributeTok{names\_to =} \StringTok{"Tipo\_de\_Comida\_Raw"}\NormalTok{, }
    \AttributeTok{values\_to =} \StringTok{"Respuesta"} 
\NormalTok{  ) }\SpecialCharTok{\%\textgreater{}\%}
\NormalTok{  dplyr}\SpecialCharTok{::}\FunctionTok{count}\NormalTok{(Tipo\_de\_Comida\_Raw, Respuesta, }\AttributeTok{name =} \StringTok{"Frecuencia"}\NormalTok{) }\SpecialCharTok{\%\textgreater{}\%}
\NormalTok{  dplyr}\SpecialCharTok{::}\FunctionTok{mutate}\NormalTok{(}
    \AttributeTok{Porcentaje =}\NormalTok{ (Frecuencia }\SpecialCharTok{/}\NormalTok{ n.filas) }\SpecialCharTok{*} \DecValTok{100}
\NormalTok{  ) }\SpecialCharTok{\%\textgreater{}\%}
\NormalTok{  tidyr}\SpecialCharTok{::}\FunctionTok{pivot\_wider}\NormalTok{(}
    \AttributeTok{names\_from =}\NormalTok{ Respuesta, }
    \AttributeTok{values\_from =} \FunctionTok{c}\NormalTok{(Frecuencia, Porcentaje),}
    \AttributeTok{values\_fill =} \DecValTok{0} 
\NormalTok{  ) }\SpecialCharTok{\%\textgreater{}\%}
\NormalTok{  dplyr}\SpecialCharTok{::}\FunctionTok{mutate}\NormalTok{(}
    \AttributeTok{Tipo\_de\_Comida =} \FunctionTok{gsub}\NormalTok{(}\StringTok{"f\_Tipo.Comida."}\NormalTok{, }\StringTok{""}\NormalTok{, Tipo\_de\_Comida\_Raw, }\AttributeTok{fixed =} \ConstantTok{TRUE}\NormalTok{),}
    \AttributeTok{Tipo\_de\_Comida =}\NormalTok{ tools}\SpecialCharTok{::}\FunctionTok{toTitleCase}\NormalTok{(Tipo\_de\_Comida)}
\NormalTok{  ) }\SpecialCharTok{\%\textgreater{}\%}
\NormalTok{  dplyr}\SpecialCharTok{::}\FunctionTok{select}\NormalTok{(}
\NormalTok{    Tipo\_de\_Comida,}
    \StringTok{\textasciigrave{}}\AttributeTok{Frecuencia\_Sí}\StringTok{\textasciigrave{}}\NormalTok{, }\StringTok{\textasciigrave{}}\AttributeTok{Porcentaje\_Sí}\StringTok{\textasciigrave{}}\NormalTok{,}
    \StringTok{\textasciigrave{}}\AttributeTok{Frecuencia\_No}\StringTok{\textasciigrave{}}\NormalTok{, }\StringTok{\textasciigrave{}}\AttributeTok{Porcentaje\_No}\StringTok{\textasciigrave{}}
\NormalTok{  )}


\CommentTok{\# Fila de Total}
\NormalTok{df\_total }\OtherTok{\textless{}{-}} \FunctionTok{data.frame}\NormalTok{(}
  \AttributeTok{Tipo\_de\_Comida =} \StringTok{"Total"}\NormalTok{,}
\NormalTok{  Frecuencia\_Sí }\OtherTok{=} \FunctionTok{sum}\NormalTok{(df\_tabla\_temp}\SpecialCharTok{$}\NormalTok{Frecuencia\_Sí),}
\NormalTok{  Porcentaje\_Sí }\OtherTok{=} \ConstantTok{NA}\NormalTok{, }\CommentTok{\# No se suma, sería engañoso}
  \AttributeTok{Frecuencia\_No =} \FunctionTok{sum}\NormalTok{(df\_tabla\_temp}\SpecialCharTok{$}\NormalTok{Frecuencia\_No),}
  \AttributeTok{Porcentaje\_No =} \ConstantTok{NA} \CommentTok{\# No se suma, sería engañoso}
\NormalTok{)}

\CommentTok{\# Unir el data frame con la fila total}
\NormalTok{df\_tabla\_final }\OtherTok{\textless{}{-}} \FunctionTok{bind\_rows}\NormalTok{(df\_tabla\_temp, df\_total)}


\CommentTok{\# Generación de la tabla (flextable)}
\NormalTok{ft\_si\_no }\OtherTok{\textless{}{-}}\NormalTok{ flextable}\SpecialCharTok{::}\FunctionTok{flextable}\NormalTok{(df\_tabla\_final) }\SpecialCharTok{\%\textgreater{}\%}
  
  \CommentTok{\# Añadir Título}
\NormalTok{  flextable}\SpecialCharTok{::}\FunctionTok{set\_caption}\NormalTok{(}\AttributeTok{caption =} \StringTok{"Tabla 3. Distribución de Frecuencias (Sí/No) para Tipos de Comida"}\NormalTok{) }\SpecialCharTok{\%\textgreater{}\%}
  
  \CommentTok{\# Renombrar cabeceras (se usan para la segunda fila de la cabecera)}
\NormalTok{  flextable}\SpecialCharTok{::}\FunctionTok{set\_header\_labels}\NormalTok{(}
    \AttributeTok{Tipo\_de\_Comida =} \StringTok{"Tipo de Comida"}\NormalTok{,}
\NormalTok{    Frecuencia\_Sí }\OtherTok{=} \StringTok{"Frecuencia (n)"}\NormalTok{,}
\NormalTok{    Porcentaje\_Sí }\OtherTok{=} \StringTok{"Porcentaje"}\NormalTok{,}
    \AttributeTok{Frecuencia\_No =} \StringTok{"Frecuencia (n)"}\NormalTok{,}
    \AttributeTok{Porcentaje\_No =} \StringTok{"Porcentaje"}
\NormalTok{  ) }\SpecialCharTok{\%\textgreater{}\%}
  
  \CommentTok{\# Agrupar las columnas \textquotesingle{}Sí\textquotesingle{} y \textquotesingle{}No\textquotesingle{} en una fila superior}
\NormalTok{  flextable}\SpecialCharTok{::}\FunctionTok{add\_header\_row}\NormalTok{(}
    \AttributeTok{values =} \FunctionTok{c}\NormalTok{(}\StringTok{" "}\NormalTok{, }\StringTok{"SÍ"}\NormalTok{, }\StringTok{"NO"}\NormalTok{),}
    \AttributeTok{colwidths =} \FunctionTok{c}\NormalTok{(}\DecValTok{1}\NormalTok{, }\DecValTok{2}\NormalTok{, }\DecValTok{2}\NormalTok{)}
\NormalTok{  ) }\SpecialCharTok{\%\textgreater{}\%}
  
  \CommentTok{\# Formato y Tema}
\NormalTok{  flextable}\SpecialCharTok{::}\FunctionTok{colformat\_double}\NormalTok{(}\AttributeTok{j =} \FunctionTok{c}\NormalTok{(}\DecValTok{3}\NormalTok{, }\DecValTok{5}\NormalTok{), }\AttributeTok{digits =} \DecValTok{1}\NormalTok{, }\AttributeTok{suffix =} \StringTok{"\%"}\NormalTok{, }\AttributeTok{na\_str =} \StringTok{""}\NormalTok{) }\SpecialCharTok{\%\textgreater{}\%} 
  
\NormalTok{  flextable}\SpecialCharTok{::}\FunctionTok{theme\_booktabs}\NormalTok{() }\SpecialCharTok{\%\textgreater{}\%} 
  
  \CommentTok{\# Alineación}
  \CommentTok{\# Aplicar negrita y bordes a la fila "Total" (la última fila)}
\NormalTok{  flextable}\SpecialCharTok{::}\FunctionTok{bold}\NormalTok{(}\AttributeTok{i =} \FunctionTok{nrow}\NormalTok{(df\_tabla\_final), }\AttributeTok{part =} \StringTok{"body"}\NormalTok{) }\SpecialCharTok{\%\textgreater{}\%} 
\NormalTok{  flextable}\SpecialCharTok{::}\FunctionTok{hline}\NormalTok{(}\AttributeTok{i =} \FunctionTok{nrow}\NormalTok{(df\_tabla\_final) }\SpecialCharTok{{-}} \DecValTok{1}\NormalTok{, }\AttributeTok{border =}\NormalTok{ officer}\SpecialCharTok{::}\FunctionTok{fp\_border}\NormalTok{(}\AttributeTok{width =} \FloatTok{1.5}\NormalTok{, }\AttributeTok{color =} \StringTok{"black"}\NormalTok{)) }\SpecialCharTok{\%\textgreater{}\%}
  
  \CommentTok{\# Alineación}
\NormalTok{  flextable}\SpecialCharTok{::}\FunctionTok{align}\NormalTok{(}\AttributeTok{align =} \StringTok{"center"}\NormalTok{, }\AttributeTok{part =} \StringTok{"all"}\NormalTok{) }\SpecialCharTok{\%\textgreater{}\%}
\NormalTok{  flextable}\SpecialCharTok{::}\FunctionTok{align}\NormalTok{(}\AttributeTok{j =} \DecValTok{1}\NormalTok{, }\AttributeTok{align =} \StringTok{"left"}\NormalTok{, }\AttributeTok{part =} \StringTok{"body"}\NormalTok{) }\SpecialCharTok{\%\textgreater{}\%}
  
  \CommentTok{\# Ajustar}
\NormalTok{  flextable}\SpecialCharTok{::}\FunctionTok{autofit}\NormalTok{()}

\CommentTok{\# Mostrar la tabla}
\NormalTok{ft\_si\_no}
\end{Highlighting}
\end{Shaded}

\global\setlength{\Oldarrayrulewidth}{\arrayrulewidth}

\global\setlength{\Oldtabcolsep}{\tabcolsep}

\setlength{\tabcolsep}{2pt}

\renewcommand*{\arraystretch}{1.5}



\providecommand{\ascline}[3]{\noalign{\global\arrayrulewidth #1}\arrayrulecolor[HTML]{#2}\cline{#3}}

\begin{longtable}[c]{|p{1.36in}|p{1.27in}|p{1.02in}|p{1.27in}|p{1.02in}}

\caption{Tabla\ 3.\ Distribución\ de\ Frecuencias\ (Sí/No)\ para\ Tipos\ de\ Comida}\\

\ascline{1.5pt}{666666}{1-5}

\multicolumn{1}{>{\centering}m{\dimexpr 1.36in+0\tabcolsep}}{\textcolor[HTML]{000000}{\fontsize{11}{11}\selectfont{\ }}} & \multicolumn{2}{>{\centering}m{\dimexpr 2.29in+2\tabcolsep}}{\textcolor[HTML]{000000}{\fontsize{11}{11}\selectfont{SÍ}}} & \multicolumn{2}{>{\centering}m{\dimexpr 2.29in+2\tabcolsep}}{\textcolor[HTML]{000000}{\fontsize{11}{11}\selectfont{NO}}} \\





\multicolumn{1}{>{\centering}m{\dimexpr 1.36in+0\tabcolsep}}{\textcolor[HTML]{000000}{\fontsize{11}{11}\selectfont{Tipo\ de\ Comida}}} & \multicolumn{1}{>{\centering}m{\dimexpr 1.27in+0\tabcolsep}}{\textcolor[HTML]{000000}{\fontsize{11}{11}\selectfont{Frecuencia\ (n)}}} & \multicolumn{1}{>{\centering}m{\dimexpr 1.02in+0\tabcolsep}}{\textcolor[HTML]{000000}{\fontsize{11}{11}\selectfont{Porcentaje}}} & \multicolumn{1}{>{\centering}m{\dimexpr 1.27in+0\tabcolsep}}{\textcolor[HTML]{000000}{\fontsize{11}{11}\selectfont{Frecuencia\ (n)}}} & \multicolumn{1}{>{\centering}m{\dimexpr 1.02in+0\tabcolsep}}{\textcolor[HTML]{000000}{\fontsize{11}{11}\selectfont{Porcentaje}}} \\

\ascline{1.5pt}{666666}{1-5}\endfirsthead \caption[]{Tabla\ 3.\ Distribución\ de\ Frecuencias\ (Sí/No)\ para\ Tipos\ de\ Comida}\\

\ascline{1.5pt}{666666}{1-5}

\multicolumn{1}{>{\centering}m{\dimexpr 1.36in+0\tabcolsep}}{\textcolor[HTML]{000000}{\fontsize{11}{11}\selectfont{\ }}} & \multicolumn{2}{>{\centering}m{\dimexpr 2.29in+2\tabcolsep}}{\textcolor[HTML]{000000}{\fontsize{11}{11}\selectfont{SÍ}}} & \multicolumn{2}{>{\centering}m{\dimexpr 2.29in+2\tabcolsep}}{\textcolor[HTML]{000000}{\fontsize{11}{11}\selectfont{NO}}} \\





\multicolumn{1}{>{\centering}m{\dimexpr 1.36in+0\tabcolsep}}{\textcolor[HTML]{000000}{\fontsize{11}{11}\selectfont{Tipo\ de\ Comida}}} & \multicolumn{1}{>{\centering}m{\dimexpr 1.27in+0\tabcolsep}}{\textcolor[HTML]{000000}{\fontsize{11}{11}\selectfont{Frecuencia\ (n)}}} & \multicolumn{1}{>{\centering}m{\dimexpr 1.02in+0\tabcolsep}}{\textcolor[HTML]{000000}{\fontsize{11}{11}\selectfont{Porcentaje}}} & \multicolumn{1}{>{\centering}m{\dimexpr 1.27in+0\tabcolsep}}{\textcolor[HTML]{000000}{\fontsize{11}{11}\selectfont{Frecuencia\ (n)}}} & \multicolumn{1}{>{\centering}m{\dimexpr 1.02in+0\tabcolsep}}{\textcolor[HTML]{000000}{\fontsize{11}{11}\selectfont{Porcentaje}}} \\

\ascline{1.5pt}{666666}{1-5}\endhead



\multicolumn{1}{>{\raggedright}m{\dimexpr 1.36in+0\tabcolsep}}{\textcolor[HTML]{000000}{\fontsize{11}{11}\selectfont{Propia}}} & \multicolumn{1}{>{\centering}m{\dimexpr 1.27in+0\tabcolsep}}{\textcolor[HTML]{000000}{\fontsize{11}{11}\selectfont{19}}} & \multicolumn{1}{>{\centering}m{\dimexpr 1.02in+0\tabcolsep}}{\textcolor[HTML]{000000}{\fontsize{11}{11}\selectfont{30.2\%}}} & \multicolumn{1}{>{\centering}m{\dimexpr 1.27in+0\tabcolsep}}{\textcolor[HTML]{000000}{\fontsize{11}{11}\selectfont{44}}} & \multicolumn{1}{>{\centering}m{\dimexpr 1.02in+0\tabcolsep}}{\textcolor[HTML]{000000}{\fontsize{11}{11}\selectfont{69.8\%}}} \\





\multicolumn{1}{>{\raggedright}m{\dimexpr 1.36in+0\tabcolsep}}{\textcolor[HTML]{000000}{\fontsize{11}{11}\selectfont{Rapida}}} & \multicolumn{1}{>{\centering}m{\dimexpr 1.27in+0\tabcolsep}}{\textcolor[HTML]{000000}{\fontsize{11}{11}\selectfont{18}}} & \multicolumn{1}{>{\centering}m{\dimexpr 1.02in+0\tabcolsep}}{\textcolor[HTML]{000000}{\fontsize{11}{11}\selectfont{28.6\%}}} & \multicolumn{1}{>{\centering}m{\dimexpr 1.27in+0\tabcolsep}}{\textcolor[HTML]{000000}{\fontsize{11}{11}\selectfont{45}}} & \multicolumn{1}{>{\centering}m{\dimexpr 1.02in+0\tabcolsep}}{\textcolor[HTML]{000000}{\fontsize{11}{11}\selectfont{71.4\%}}} \\





\multicolumn{1}{>{\raggedright}m{\dimexpr 1.36in+0\tabcolsep}}{\textcolor[HTML]{000000}{\fontsize{11}{11}\selectfont{Restaurante}}} & \multicolumn{1}{>{\centering}m{\dimexpr 1.27in+0\tabcolsep}}{\textcolor[HTML]{000000}{\fontsize{11}{11}\selectfont{43}}} & \multicolumn{1}{>{\centering}m{\dimexpr 1.02in+0\tabcolsep}}{\textcolor[HTML]{000000}{\fontsize{11}{11}\selectfont{68.3\%}}} & \multicolumn{1}{>{\centering}m{\dimexpr 1.27in+0\tabcolsep}}{\textcolor[HTML]{000000}{\fontsize{11}{11}\selectfont{20}}} & \multicolumn{1}{>{\centering}m{\dimexpr 1.02in+0\tabcolsep}}{\textcolor[HTML]{000000}{\fontsize{11}{11}\selectfont{31.7\%}}} \\





\multicolumn{1}{>{\raggedright}m{\dimexpr 1.36in+0\tabcolsep}}{\textcolor[HTML]{000000}{\fontsize{11}{11}\selectfont{Super}}} & \multicolumn{1}{>{\centering}m{\dimexpr 1.27in+0\tabcolsep}}{\textcolor[HTML]{000000}{\fontsize{11}{11}\selectfont{13}}} & \multicolumn{1}{>{\centering}m{\dimexpr 1.02in+0\tabcolsep}}{\textcolor[HTML]{000000}{\fontsize{11}{11}\selectfont{20.6\%}}} & \multicolumn{1}{>{\centering}m{\dimexpr 1.27in+0\tabcolsep}}{\textcolor[HTML]{000000}{\fontsize{11}{11}\selectfont{50}}} & \multicolumn{1}{>{\centering}m{\dimexpr 1.02in+0\tabcolsep}}{\textcolor[HTML]{000000}{\fontsize{11}{11}\selectfont{79.4\%}}} \\

\ascline{1.5pt}{000000}{1-5}



\multicolumn{1}{>{\raggedright}m{\dimexpr 1.36in+0\tabcolsep}}{\textcolor[HTML]{000000}{\fontsize{11}{11}\selectfont{\textbf{Total}}}} & \multicolumn{1}{>{\centering}m{\dimexpr 1.27in+0\tabcolsep}}{\textcolor[HTML]{000000}{\fontsize{11}{11}\selectfont{\textbf{93}}}} & \multicolumn{1}{>{\centering}m{\dimexpr 1.02in+0\tabcolsep}}{\textcolor[HTML]{000000}{\fontsize{11}{11}\selectfont{\textbf{}}}} & \multicolumn{1}{>{\centering}m{\dimexpr 1.27in+0\tabcolsep}}{\textcolor[HTML]{000000}{\fontsize{11}{11}\selectfont{\textbf{159}}}} & \multicolumn{1}{>{\centering}m{\dimexpr 1.02in+0\tabcolsep}}{\textcolor[HTML]{000000}{\fontsize{11}{11}\selectfont{\textbf{}}}} \\

\ascline{1.5pt}{666666}{1-5}



\end{longtable}



\arrayrulecolor[HTML]{000000}

\global\setlength{\arrayrulewidth}{\Oldarrayrulewidth}

\global\setlength{\tabcolsep}{\Oldtabcolsep}

\renewcommand*{\arraystretch}{1}

\hypertarget{f_supermercado.habitual}{%
\paragraph{\texorpdfstring{\texttt{f\_Supermercado.Habitual}}{f\_Supermercado.Habitual}}\label{f_supermercado.habitual}}

Las categorías de \texttt{f\_Supermercado.Habitual} hacen referencia al
supermercado en el que los encuestados suelen comprar comida. Es
importante destacar que esta variable proviene de una pregunta con
respuesta múltiple por lo que optamos por un gráfico de barras múltiple
basándonos en si ha se ha seleccionado cada respuesta o no. También
destacamos que esta pregunta es de control y no tiene más trascendencia.

\begin{Shaded}
\begin{Highlighting}[]
\CommentTok{\# Definir las variables a analizar (P6: Supermercado)}
\NormalTok{multi\_vars }\OtherTok{\textless{}{-}} \FunctionTok{c}\NormalTok{(}\StringTok{"f\_Super.Carrefour"}\NormalTok{, }\StringTok{"f\_Super.Lidl"}\NormalTok{, }\StringTok{"f\_Super.Mercadona"}\NormalTok{, }\StringTok{"f\_Super.Consum"}\NormalTok{, }\StringTok{"f\_Super.Aldi"}\NormalTok{, }\StringTok{"f\_Super.Masymas"}\NormalTok{, }\StringTok{"f\_Super.Alcampo"}\NormalTok{, }\StringTok{"f\_Super.DIA"}\NormalTok{, }\StringTok{"f\_Super.Dialprix"}\NormalTok{)}
\CommentTok{\#FALTA GADIS PORQUE NO SE HA ELEGIDO}

\CommentTok{\# Transformar a formato largo y contar frecuencias de SÍ/NO}
\NormalTok{df\_barras\_dobles }\OtherTok{\textless{}{-}}\NormalTok{ datos }\SpecialCharTok{\%\textgreater{}\%}
\NormalTok{  dplyr}\SpecialCharTok{::}\FunctionTok{select}\NormalTok{(}\FunctionTok{all\_of}\NormalTok{(multi\_vars)) }\SpecialCharTok{\%\textgreater{}\%}
\NormalTok{  tidyr}\SpecialCharTok{::}\FunctionTok{pivot\_longer}\NormalTok{(}
    \AttributeTok{cols =} \FunctionTok{everything}\NormalTok{(), }
    \AttributeTok{names\_to =} \StringTok{"Supermercado\_Habitual"}\NormalTok{, }
    \AttributeTok{values\_to =} \StringTok{"Respuesta"}
\NormalTok{  ) }\SpecialCharTok{\%\textgreater{}\%}
\NormalTok{  dplyr}\SpecialCharTok{::}\FunctionTok{count}\NormalTok{(Supermercado\_Habitual, Respuesta, }\AttributeTok{name =} \StringTok{"Frecuencia"}\NormalTok{) }\SpecialCharTok{\%\textgreater{}\%}
\NormalTok{  dplyr}\SpecialCharTok{::}\FunctionTok{mutate}\NormalTok{(}
    \AttributeTok{Supermercado\_Habitual =} \FunctionTok{gsub}\NormalTok{(}\StringTok{"f\_Super."}\NormalTok{, }\StringTok{""}\NormalTok{, Supermercado\_Habitual, }\AttributeTok{fixed =} \ConstantTok{TRUE}\NormalTok{),}
    \AttributeTok{Supermercado\_Habitual =}\NormalTok{ tools}\SpecialCharTok{::}\FunctionTok{toTitleCase}\NormalTok{(Supermercado\_Habitual),}
    \AttributeTok{Respuesta =} \FunctionTok{factor}\NormalTok{(Respuesta, }\AttributeTok{levels =} \FunctionTok{c}\NormalTok{(}\StringTok{"No"}\NormalTok{, }\StringTok{"Sí"}\NormalTok{)) }
\NormalTok{  )}

\CommentTok{\# Generación del Gráfico de Barras Agrupadas (Horizontal)}
\FunctionTok{ggplot}\NormalTok{(df\_barras\_dobles, }\FunctionTok{aes}\NormalTok{(}\AttributeTok{x =}\NormalTok{ Supermercado\_Habitual, }\AttributeTok{y =}\NormalTok{ Frecuencia, }\AttributeTok{fill =}\NormalTok{ Respuesta)) }\SpecialCharTok{+}
  
  \CommentTok{\# Barras agrupadas}
  \FunctionTok{geom\_bar}\NormalTok{(}\AttributeTok{stat =} \StringTok{"identity"}\NormalTok{, }\AttributeTok{position =} \FunctionTok{position\_dodge}\NormalTok{(}\AttributeTok{width =} \FloatTok{0.9}\NormalTok{)) }\SpecialCharTok{+} 
  
  \CommentTok{\# Etiquetas de Frecuencia}
  \FunctionTok{geom\_text}\NormalTok{(}
    \FunctionTok{aes}\NormalTok{(}\AttributeTok{label =}\NormalTok{ Frecuencia),}
    \AttributeTok{position =} \FunctionTok{position\_dodge}\NormalTok{(}\AttributeTok{width =} \FloatTok{0.9}\NormalTok{),}
    \AttributeTok{hjust =} \SpecialCharTok{{-}}\FloatTok{0.2}\NormalTok{, }\CommentTok{\# Mueve la etiqueta {-}0.2 unidades a lo largo del eje X (separa de la barra)}
    \AttributeTok{size =} \FloatTok{3.5}
\NormalTok{  ) }\SpecialCharTok{+}
  
  \CommentTok{\# Títulos y Ejes}
  \FunctionTok{labs}\NormalTok{(}
    \AttributeTok{title =} \StringTok{"Figura 4. Gráfico de barras (Sí/No) por Supermercado"}\NormalTok{,}
    \AttributeTok{x =} \StringTok{"Supermercado Habitual"}\NormalTok{,}
    \AttributeTok{y =} \StringTok{"Número de Respuestas"}\NormalTok{,}
    \AttributeTok{fill =} \StringTok{"Elegido"}
\NormalTok{  ) }\SpecialCharTok{+}
  
  \CommentTok{\# Colores}
  \FunctionTok{scale\_fill\_manual}\NormalTok{(}\AttributeTok{values =} \FunctionTok{c}\NormalTok{(}\StringTok{"No"} \OtherTok{=} \StringTok{"\#a6cee3"}\NormalTok{, }\StringTok{"Sí"} \OtherTok{=} \StringTok{"\#1f78b4"}\NormalTok{)) }\SpecialCharTok{+} 
  
  \CommentTok{\# Invertir coordenadas para hacerlo horizontal}
  \FunctionTok{coord\_flip}\NormalTok{() }\SpecialCharTok{+} 
  
  \CommentTok{\# Ajuste del eje para que no se corten las etiquetas (Aumentamos el espacio)}
  \FunctionTok{scale\_y\_continuous}\NormalTok{(}\AttributeTok{expand =} \FunctionTok{expansion}\NormalTok{(}\AttributeTok{mult =} \FunctionTok{c}\NormalTok{(}\DecValTok{0}\NormalTok{, }\FloatTok{0.20}\NormalTok{))) }\SpecialCharTok{+} 
  
  \CommentTok{\# Ajuste de Tema}
  \FunctionTok{theme}\NormalTok{(}\AttributeTok{legend.position =} \StringTok{"bottom"}\NormalTok{)}
\end{Highlighting}
\end{Shaded}

\begin{center}\includegraphics{ICO-analisis_files/figure-latex/P6-barras-1} \end{center}

Se observa una clara dominancia de Mercadona (con 51 respuestas
afirmativas), seguido de Consum (con 30 respuestas afirmativas). A
continuación presentamos los resultados en formato de tabla:

\begin{Shaded}
\begin{Highlighting}[]
\CommentTok{\# Transformar a formato largo, contar frecuencias de SÍ/NO y calcular porcentajes}
\NormalTok{df\_tabla\_temp }\OtherTok{\textless{}{-}}\NormalTok{ datos }\SpecialCharTok{\%\textgreater{}\%}
\NormalTok{  dplyr}\SpecialCharTok{::}\FunctionTok{select}\NormalTok{(}\FunctionTok{all\_of}\NormalTok{(multi\_vars)) }\SpecialCharTok{\%\textgreater{}\%}
\NormalTok{  tidyr}\SpecialCharTok{::}\FunctionTok{pivot\_longer}\NormalTok{(}
    \AttributeTok{cols =} \FunctionTok{everything}\NormalTok{(), }
    \AttributeTok{names\_to =} \StringTok{"Supermercado\_Habitual\_Raw"}\NormalTok{, }
    \AttributeTok{values\_to =} \StringTok{"Respuesta"} 
\NormalTok{  ) }\SpecialCharTok{\%\textgreater{}\%}
\NormalTok{  dplyr}\SpecialCharTok{::}\FunctionTok{count}\NormalTok{(Supermercado\_Habitual\_Raw, Respuesta, }\AttributeTok{name =} \StringTok{"Frecuencia"}\NormalTok{) }\SpecialCharTok{\%\textgreater{}\%}
\NormalTok{  dplyr}\SpecialCharTok{::}\FunctionTok{mutate}\NormalTok{(}
    \AttributeTok{Porcentaje =}\NormalTok{ (Frecuencia }\SpecialCharTok{/}\NormalTok{ n.filas) }\SpecialCharTok{*} \DecValTok{100}
\NormalTok{  ) }\SpecialCharTok{\%\textgreater{}\%}
\NormalTok{  tidyr}\SpecialCharTok{::}\FunctionTok{pivot\_wider}\NormalTok{(}
    \AttributeTok{names\_from =}\NormalTok{ Respuesta, }
    \AttributeTok{values\_from =} \FunctionTok{c}\NormalTok{(Frecuencia, Porcentaje),}
    \AttributeTok{values\_fill =} \DecValTok{0} 
\NormalTok{  ) }\SpecialCharTok{\%\textgreater{}\%}
\NormalTok{  dplyr}\SpecialCharTok{::}\FunctionTok{mutate}\NormalTok{(}
    \AttributeTok{Supermercado\_Habitual =} \FunctionTok{gsub}\NormalTok{(}\StringTok{"f\_Super."}\NormalTok{, }\StringTok{""}\NormalTok{, Supermercado\_Habitual\_Raw, }\AttributeTok{fixed =} \ConstantTok{TRUE}\NormalTok{),}
    \AttributeTok{Supermercado\_Habitual =}\NormalTok{ tools}\SpecialCharTok{::}\FunctionTok{toTitleCase}\NormalTok{(Supermercado\_Habitual)}
\NormalTok{  ) }\SpecialCharTok{\%\textgreater{}\%}
\NormalTok{  dplyr}\SpecialCharTok{::}\FunctionTok{select}\NormalTok{(}
\NormalTok{    Supermercado\_Habitual,}
    \StringTok{\textasciigrave{}}\AttributeTok{Frecuencia\_Sí}\StringTok{\textasciigrave{}}\NormalTok{, }\StringTok{\textasciigrave{}}\AttributeTok{Porcentaje\_Sí}\StringTok{\textasciigrave{}}\NormalTok{,}
    \StringTok{\textasciigrave{}}\AttributeTok{Frecuencia\_No}\StringTok{\textasciigrave{}}\NormalTok{, }\StringTok{\textasciigrave{}}\AttributeTok{Porcentaje\_No}\StringTok{\textasciigrave{}}
\NormalTok{  )}


\CommentTok{\# Fila de Total}
\NormalTok{df\_total }\OtherTok{\textless{}{-}} \FunctionTok{data.frame}\NormalTok{(}
  \AttributeTok{Supermercado\_Habitual =} \StringTok{"Total"}\NormalTok{,}
\NormalTok{  Frecuencia\_Sí }\OtherTok{=} \FunctionTok{sum}\NormalTok{(df\_tabla\_temp}\SpecialCharTok{$}\NormalTok{Frecuencia\_Sí),}
\NormalTok{  Porcentaje\_Sí }\OtherTok{=} \ConstantTok{NA}\NormalTok{, }\CommentTok{\# No se suma, sería engañoso}
  \AttributeTok{Frecuencia\_No =} \FunctionTok{sum}\NormalTok{(df\_tabla\_temp}\SpecialCharTok{$}\NormalTok{Frecuencia\_No),}
  \AttributeTok{Porcentaje\_No =} \ConstantTok{NA} \CommentTok{\# No se suma, sería engañoso}
\NormalTok{)}

\CommentTok{\# Unir el data frame con la fila total}
\NormalTok{df\_tabla\_final }\OtherTok{\textless{}{-}} \FunctionTok{bind\_rows}\NormalTok{(df\_tabla\_temp, df\_total)}


\CommentTok{\# Generación de la tabla (flextable)}
\NormalTok{ft\_si\_no }\OtherTok{\textless{}{-}}\NormalTok{ flextable}\SpecialCharTok{::}\FunctionTok{flextable}\NormalTok{(df\_tabla\_final) }\SpecialCharTok{\%\textgreater{}\%}
  
  \CommentTok{\# Añadir Título}
\NormalTok{  flextable}\SpecialCharTok{::}\FunctionTok{set\_caption}\NormalTok{(}\AttributeTok{caption =} \StringTok{"Tabla 4. Distribución de Frecuencias (Sí/No) para Supermercados"}\NormalTok{) }\SpecialCharTok{\%\textgreater{}\%}
  
  \CommentTok{\# Renombrar cabeceras (se usan para la segunda fila de la cabecera)}
\NormalTok{  flextable}\SpecialCharTok{::}\FunctionTok{set\_header\_labels}\NormalTok{(}
    \AttributeTok{Supermercado\_Habitual =} \StringTok{"Supermercado Habitual"}\NormalTok{,}
\NormalTok{    Frecuencia\_Sí }\OtherTok{=} \StringTok{"Frecuencia (n)"}\NormalTok{,}
\NormalTok{    Porcentaje\_Sí }\OtherTok{=} \StringTok{"Porcentaje"}\NormalTok{,}
    \AttributeTok{Frecuencia\_No =} \StringTok{"Frecuencia (n)"}\NormalTok{,}
    \AttributeTok{Porcentaje\_No =} \StringTok{"Porcentaje"}
\NormalTok{  ) }\SpecialCharTok{\%\textgreater{}\%}
  
  \CommentTok{\# Agrupar las columnas \textquotesingle{}Sí\textquotesingle{} y \textquotesingle{}No\textquotesingle{} en una fila superior}
\NormalTok{  flextable}\SpecialCharTok{::}\FunctionTok{add\_header\_row}\NormalTok{(}
    \AttributeTok{values =} \FunctionTok{c}\NormalTok{(}\StringTok{" "}\NormalTok{, }\StringTok{"SÍ"}\NormalTok{, }\StringTok{"NO"}\NormalTok{),}
    \AttributeTok{colwidths =} \FunctionTok{c}\NormalTok{(}\DecValTok{1}\NormalTok{, }\DecValTok{2}\NormalTok{, }\DecValTok{2}\NormalTok{)}
\NormalTok{  ) }\SpecialCharTok{\%\textgreater{}\%}
  
  \CommentTok{\# Formato y Tema}
\NormalTok{  flextable}\SpecialCharTok{::}\FunctionTok{colformat\_double}\NormalTok{(}\AttributeTok{j =} \FunctionTok{c}\NormalTok{(}\DecValTok{3}\NormalTok{, }\DecValTok{5}\NormalTok{), }\AttributeTok{digits =} \DecValTok{1}\NormalTok{, }\AttributeTok{suffix =} \StringTok{"\%"}\NormalTok{, }\AttributeTok{na\_str =} \StringTok{""}\NormalTok{) }\SpecialCharTok{\%\textgreater{}\%} 
  
\NormalTok{  flextable}\SpecialCharTok{::}\FunctionTok{theme\_booktabs}\NormalTok{() }\SpecialCharTok{\%\textgreater{}\%} 
  
  \CommentTok{\# Alineación}
  \CommentTok{\# Aplicar negrita y bordes a la fila "Total" (la última fila)}
\NormalTok{  flextable}\SpecialCharTok{::}\FunctionTok{bold}\NormalTok{(}\AttributeTok{i =} \FunctionTok{nrow}\NormalTok{(df\_tabla\_final), }\AttributeTok{part =} \StringTok{"body"}\NormalTok{) }\SpecialCharTok{\%\textgreater{}\%} 
\NormalTok{  flextable}\SpecialCharTok{::}\FunctionTok{hline}\NormalTok{(}\AttributeTok{i =} \FunctionTok{nrow}\NormalTok{(df\_tabla\_final) }\SpecialCharTok{{-}} \DecValTok{1}\NormalTok{, }\AttributeTok{border =}\NormalTok{ officer}\SpecialCharTok{::}\FunctionTok{fp\_border}\NormalTok{(}\AttributeTok{width =} \FloatTok{1.5}\NormalTok{, }\AttributeTok{color =} \StringTok{"black"}\NormalTok{)) }\SpecialCharTok{\%\textgreater{}\%}
  
  \CommentTok{\# Alineación}
\NormalTok{  flextable}\SpecialCharTok{::}\FunctionTok{align}\NormalTok{(}\AttributeTok{align =} \StringTok{"center"}\NormalTok{, }\AttributeTok{part =} \StringTok{"all"}\NormalTok{) }\SpecialCharTok{\%\textgreater{}\%}
\NormalTok{  flextable}\SpecialCharTok{::}\FunctionTok{align}\NormalTok{(}\AttributeTok{j =} \DecValTok{1}\NormalTok{, }\AttributeTok{align =} \StringTok{"left"}\NormalTok{, }\AttributeTok{part =} \StringTok{"body"}\NormalTok{) }\SpecialCharTok{\%\textgreater{}\%}
  
  \CommentTok{\# Ajustar}
\NormalTok{  flextable}\SpecialCharTok{::}\FunctionTok{autofit}\NormalTok{()}

\CommentTok{\# Mostrar la tabla}
\NormalTok{ft\_si\_no}
\end{Highlighting}
\end{Shaded}

\global\setlength{\Oldarrayrulewidth}{\arrayrulewidth}

\global\setlength{\Oldtabcolsep}{\tabcolsep}

\setlength{\tabcolsep}{2pt}

\renewcommand*{\arraystretch}{1.5}



\providecommand{\ascline}[3]{\noalign{\global\arrayrulewidth #1}\arrayrulecolor[HTML]{#2}\cline{#3}}

\begin{longtable}[c]{|p{1.89in}|p{1.27in}|p{1.02in}|p{1.27in}|p{1.02in}}

\caption{Tabla\ 4.\ Distribución\ de\ Frecuencias\ (Sí/No)\ para\ Supermercados}\\

\ascline{1.5pt}{666666}{1-5}

\multicolumn{1}{>{\centering}m{\dimexpr 1.89in+0\tabcolsep}}{\textcolor[HTML]{000000}{\fontsize{11}{11}\selectfont{\ }}} & \multicolumn{2}{>{\centering}m{\dimexpr 2.29in+2\tabcolsep}}{\textcolor[HTML]{000000}{\fontsize{11}{11}\selectfont{SÍ}}} & \multicolumn{2}{>{\centering}m{\dimexpr 2.29in+2\tabcolsep}}{\textcolor[HTML]{000000}{\fontsize{11}{11}\selectfont{NO}}} \\





\multicolumn{1}{>{\centering}m{\dimexpr 1.89in+0\tabcolsep}}{\textcolor[HTML]{000000}{\fontsize{11}{11}\selectfont{Supermercado\ Habitual}}} & \multicolumn{1}{>{\centering}m{\dimexpr 1.27in+0\tabcolsep}}{\textcolor[HTML]{000000}{\fontsize{11}{11}\selectfont{Frecuencia\ (n)}}} & \multicolumn{1}{>{\centering}m{\dimexpr 1.02in+0\tabcolsep}}{\textcolor[HTML]{000000}{\fontsize{11}{11}\selectfont{Porcentaje}}} & \multicolumn{1}{>{\centering}m{\dimexpr 1.27in+0\tabcolsep}}{\textcolor[HTML]{000000}{\fontsize{11}{11}\selectfont{Frecuencia\ (n)}}} & \multicolumn{1}{>{\centering}m{\dimexpr 1.02in+0\tabcolsep}}{\textcolor[HTML]{000000}{\fontsize{11}{11}\selectfont{Porcentaje}}} \\

\ascline{1.5pt}{666666}{1-5}\endfirsthead \caption[]{Tabla\ 4.\ Distribución\ de\ Frecuencias\ (Sí/No)\ para\ Supermercados}\\

\ascline{1.5pt}{666666}{1-5}

\multicolumn{1}{>{\centering}m{\dimexpr 1.89in+0\tabcolsep}}{\textcolor[HTML]{000000}{\fontsize{11}{11}\selectfont{\ }}} & \multicolumn{2}{>{\centering}m{\dimexpr 2.29in+2\tabcolsep}}{\textcolor[HTML]{000000}{\fontsize{11}{11}\selectfont{SÍ}}} & \multicolumn{2}{>{\centering}m{\dimexpr 2.29in+2\tabcolsep}}{\textcolor[HTML]{000000}{\fontsize{11}{11}\selectfont{NO}}} \\





\multicolumn{1}{>{\centering}m{\dimexpr 1.89in+0\tabcolsep}}{\textcolor[HTML]{000000}{\fontsize{11}{11}\selectfont{Supermercado\ Habitual}}} & \multicolumn{1}{>{\centering}m{\dimexpr 1.27in+0\tabcolsep}}{\textcolor[HTML]{000000}{\fontsize{11}{11}\selectfont{Frecuencia\ (n)}}} & \multicolumn{1}{>{\centering}m{\dimexpr 1.02in+0\tabcolsep}}{\textcolor[HTML]{000000}{\fontsize{11}{11}\selectfont{Porcentaje}}} & \multicolumn{1}{>{\centering}m{\dimexpr 1.27in+0\tabcolsep}}{\textcolor[HTML]{000000}{\fontsize{11}{11}\selectfont{Frecuencia\ (n)}}} & \multicolumn{1}{>{\centering}m{\dimexpr 1.02in+0\tabcolsep}}{\textcolor[HTML]{000000}{\fontsize{11}{11}\selectfont{Porcentaje}}} \\

\ascline{1.5pt}{666666}{1-5}\endhead



\multicolumn{1}{>{\raggedright}m{\dimexpr 1.89in+0\tabcolsep}}{\textcolor[HTML]{000000}{\fontsize{11}{11}\selectfont{Alcampo}}} & \multicolumn{1}{>{\centering}m{\dimexpr 1.27in+0\tabcolsep}}{\textcolor[HTML]{000000}{\fontsize{11}{11}\selectfont{5}}} & \multicolumn{1}{>{\centering}m{\dimexpr 1.02in+0\tabcolsep}}{\textcolor[HTML]{000000}{\fontsize{11}{11}\selectfont{7.9\%}}} & \multicolumn{1}{>{\centering}m{\dimexpr 1.27in+0\tabcolsep}}{\textcolor[HTML]{000000}{\fontsize{11}{11}\selectfont{58}}} & \multicolumn{1}{>{\centering}m{\dimexpr 1.02in+0\tabcolsep}}{\textcolor[HTML]{000000}{\fontsize{11}{11}\selectfont{92.1\%}}} \\





\multicolumn{1}{>{\raggedright}m{\dimexpr 1.89in+0\tabcolsep}}{\textcolor[HTML]{000000}{\fontsize{11}{11}\selectfont{Aldi}}} & \multicolumn{1}{>{\centering}m{\dimexpr 1.27in+0\tabcolsep}}{\textcolor[HTML]{000000}{\fontsize{11}{11}\selectfont{8}}} & \multicolumn{1}{>{\centering}m{\dimexpr 1.02in+0\tabcolsep}}{\textcolor[HTML]{000000}{\fontsize{11}{11}\selectfont{12.7\%}}} & \multicolumn{1}{>{\centering}m{\dimexpr 1.27in+0\tabcolsep}}{\textcolor[HTML]{000000}{\fontsize{11}{11}\selectfont{55}}} & \multicolumn{1}{>{\centering}m{\dimexpr 1.02in+0\tabcolsep}}{\textcolor[HTML]{000000}{\fontsize{11}{11}\selectfont{87.3\%}}} \\





\multicolumn{1}{>{\raggedright}m{\dimexpr 1.89in+0\tabcolsep}}{\textcolor[HTML]{000000}{\fontsize{11}{11}\selectfont{Carrefour}}} & \multicolumn{1}{>{\centering}m{\dimexpr 1.27in+0\tabcolsep}}{\textcolor[HTML]{000000}{\fontsize{11}{11}\selectfont{19}}} & \multicolumn{1}{>{\centering}m{\dimexpr 1.02in+0\tabcolsep}}{\textcolor[HTML]{000000}{\fontsize{11}{11}\selectfont{30.2\%}}} & \multicolumn{1}{>{\centering}m{\dimexpr 1.27in+0\tabcolsep}}{\textcolor[HTML]{000000}{\fontsize{11}{11}\selectfont{44}}} & \multicolumn{1}{>{\centering}m{\dimexpr 1.02in+0\tabcolsep}}{\textcolor[HTML]{000000}{\fontsize{11}{11}\selectfont{69.8\%}}} \\





\multicolumn{1}{>{\raggedright}m{\dimexpr 1.89in+0\tabcolsep}}{\textcolor[HTML]{000000}{\fontsize{11}{11}\selectfont{Consum}}} & \multicolumn{1}{>{\centering}m{\dimexpr 1.27in+0\tabcolsep}}{\textcolor[HTML]{000000}{\fontsize{11}{11}\selectfont{30}}} & \multicolumn{1}{>{\centering}m{\dimexpr 1.02in+0\tabcolsep}}{\textcolor[HTML]{000000}{\fontsize{11}{11}\selectfont{47.6\%}}} & \multicolumn{1}{>{\centering}m{\dimexpr 1.27in+0\tabcolsep}}{\textcolor[HTML]{000000}{\fontsize{11}{11}\selectfont{33}}} & \multicolumn{1}{>{\centering}m{\dimexpr 1.02in+0\tabcolsep}}{\textcolor[HTML]{000000}{\fontsize{11}{11}\selectfont{52.4\%}}} \\





\multicolumn{1}{>{\raggedright}m{\dimexpr 1.89in+0\tabcolsep}}{\textcolor[HTML]{000000}{\fontsize{11}{11}\selectfont{DIA}}} & \multicolumn{1}{>{\centering}m{\dimexpr 1.27in+0\tabcolsep}}{\textcolor[HTML]{000000}{\fontsize{11}{11}\selectfont{3}}} & \multicolumn{1}{>{\centering}m{\dimexpr 1.02in+0\tabcolsep}}{\textcolor[HTML]{000000}{\fontsize{11}{11}\selectfont{4.8\%}}} & \multicolumn{1}{>{\centering}m{\dimexpr 1.27in+0\tabcolsep}}{\textcolor[HTML]{000000}{\fontsize{11}{11}\selectfont{60}}} & \multicolumn{1}{>{\centering}m{\dimexpr 1.02in+0\tabcolsep}}{\textcolor[HTML]{000000}{\fontsize{11}{11}\selectfont{95.2\%}}} \\





\multicolumn{1}{>{\raggedright}m{\dimexpr 1.89in+0\tabcolsep}}{\textcolor[HTML]{000000}{\fontsize{11}{11}\selectfont{Dialprix}}} & \multicolumn{1}{>{\centering}m{\dimexpr 1.27in+0\tabcolsep}}{\textcolor[HTML]{000000}{\fontsize{11}{11}\selectfont{2}}} & \multicolumn{1}{>{\centering}m{\dimexpr 1.02in+0\tabcolsep}}{\textcolor[HTML]{000000}{\fontsize{11}{11}\selectfont{3.2\%}}} & \multicolumn{1}{>{\centering}m{\dimexpr 1.27in+0\tabcolsep}}{\textcolor[HTML]{000000}{\fontsize{11}{11}\selectfont{61}}} & \multicolumn{1}{>{\centering}m{\dimexpr 1.02in+0\tabcolsep}}{\textcolor[HTML]{000000}{\fontsize{11}{11}\selectfont{96.8\%}}} \\





\multicolumn{1}{>{\raggedright}m{\dimexpr 1.89in+0\tabcolsep}}{\textcolor[HTML]{000000}{\fontsize{11}{11}\selectfont{Lidl}}} & \multicolumn{1}{>{\centering}m{\dimexpr 1.27in+0\tabcolsep}}{\textcolor[HTML]{000000}{\fontsize{11}{11}\selectfont{16}}} & \multicolumn{1}{>{\centering}m{\dimexpr 1.02in+0\tabcolsep}}{\textcolor[HTML]{000000}{\fontsize{11}{11}\selectfont{25.4\%}}} & \multicolumn{1}{>{\centering}m{\dimexpr 1.27in+0\tabcolsep}}{\textcolor[HTML]{000000}{\fontsize{11}{11}\selectfont{47}}} & \multicolumn{1}{>{\centering}m{\dimexpr 1.02in+0\tabcolsep}}{\textcolor[HTML]{000000}{\fontsize{11}{11}\selectfont{74.6\%}}} \\





\multicolumn{1}{>{\raggedright}m{\dimexpr 1.89in+0\tabcolsep}}{\textcolor[HTML]{000000}{\fontsize{11}{11}\selectfont{Masymas}}} & \multicolumn{1}{>{\centering}m{\dimexpr 1.27in+0\tabcolsep}}{\textcolor[HTML]{000000}{\fontsize{11}{11}\selectfont{4}}} & \multicolumn{1}{>{\centering}m{\dimexpr 1.02in+0\tabcolsep}}{\textcolor[HTML]{000000}{\fontsize{11}{11}\selectfont{6.3\%}}} & \multicolumn{1}{>{\centering}m{\dimexpr 1.27in+0\tabcolsep}}{\textcolor[HTML]{000000}{\fontsize{11}{11}\selectfont{59}}} & \multicolumn{1}{>{\centering}m{\dimexpr 1.02in+0\tabcolsep}}{\textcolor[HTML]{000000}{\fontsize{11}{11}\selectfont{93.7\%}}} \\





\multicolumn{1}{>{\raggedright}m{\dimexpr 1.89in+0\tabcolsep}}{\textcolor[HTML]{000000}{\fontsize{11}{11}\selectfont{Mercadona}}} & \multicolumn{1}{>{\centering}m{\dimexpr 1.27in+0\tabcolsep}}{\textcolor[HTML]{000000}{\fontsize{11}{11}\selectfont{51}}} & \multicolumn{1}{>{\centering}m{\dimexpr 1.02in+0\tabcolsep}}{\textcolor[HTML]{000000}{\fontsize{11}{11}\selectfont{81.0\%}}} & \multicolumn{1}{>{\centering}m{\dimexpr 1.27in+0\tabcolsep}}{\textcolor[HTML]{000000}{\fontsize{11}{11}\selectfont{12}}} & \multicolumn{1}{>{\centering}m{\dimexpr 1.02in+0\tabcolsep}}{\textcolor[HTML]{000000}{\fontsize{11}{11}\selectfont{19.0\%}}} \\

\ascline{1.5pt}{000000}{1-5}



\multicolumn{1}{>{\raggedright}m{\dimexpr 1.89in+0\tabcolsep}}{\textcolor[HTML]{000000}{\fontsize{11}{11}\selectfont{\textbf{Total}}}} & \multicolumn{1}{>{\centering}m{\dimexpr 1.27in+0\tabcolsep}}{\textcolor[HTML]{000000}{\fontsize{11}{11}\selectfont{\textbf{138}}}} & \multicolumn{1}{>{\centering}m{\dimexpr 1.02in+0\tabcolsep}}{\textcolor[HTML]{000000}{\fontsize{11}{11}\selectfont{\textbf{}}}} & \multicolumn{1}{>{\centering}m{\dimexpr 1.27in+0\tabcolsep}}{\textcolor[HTML]{000000}{\fontsize{11}{11}\selectfont{\textbf{429}}}} & \multicolumn{1}{>{\centering}m{\dimexpr 1.02in+0\tabcolsep}}{\textcolor[HTML]{000000}{\fontsize{11}{11}\selectfont{\textbf{}}}} \\

\ascline{1.5pt}{666666}{1-5}



\end{longtable}



\arrayrulecolor[HTML]{000000}

\global\setlength{\arrayrulewidth}{\Oldarrayrulewidth}

\global\setlength{\tabcolsep}{\Oldtabcolsep}

\renewcommand*{\arraystretch}{1}

\hypertarget{f_experiencia.barra}{%
\paragraph{\texorpdfstring{\texttt{f\_Experiencia.Barra}}{f\_Experiencia.Barra}}\label{f_experiencia.barra}}

Las categorías de \texttt{f\_Experiencia.Barra} hacen referencia a si el
encuestado a probado una barra de ensaladas previamente. Se trata de una
variable cuyas categorías no guardan una relación de orden entre sí.

\begin{Shaded}
\begin{Highlighting}[]
\CommentTok{\# Crear el data frame de frecuencias y porcentajes usando dplyr}
\NormalTok{datos\_grafico }\OtherTok{\textless{}{-}}\NormalTok{ datos }\SpecialCharTok{\%\textgreater{}\%}
  \CommentTok{\# Contar la frecuencia de cada nivel de la variable f\_Experiencia.Barra}
  \FunctionTok{count}\NormalTok{(f\_Experiencia.Barra, }\AttributeTok{name =} \StringTok{"Frecuencia"}\NormalTok{) }\SpecialCharTok{\%\textgreater{}\%}
  \CommentTok{\# Calcular el porcentaje y el texto de la etiqueta}
  \FunctionTok{mutate}\NormalTok{(}
    \AttributeTok{Porcentaje =}\NormalTok{ Frecuencia }\SpecialCharTok{/} \FunctionTok{sum}\NormalTok{(Frecuencia) }\SpecialCharTok{*} \DecValTok{100}\NormalTok{,}
    \AttributeTok{Etiquetas =} \FunctionTok{paste0}\NormalTok{(f\_Experiencia.Barra, }\StringTok{"}\SpecialCharTok{\textbackslash{}n}\StringTok{("}\NormalTok{, }\FunctionTok{round}\NormalTok{(Porcentaje, }\DecValTok{1}\NormalTok{), }\StringTok{"\%)"}\NormalTok{) }\CommentTok{\# Etiqueta para el gráfico}
\NormalTok{  ) }\SpecialCharTok{\%\textgreater{}\%}
  \CommentTok{\# Eliminar filas con 0\% si existen, y categorías no deseadas}
  \FunctionTok{filter}\NormalTok{(Porcentaje }\SpecialCharTok{\textgreater{}} \DecValTok{0} \SpecialCharTok{\&}\NormalTok{ f\_Experiencia.Barra }\SpecialCharTok{!=} \StringTok{"Total"}\NormalTok{) }\CommentTok{\# Se filtra \textquotesingle{}Total\textquotesingle{} aunque dplyr no lo añade}

\CommentTok{\# Usamos el vector de Porcentaje para el tamaño de las porciones}
\FunctionTok{pie}\NormalTok{(datos\_grafico}\SpecialCharTok{$}\NormalTok{Porcentaje,}
    \CommentTok{\# Usamos las Etiquetas que combinan nombre y porcentaje}
    \AttributeTok{labels =}\NormalTok{ datos\_grafico}\SpecialCharTok{$}\NormalTok{Etiquetas,}
    \CommentTok{\# Título}
    \AttributeTok{main =} \StringTok{\textquotesingle{}Figura 5. Experiencia con barra de ensaldas.\textquotesingle{}}\NormalTok{,}
    \CommentTok{\# Asignar un color diferente a cada porción}
    \AttributeTok{col =}\NormalTok{ RColorBrewer}\SpecialCharTok{::}\FunctionTok{brewer.pal}\NormalTok{(}\AttributeTok{n =} \FunctionTok{nrow}\NormalTok{(datos\_grafico), }\AttributeTok{name =} \StringTok{"Set3"}\NormalTok{))}
\end{Highlighting}
\end{Shaded}

\begin{center}\includegraphics{ICO-analisis_files/figure-latex/P8-tarta-1} \end{center}

Se observa que la mayoría de los encuestados no habían probado una barra
de ensaladas (73\%), frente a los que si (27\%). A continuación
presentamos los resultados en formato de tabla:

\begin{Shaded}
\begin{Highlighting}[]
\CommentTok{\# Preparación de datos}
\NormalTok{df\_tabla\_experiencia }\OtherTok{\textless{}{-}}\NormalTok{ datos }\SpecialCharTok{\%\textgreater{}\%}
  \CommentTok{\# Contar la frecuencia de cada nivel}
\NormalTok{  dplyr}\SpecialCharTok{::}\FunctionTok{count}\NormalTok{(f\_Experiencia.Barra, }\AttributeTok{name =} \StringTok{"Frecuencia"}\NormalTok{) }\SpecialCharTok{\%\textgreater{}\%}
  \CommentTok{\# Calcular porcentajes}
\NormalTok{  dplyr}\SpecialCharTok{::}\FunctionTok{mutate}\NormalTok{(}
    \AttributeTok{Porcentaje =}\NormalTok{ Frecuencia }\SpecialCharTok{/} \FunctionTok{sum}\NormalTok{(Frecuencia) }\SpecialCharTok{*} \DecValTok{100}
\NormalTok{  ) }\SpecialCharTok{\%\textgreater{}\%}
  \CommentTok{\# Redondear y formatear los porcentajes}
\NormalTok{  dplyr}\SpecialCharTok{::}\FunctionTok{mutate}\NormalTok{(}
    \AttributeTok{Porcentaje =} \FunctionTok{paste0}\NormalTok{(}\FunctionTok{round}\NormalTok{(Porcentaje, }\DecValTok{1}\NormalTok{), }\StringTok{"\%"}\NormalTok{)}
\NormalTok{  )}

\CommentTok{\# Calcular y añadir la fila total}
\NormalTok{df\_total }\OtherTok{\textless{}{-}} \FunctionTok{data.frame}\NormalTok{(}
  \AttributeTok{f\_Experiencia.Barra =} \StringTok{"Total"}\NormalTok{,}
  \AttributeTok{Frecuencia =} \FunctionTok{sum}\NormalTok{(df\_tabla\_experiencia}\SpecialCharTok{$}\NormalTok{Frecuencia),}
  \AttributeTok{Porcentaje =} \StringTok{"100.0\%"}
\NormalTok{)}

\CommentTok{\# Unir el data frame de frecuencias con la fila total}
\NormalTok{df\_tabla\_final }\OtherTok{\textless{}{-}} \FunctionTok{bind\_rows}\NormalTok{(df\_tabla\_experiencia, df\_total)}

\CommentTok{\# Generación de la tabla (flextable)}
\NormalTok{ft }\OtherTok{\textless{}{-}}\NormalTok{ flextable}\SpecialCharTok{::}\FunctionTok{flextable}\NormalTok{(df\_tabla\_final) }\SpecialCharTok{\%\textgreater{}\%}

\NormalTok{flextable}\SpecialCharTok{::}\FunctionTok{set\_caption}\NormalTok{(}\AttributeTok{caption =} \StringTok{"Tabla 5. Distribución de frecuencias para Experiencia previa con una barra de ensaladas"}\NormalTok{) }\SpecialCharTok{\%\textgreater{}\%}
  
  \CommentTok{\# Renombrar las cabeceras}
\NormalTok{  flextable}\SpecialCharTok{::}\FunctionTok{set\_header\_labels}\NormalTok{(}
    \AttributeTok{f\_Experiencia.Barra =} \StringTok{"Experiencia con una barra de ensaladas"}\NormalTok{,}
    \AttributeTok{Frecuencia =} \StringTok{"Frecuencia (n)"}\NormalTok{,}
    \AttributeTok{Porcentaje =} \StringTok{"Porcentaje"}
\NormalTok{  ) }\SpecialCharTok{\%\textgreater{}\%}
  
  \CommentTok{\# Aplicar negrita y bordes a la fila "Total"}
\NormalTok{  flextable}\SpecialCharTok{::}\FunctionTok{bold}\NormalTok{(}\AttributeTok{i =} \FunctionTok{nrow}\NormalTok{(df\_tabla\_final), }\AttributeTok{part =} \StringTok{"body"}\NormalTok{) }\SpecialCharTok{\%\textgreater{}\%} \CommentTok{\# Negrita a la última fila (Total)}
\NormalTok{  flextable}\SpecialCharTok{::}\FunctionTok{border\_remove}\NormalTok{() }\SpecialCharTok{\%\textgreater{}\%} \CommentTok{\# Quitar bordes predeterminados}
\NormalTok{  flextable}\SpecialCharTok{::}\FunctionTok{theme\_booktabs}\NormalTok{() }\SpecialCharTok{\%\textgreater{}\%} \CommentTok{\# Aplicar un tema con líneas horizontales}
  
  \CommentTok{\# Formato de alineación y cabecera}
\NormalTok{  flextable}\SpecialCharTok{::}\FunctionTok{align}\NormalTok{(}\AttributeTok{j =} \DecValTok{1}\NormalTok{, }\AttributeTok{align =} \StringTok{"left"}\NormalTok{, }\AttributeTok{part =} \StringTok{"body"}\NormalTok{) }\SpecialCharTok{\%\textgreater{}\%}
\NormalTok{  flextable}\SpecialCharTok{::}\FunctionTok{align}\NormalTok{(}\AttributeTok{j =} \DecValTok{2}\SpecialCharTok{:}\DecValTok{3}\NormalTok{, }\AttributeTok{align =} \StringTok{"center"}\NormalTok{, }\AttributeTok{part =} \StringTok{"all"}\NormalTok{) }\SpecialCharTok{\%\textgreater{}\%} \CommentTok{\# Columnas 2 y 3 (Datos) CENTRADAS}
\NormalTok{  flextable}\SpecialCharTok{::}\FunctionTok{align}\NormalTok{(}\AttributeTok{align =} \StringTok{"center"}\NormalTok{, }\AttributeTok{part =} \StringTok{"header"}\NormalTok{) }\SpecialCharTok{\%\textgreater{}\%}        \CommentTok{\# Encabezados CENTRADOS}
  
  \CommentTok{\# Añadir una línea superior a la fila "Total" para separarla}
\NormalTok{  flextable}\SpecialCharTok{::}\FunctionTok{hline}\NormalTok{(}\AttributeTok{i =} \FunctionTok{nrow}\NormalTok{(df\_tabla\_final) }\SpecialCharTok{{-}} \DecValTok{1}\NormalTok{, }\AttributeTok{border =}\NormalTok{ officer}\SpecialCharTok{::}\FunctionTok{fp\_border}\NormalTok{(}\AttributeTok{width =} \FloatTok{1.5}\NormalTok{, }\AttributeTok{color =} \StringTok{"black"}\NormalTok{)) }\SpecialCharTok{\%\textgreater{}\%}
  
  \CommentTok{\# Ajustar el ancho de las columnas}
\NormalTok{  flextable}\SpecialCharTok{::}\FunctionTok{autofit}\NormalTok{()}

\CommentTok{\# Mostrar la tabla}
\NormalTok{ft}
\end{Highlighting}
\end{Shaded}

\global\setlength{\Oldarrayrulewidth}{\arrayrulewidth}

\global\setlength{\Oldtabcolsep}{\tabcolsep}

\setlength{\tabcolsep}{2pt}

\renewcommand*{\arraystretch}{1.5}



\providecommand{\ascline}[3]{\noalign{\global\arrayrulewidth #1}\arrayrulecolor[HTML]{#2}\cline{#3}}

\begin{longtable}[c]{|p{3.02in}|p{1.27in}|p{1.02in}}

\caption{Tabla\ 5.\ Distribución\ de\ frecuencias\ para\ Experiencia\ previa\ con\ una\ barra\ de\ ensaladas}\\

\ascline{1.5pt}{666666}{1-3}

\multicolumn{1}{>{\centering}m{\dimexpr 3.02in+0\tabcolsep}}{\textcolor[HTML]{000000}{\fontsize{11}{11}\selectfont{Experiencia\ con\ una\ barra\ de\ ensaladas}}} & \multicolumn{1}{>{\centering}m{\dimexpr 1.27in+0\tabcolsep}}{\textcolor[HTML]{000000}{\fontsize{11}{11}\selectfont{Frecuencia\ (n)}}} & \multicolumn{1}{>{\centering}m{\dimexpr 1.02in+0\tabcolsep}}{\textcolor[HTML]{000000}{\fontsize{11}{11}\selectfont{Porcentaje}}} \\

\ascline{1.5pt}{666666}{1-3}\endfirsthead \caption[]{Tabla\ 5.\ Distribución\ de\ frecuencias\ para\ Experiencia\ previa\ con\ una\ barra\ de\ ensaladas}\\

\ascline{1.5pt}{666666}{1-3}

\multicolumn{1}{>{\centering}m{\dimexpr 3.02in+0\tabcolsep}}{\textcolor[HTML]{000000}{\fontsize{11}{11}\selectfont{Experiencia\ con\ una\ barra\ de\ ensaladas}}} & \multicolumn{1}{>{\centering}m{\dimexpr 1.27in+0\tabcolsep}}{\textcolor[HTML]{000000}{\fontsize{11}{11}\selectfont{Frecuencia\ (n)}}} & \multicolumn{1}{>{\centering}m{\dimexpr 1.02in+0\tabcolsep}}{\textcolor[HTML]{000000}{\fontsize{11}{11}\selectfont{Porcentaje}}} \\

\ascline{1.5pt}{666666}{1-3}\endhead



\multicolumn{1}{>{\raggedright}m{\dimexpr 3.02in+0\tabcolsep}}{\textcolor[HTML]{000000}{\fontsize{11}{11}\selectfont{Sí}}} & \multicolumn{1}{>{\centering}m{\dimexpr 1.27in+0\tabcolsep}}{\textcolor[HTML]{000000}{\fontsize{11}{11}\selectfont{17}}} & \multicolumn{1}{>{\centering}m{\dimexpr 1.02in+0\tabcolsep}}{\textcolor[HTML]{000000}{\fontsize{11}{11}\selectfont{27\%}}} \\





\multicolumn{1}{>{\raggedright}m{\dimexpr 3.02in+0\tabcolsep}}{\textcolor[HTML]{000000}{\fontsize{11}{11}\selectfont{No}}} & \multicolumn{1}{>{\centering}m{\dimexpr 1.27in+0\tabcolsep}}{\textcolor[HTML]{000000}{\fontsize{11}{11}\selectfont{46}}} & \multicolumn{1}{>{\centering}m{\dimexpr 1.02in+0\tabcolsep}}{\textcolor[HTML]{000000}{\fontsize{11}{11}\selectfont{73\%}}} \\

\ascline{1.5pt}{000000}{1-3}



\multicolumn{1}{>{\raggedright}m{\dimexpr 3.02in+0\tabcolsep}}{\textcolor[HTML]{000000}{\fontsize{11}{11}\selectfont{\textbf{Total}}}} & \multicolumn{1}{>{\centering}m{\dimexpr 1.27in+0\tabcolsep}}{\textcolor[HTML]{000000}{\fontsize{11}{11}\selectfont{\textbf{63}}}} & \multicolumn{1}{>{\centering}m{\dimexpr 1.02in+0\tabcolsep}}{\textcolor[HTML]{000000}{\fontsize{11}{11}\selectfont{\textbf{100.0\%}}}} \\

\ascline{1.5pt}{666666}{1-3}



\end{longtable}



\arrayrulecolor[HTML]{000000}

\global\setlength{\arrayrulewidth}{\Oldarrayrulewidth}

\global\setlength{\tabcolsep}{\Oldtabcolsep}

\renewcommand*{\arraystretch}{1}

\hypertarget{f_situaciones.uso}{%
\paragraph{\texorpdfstring{\texttt{f\_Situaciones.Uso}}{f\_Situaciones.Uso}}\label{f_situaciones.uso}}

Las categorías de \texttt{f\_Situaciones.Uso} hacen referencia a las
situaciones en que los encuestados utilizarían la barra de ensaladas. Es
importante destacar que esta variable proviene de una pregunta con
respuesta múltiple por lo que optamos por un gráfico de barras múltiple
basándonos en si ha se ha seleccionado cada respuesta o no.

\begin{Shaded}
\begin{Highlighting}[]
\CommentTok{\# Definir las variables a analizar (P10: Situaciones de Uso)}
\NormalTok{multi\_vars }\OtherTok{\textless{}{-}} \FunctionTok{c}\NormalTok{(}\StringTok{"f\_Uso.Momento"}\NormalTok{, }\StringTok{"f\_Uso.Llevar"}\NormalTok{, }\StringTok{"f\_Uso.Cena"}\NormalTok{, }\StringTok{"f\_Uso.Complem"}\NormalTok{, }\StringTok{"f\_Uso.Nunca"}\NormalTok{)}

\CommentTok{\# Transformar a formato largo y contar frecuencias de SÍ/NO}
\NormalTok{df\_barras\_dobles }\OtherTok{\textless{}{-}}\NormalTok{ datos }\SpecialCharTok{\%\textgreater{}\%}
\NormalTok{  dplyr}\SpecialCharTok{::}\FunctionTok{select}\NormalTok{(}\FunctionTok{all\_of}\NormalTok{(multi\_vars)) }\SpecialCharTok{\%\textgreater{}\%}
\NormalTok{  tidyr}\SpecialCharTok{::}\FunctionTok{pivot\_longer}\NormalTok{(}
    \AttributeTok{cols =} \FunctionTok{everything}\NormalTok{(), }
    \AttributeTok{names\_to =} \StringTok{"Situaciones\_Uso"}\NormalTok{, }
    \AttributeTok{values\_to =} \StringTok{"Respuesta"}
\NormalTok{  ) }\SpecialCharTok{\%\textgreater{}\%}
\NormalTok{  dplyr}\SpecialCharTok{::}\FunctionTok{count}\NormalTok{(Situaciones\_Uso, Respuesta, }\AttributeTok{name =} \StringTok{"Frecuencia"}\NormalTok{) }\SpecialCharTok{\%\textgreater{}\%}
\NormalTok{  dplyr}\SpecialCharTok{::}\FunctionTok{mutate}\NormalTok{(}
    \AttributeTok{Situaciones\_Uso =} \FunctionTok{gsub}\NormalTok{(}\StringTok{"f\_Uso."}\NormalTok{, }\StringTok{""}\NormalTok{, Situaciones\_Uso, }\AttributeTok{fixed =} \ConstantTok{TRUE}\NormalTok{),}
    \AttributeTok{Situaciones\_Uso =}\NormalTok{ tools}\SpecialCharTok{::}\FunctionTok{toTitleCase}\NormalTok{(Situaciones\_Uso),}
    \AttributeTok{Respuesta =} \FunctionTok{factor}\NormalTok{(Respuesta, }\AttributeTok{levels =} \FunctionTok{c}\NormalTok{(}\StringTok{"No"}\NormalTok{, }\StringTok{"Sí"}\NormalTok{)) }
\NormalTok{  )}

\CommentTok{\# Generación del Gráfico de Barras Agrupadas (Horizontal)}
\FunctionTok{ggplot}\NormalTok{(df\_barras\_dobles, }\FunctionTok{aes}\NormalTok{(}\AttributeTok{x =}\NormalTok{ Situaciones\_Uso, }\AttributeTok{y =}\NormalTok{ Frecuencia, }\AttributeTok{fill =}\NormalTok{ Respuesta)) }\SpecialCharTok{+}
  
  \CommentTok{\# Barras agrupadas}
  \FunctionTok{geom\_bar}\NormalTok{(}\AttributeTok{stat =} \StringTok{"identity"}\NormalTok{, }\AttributeTok{position =} \FunctionTok{position\_dodge}\NormalTok{(}\AttributeTok{width =} \FloatTok{0.9}\NormalTok{)) }\SpecialCharTok{+} 
  
  \CommentTok{\# Etiquetas de Frecuencia}
  \FunctionTok{geom\_text}\NormalTok{(}
    \FunctionTok{aes}\NormalTok{(}\AttributeTok{label =}\NormalTok{ Frecuencia),}
    \AttributeTok{position =} \FunctionTok{position\_dodge}\NormalTok{(}\AttributeTok{width =} \FloatTok{0.9}\NormalTok{),}
    \AttributeTok{hjust =} \SpecialCharTok{{-}}\FloatTok{0.2}\NormalTok{, }\CommentTok{\# Mueve la etiqueta {-}0.2 unidades a lo largo del eje X (separa de la barra)}
    \AttributeTok{size =} \FloatTok{3.5}
\NormalTok{  ) }\SpecialCharTok{+}
  
  \CommentTok{\# Títulos y Ejes}
  \FunctionTok{labs}\NormalTok{(}
    \AttributeTok{title =} \StringTok{"Figura 6. Gráfico de barras (Sí/No) por Situaciones de Uso"}\NormalTok{,}
    \AttributeTok{x =} \StringTok{"Situación de Uso"}\NormalTok{,}
    \AttributeTok{y =} \StringTok{"Número de Respuestas"}\NormalTok{,}
    \AttributeTok{fill =} \StringTok{"Usaría"}
\NormalTok{  ) }\SpecialCharTok{+}
  
  \CommentTok{\# Colores}
  \FunctionTok{scale\_fill\_manual}\NormalTok{(}\AttributeTok{values =} \FunctionTok{c}\NormalTok{(}\StringTok{"No"} \OtherTok{=} \StringTok{"\#a6cee3"}\NormalTok{, }\StringTok{"Sí"} \OtherTok{=} \StringTok{"\#1f78b4"}\NormalTok{)) }\SpecialCharTok{+} 
  
  \CommentTok{\# Invertir coordenadas para hacerlo horizontal}
  \FunctionTok{coord\_flip}\NormalTok{() }\SpecialCharTok{+} 
  
  \CommentTok{\# Ajuste del eje para que no se corten las etiquetas (Aumentamos el espacio)}
  \FunctionTok{scale\_y\_continuous}\NormalTok{(}\AttributeTok{expand =} \FunctionTok{expansion}\NormalTok{(}\AttributeTok{mult =} \FunctionTok{c}\NormalTok{(}\DecValTok{0}\NormalTok{, }\FloatTok{0.20}\NormalTok{))) }\SpecialCharTok{+} 
  
  \CommentTok{\# Ajuste de Tema}
  \FunctionTok{theme}\NormalTok{(}\AttributeTok{legend.position =} \StringTok{"bottom"}\NormalTok{)}
\end{Highlighting}
\end{Shaded}

\begin{center}\includegraphics{ICO-analisis_files/figure-latex/P10-barras-1} \end{center}

Se observa que se optaría por tomar la ensalada para llevar al
trabajo/universidad (con 25 respuestas afirmativas) o bien para comer en
el momento (con 30 respuestas afirmativas). A continuación presentamos
los resultados en formato de tabla:

\begin{Shaded}
\begin{Highlighting}[]
\CommentTok{\# Transformar a formato largo, contar frecuencias de SÍ/NO y calcular porcentajes}
\NormalTok{df\_tabla\_temp }\OtherTok{\textless{}{-}}\NormalTok{ datos }\SpecialCharTok{\%\textgreater{}\%}
\NormalTok{  dplyr}\SpecialCharTok{::}\FunctionTok{select}\NormalTok{(}\FunctionTok{all\_of}\NormalTok{(multi\_vars)) }\SpecialCharTok{\%\textgreater{}\%}
\NormalTok{  tidyr}\SpecialCharTok{::}\FunctionTok{pivot\_longer}\NormalTok{(}
    \AttributeTok{cols =} \FunctionTok{everything}\NormalTok{(), }
    \AttributeTok{names\_to =} \StringTok{"Situaciones\_Uso\_Raw"}\NormalTok{, }
    \AttributeTok{values\_to =} \StringTok{"Respuesta"} 
\NormalTok{  ) }\SpecialCharTok{\%\textgreater{}\%}
\NormalTok{  dplyr}\SpecialCharTok{::}\FunctionTok{count}\NormalTok{(Situaciones\_Uso\_Raw, Respuesta, }\AttributeTok{name =} \StringTok{"Frecuencia"}\NormalTok{) }\SpecialCharTok{\%\textgreater{}\%}
\NormalTok{  dplyr}\SpecialCharTok{::}\FunctionTok{mutate}\NormalTok{(}
    \AttributeTok{Porcentaje =}\NormalTok{ (Frecuencia }\SpecialCharTok{/}\NormalTok{ n.filas) }\SpecialCharTok{*} \DecValTok{100}
\NormalTok{  ) }\SpecialCharTok{\%\textgreater{}\%}
\NormalTok{  tidyr}\SpecialCharTok{::}\FunctionTok{pivot\_wider}\NormalTok{(}
    \AttributeTok{names\_from =}\NormalTok{ Respuesta, }
    \AttributeTok{values\_from =} \FunctionTok{c}\NormalTok{(Frecuencia, Porcentaje),}
    \AttributeTok{values\_fill =} \DecValTok{0} 
\NormalTok{  ) }\SpecialCharTok{\%\textgreater{}\%}
\NormalTok{  dplyr}\SpecialCharTok{::}\FunctionTok{mutate}\NormalTok{(}
    \AttributeTok{Situaciones\_Uso =} \FunctionTok{gsub}\NormalTok{(}\StringTok{"f\_Uso."}\NormalTok{, }\StringTok{""}\NormalTok{, Situaciones\_Uso\_Raw, }\AttributeTok{fixed =} \ConstantTok{TRUE}\NormalTok{),}
    \AttributeTok{Situaciones\_Uso =}\NormalTok{ tools}\SpecialCharTok{::}\FunctionTok{toTitleCase}\NormalTok{(Situaciones\_Uso)}
\NormalTok{  ) }\SpecialCharTok{\%\textgreater{}\%}
\NormalTok{  dplyr}\SpecialCharTok{::}\FunctionTok{select}\NormalTok{(}
\NormalTok{    Situaciones\_Uso,}
    \StringTok{\textasciigrave{}}\AttributeTok{Frecuencia\_Sí}\StringTok{\textasciigrave{}}\NormalTok{, }\StringTok{\textasciigrave{}}\AttributeTok{Porcentaje\_Sí}\StringTok{\textasciigrave{}}\NormalTok{,}
    \StringTok{\textasciigrave{}}\AttributeTok{Frecuencia\_No}\StringTok{\textasciigrave{}}\NormalTok{, }\StringTok{\textasciigrave{}}\AttributeTok{Porcentaje\_No}\StringTok{\textasciigrave{}}
\NormalTok{  )}


\CommentTok{\# Fila de Total}
\NormalTok{df\_total }\OtherTok{\textless{}{-}} \FunctionTok{data.frame}\NormalTok{(}
  \AttributeTok{Situaciones\_Uso =} \StringTok{"Total"}\NormalTok{,}
\NormalTok{  Frecuencia\_Sí }\OtherTok{=} \FunctionTok{sum}\NormalTok{(df\_tabla\_temp}\SpecialCharTok{$}\NormalTok{Frecuencia\_Sí),}
\NormalTok{  Porcentaje\_Sí }\OtherTok{=} \ConstantTok{NA}\NormalTok{, }\CommentTok{\# No se suma, sería engañoso}
  \AttributeTok{Frecuencia\_No =} \FunctionTok{sum}\NormalTok{(df\_tabla\_temp}\SpecialCharTok{$}\NormalTok{Frecuencia\_No),}
  \AttributeTok{Porcentaje\_No =} \ConstantTok{NA} \CommentTok{\# No se suma, sería engañoso}
\NormalTok{)}

\CommentTok{\# Unir el data frame con la fila total}
\NormalTok{df\_tabla\_final }\OtherTok{\textless{}{-}} \FunctionTok{bind\_rows}\NormalTok{(df\_tabla\_temp, df\_total)}


\CommentTok{\# Generación de la tabla (flextable)}
\NormalTok{ft\_si\_no }\OtherTok{\textless{}{-}}\NormalTok{ flextable}\SpecialCharTok{::}\FunctionTok{flextable}\NormalTok{(df\_tabla\_final) }\SpecialCharTok{\%\textgreater{}\%}
  
  \CommentTok{\# Añadir Título}
\NormalTok{  flextable}\SpecialCharTok{::}\FunctionTok{set\_caption}\NormalTok{(}\AttributeTok{caption =} \StringTok{"Tabla 6. Distribución de Frecuencias (Sí/No) para Situaciones de Uso"}\NormalTok{) }\SpecialCharTok{\%\textgreater{}\%}
  
  \CommentTok{\# Renombrar cabeceras (se usan para la segunda fila de la cabecera)}
\NormalTok{  flextable}\SpecialCharTok{::}\FunctionTok{set\_header\_labels}\NormalTok{(}
    \AttributeTok{Situaciones\_Uso =} \StringTok{"Situación de Uso"}\NormalTok{,}
\NormalTok{    Frecuencia\_Sí }\OtherTok{=} \StringTok{"Frecuencia (n)"}\NormalTok{,}
\NormalTok{    Porcentaje\_Sí }\OtherTok{=} \StringTok{"Porcentaje"}\NormalTok{,}
    \AttributeTok{Frecuencia\_No =} \StringTok{"Frecuencia (n)"}\NormalTok{,}
    \AttributeTok{Porcentaje\_No =} \StringTok{"Porcentaje"}
\NormalTok{  ) }\SpecialCharTok{\%\textgreater{}\%}
  
  \CommentTok{\# Agrupar las columnas \textquotesingle{}Sí\textquotesingle{} y \textquotesingle{}No\textquotesingle{} en una fila superior}
\NormalTok{  flextable}\SpecialCharTok{::}\FunctionTok{add\_header\_row}\NormalTok{(}
    \AttributeTok{values =} \FunctionTok{c}\NormalTok{(}\StringTok{" "}\NormalTok{, }\StringTok{"SÍ"}\NormalTok{, }\StringTok{"NO"}\NormalTok{),}
    \AttributeTok{colwidths =} \FunctionTok{c}\NormalTok{(}\DecValTok{1}\NormalTok{, }\DecValTok{2}\NormalTok{, }\DecValTok{2}\NormalTok{)}
\NormalTok{  ) }\SpecialCharTok{\%\textgreater{}\%}
  
  \CommentTok{\# Formato y Tema}
\NormalTok{  flextable}\SpecialCharTok{::}\FunctionTok{colformat\_double}\NormalTok{(}\AttributeTok{j =} \FunctionTok{c}\NormalTok{(}\DecValTok{3}\NormalTok{, }\DecValTok{5}\NormalTok{), }\AttributeTok{digits =} \DecValTok{1}\NormalTok{, }\AttributeTok{suffix =} \StringTok{"\%"}\NormalTok{, }\AttributeTok{na\_str =} \StringTok{""}\NormalTok{) }\SpecialCharTok{\%\textgreater{}\%} 
  
\NormalTok{  flextable}\SpecialCharTok{::}\FunctionTok{theme\_booktabs}\NormalTok{() }\SpecialCharTok{\%\textgreater{}\%} 
  
  \CommentTok{\# Alineación}
  \CommentTok{\# Aplicar negrita y bordes a la fila "Total" (la última fila)}
\NormalTok{  flextable}\SpecialCharTok{::}\FunctionTok{bold}\NormalTok{(}\AttributeTok{i =} \FunctionTok{nrow}\NormalTok{(df\_tabla\_final), }\AttributeTok{part =} \StringTok{"body"}\NormalTok{) }\SpecialCharTok{\%\textgreater{}\%} 
\NormalTok{  flextable}\SpecialCharTok{::}\FunctionTok{hline}\NormalTok{(}\AttributeTok{i =} \FunctionTok{nrow}\NormalTok{(df\_tabla\_final) }\SpecialCharTok{{-}} \DecValTok{1}\NormalTok{, }\AttributeTok{border =}\NormalTok{ officer}\SpecialCharTok{::}\FunctionTok{fp\_border}\NormalTok{(}\AttributeTok{width =} \FloatTok{1.5}\NormalTok{, }\AttributeTok{color =} \StringTok{"black"}\NormalTok{)) }\SpecialCharTok{\%\textgreater{}\%}
  
  \CommentTok{\# Alineación}
\NormalTok{  flextable}\SpecialCharTok{::}\FunctionTok{align}\NormalTok{(}\AttributeTok{align =} \StringTok{"center"}\NormalTok{, }\AttributeTok{part =} \StringTok{"all"}\NormalTok{) }\SpecialCharTok{\%\textgreater{}\%}
\NormalTok{  flextable}\SpecialCharTok{::}\FunctionTok{align}\NormalTok{(}\AttributeTok{j =} \DecValTok{1}\NormalTok{, }\AttributeTok{align =} \StringTok{"left"}\NormalTok{, }\AttributeTok{part =} \StringTok{"body"}\NormalTok{) }\SpecialCharTok{\%\textgreater{}\%}
  
  \CommentTok{\# Ajustar}
\NormalTok{  flextable}\SpecialCharTok{::}\FunctionTok{autofit}\NormalTok{()}

\CommentTok{\# Mostrar la tabla}
\NormalTok{ft\_si\_no}
\end{Highlighting}
\end{Shaded}

\global\setlength{\Oldarrayrulewidth}{\arrayrulewidth}

\global\setlength{\Oldtabcolsep}{\tabcolsep}

\setlength{\tabcolsep}{2pt}

\renewcommand*{\arraystretch}{1.5}



\providecommand{\ascline}[3]{\noalign{\global\arrayrulewidth #1}\arrayrulecolor[HTML]{#2}\cline{#3}}

\begin{longtable}[c]{|p{1.44in}|p{1.27in}|p{1.02in}|p{1.27in}|p{1.02in}}

\caption{Tabla\ 6.\ Distribución\ de\ Frecuencias\ (Sí/No)\ para\ Situaciones\ de\ Uso}\\

\ascline{1.5pt}{666666}{1-5}

\multicolumn{1}{>{\centering}m{\dimexpr 1.44in+0\tabcolsep}}{\textcolor[HTML]{000000}{\fontsize{11}{11}\selectfont{\ }}} & \multicolumn{2}{>{\centering}m{\dimexpr 2.29in+2\tabcolsep}}{\textcolor[HTML]{000000}{\fontsize{11}{11}\selectfont{SÍ}}} & \multicolumn{2}{>{\centering}m{\dimexpr 2.29in+2\tabcolsep}}{\textcolor[HTML]{000000}{\fontsize{11}{11}\selectfont{NO}}} \\





\multicolumn{1}{>{\centering}m{\dimexpr 1.44in+0\tabcolsep}}{\textcolor[HTML]{000000}{\fontsize{11}{11}\selectfont{Situación\ de\ Uso}}} & \multicolumn{1}{>{\centering}m{\dimexpr 1.27in+0\tabcolsep}}{\textcolor[HTML]{000000}{\fontsize{11}{11}\selectfont{Frecuencia\ (n)}}} & \multicolumn{1}{>{\centering}m{\dimexpr 1.02in+0\tabcolsep}}{\textcolor[HTML]{000000}{\fontsize{11}{11}\selectfont{Porcentaje}}} & \multicolumn{1}{>{\centering}m{\dimexpr 1.27in+0\tabcolsep}}{\textcolor[HTML]{000000}{\fontsize{11}{11}\selectfont{Frecuencia\ (n)}}} & \multicolumn{1}{>{\centering}m{\dimexpr 1.02in+0\tabcolsep}}{\textcolor[HTML]{000000}{\fontsize{11}{11}\selectfont{Porcentaje}}} \\

\ascline{1.5pt}{666666}{1-5}\endfirsthead \caption[]{Tabla\ 6.\ Distribución\ de\ Frecuencias\ (Sí/No)\ para\ Situaciones\ de\ Uso}\\

\ascline{1.5pt}{666666}{1-5}

\multicolumn{1}{>{\centering}m{\dimexpr 1.44in+0\tabcolsep}}{\textcolor[HTML]{000000}{\fontsize{11}{11}\selectfont{\ }}} & \multicolumn{2}{>{\centering}m{\dimexpr 2.29in+2\tabcolsep}}{\textcolor[HTML]{000000}{\fontsize{11}{11}\selectfont{SÍ}}} & \multicolumn{2}{>{\centering}m{\dimexpr 2.29in+2\tabcolsep}}{\textcolor[HTML]{000000}{\fontsize{11}{11}\selectfont{NO}}} \\





\multicolumn{1}{>{\centering}m{\dimexpr 1.44in+0\tabcolsep}}{\textcolor[HTML]{000000}{\fontsize{11}{11}\selectfont{Situación\ de\ Uso}}} & \multicolumn{1}{>{\centering}m{\dimexpr 1.27in+0\tabcolsep}}{\textcolor[HTML]{000000}{\fontsize{11}{11}\selectfont{Frecuencia\ (n)}}} & \multicolumn{1}{>{\centering}m{\dimexpr 1.02in+0\tabcolsep}}{\textcolor[HTML]{000000}{\fontsize{11}{11}\selectfont{Porcentaje}}} & \multicolumn{1}{>{\centering}m{\dimexpr 1.27in+0\tabcolsep}}{\textcolor[HTML]{000000}{\fontsize{11}{11}\selectfont{Frecuencia\ (n)}}} & \multicolumn{1}{>{\centering}m{\dimexpr 1.02in+0\tabcolsep}}{\textcolor[HTML]{000000}{\fontsize{11}{11}\selectfont{Porcentaje}}} \\

\ascline{1.5pt}{666666}{1-5}\endhead



\multicolumn{1}{>{\raggedright}m{\dimexpr 1.44in+0\tabcolsep}}{\textcolor[HTML]{000000}{\fontsize{11}{11}\selectfont{Cena}}} & \multicolumn{1}{>{\centering}m{\dimexpr 1.27in+0\tabcolsep}}{\textcolor[HTML]{000000}{\fontsize{11}{11}\selectfont{20}}} & \multicolumn{1}{>{\centering}m{\dimexpr 1.02in+0\tabcolsep}}{\textcolor[HTML]{000000}{\fontsize{11}{11}\selectfont{31.7\%}}} & \multicolumn{1}{>{\centering}m{\dimexpr 1.27in+0\tabcolsep}}{\textcolor[HTML]{000000}{\fontsize{11}{11}\selectfont{43}}} & \multicolumn{1}{>{\centering}m{\dimexpr 1.02in+0\tabcolsep}}{\textcolor[HTML]{000000}{\fontsize{11}{11}\selectfont{68.3\%}}} \\





\multicolumn{1}{>{\raggedright}m{\dimexpr 1.44in+0\tabcolsep}}{\textcolor[HTML]{000000}{\fontsize{11}{11}\selectfont{Complem}}} & \multicolumn{1}{>{\centering}m{\dimexpr 1.27in+0\tabcolsep}}{\textcolor[HTML]{000000}{\fontsize{11}{11}\selectfont{19}}} & \multicolumn{1}{>{\centering}m{\dimexpr 1.02in+0\tabcolsep}}{\textcolor[HTML]{000000}{\fontsize{11}{11}\selectfont{30.2\%}}} & \multicolumn{1}{>{\centering}m{\dimexpr 1.27in+0\tabcolsep}}{\textcolor[HTML]{000000}{\fontsize{11}{11}\selectfont{44}}} & \multicolumn{1}{>{\centering}m{\dimexpr 1.02in+0\tabcolsep}}{\textcolor[HTML]{000000}{\fontsize{11}{11}\selectfont{69.8\%}}} \\





\multicolumn{1}{>{\raggedright}m{\dimexpr 1.44in+0\tabcolsep}}{\textcolor[HTML]{000000}{\fontsize{11}{11}\selectfont{Llevar}}} & \multicolumn{1}{>{\centering}m{\dimexpr 1.27in+0\tabcolsep}}{\textcolor[HTML]{000000}{\fontsize{11}{11}\selectfont{25}}} & \multicolumn{1}{>{\centering}m{\dimexpr 1.02in+0\tabcolsep}}{\textcolor[HTML]{000000}{\fontsize{11}{11}\selectfont{39.7\%}}} & \multicolumn{1}{>{\centering}m{\dimexpr 1.27in+0\tabcolsep}}{\textcolor[HTML]{000000}{\fontsize{11}{11}\selectfont{38}}} & \multicolumn{1}{>{\centering}m{\dimexpr 1.02in+0\tabcolsep}}{\textcolor[HTML]{000000}{\fontsize{11}{11}\selectfont{60.3\%}}} \\





\multicolumn{1}{>{\raggedright}m{\dimexpr 1.44in+0\tabcolsep}}{\textcolor[HTML]{000000}{\fontsize{11}{11}\selectfont{Momento}}} & \multicolumn{1}{>{\centering}m{\dimexpr 1.27in+0\tabcolsep}}{\textcolor[HTML]{000000}{\fontsize{11}{11}\selectfont{23}}} & \multicolumn{1}{>{\centering}m{\dimexpr 1.02in+0\tabcolsep}}{\textcolor[HTML]{000000}{\fontsize{11}{11}\selectfont{36.5\%}}} & \multicolumn{1}{>{\centering}m{\dimexpr 1.27in+0\tabcolsep}}{\textcolor[HTML]{000000}{\fontsize{11}{11}\selectfont{40}}} & \multicolumn{1}{>{\centering}m{\dimexpr 1.02in+0\tabcolsep}}{\textcolor[HTML]{000000}{\fontsize{11}{11}\selectfont{63.5\%}}} \\





\multicolumn{1}{>{\raggedright}m{\dimexpr 1.44in+0\tabcolsep}}{\textcolor[HTML]{000000}{\fontsize{11}{11}\selectfont{Nunca}}} & \multicolumn{1}{>{\centering}m{\dimexpr 1.27in+0\tabcolsep}}{\textcolor[HTML]{000000}{\fontsize{11}{11}\selectfont{12}}} & \multicolumn{1}{>{\centering}m{\dimexpr 1.02in+0\tabcolsep}}{\textcolor[HTML]{000000}{\fontsize{11}{11}\selectfont{19.0\%}}} & \multicolumn{1}{>{\centering}m{\dimexpr 1.27in+0\tabcolsep}}{\textcolor[HTML]{000000}{\fontsize{11}{11}\selectfont{51}}} & \multicolumn{1}{>{\centering}m{\dimexpr 1.02in+0\tabcolsep}}{\textcolor[HTML]{000000}{\fontsize{11}{11}\selectfont{81.0\%}}} \\

\ascline{1.5pt}{000000}{1-5}



\multicolumn{1}{>{\raggedright}m{\dimexpr 1.44in+0\tabcolsep}}{\textcolor[HTML]{000000}{\fontsize{11}{11}\selectfont{\textbf{Total}}}} & \multicolumn{1}{>{\centering}m{\dimexpr 1.27in+0\tabcolsep}}{\textcolor[HTML]{000000}{\fontsize{11}{11}\selectfont{\textbf{99}}}} & \multicolumn{1}{>{\centering}m{\dimexpr 1.02in+0\tabcolsep}}{\textcolor[HTML]{000000}{\fontsize{11}{11}\selectfont{\textbf{}}}} & \multicolumn{1}{>{\centering}m{\dimexpr 1.27in+0\tabcolsep}}{\textcolor[HTML]{000000}{\fontsize{11}{11}\selectfont{\textbf{216}}}} & \multicolumn{1}{>{\centering}m{\dimexpr 1.02in+0\tabcolsep}}{\textcolor[HTML]{000000}{\fontsize{11}{11}\selectfont{\textbf{}}}} \\

\ascline{1.5pt}{666666}{1-5}



\end{longtable}



\arrayrulecolor[HTML]{000000}

\global\setlength{\arrayrulewidth}{\Oldarrayrulewidth}

\global\setlength{\tabcolsep}{\Oldtabcolsep}

\renewcommand*{\arraystretch}{1}

\hypertarget{f_formato.preferido}{%
\paragraph{\texorpdfstring{\texttt{f\_Formato.Preferido}}{f\_Formato.Preferido}}\label{f_formato.preferido}}

Las categorías de \texttt{f\_Formato.Preferido} se diferencian en si el
encuestado prefiere que las ensaladas se paguen por peso o tengan un
precio fijado. Se trata de una variable cuyas categorías no guardan una
relación de orden entre sí.

\begin{Shaded}
\begin{Highlighting}[]
\CommentTok{\# Crear el data frame de frecuencias y porcentajes usando dplyr}
\NormalTok{datos\_grafico }\OtherTok{\textless{}{-}}\NormalTok{ datos }\SpecialCharTok{\%\textgreater{}\%}
  \CommentTok{\# Contar la frecuencia de cada nivel de la variable f\_Formato.Preferido}
  \FunctionTok{count}\NormalTok{(f\_Formato.Preferido, }\AttributeTok{name =} \StringTok{"Frecuencia"}\NormalTok{) }\SpecialCharTok{\%\textgreater{}\%}
  \CommentTok{\# Calcular el porcentaje y el texto de la etiqueta}
  \FunctionTok{mutate}\NormalTok{(}
    \AttributeTok{Porcentaje =}\NormalTok{ Frecuencia }\SpecialCharTok{/} \FunctionTok{sum}\NormalTok{(Frecuencia) }\SpecialCharTok{*} \DecValTok{100}\NormalTok{,}
    \AttributeTok{Etiquetas =} \FunctionTok{paste0}\NormalTok{(f\_Formato.Preferido, }\StringTok{"}\SpecialCharTok{\textbackslash{}n}\StringTok{("}\NormalTok{, }\FunctionTok{round}\NormalTok{(Porcentaje, }\DecValTok{1}\NormalTok{), }\StringTok{"\%)"}\NormalTok{) }\CommentTok{\# Etiqueta para el gráfico}
\NormalTok{  ) }\SpecialCharTok{\%\textgreater{}\%}
  \CommentTok{\# Eliminar filas con 0\% si existen, y categorías no deseadas}
  \FunctionTok{filter}\NormalTok{(Porcentaje }\SpecialCharTok{\textgreater{}} \DecValTok{0} \SpecialCharTok{\&}\NormalTok{ f\_Formato.Preferido }\SpecialCharTok{!=} \StringTok{"Total"}\NormalTok{) }\CommentTok{\# Se filtra \textquotesingle{}Total\textquotesingle{} aunque dplyr no lo añade}

\CommentTok{\# Usamos el vector de Porcentaje para el tamaño de las porciones}
\FunctionTok{pie}\NormalTok{(datos\_grafico}\SpecialCharTok{$}\NormalTok{Porcentaje,}
    \CommentTok{\# Usamos las Etiquetas que combinan nombre y porcentaje}
    \AttributeTok{labels =}\NormalTok{ datos\_grafico}\SpecialCharTok{$}\NormalTok{Etiquetas,}
    \CommentTok{\# Título}
    \AttributeTok{main =} \StringTok{\textquotesingle{}Figura 7. Formato preferido\textquotesingle{}}\NormalTok{,}
    \CommentTok{\# Asignar un color diferente a cada porción}
    \AttributeTok{col =}\NormalTok{ RColorBrewer}\SpecialCharTok{::}\FunctionTok{brewer.pal}\NormalTok{(}\AttributeTok{n =} \FunctionTok{nrow}\NormalTok{(datos\_grafico), }\AttributeTok{name =} \StringTok{"Set3"}\NormalTok{))}
\end{Highlighting}
\end{Shaded}

\begin{center}\includegraphics{ICO-analisis_files/figure-latex/P11-tarta-1} \end{center}

Se observa que la mayoría optaría por un formato donde el precio de las
ensaldas es fijo (81\%), frente a los que prefieren que se pagara la
ensalada por peso (19\%).

\begin{Shaded}
\begin{Highlighting}[]
\CommentTok{\# Preparación de datos}
\NormalTok{df\_tabla\_formato }\OtherTok{\textless{}{-}}\NormalTok{ datos }\SpecialCharTok{\%\textgreater{}\%}
  \CommentTok{\# Contar la frecuencia de cada nivel}
\NormalTok{  dplyr}\SpecialCharTok{::}\FunctionTok{count}\NormalTok{(f\_Formato.Preferido, }\AttributeTok{name =} \StringTok{"Frecuencia"}\NormalTok{) }\SpecialCharTok{\%\textgreater{}\%}
  \CommentTok{\# Calcular porcentajes}
\NormalTok{  dplyr}\SpecialCharTok{::}\FunctionTok{mutate}\NormalTok{(}
    \AttributeTok{Porcentaje =}\NormalTok{ Frecuencia }\SpecialCharTok{/} \FunctionTok{sum}\NormalTok{(Frecuencia) }\SpecialCharTok{*} \DecValTok{100}
\NormalTok{  ) }\SpecialCharTok{\%\textgreater{}\%}
  \CommentTok{\# Redondear y formatear los porcentajes}
\NormalTok{  dplyr}\SpecialCharTok{::}\FunctionTok{mutate}\NormalTok{(}
    \AttributeTok{Porcentaje =} \FunctionTok{paste0}\NormalTok{(}\FunctionTok{round}\NormalTok{(Porcentaje, }\DecValTok{1}\NormalTok{), }\StringTok{"\%"}\NormalTok{)}
\NormalTok{  )}

\CommentTok{\# Calcular y añadir la fila total}
\NormalTok{df\_total }\OtherTok{\textless{}{-}} \FunctionTok{data.frame}\NormalTok{(}
  \AttributeTok{f\_Formato.Preferido =} \StringTok{"Total"}\NormalTok{,}
  \AttributeTok{Frecuencia =} \FunctionTok{sum}\NormalTok{(df\_tabla\_formato}\SpecialCharTok{$}\NormalTok{Frecuencia),}
  \AttributeTok{Porcentaje =} \StringTok{"100.0\%"}
\NormalTok{)}

\CommentTok{\# Unir el data frame de frecuencias con la fila total}
\NormalTok{df\_tabla\_final }\OtherTok{\textless{}{-}} \FunctionTok{bind\_rows}\NormalTok{(df\_tabla\_formato, df\_total)}

\CommentTok{\# Generación de la tabla (flextable)}
\NormalTok{ft }\OtherTok{\textless{}{-}}\NormalTok{ flextable}\SpecialCharTok{::}\FunctionTok{flextable}\NormalTok{(df\_tabla\_final) }\SpecialCharTok{\%\textgreater{}\%}

\NormalTok{flextable}\SpecialCharTok{::}\FunctionTok{set\_caption}\NormalTok{(}\AttributeTok{caption =} \StringTok{"Tabla 7. Distribución de frecuencias para Formato preferido"}\NormalTok{) }\SpecialCharTok{\%\textgreater{}\%}
  
  \CommentTok{\# Renombrar las cabeceras}
\NormalTok{  flextable}\SpecialCharTok{::}\FunctionTok{set\_header\_labels}\NormalTok{(}
    \AttributeTok{f\_Formato.Preferido =} \StringTok{"Formato Preferido"}\NormalTok{,}
    \AttributeTok{Frecuencia =} \StringTok{"Frecuencia (n)"}\NormalTok{,}
    \AttributeTok{Porcentaje =} \StringTok{"Porcentaje"}
\NormalTok{  ) }\SpecialCharTok{\%\textgreater{}\%}
  
  \CommentTok{\# Aplicar negrita y bordes a la fila "Total"}
\NormalTok{  flextable}\SpecialCharTok{::}\FunctionTok{bold}\NormalTok{(}\AttributeTok{i =} \FunctionTok{nrow}\NormalTok{(df\_tabla\_final), }\AttributeTok{part =} \StringTok{"body"}\NormalTok{) }\SpecialCharTok{\%\textgreater{}\%} \CommentTok{\# Negrita a la última fila (Total)}
\NormalTok{  flextable}\SpecialCharTok{::}\FunctionTok{border\_remove}\NormalTok{() }\SpecialCharTok{\%\textgreater{}\%} \CommentTok{\# Quitar bordes predeterminados}
\NormalTok{  flextable}\SpecialCharTok{::}\FunctionTok{theme\_booktabs}\NormalTok{() }\SpecialCharTok{\%\textgreater{}\%} \CommentTok{\# Aplicar un tema con líneas horizontales}
  
  \CommentTok{\# Formato de alineación y cabecera}
\NormalTok{  flextable}\SpecialCharTok{::}\FunctionTok{align}\NormalTok{(}\AttributeTok{j =} \DecValTok{1}\NormalTok{, }\AttributeTok{align =} \StringTok{"left"}\NormalTok{, }\AttributeTok{part =} \StringTok{"body"}\NormalTok{) }\SpecialCharTok{\%\textgreater{}\%}
\NormalTok{  flextable}\SpecialCharTok{::}\FunctionTok{align}\NormalTok{(}\AttributeTok{j =} \DecValTok{2}\SpecialCharTok{:}\DecValTok{3}\NormalTok{, }\AttributeTok{align =} \StringTok{"center"}\NormalTok{, }\AttributeTok{part =} \StringTok{"all"}\NormalTok{) }\SpecialCharTok{\%\textgreater{}\%} \CommentTok{\# Columnas 2 y 3 (Datos) CENTRADAS}
\NormalTok{  flextable}\SpecialCharTok{::}\FunctionTok{align}\NormalTok{(}\AttributeTok{align =} \StringTok{"center"}\NormalTok{, }\AttributeTok{part =} \StringTok{"header"}\NormalTok{) }\SpecialCharTok{\%\textgreater{}\%}        \CommentTok{\# Encabezados CENTRADOS}
  
  \CommentTok{\# Añadir una línea superior a la fila "Total" para separarla}
\NormalTok{  flextable}\SpecialCharTok{::}\FunctionTok{hline}\NormalTok{(}\AttributeTok{i =} \FunctionTok{nrow}\NormalTok{(df\_tabla\_final) }\SpecialCharTok{{-}} \DecValTok{1}\NormalTok{, }\AttributeTok{border =}\NormalTok{ officer}\SpecialCharTok{::}\FunctionTok{fp\_border}\NormalTok{(}\AttributeTok{width =} \FloatTok{1.5}\NormalTok{, }\AttributeTok{color =} \StringTok{"black"}\NormalTok{)) }\SpecialCharTok{\%\textgreater{}\%}
  
  \CommentTok{\# Ajustar el ancho de las columnas}
\NormalTok{  flextable}\SpecialCharTok{::}\FunctionTok{autofit}\NormalTok{()}

\CommentTok{\# Mostrar la tabla}
\NormalTok{ft}
\end{Highlighting}
\end{Shaded}

\global\setlength{\Oldarrayrulewidth}{\arrayrulewidth}

\global\setlength{\Oldtabcolsep}{\tabcolsep}

\setlength{\tabcolsep}{2pt}

\renewcommand*{\arraystretch}{1.5}



\providecommand{\ascline}[3]{\noalign{\global\arrayrulewidth #1}\arrayrulecolor[HTML]{#2}\cline{#3}}

\begin{longtable}[c]{|p{1.52in}|p{1.27in}|p{1.02in}}

\caption{Tabla\ 7.\ Distribución\ de\ frecuencias\ para\ Formato\ preferido}\\

\ascline{1.5pt}{666666}{1-3}

\multicolumn{1}{>{\centering}m{\dimexpr 1.52in+0\tabcolsep}}{\textcolor[HTML]{000000}{\fontsize{11}{11}\selectfont{Formato\ Preferido}}} & \multicolumn{1}{>{\centering}m{\dimexpr 1.27in+0\tabcolsep}}{\textcolor[HTML]{000000}{\fontsize{11}{11}\selectfont{Frecuencia\ (n)}}} & \multicolumn{1}{>{\centering}m{\dimexpr 1.02in+0\tabcolsep}}{\textcolor[HTML]{000000}{\fontsize{11}{11}\selectfont{Porcentaje}}} \\

\ascline{1.5pt}{666666}{1-3}\endfirsthead \caption[]{Tabla\ 7.\ Distribución\ de\ frecuencias\ para\ Formato\ preferido}\\

\ascline{1.5pt}{666666}{1-3}

\multicolumn{1}{>{\centering}m{\dimexpr 1.52in+0\tabcolsep}}{\textcolor[HTML]{000000}{\fontsize{11}{11}\selectfont{Formato\ Preferido}}} & \multicolumn{1}{>{\centering}m{\dimexpr 1.27in+0\tabcolsep}}{\textcolor[HTML]{000000}{\fontsize{11}{11}\selectfont{Frecuencia\ (n)}}} & \multicolumn{1}{>{\centering}m{\dimexpr 1.02in+0\tabcolsep}}{\textcolor[HTML]{000000}{\fontsize{11}{11}\selectfont{Porcentaje}}} \\

\ascline{1.5pt}{666666}{1-3}\endhead



\multicolumn{1}{>{\raggedright}m{\dimexpr 1.52in+0\tabcolsep}}{\textcolor[HTML]{000000}{\fontsize{11}{11}\selectfont{Precio\ Fijo}}} & \multicolumn{1}{>{\centering}m{\dimexpr 1.27in+0\tabcolsep}}{\textcolor[HTML]{000000}{\fontsize{11}{11}\selectfont{51}}} & \multicolumn{1}{>{\centering}m{\dimexpr 1.02in+0\tabcolsep}}{\textcolor[HTML]{000000}{\fontsize{11}{11}\selectfont{81\%}}} \\





\multicolumn{1}{>{\raggedright}m{\dimexpr 1.52in+0\tabcolsep}}{\textcolor[HTML]{000000}{\fontsize{11}{11}\selectfont{Pago\ Peso}}} & \multicolumn{1}{>{\centering}m{\dimexpr 1.27in+0\tabcolsep}}{\textcolor[HTML]{000000}{\fontsize{11}{11}\selectfont{12}}} & \multicolumn{1}{>{\centering}m{\dimexpr 1.02in+0\tabcolsep}}{\textcolor[HTML]{000000}{\fontsize{11}{11}\selectfont{19\%}}} \\

\ascline{1.5pt}{000000}{1-3}



\multicolumn{1}{>{\raggedright}m{\dimexpr 1.52in+0\tabcolsep}}{\textcolor[HTML]{000000}{\fontsize{11}{11}\selectfont{\textbf{Total}}}} & \multicolumn{1}{>{\centering}m{\dimexpr 1.27in+0\tabcolsep}}{\textcolor[HTML]{000000}{\fontsize{11}{11}\selectfont{\textbf{63}}}} & \multicolumn{1}{>{\centering}m{\dimexpr 1.02in+0\tabcolsep}}{\textcolor[HTML]{000000}{\fontsize{11}{11}\selectfont{\textbf{100.0\%}}}} \\

\ascline{1.5pt}{666666}{1-3}



\end{longtable}



\arrayrulecolor[HTML]{000000}

\global\setlength{\arrayrulewidth}{\Oldarrayrulewidth}

\global\setlength{\tabcolsep}{\Oldtabcolsep}

\renewcommand*{\arraystretch}{1}

\hypertarget{f_preferencia.base}{%
\paragraph{\texorpdfstring{\texttt{f\_Preferencia.Base}}{f\_Preferencia.Base}}\label{f_preferencia.base}}

Las categorías de \texttt{f\_Preferencia.Base} hacen referencia a las
bases para la ensalada que los encuestados aplicarían a sus ensaladas.
Es importante destacar que esta variable proviene de una pregunta con
respuesta múltiple por lo que optamos por un gráfico de barras múltiple
basándonos en si ha se ha seleccionado cada respuesta o no.

\begin{Shaded}
\begin{Highlighting}[]
\CommentTok{\# Definir las variables a analizar (P10: Situaciones de Uso)}
\NormalTok{multi\_vars }\OtherTok{\textless{}{-}} \FunctionTok{c}\NormalTok{(}\StringTok{"f\_Base.Lechuga"}\NormalTok{, }\StringTok{"f\_Base.Verde"}\NormalTok{, }\StringTok{"f\_Base.Pasta"}\NormalTok{, }\StringTok{"f\_Base.Quinoa"}\NormalTok{, }\StringTok{"f\_Base.Cuscus"}\NormalTok{, }\StringTok{"f\_Base.Legum"}\NormalTok{)}

\CommentTok{\# Transformar a formato largo y contar frecuencias de SÍ/NO}
\NormalTok{df\_barras\_dobles }\OtherTok{\textless{}{-}}\NormalTok{ datos }\SpecialCharTok{\%\textgreater{}\%}
\NormalTok{  dplyr}\SpecialCharTok{::}\FunctionTok{select}\NormalTok{(}\FunctionTok{all\_of}\NormalTok{(multi\_vars)) }\SpecialCharTok{\%\textgreater{}\%}
\NormalTok{  tidyr}\SpecialCharTok{::}\FunctionTok{pivot\_longer}\NormalTok{(}
    \AttributeTok{cols =} \FunctionTok{everything}\NormalTok{(), }
    \AttributeTok{names\_to =} \StringTok{"Preferencia\_Base"}\NormalTok{, }
    \AttributeTok{values\_to =} \StringTok{"Respuesta"}
\NormalTok{  ) }\SpecialCharTok{\%\textgreater{}\%}
\NormalTok{  dplyr}\SpecialCharTok{::}\FunctionTok{count}\NormalTok{(Preferencia\_Base, Respuesta, }\AttributeTok{name =} \StringTok{"Frecuencia"}\NormalTok{) }\SpecialCharTok{\%\textgreater{}\%}
\NormalTok{  dplyr}\SpecialCharTok{::}\FunctionTok{mutate}\NormalTok{(}
    \AttributeTok{Preferencia\_Base =} \FunctionTok{gsub}\NormalTok{(}\StringTok{"f\_Base."}\NormalTok{, }\StringTok{""}\NormalTok{, Preferencia\_Base, }\AttributeTok{fixed =} \ConstantTok{TRUE}\NormalTok{),}
    \AttributeTok{Preferencia\_Base =}\NormalTok{ tools}\SpecialCharTok{::}\FunctionTok{toTitleCase}\NormalTok{(Preferencia\_Base),}
    \AttributeTok{Respuesta =} \FunctionTok{factor}\NormalTok{(Respuesta, }\AttributeTok{levels =} \FunctionTok{c}\NormalTok{(}\StringTok{"No"}\NormalTok{, }\StringTok{"Sí"}\NormalTok{)) }
\NormalTok{  )}

\CommentTok{\# Generación del Gráfico de Barras Agrupadas (Horizontal)}
\FunctionTok{ggplot}\NormalTok{(df\_barras\_dobles, }\FunctionTok{aes}\NormalTok{(}\AttributeTok{x =}\NormalTok{ Preferencia\_Base, }\AttributeTok{y =}\NormalTok{ Frecuencia, }\AttributeTok{fill =}\NormalTok{ Respuesta)) }\SpecialCharTok{+}
  
  \CommentTok{\# Barras agrupadas}
  \FunctionTok{geom\_bar}\NormalTok{(}\AttributeTok{stat =} \StringTok{"identity"}\NormalTok{, }\AttributeTok{position =} \FunctionTok{position\_dodge}\NormalTok{(}\AttributeTok{width =} \FloatTok{0.9}\NormalTok{)) }\SpecialCharTok{+} 
  
  \CommentTok{\# Etiquetas de Frecuencia}
  \FunctionTok{geom\_text}\NormalTok{(}
    \FunctionTok{aes}\NormalTok{(}\AttributeTok{label =}\NormalTok{ Frecuencia),}
    \AttributeTok{position =} \FunctionTok{position\_dodge}\NormalTok{(}\AttributeTok{width =} \FloatTok{0.9}\NormalTok{),}
    \AttributeTok{hjust =} \SpecialCharTok{{-}}\FloatTok{0.2}\NormalTok{, }\CommentTok{\# Mueve la etiqueta {-}0.2 unidades a lo largo del eje X (separa de la barra)}
    \AttributeTok{size =} \FloatTok{3.5}
\NormalTok{  ) }\SpecialCharTok{+}
  
  \CommentTok{\# Títulos y Ejes}
  \FunctionTok{labs}\NormalTok{(}
    \AttributeTok{title =} \StringTok{"Figura 8. Gráfico de barras (Sí/No) por Preferencia de Base"}\NormalTok{,}
    \AttributeTok{x =} \StringTok{"Preferencia Base"}\NormalTok{,}
    \AttributeTok{y =} \StringTok{"Número de Respuestas"}\NormalTok{,}
    \AttributeTok{fill =} \StringTok{"Usaría"}
\NormalTok{  ) }\SpecialCharTok{+}
  
  \CommentTok{\# Colores}
  \FunctionTok{scale\_fill\_manual}\NormalTok{(}\AttributeTok{values =} \FunctionTok{c}\NormalTok{(}\StringTok{"No"} \OtherTok{=} \StringTok{"\#a6cee3"}\NormalTok{, }\StringTok{"Sí"} \OtherTok{=} \StringTok{"\#1f78b4"}\NormalTok{)) }\SpecialCharTok{+} 
  
  \CommentTok{\# Invertir coordenadas para hacerlo horizontal}
  \FunctionTok{coord\_flip}\NormalTok{() }\SpecialCharTok{+} 
  
  \CommentTok{\# Ajuste del eje para que no se corten las etiquetas (Aumentamos el espacio)}
  \FunctionTok{scale\_y\_continuous}\NormalTok{(}\AttributeTok{expand =} \FunctionTok{expansion}\NormalTok{(}\AttributeTok{mult =} \FunctionTok{c}\NormalTok{(}\DecValTok{0}\NormalTok{, }\FloatTok{0.20}\NormalTok{))) }\SpecialCharTok{+} 
  
  \CommentTok{\# Ajuste de Tema}
  \FunctionTok{theme}\NormalTok{(}\AttributeTok{legend.position =} \StringTok{"bottom"}\NormalTok{)}
\end{Highlighting}
\end{Shaded}

\begin{center}\includegraphics{ICO-analisis_files/figure-latex/P13-barras-1} \end{center}

Se observa que la base más que se usaría seria la mezcla verde de
rúcula, canónigos, espinacas\ldots{} (con 38 respuestas afirmativas),
seguido de pasta (con 29 respuestas afirmativas).

\begin{Shaded}
\begin{Highlighting}[]
\CommentTok{\# Transformar a formato largo, contar frecuencias de SÍ/NO y calcular porcentajes}
\NormalTok{df\_tabla\_temp }\OtherTok{\textless{}{-}}\NormalTok{ datos }\SpecialCharTok{\%\textgreater{}\%}
\NormalTok{  dplyr}\SpecialCharTok{::}\FunctionTok{select}\NormalTok{(}\FunctionTok{all\_of}\NormalTok{(multi\_vars)) }\SpecialCharTok{\%\textgreater{}\%}
\NormalTok{  tidyr}\SpecialCharTok{::}\FunctionTok{pivot\_longer}\NormalTok{(}
    \AttributeTok{cols =} \FunctionTok{everything}\NormalTok{(), }
    \AttributeTok{names\_to =} \StringTok{"Preferencia\_Base\_Raw"}\NormalTok{, }
    \AttributeTok{values\_to =} \StringTok{"Respuesta"} 
\NormalTok{  ) }\SpecialCharTok{\%\textgreater{}\%}
\NormalTok{  dplyr}\SpecialCharTok{::}\FunctionTok{count}\NormalTok{(Preferencia\_Base\_Raw, Respuesta, }\AttributeTok{name =} \StringTok{"Frecuencia"}\NormalTok{) }\SpecialCharTok{\%\textgreater{}\%}
\NormalTok{  dplyr}\SpecialCharTok{::}\FunctionTok{mutate}\NormalTok{(}
    \AttributeTok{Porcentaje =}\NormalTok{ (Frecuencia }\SpecialCharTok{/}\NormalTok{ n.filas) }\SpecialCharTok{*} \DecValTok{100}
\NormalTok{  ) }\SpecialCharTok{\%\textgreater{}\%}
\NormalTok{  tidyr}\SpecialCharTok{::}\FunctionTok{pivot\_wider}\NormalTok{(}
    \AttributeTok{names\_from =}\NormalTok{ Respuesta, }
    \AttributeTok{values\_from =} \FunctionTok{c}\NormalTok{(Frecuencia, Porcentaje),}
    \AttributeTok{values\_fill =} \DecValTok{0} 
\NormalTok{  ) }\SpecialCharTok{\%\textgreater{}\%}
\NormalTok{  dplyr}\SpecialCharTok{::}\FunctionTok{mutate}\NormalTok{(}
    \AttributeTok{Preferencia\_Base =} \FunctionTok{gsub}\NormalTok{(}\StringTok{"f\_Base."}\NormalTok{, }\StringTok{""}\NormalTok{, Preferencia\_Base\_Raw, }\AttributeTok{fixed =} \ConstantTok{TRUE}\NormalTok{),}
    \AttributeTok{Preferencia\_Base =}\NormalTok{ tools}\SpecialCharTok{::}\FunctionTok{toTitleCase}\NormalTok{(Preferencia\_Base)}
\NormalTok{  ) }\SpecialCharTok{\%\textgreater{}\%}
\NormalTok{  dplyr}\SpecialCharTok{::}\FunctionTok{select}\NormalTok{(}
\NormalTok{    Preferencia\_Base,}
    \StringTok{\textasciigrave{}}\AttributeTok{Frecuencia\_Sí}\StringTok{\textasciigrave{}}\NormalTok{, }\StringTok{\textasciigrave{}}\AttributeTok{Porcentaje\_Sí}\StringTok{\textasciigrave{}}\NormalTok{,}
    \StringTok{\textasciigrave{}}\AttributeTok{Frecuencia\_No}\StringTok{\textasciigrave{}}\NormalTok{, }\StringTok{\textasciigrave{}}\AttributeTok{Porcentaje\_No}\StringTok{\textasciigrave{}}
\NormalTok{  )}


\CommentTok{\# Fila de Total}
\NormalTok{df\_total }\OtherTok{\textless{}{-}} \FunctionTok{data.frame}\NormalTok{(}
  \AttributeTok{Preferencia\_Base =} \StringTok{"Total"}\NormalTok{,}
\NormalTok{  Frecuencia\_Sí }\OtherTok{=} \FunctionTok{sum}\NormalTok{(df\_tabla\_temp}\SpecialCharTok{$}\NormalTok{Frecuencia\_Sí),}
\NormalTok{  Porcentaje\_Sí }\OtherTok{=} \ConstantTok{NA}\NormalTok{, }\CommentTok{\# No se suma, sería engañoso}
  \AttributeTok{Frecuencia\_No =} \FunctionTok{sum}\NormalTok{(df\_tabla\_temp}\SpecialCharTok{$}\NormalTok{Frecuencia\_No),}
  \AttributeTok{Porcentaje\_No =} \ConstantTok{NA} \CommentTok{\# No se suma, sería engañoso}
\NormalTok{)}

\CommentTok{\# Unir el data frame con la fila total}
\NormalTok{df\_tabla\_final }\OtherTok{\textless{}{-}} \FunctionTok{bind\_rows}\NormalTok{(df\_tabla\_temp, df\_total)}


\CommentTok{\# Generación de la tabla (flextable)}
\NormalTok{ft\_si\_no }\OtherTok{\textless{}{-}}\NormalTok{ flextable}\SpecialCharTok{::}\FunctionTok{flextable}\NormalTok{(df\_tabla\_final) }\SpecialCharTok{\%\textgreater{}\%}
  
  \CommentTok{\# Añadir Título}
\NormalTok{  flextable}\SpecialCharTok{::}\FunctionTok{set\_caption}\NormalTok{(}\AttributeTok{caption =} \StringTok{"Tabla 8. Distribución de Frecuencias (Sí/No) para Preferencia Base"}\NormalTok{) }\SpecialCharTok{\%\textgreater{}\%}
  
  \CommentTok{\# Renombrar cabeceras (se usan para la segunda fila de la cabecera)}
\NormalTok{  flextable}\SpecialCharTok{::}\FunctionTok{set\_header\_labels}\NormalTok{(}
    \AttributeTok{Preferencia\_Base =} \StringTok{"Preferencia Base"}\NormalTok{,}
\NormalTok{    Frecuencia\_Sí }\OtherTok{=} \StringTok{"Frecuencia (n)"}\NormalTok{,}
\NormalTok{    Porcentaje\_Sí }\OtherTok{=} \StringTok{"Porcentaje"}\NormalTok{,}
    \AttributeTok{Frecuencia\_No =} \StringTok{"Frecuencia (n)"}\NormalTok{,}
    \AttributeTok{Porcentaje\_No =} \StringTok{"Porcentaje"}
\NormalTok{  ) }\SpecialCharTok{\%\textgreater{}\%}
  
  \CommentTok{\# Agrupar las columnas \textquotesingle{}Sí\textquotesingle{} y \textquotesingle{}No\textquotesingle{} en una fila superior}
\NormalTok{  flextable}\SpecialCharTok{::}\FunctionTok{add\_header\_row}\NormalTok{(}
    \AttributeTok{values =} \FunctionTok{c}\NormalTok{(}\StringTok{" "}\NormalTok{, }\StringTok{"SÍ"}\NormalTok{, }\StringTok{"NO"}\NormalTok{),}
    \AttributeTok{colwidths =} \FunctionTok{c}\NormalTok{(}\DecValTok{1}\NormalTok{, }\DecValTok{2}\NormalTok{, }\DecValTok{2}\NormalTok{)}
\NormalTok{  ) }\SpecialCharTok{\%\textgreater{}\%}
  
  \CommentTok{\# Formato y Tema}
\NormalTok{  flextable}\SpecialCharTok{::}\FunctionTok{colformat\_double}\NormalTok{(}\AttributeTok{j =} \FunctionTok{c}\NormalTok{(}\DecValTok{3}\NormalTok{, }\DecValTok{5}\NormalTok{), }\AttributeTok{digits =} \DecValTok{1}\NormalTok{, }\AttributeTok{suffix =} \StringTok{"\%"}\NormalTok{, }\AttributeTok{na\_str =} \StringTok{""}\NormalTok{) }\SpecialCharTok{\%\textgreater{}\%} 
  
\NormalTok{  flextable}\SpecialCharTok{::}\FunctionTok{theme\_booktabs}\NormalTok{() }\SpecialCharTok{\%\textgreater{}\%} 
  
  \CommentTok{\# Alineación}
  \CommentTok{\# Aplicar negrita y bordes a la fila "Total" (la última fila)}
\NormalTok{  flextable}\SpecialCharTok{::}\FunctionTok{bold}\NormalTok{(}\AttributeTok{i =} \FunctionTok{nrow}\NormalTok{(df\_tabla\_final), }\AttributeTok{part =} \StringTok{"body"}\NormalTok{) }\SpecialCharTok{\%\textgreater{}\%} 
\NormalTok{  flextable}\SpecialCharTok{::}\FunctionTok{hline}\NormalTok{(}\AttributeTok{i =} \FunctionTok{nrow}\NormalTok{(df\_tabla\_final) }\SpecialCharTok{{-}} \DecValTok{1}\NormalTok{, }\AttributeTok{border =}\NormalTok{ officer}\SpecialCharTok{::}\FunctionTok{fp\_border}\NormalTok{(}\AttributeTok{width =} \FloatTok{1.5}\NormalTok{, }\AttributeTok{color =} \StringTok{"black"}\NormalTok{)) }\SpecialCharTok{\%\textgreater{}\%}
  
  \CommentTok{\# Alineación}
\NormalTok{  flextable}\SpecialCharTok{::}\FunctionTok{align}\NormalTok{(}\AttributeTok{align =} \StringTok{"center"}\NormalTok{, }\AttributeTok{part =} \StringTok{"all"}\NormalTok{) }\SpecialCharTok{\%\textgreater{}\%}
\NormalTok{  flextable}\SpecialCharTok{::}\FunctionTok{align}\NormalTok{(}\AttributeTok{j =} \DecValTok{1}\NormalTok{, }\AttributeTok{align =} \StringTok{"left"}\NormalTok{, }\AttributeTok{part =} \StringTok{"body"}\NormalTok{) }\SpecialCharTok{\%\textgreater{}\%}
  
  \CommentTok{\# Ajustar}
\NormalTok{  flextable}\SpecialCharTok{::}\FunctionTok{autofit}\NormalTok{()}

\CommentTok{\# Mostrar la tabla}
\NormalTok{ft\_si\_no}
\end{Highlighting}
\end{Shaded}

\global\setlength{\Oldarrayrulewidth}{\arrayrulewidth}

\global\setlength{\Oldtabcolsep}{\tabcolsep}

\setlength{\tabcolsep}{2pt}

\renewcommand*{\arraystretch}{1.5}



\providecommand{\ascline}[3]{\noalign{\global\arrayrulewidth #1}\arrayrulecolor[HTML]{#2}\cline{#3}}

\begin{longtable}[c]{|p{1.46in}|p{1.27in}|p{1.02in}|p{1.27in}|p{1.02in}}

\caption{Tabla\ 8.\ Distribución\ de\ Frecuencias\ (Sí/No)\ para\ Preferencia\ Base}\\

\ascline{1.5pt}{666666}{1-5}

\multicolumn{1}{>{\centering}m{\dimexpr 1.46in+0\tabcolsep}}{\textcolor[HTML]{000000}{\fontsize{11}{11}\selectfont{\ }}} & \multicolumn{2}{>{\centering}m{\dimexpr 2.29in+2\tabcolsep}}{\textcolor[HTML]{000000}{\fontsize{11}{11}\selectfont{SÍ}}} & \multicolumn{2}{>{\centering}m{\dimexpr 2.29in+2\tabcolsep}}{\textcolor[HTML]{000000}{\fontsize{11}{11}\selectfont{NO}}} \\





\multicolumn{1}{>{\centering}m{\dimexpr 1.46in+0\tabcolsep}}{\textcolor[HTML]{000000}{\fontsize{11}{11}\selectfont{Preferencia\ Base}}} & \multicolumn{1}{>{\centering}m{\dimexpr 1.27in+0\tabcolsep}}{\textcolor[HTML]{000000}{\fontsize{11}{11}\selectfont{Frecuencia\ (n)}}} & \multicolumn{1}{>{\centering}m{\dimexpr 1.02in+0\tabcolsep}}{\textcolor[HTML]{000000}{\fontsize{11}{11}\selectfont{Porcentaje}}} & \multicolumn{1}{>{\centering}m{\dimexpr 1.27in+0\tabcolsep}}{\textcolor[HTML]{000000}{\fontsize{11}{11}\selectfont{Frecuencia\ (n)}}} & \multicolumn{1}{>{\centering}m{\dimexpr 1.02in+0\tabcolsep}}{\textcolor[HTML]{000000}{\fontsize{11}{11}\selectfont{Porcentaje}}} \\

\ascline{1.5pt}{666666}{1-5}\endfirsthead \caption[]{Tabla\ 8.\ Distribución\ de\ Frecuencias\ (Sí/No)\ para\ Preferencia\ Base}\\

\ascline{1.5pt}{666666}{1-5}

\multicolumn{1}{>{\centering}m{\dimexpr 1.46in+0\tabcolsep}}{\textcolor[HTML]{000000}{\fontsize{11}{11}\selectfont{\ }}} & \multicolumn{2}{>{\centering}m{\dimexpr 2.29in+2\tabcolsep}}{\textcolor[HTML]{000000}{\fontsize{11}{11}\selectfont{SÍ}}} & \multicolumn{2}{>{\centering}m{\dimexpr 2.29in+2\tabcolsep}}{\textcolor[HTML]{000000}{\fontsize{11}{11}\selectfont{NO}}} \\





\multicolumn{1}{>{\centering}m{\dimexpr 1.46in+0\tabcolsep}}{\textcolor[HTML]{000000}{\fontsize{11}{11}\selectfont{Preferencia\ Base}}} & \multicolumn{1}{>{\centering}m{\dimexpr 1.27in+0\tabcolsep}}{\textcolor[HTML]{000000}{\fontsize{11}{11}\selectfont{Frecuencia\ (n)}}} & \multicolumn{1}{>{\centering}m{\dimexpr 1.02in+0\tabcolsep}}{\textcolor[HTML]{000000}{\fontsize{11}{11}\selectfont{Porcentaje}}} & \multicolumn{1}{>{\centering}m{\dimexpr 1.27in+0\tabcolsep}}{\textcolor[HTML]{000000}{\fontsize{11}{11}\selectfont{Frecuencia\ (n)}}} & \multicolumn{1}{>{\centering}m{\dimexpr 1.02in+0\tabcolsep}}{\textcolor[HTML]{000000}{\fontsize{11}{11}\selectfont{Porcentaje}}} \\

\ascline{1.5pt}{666666}{1-5}\endhead



\multicolumn{1}{>{\raggedright}m{\dimexpr 1.46in+0\tabcolsep}}{\textcolor[HTML]{000000}{\fontsize{11}{11}\selectfont{Cuscus}}} & \multicolumn{1}{>{\centering}m{\dimexpr 1.27in+0\tabcolsep}}{\textcolor[HTML]{000000}{\fontsize{11}{11}\selectfont{5}}} & \multicolumn{1}{>{\centering}m{\dimexpr 1.02in+0\tabcolsep}}{\textcolor[HTML]{000000}{\fontsize{11}{11}\selectfont{7.9\%}}} & \multicolumn{1}{>{\centering}m{\dimexpr 1.27in+0\tabcolsep}}{\textcolor[HTML]{000000}{\fontsize{11}{11}\selectfont{58}}} & \multicolumn{1}{>{\centering}m{\dimexpr 1.02in+0\tabcolsep}}{\textcolor[HTML]{000000}{\fontsize{11}{11}\selectfont{92.1\%}}} \\





\multicolumn{1}{>{\raggedright}m{\dimexpr 1.46in+0\tabcolsep}}{\textcolor[HTML]{000000}{\fontsize{11}{11}\selectfont{Lechuga}}} & \multicolumn{1}{>{\centering}m{\dimexpr 1.27in+0\tabcolsep}}{\textcolor[HTML]{000000}{\fontsize{11}{11}\selectfont{25}}} & \multicolumn{1}{>{\centering}m{\dimexpr 1.02in+0\tabcolsep}}{\textcolor[HTML]{000000}{\fontsize{11}{11}\selectfont{39.7\%}}} & \multicolumn{1}{>{\centering}m{\dimexpr 1.27in+0\tabcolsep}}{\textcolor[HTML]{000000}{\fontsize{11}{11}\selectfont{38}}} & \multicolumn{1}{>{\centering}m{\dimexpr 1.02in+0\tabcolsep}}{\textcolor[HTML]{000000}{\fontsize{11}{11}\selectfont{60.3\%}}} \\





\multicolumn{1}{>{\raggedright}m{\dimexpr 1.46in+0\tabcolsep}}{\textcolor[HTML]{000000}{\fontsize{11}{11}\selectfont{Legum}}} & \multicolumn{1}{>{\centering}m{\dimexpr 1.27in+0\tabcolsep}}{\textcolor[HTML]{000000}{\fontsize{11}{11}\selectfont{7}}} & \multicolumn{1}{>{\centering}m{\dimexpr 1.02in+0\tabcolsep}}{\textcolor[HTML]{000000}{\fontsize{11}{11}\selectfont{11.1\%}}} & \multicolumn{1}{>{\centering}m{\dimexpr 1.27in+0\tabcolsep}}{\textcolor[HTML]{000000}{\fontsize{11}{11}\selectfont{56}}} & \multicolumn{1}{>{\centering}m{\dimexpr 1.02in+0\tabcolsep}}{\textcolor[HTML]{000000}{\fontsize{11}{11}\selectfont{88.9\%}}} \\





\multicolumn{1}{>{\raggedright}m{\dimexpr 1.46in+0\tabcolsep}}{\textcolor[HTML]{000000}{\fontsize{11}{11}\selectfont{Pasta}}} & \multicolumn{1}{>{\centering}m{\dimexpr 1.27in+0\tabcolsep}}{\textcolor[HTML]{000000}{\fontsize{11}{11}\selectfont{29}}} & \multicolumn{1}{>{\centering}m{\dimexpr 1.02in+0\tabcolsep}}{\textcolor[HTML]{000000}{\fontsize{11}{11}\selectfont{46.0\%}}} & \multicolumn{1}{>{\centering}m{\dimexpr 1.27in+0\tabcolsep}}{\textcolor[HTML]{000000}{\fontsize{11}{11}\selectfont{34}}} & \multicolumn{1}{>{\centering}m{\dimexpr 1.02in+0\tabcolsep}}{\textcolor[HTML]{000000}{\fontsize{11}{11}\selectfont{54.0\%}}} \\





\multicolumn{1}{>{\raggedright}m{\dimexpr 1.46in+0\tabcolsep}}{\textcolor[HTML]{000000}{\fontsize{11}{11}\selectfont{Quinoa}}} & \multicolumn{1}{>{\centering}m{\dimexpr 1.27in+0\tabcolsep}}{\textcolor[HTML]{000000}{\fontsize{11}{11}\selectfont{7}}} & \multicolumn{1}{>{\centering}m{\dimexpr 1.02in+0\tabcolsep}}{\textcolor[HTML]{000000}{\fontsize{11}{11}\selectfont{11.1\%}}} & \multicolumn{1}{>{\centering}m{\dimexpr 1.27in+0\tabcolsep}}{\textcolor[HTML]{000000}{\fontsize{11}{11}\selectfont{56}}} & \multicolumn{1}{>{\centering}m{\dimexpr 1.02in+0\tabcolsep}}{\textcolor[HTML]{000000}{\fontsize{11}{11}\selectfont{88.9\%}}} \\





\multicolumn{1}{>{\raggedright}m{\dimexpr 1.46in+0\tabcolsep}}{\textcolor[HTML]{000000}{\fontsize{11}{11}\selectfont{Verde}}} & \multicolumn{1}{>{\centering}m{\dimexpr 1.27in+0\tabcolsep}}{\textcolor[HTML]{000000}{\fontsize{11}{11}\selectfont{38}}} & \multicolumn{1}{>{\centering}m{\dimexpr 1.02in+0\tabcolsep}}{\textcolor[HTML]{000000}{\fontsize{11}{11}\selectfont{60.3\%}}} & \multicolumn{1}{>{\centering}m{\dimexpr 1.27in+0\tabcolsep}}{\textcolor[HTML]{000000}{\fontsize{11}{11}\selectfont{25}}} & \multicolumn{1}{>{\centering}m{\dimexpr 1.02in+0\tabcolsep}}{\textcolor[HTML]{000000}{\fontsize{11}{11}\selectfont{39.7\%}}} \\

\ascline{1.5pt}{000000}{1-5}



\multicolumn{1}{>{\raggedright}m{\dimexpr 1.46in+0\tabcolsep}}{\textcolor[HTML]{000000}{\fontsize{11}{11}\selectfont{\textbf{Total}}}} & \multicolumn{1}{>{\centering}m{\dimexpr 1.27in+0\tabcolsep}}{\textcolor[HTML]{000000}{\fontsize{11}{11}\selectfont{\textbf{111}}}} & \multicolumn{1}{>{\centering}m{\dimexpr 1.02in+0\tabcolsep}}{\textcolor[HTML]{000000}{\fontsize{11}{11}\selectfont{\textbf{}}}} & \multicolumn{1}{>{\centering}m{\dimexpr 1.27in+0\tabcolsep}}{\textcolor[HTML]{000000}{\fontsize{11}{11}\selectfont{\textbf{267}}}} & \multicolumn{1}{>{\centering}m{\dimexpr 1.02in+0\tabcolsep}}{\textcolor[HTML]{000000}{\fontsize{11}{11}\selectfont{\textbf{}}}} \\

\ascline{1.5pt}{666666}{1-5}



\end{longtable}



\arrayrulecolor[HTML]{000000}

\global\setlength{\arrayrulewidth}{\Oldarrayrulewidth}

\global\setlength{\tabcolsep}{\Oldtabcolsep}

\renewcommand*{\arraystretch}{1}

\hypertarget{f_preferencia.proteina}{%
\paragraph{\texorpdfstring{\texttt{f\_Preferencia.Proteina}}{f\_Preferencia.Proteina}}\label{f_preferencia.proteina}}

Las categorías de \texttt{f\_Preferencia.Proteina} hacen referencia a
las proteínas que los encuestados utilizarían pondrían en sus ensaladas.
Es importante destacar que esta variable proviene de una pregunta con
respuesta múltiple por lo que optamos por un gráfico de barras múltiple
basándonos en si ha se ha seleccionado cada respuesta o no.

\begin{Shaded}
\begin{Highlighting}[]
\CommentTok{\# Definir las variables a analizar (P10: Situaciones de Uso)}
\NormalTok{multi\_vars }\OtherTok{\textless{}{-}} \FunctionTok{c}\NormalTok{(}\StringTok{"f\_Prot.Pollo"}\NormalTok{, }\StringTok{"f\_Prot.Pescado"}\NormalTok{, }\StringTok{"f\_Prot.Huevo"}\NormalTok{, }\StringTok{"f\_Prot.Queso"}\NormalTok{, }\StringTok{"f\_Prot.Tofu"}\NormalTok{, }\StringTok{"f\_Prot.Marisco"}\NormalTok{, }\StringTok{"f\_Prot.Carne"}\NormalTok{)}

\CommentTok{\# Transformar a formato largo y contar frecuencias de SÍ/NO}
\NormalTok{df\_barras\_dobles }\OtherTok{\textless{}{-}}\NormalTok{ datos }\SpecialCharTok{\%\textgreater{}\%}
\NormalTok{  dplyr}\SpecialCharTok{::}\FunctionTok{select}\NormalTok{(}\FunctionTok{all\_of}\NormalTok{(multi\_vars)) }\SpecialCharTok{\%\textgreater{}\%}
\NormalTok{  tidyr}\SpecialCharTok{::}\FunctionTok{pivot\_longer}\NormalTok{(}
    \AttributeTok{cols =} \FunctionTok{everything}\NormalTok{(), }
    \AttributeTok{names\_to =} \StringTok{"Prerencia\_Proteina"}\NormalTok{, }
    \AttributeTok{values\_to =} \StringTok{"Respuesta"}
\NormalTok{  ) }\SpecialCharTok{\%\textgreater{}\%}
\NormalTok{  dplyr}\SpecialCharTok{::}\FunctionTok{count}\NormalTok{(Prerencia\_Proteina, Respuesta, }\AttributeTok{name =} \StringTok{"Frecuencia"}\NormalTok{) }\SpecialCharTok{\%\textgreater{}\%}
\NormalTok{  dplyr}\SpecialCharTok{::}\FunctionTok{mutate}\NormalTok{(}
    \AttributeTok{Prerencia\_Proteina =} \FunctionTok{gsub}\NormalTok{(}\StringTok{"f\_Prot."}\NormalTok{, }\StringTok{""}\NormalTok{, Prerencia\_Proteina, }\AttributeTok{fixed =} \ConstantTok{TRUE}\NormalTok{),}
    \AttributeTok{Prerencia\_Proteina =}\NormalTok{ tools}\SpecialCharTok{::}\FunctionTok{toTitleCase}\NormalTok{(Prerencia\_Proteina),}
    \AttributeTok{Respuesta =} \FunctionTok{factor}\NormalTok{(Respuesta, }\AttributeTok{levels =} \FunctionTok{c}\NormalTok{(}\StringTok{"No"}\NormalTok{, }\StringTok{"Sí"}\NormalTok{)) }
\NormalTok{  )}

\CommentTok{\# Generación del Gráfico de Barras Agrupadas (Horizontal)}
\FunctionTok{ggplot}\NormalTok{(df\_barras\_dobles, }\FunctionTok{aes}\NormalTok{(}\AttributeTok{x =}\NormalTok{ Prerencia\_Proteina, }\AttributeTok{y =}\NormalTok{ Frecuencia, }\AttributeTok{fill =}\NormalTok{ Respuesta)) }\SpecialCharTok{+}
  
  \CommentTok{\# Barras agrupadas}
  \FunctionTok{geom\_bar}\NormalTok{(}\AttributeTok{stat =} \StringTok{"identity"}\NormalTok{, }\AttributeTok{position =} \FunctionTok{position\_dodge}\NormalTok{(}\AttributeTok{width =} \FloatTok{0.9}\NormalTok{)) }\SpecialCharTok{+} 
  
  \CommentTok{\# Etiquetas de Frecuencia}
  \FunctionTok{geom\_text}\NormalTok{(}
    \FunctionTok{aes}\NormalTok{(}\AttributeTok{label =}\NormalTok{ Frecuencia),}
    \AttributeTok{position =} \FunctionTok{position\_dodge}\NormalTok{(}\AttributeTok{width =} \FloatTok{0.9}\NormalTok{),}
    \AttributeTok{hjust =} \SpecialCharTok{{-}}\FloatTok{0.2}\NormalTok{, }\CommentTok{\# Mueve la etiqueta {-}0.2 unidades a lo largo del eje X (separa de la barra)}
    \AttributeTok{size =} \FloatTok{3.5}
\NormalTok{  ) }\SpecialCharTok{+}
  
  \CommentTok{\# Títulos y Ejes}
  \FunctionTok{labs}\NormalTok{(}
    \AttributeTok{title =} \StringTok{"Figura 9. Gráfico de barras (Sí/No) por Preferencia de Proteínas"}\NormalTok{,}
    \AttributeTok{x =} \StringTok{"Preferencia de Proteínas"}\NormalTok{,}
    \AttributeTok{y =} \StringTok{"Número de Respuestas"}\NormalTok{,}
    \AttributeTok{fill =} \StringTok{"Pondría"}
\NormalTok{  ) }\SpecialCharTok{+}
  
  \CommentTok{\# Colores}
  \FunctionTok{scale\_fill\_manual}\NormalTok{(}\AttributeTok{values =} \FunctionTok{c}\NormalTok{(}\StringTok{"No"} \OtherTok{=} \StringTok{"\#a6cee3"}\NormalTok{, }\StringTok{"Sí"} \OtherTok{=} \StringTok{"\#1f78b4"}\NormalTok{)) }\SpecialCharTok{+} 
  
  \CommentTok{\# Invertir coordenadas para hacerlo horizontal}
  \FunctionTok{coord\_flip}\NormalTok{() }\SpecialCharTok{+} 
  
  \CommentTok{\# Ajuste del eje para que no se corten las etiquetas (Aumentamos el espacio)}
  \FunctionTok{scale\_y\_continuous}\NormalTok{(}\AttributeTok{expand =} \FunctionTok{expansion}\NormalTok{(}\AttributeTok{mult =} \FunctionTok{c}\NormalTok{(}\DecValTok{0}\NormalTok{, }\FloatTok{0.20}\NormalTok{))) }\SpecialCharTok{+} 
  
  \CommentTok{\# Ajuste de Tema}
  \FunctionTok{theme}\NormalTok{(}\AttributeTok{legend.position =} \StringTok{"bottom"}\NormalTok{)}
\end{Highlighting}
\end{Shaded}

\begin{center}\includegraphics{ICO-analisis_files/figure-latex/P14-barras-1} \end{center}

Se observa que la proteína que más se pondría en la ensalada sería pollo
(con 40 respuestas afirmativas), seguido de huevo (con 29 respuestas
afirmativas).

\begin{Shaded}
\begin{Highlighting}[]
\CommentTok{\# Transformar a formato largo, contar frecuencias de SÍ/NO y calcular porcentajes}
\NormalTok{df\_tabla\_temp }\OtherTok{\textless{}{-}}\NormalTok{ datos }\SpecialCharTok{\%\textgreater{}\%}
\NormalTok{  dplyr}\SpecialCharTok{::}\FunctionTok{select}\NormalTok{(}\FunctionTok{all\_of}\NormalTok{(multi\_vars)) }\SpecialCharTok{\%\textgreater{}\%}
\NormalTok{  tidyr}\SpecialCharTok{::}\FunctionTok{pivot\_longer}\NormalTok{(}
    \AttributeTok{cols =} \FunctionTok{everything}\NormalTok{(), }
    \AttributeTok{names\_to =} \StringTok{"Preferencia\_Proteina\_Raw"}\NormalTok{, }
    \AttributeTok{values\_to =} \StringTok{"Respuesta"} 
\NormalTok{  ) }\SpecialCharTok{\%\textgreater{}\%}
\NormalTok{  dplyr}\SpecialCharTok{::}\FunctionTok{count}\NormalTok{(Preferencia\_Proteina\_Raw, Respuesta, }\AttributeTok{name =} \StringTok{"Frecuencia"}\NormalTok{) }\SpecialCharTok{\%\textgreater{}\%}
\NormalTok{  dplyr}\SpecialCharTok{::}\FunctionTok{mutate}\NormalTok{(}
    \AttributeTok{Porcentaje =}\NormalTok{ (Frecuencia }\SpecialCharTok{/}\NormalTok{ n.filas) }\SpecialCharTok{*} \DecValTok{100}
\NormalTok{  ) }\SpecialCharTok{\%\textgreater{}\%}
\NormalTok{  tidyr}\SpecialCharTok{::}\FunctionTok{pivot\_wider}\NormalTok{(}
    \AttributeTok{names\_from =}\NormalTok{ Respuesta, }
    \AttributeTok{values\_from =} \FunctionTok{c}\NormalTok{(Frecuencia, Porcentaje),}
    \AttributeTok{values\_fill =} \DecValTok{0} 
\NormalTok{  ) }\SpecialCharTok{\%\textgreater{}\%}
\NormalTok{  dplyr}\SpecialCharTok{::}\FunctionTok{mutate}\NormalTok{(}
    \AttributeTok{Preferencia\_Proteina =} \FunctionTok{gsub}\NormalTok{(}\StringTok{"f\_Prot."}\NormalTok{, }\StringTok{""}\NormalTok{, Preferencia\_Proteina\_Raw, }\AttributeTok{fixed =} \ConstantTok{TRUE}\NormalTok{),}
    \AttributeTok{Preferencia\_Proteina =}\NormalTok{ tools}\SpecialCharTok{::}\FunctionTok{toTitleCase}\NormalTok{(Preferencia\_Proteina)}
\NormalTok{  ) }\SpecialCharTok{\%\textgreater{}\%}
\NormalTok{  dplyr}\SpecialCharTok{::}\FunctionTok{select}\NormalTok{(}
\NormalTok{    Preferencia\_Proteina,}
    \StringTok{\textasciigrave{}}\AttributeTok{Frecuencia\_Sí}\StringTok{\textasciigrave{}}\NormalTok{, }\StringTok{\textasciigrave{}}\AttributeTok{Porcentaje\_Sí}\StringTok{\textasciigrave{}}\NormalTok{,}
    \StringTok{\textasciigrave{}}\AttributeTok{Frecuencia\_No}\StringTok{\textasciigrave{}}\NormalTok{, }\StringTok{\textasciigrave{}}\AttributeTok{Porcentaje\_No}\StringTok{\textasciigrave{}}
\NormalTok{  )}


\CommentTok{\# Fila de Total}
\NormalTok{df\_total }\OtherTok{\textless{}{-}} \FunctionTok{data.frame}\NormalTok{(}
  \AttributeTok{Preferencia\_Proteina =} \StringTok{"Total"}\NormalTok{,}
\NormalTok{  Frecuencia\_Sí }\OtherTok{=} \FunctionTok{sum}\NormalTok{(df\_tabla\_temp}\SpecialCharTok{$}\NormalTok{Frecuencia\_Sí),}
\NormalTok{  Porcentaje\_Sí }\OtherTok{=} \ConstantTok{NA}\NormalTok{, }\CommentTok{\# No se suma, sería engañoso}
  \AttributeTok{Frecuencia\_No =} \FunctionTok{sum}\NormalTok{(df\_tabla\_temp}\SpecialCharTok{$}\NormalTok{Frecuencia\_No),}
  \AttributeTok{Porcentaje\_No =} \ConstantTok{NA} \CommentTok{\# No se suma, sería engañoso}
\NormalTok{)}

\CommentTok{\# Unir el data frame con la fila total}
\NormalTok{df\_tabla\_final }\OtherTok{\textless{}{-}} \FunctionTok{bind\_rows}\NormalTok{(df\_tabla\_temp, df\_total)}


\CommentTok{\# Generación de la tabla (flextable)}
\NormalTok{ft\_si\_no }\OtherTok{\textless{}{-}}\NormalTok{ flextable}\SpecialCharTok{::}\FunctionTok{flextable}\NormalTok{(df\_tabla\_final) }\SpecialCharTok{\%\textgreater{}\%}
  
  \CommentTok{\# Añadir Título}
\NormalTok{  flextable}\SpecialCharTok{::}\FunctionTok{set\_caption}\NormalTok{(}\AttributeTok{caption =} \StringTok{"Tabla 9. Distribución de Frecuencias (Sí/No) para Preferencia de Proteína"}\NormalTok{) }\SpecialCharTok{\%\textgreater{}\%}
  
  \CommentTok{\# Renombrar cabeceras (se usan para la segunda fila de la cabecera)}
\NormalTok{  flextable}\SpecialCharTok{::}\FunctionTok{set\_header\_labels}\NormalTok{(}
    \AttributeTok{Preferencia\_Proteina =} \StringTok{"Preferencia de Proteína"}\NormalTok{,}
\NormalTok{    Frecuencia\_Sí }\OtherTok{=} \StringTok{"Frecuencia (n)"}\NormalTok{,}
\NormalTok{    Porcentaje\_Sí }\OtherTok{=} \StringTok{"Porcentaje"}\NormalTok{,}
    \AttributeTok{Frecuencia\_No =} \StringTok{"Frecuencia (n)"}\NormalTok{,}
    \AttributeTok{Porcentaje\_No =} \StringTok{"Porcentaje"}
\NormalTok{  ) }\SpecialCharTok{\%\textgreater{}\%}
  
  \CommentTok{\# Agrupar las columnas \textquotesingle{}Sí\textquotesingle{} y \textquotesingle{}No\textquotesingle{} en una fila superior}
\NormalTok{  flextable}\SpecialCharTok{::}\FunctionTok{add\_header\_row}\NormalTok{(}
    \AttributeTok{values =} \FunctionTok{c}\NormalTok{(}\StringTok{" "}\NormalTok{, }\StringTok{"SÍ"}\NormalTok{, }\StringTok{"NO"}\NormalTok{),}
    \AttributeTok{colwidths =} \FunctionTok{c}\NormalTok{(}\DecValTok{1}\NormalTok{, }\DecValTok{2}\NormalTok{, }\DecValTok{2}\NormalTok{)}
\NormalTok{  ) }\SpecialCharTok{\%\textgreater{}\%}
  
  \CommentTok{\# Formato y Tema}
\NormalTok{  flextable}\SpecialCharTok{::}\FunctionTok{colformat\_double}\NormalTok{(}\AttributeTok{j =} \FunctionTok{c}\NormalTok{(}\DecValTok{3}\NormalTok{, }\DecValTok{5}\NormalTok{), }\AttributeTok{digits =} \DecValTok{1}\NormalTok{, }\AttributeTok{suffix =} \StringTok{"\%"}\NormalTok{, }\AttributeTok{na\_str =} \StringTok{""}\NormalTok{) }\SpecialCharTok{\%\textgreater{}\%} 
  
\NormalTok{  flextable}\SpecialCharTok{::}\FunctionTok{theme\_booktabs}\NormalTok{() }\SpecialCharTok{\%\textgreater{}\%} 
  
  \CommentTok{\# Alineación}
  \CommentTok{\# Aplicar negrita y bordes a la fila "Total" (la última fila)}
\NormalTok{  flextable}\SpecialCharTok{::}\FunctionTok{bold}\NormalTok{(}\AttributeTok{i =} \FunctionTok{nrow}\NormalTok{(df\_tabla\_final), }\AttributeTok{part =} \StringTok{"body"}\NormalTok{) }\SpecialCharTok{\%\textgreater{}\%} 
\NormalTok{  flextable}\SpecialCharTok{::}\FunctionTok{hline}\NormalTok{(}\AttributeTok{i =} \FunctionTok{nrow}\NormalTok{(df\_tabla\_final) }\SpecialCharTok{{-}} \DecValTok{1}\NormalTok{, }\AttributeTok{border =}\NormalTok{ officer}\SpecialCharTok{::}\FunctionTok{fp\_border}\NormalTok{(}\AttributeTok{width =} \FloatTok{1.5}\NormalTok{, }\AttributeTok{color =} \StringTok{"black"}\NormalTok{)) }\SpecialCharTok{\%\textgreater{}\%}
  
  \CommentTok{\# Alineación}
\NormalTok{  flextable}\SpecialCharTok{::}\FunctionTok{align}\NormalTok{(}\AttributeTok{align =} \StringTok{"center"}\NormalTok{, }\AttributeTok{part =} \StringTok{"all"}\NormalTok{) }\SpecialCharTok{\%\textgreater{}\%}
\NormalTok{  flextable}\SpecialCharTok{::}\FunctionTok{align}\NormalTok{(}\AttributeTok{j =} \DecValTok{1}\NormalTok{, }\AttributeTok{align =} \StringTok{"left"}\NormalTok{, }\AttributeTok{part =} \StringTok{"body"}\NormalTok{) }\SpecialCharTok{\%\textgreater{}\%}
  
  \CommentTok{\# Ajustar}
\NormalTok{  flextable}\SpecialCharTok{::}\FunctionTok{autofit}\NormalTok{()}

\CommentTok{\# Mostrar la tabla}
\NormalTok{ft\_si\_no}
\end{Highlighting}
\end{Shaded}

\global\setlength{\Oldarrayrulewidth}{\arrayrulewidth}

\global\setlength{\Oldtabcolsep}{\tabcolsep}

\setlength{\tabcolsep}{2pt}

\renewcommand*{\arraystretch}{1.5}



\providecommand{\ascline}[3]{\noalign{\global\arrayrulewidth #1}\arrayrulecolor[HTML]{#2}\cline{#3}}

\begin{longtable}[c]{|p{1.90in}|p{1.27in}|p{1.02in}|p{1.27in}|p{1.02in}}

\caption{Tabla\ 9.\ Distribución\ de\ Frecuencias\ (Sí/No)\ para\ Preferencia\ de\ Proteína}\\

\ascline{1.5pt}{666666}{1-5}

\multicolumn{1}{>{\centering}m{\dimexpr 1.9in+0\tabcolsep}}{\textcolor[HTML]{000000}{\fontsize{11}{11}\selectfont{\ }}} & \multicolumn{2}{>{\centering}m{\dimexpr 2.29in+2\tabcolsep}}{\textcolor[HTML]{000000}{\fontsize{11}{11}\selectfont{SÍ}}} & \multicolumn{2}{>{\centering}m{\dimexpr 2.29in+2\tabcolsep}}{\textcolor[HTML]{000000}{\fontsize{11}{11}\selectfont{NO}}} \\





\multicolumn{1}{>{\centering}m{\dimexpr 1.9in+0\tabcolsep}}{\textcolor[HTML]{000000}{\fontsize{11}{11}\selectfont{Preferencia\ de\ Proteína}}} & \multicolumn{1}{>{\centering}m{\dimexpr 1.27in+0\tabcolsep}}{\textcolor[HTML]{000000}{\fontsize{11}{11}\selectfont{Frecuencia\ (n)}}} & \multicolumn{1}{>{\centering}m{\dimexpr 1.02in+0\tabcolsep}}{\textcolor[HTML]{000000}{\fontsize{11}{11}\selectfont{Porcentaje}}} & \multicolumn{1}{>{\centering}m{\dimexpr 1.27in+0\tabcolsep}}{\textcolor[HTML]{000000}{\fontsize{11}{11}\selectfont{Frecuencia\ (n)}}} & \multicolumn{1}{>{\centering}m{\dimexpr 1.02in+0\tabcolsep}}{\textcolor[HTML]{000000}{\fontsize{11}{11}\selectfont{Porcentaje}}} \\

\ascline{1.5pt}{666666}{1-5}\endfirsthead \caption[]{Tabla\ 9.\ Distribución\ de\ Frecuencias\ (Sí/No)\ para\ Preferencia\ de\ Proteína}\\

\ascline{1.5pt}{666666}{1-5}

\multicolumn{1}{>{\centering}m{\dimexpr 1.9in+0\tabcolsep}}{\textcolor[HTML]{000000}{\fontsize{11}{11}\selectfont{\ }}} & \multicolumn{2}{>{\centering}m{\dimexpr 2.29in+2\tabcolsep}}{\textcolor[HTML]{000000}{\fontsize{11}{11}\selectfont{SÍ}}} & \multicolumn{2}{>{\centering}m{\dimexpr 2.29in+2\tabcolsep}}{\textcolor[HTML]{000000}{\fontsize{11}{11}\selectfont{NO}}} \\





\multicolumn{1}{>{\centering}m{\dimexpr 1.9in+0\tabcolsep}}{\textcolor[HTML]{000000}{\fontsize{11}{11}\selectfont{Preferencia\ de\ Proteína}}} & \multicolumn{1}{>{\centering}m{\dimexpr 1.27in+0\tabcolsep}}{\textcolor[HTML]{000000}{\fontsize{11}{11}\selectfont{Frecuencia\ (n)}}} & \multicolumn{1}{>{\centering}m{\dimexpr 1.02in+0\tabcolsep}}{\textcolor[HTML]{000000}{\fontsize{11}{11}\selectfont{Porcentaje}}} & \multicolumn{1}{>{\centering}m{\dimexpr 1.27in+0\tabcolsep}}{\textcolor[HTML]{000000}{\fontsize{11}{11}\selectfont{Frecuencia\ (n)}}} & \multicolumn{1}{>{\centering}m{\dimexpr 1.02in+0\tabcolsep}}{\textcolor[HTML]{000000}{\fontsize{11}{11}\selectfont{Porcentaje}}} \\

\ascline{1.5pt}{666666}{1-5}\endhead



\multicolumn{1}{>{\raggedright}m{\dimexpr 1.9in+0\tabcolsep}}{\textcolor[HTML]{000000}{\fontsize{11}{11}\selectfont{Carne}}} & \multicolumn{1}{>{\centering}m{\dimexpr 1.27in+0\tabcolsep}}{\textcolor[HTML]{000000}{\fontsize{11}{11}\selectfont{10}}} & \multicolumn{1}{>{\centering}m{\dimexpr 1.02in+0\tabcolsep}}{\textcolor[HTML]{000000}{\fontsize{11}{11}\selectfont{15.9\%}}} & \multicolumn{1}{>{\centering}m{\dimexpr 1.27in+0\tabcolsep}}{\textcolor[HTML]{000000}{\fontsize{11}{11}\selectfont{53}}} & \multicolumn{1}{>{\centering}m{\dimexpr 1.02in+0\tabcolsep}}{\textcolor[HTML]{000000}{\fontsize{11}{11}\selectfont{84.1\%}}} \\





\multicolumn{1}{>{\raggedright}m{\dimexpr 1.9in+0\tabcolsep}}{\textcolor[HTML]{000000}{\fontsize{11}{11}\selectfont{Huevo}}} & \multicolumn{1}{>{\centering}m{\dimexpr 1.27in+0\tabcolsep}}{\textcolor[HTML]{000000}{\fontsize{11}{11}\selectfont{29}}} & \multicolumn{1}{>{\centering}m{\dimexpr 1.02in+0\tabcolsep}}{\textcolor[HTML]{000000}{\fontsize{11}{11}\selectfont{46.0\%}}} & \multicolumn{1}{>{\centering}m{\dimexpr 1.27in+0\tabcolsep}}{\textcolor[HTML]{000000}{\fontsize{11}{11}\selectfont{34}}} & \multicolumn{1}{>{\centering}m{\dimexpr 1.02in+0\tabcolsep}}{\textcolor[HTML]{000000}{\fontsize{11}{11}\selectfont{54.0\%}}} \\





\multicolumn{1}{>{\raggedright}m{\dimexpr 1.9in+0\tabcolsep}}{\textcolor[HTML]{000000}{\fontsize{11}{11}\selectfont{Marisco}}} & \multicolumn{1}{>{\centering}m{\dimexpr 1.27in+0\tabcolsep}}{\textcolor[HTML]{000000}{\fontsize{11}{11}\selectfont{5}}} & \multicolumn{1}{>{\centering}m{\dimexpr 1.02in+0\tabcolsep}}{\textcolor[HTML]{000000}{\fontsize{11}{11}\selectfont{7.9\%}}} & \multicolumn{1}{>{\centering}m{\dimexpr 1.27in+0\tabcolsep}}{\textcolor[HTML]{000000}{\fontsize{11}{11}\selectfont{58}}} & \multicolumn{1}{>{\centering}m{\dimexpr 1.02in+0\tabcolsep}}{\textcolor[HTML]{000000}{\fontsize{11}{11}\selectfont{92.1\%}}} \\





\multicolumn{1}{>{\raggedright}m{\dimexpr 1.9in+0\tabcolsep}}{\textcolor[HTML]{000000}{\fontsize{11}{11}\selectfont{Pescado}}} & \multicolumn{1}{>{\centering}m{\dimexpr 1.27in+0\tabcolsep}}{\textcolor[HTML]{000000}{\fontsize{11}{11}\selectfont{12}}} & \multicolumn{1}{>{\centering}m{\dimexpr 1.02in+0\tabcolsep}}{\textcolor[HTML]{000000}{\fontsize{11}{11}\selectfont{19.0\%}}} & \multicolumn{1}{>{\centering}m{\dimexpr 1.27in+0\tabcolsep}}{\textcolor[HTML]{000000}{\fontsize{11}{11}\selectfont{51}}} & \multicolumn{1}{>{\centering}m{\dimexpr 1.02in+0\tabcolsep}}{\textcolor[HTML]{000000}{\fontsize{11}{11}\selectfont{81.0\%}}} \\





\multicolumn{1}{>{\raggedright}m{\dimexpr 1.9in+0\tabcolsep}}{\textcolor[HTML]{000000}{\fontsize{11}{11}\selectfont{Pollo}}} & \multicolumn{1}{>{\centering}m{\dimexpr 1.27in+0\tabcolsep}}{\textcolor[HTML]{000000}{\fontsize{11}{11}\selectfont{40}}} & \multicolumn{1}{>{\centering}m{\dimexpr 1.02in+0\tabcolsep}}{\textcolor[HTML]{000000}{\fontsize{11}{11}\selectfont{63.5\%}}} & \multicolumn{1}{>{\centering}m{\dimexpr 1.27in+0\tabcolsep}}{\textcolor[HTML]{000000}{\fontsize{11}{11}\selectfont{23}}} & \multicolumn{1}{>{\centering}m{\dimexpr 1.02in+0\tabcolsep}}{\textcolor[HTML]{000000}{\fontsize{11}{11}\selectfont{36.5\%}}} \\





\multicolumn{1}{>{\raggedright}m{\dimexpr 1.9in+0\tabcolsep}}{\textcolor[HTML]{000000}{\fontsize{11}{11}\selectfont{Queso}}} & \multicolumn{1}{>{\centering}m{\dimexpr 1.27in+0\tabcolsep}}{\textcolor[HTML]{000000}{\fontsize{11}{11}\selectfont{20}}} & \multicolumn{1}{>{\centering}m{\dimexpr 1.02in+0\tabcolsep}}{\textcolor[HTML]{000000}{\fontsize{11}{11}\selectfont{31.7\%}}} & \multicolumn{1}{>{\centering}m{\dimexpr 1.27in+0\tabcolsep}}{\textcolor[HTML]{000000}{\fontsize{11}{11}\selectfont{43}}} & \multicolumn{1}{>{\centering}m{\dimexpr 1.02in+0\tabcolsep}}{\textcolor[HTML]{000000}{\fontsize{11}{11}\selectfont{68.3\%}}} \\





\multicolumn{1}{>{\raggedright}m{\dimexpr 1.9in+0\tabcolsep}}{\textcolor[HTML]{000000}{\fontsize{11}{11}\selectfont{Tofu}}} & \multicolumn{1}{>{\centering}m{\dimexpr 1.27in+0\tabcolsep}}{\textcolor[HTML]{000000}{\fontsize{11}{11}\selectfont{4}}} & \multicolumn{1}{>{\centering}m{\dimexpr 1.02in+0\tabcolsep}}{\textcolor[HTML]{000000}{\fontsize{11}{11}\selectfont{6.3\%}}} & \multicolumn{1}{>{\centering}m{\dimexpr 1.27in+0\tabcolsep}}{\textcolor[HTML]{000000}{\fontsize{11}{11}\selectfont{59}}} & \multicolumn{1}{>{\centering}m{\dimexpr 1.02in+0\tabcolsep}}{\textcolor[HTML]{000000}{\fontsize{11}{11}\selectfont{93.7\%}}} \\

\ascline{1.5pt}{000000}{1-5}



\multicolumn{1}{>{\raggedright}m{\dimexpr 1.9in+0\tabcolsep}}{\textcolor[HTML]{000000}{\fontsize{11}{11}\selectfont{\textbf{Total}}}} & \multicolumn{1}{>{\centering}m{\dimexpr 1.27in+0\tabcolsep}}{\textcolor[HTML]{000000}{\fontsize{11}{11}\selectfont{\textbf{120}}}} & \multicolumn{1}{>{\centering}m{\dimexpr 1.02in+0\tabcolsep}}{\textcolor[HTML]{000000}{\fontsize{11}{11}\selectfont{\textbf{}}}} & \multicolumn{1}{>{\centering}m{\dimexpr 1.27in+0\tabcolsep}}{\textcolor[HTML]{000000}{\fontsize{11}{11}\selectfont{\textbf{321}}}} & \multicolumn{1}{>{\centering}m{\dimexpr 1.02in+0\tabcolsep}}{\textcolor[HTML]{000000}{\fontsize{11}{11}\selectfont{\textbf{}}}} \\

\ascline{1.5pt}{666666}{1-5}



\end{longtable}



\arrayrulecolor[HTML]{000000}

\global\setlength{\arrayrulewidth}{\Oldarrayrulewidth}

\global\setlength{\tabcolsep}{\Oldtabcolsep}

\renewcommand*{\arraystretch}{1}

\hypertarget{f_preferencia.complementos}{%
\paragraph{`f\_Preferencia.Complementos'}\label{f_preferencia.complementos}}

Las categorías de \texttt{f\_Preferencia.Complementos} hacen referencia
a los complementos que los encuestados añadirían a sus ensaladas. Es
importante destacar que esta variable proviene de una pregunta con
respuesta múltiple por lo que optamos por un gráfico de barras múltiple
basándonos en si ha se ha seleccionado cada respuesta o no.

\begin{Shaded}
\begin{Highlighting}[]
\CommentTok{\# Definir las variables a analizar (P10: Situaciones de Uso)}
\NormalTok{multi\_vars }\OtherTok{\textless{}{-}} \FunctionTok{c}\NormalTok{(}\StringTok{"f\_Complem.Frutas"}\NormalTok{, }\StringTok{"f\_Complem.Secos"}\NormalTok{, }\StringTok{"f\_Complem.Verduras"}\NormalTok{, }\StringTok{"f\_Complem.Salsas"}\NormalTok{, }\StringTok{"f\_Complem.Especias"}\NormalTok{)}

\CommentTok{\# Transformar a formato largo y contar frecuencias de SÍ/NO}
\NormalTok{df\_barras\_dobles }\OtherTok{\textless{}{-}}\NormalTok{ datos }\SpecialCharTok{\%\textgreater{}\%}
\NormalTok{  dplyr}\SpecialCharTok{::}\FunctionTok{select}\NormalTok{(}\FunctionTok{all\_of}\NormalTok{(multi\_vars)) }\SpecialCharTok{\%\textgreater{}\%}
\NormalTok{  tidyr}\SpecialCharTok{::}\FunctionTok{pivot\_longer}\NormalTok{(}
    \AttributeTok{cols =} \FunctionTok{everything}\NormalTok{(), }
    \AttributeTok{names\_to =} \StringTok{"Preferenica\_Complementos"}\NormalTok{, }
    \AttributeTok{values\_to =} \StringTok{"Respuesta"}
\NormalTok{  ) }\SpecialCharTok{\%\textgreater{}\%}
\NormalTok{  dplyr}\SpecialCharTok{::}\FunctionTok{count}\NormalTok{(Preferenica\_Complementos, Respuesta, }\AttributeTok{name =} \StringTok{"Frecuencia"}\NormalTok{) }\SpecialCharTok{\%\textgreater{}\%}
\NormalTok{  dplyr}\SpecialCharTok{::}\FunctionTok{mutate}\NormalTok{(}
    \AttributeTok{Preferenica\_Complementos =} \FunctionTok{gsub}\NormalTok{(}\StringTok{"f\_Complem."}\NormalTok{, }\StringTok{""}\NormalTok{, Preferenica\_Complementos, }\AttributeTok{fixed =} \ConstantTok{TRUE}\NormalTok{),}
    \AttributeTok{Preferenica\_Complementos =}\NormalTok{ tools}\SpecialCharTok{::}\FunctionTok{toTitleCase}\NormalTok{(Preferenica\_Complementos),}
    \AttributeTok{Respuesta =} \FunctionTok{factor}\NormalTok{(Respuesta, }\AttributeTok{levels =} \FunctionTok{c}\NormalTok{(}\StringTok{"No"}\NormalTok{, }\StringTok{"Sí"}\NormalTok{)) }
\NormalTok{  )}

\CommentTok{\# Generación del Gráfico de Barras Agrupadas (Horizontal)}
\FunctionTok{ggplot}\NormalTok{(df\_barras\_dobles, }\FunctionTok{aes}\NormalTok{(}\AttributeTok{x =}\NormalTok{ Preferenica\_Complementos, }\AttributeTok{y =}\NormalTok{ Frecuencia, }\AttributeTok{fill =}\NormalTok{ Respuesta)) }\SpecialCharTok{+}
  
  \CommentTok{\# Barras agrupadas}
  \FunctionTok{geom\_bar}\NormalTok{(}\AttributeTok{stat =} \StringTok{"identity"}\NormalTok{, }\AttributeTok{position =} \FunctionTok{position\_dodge}\NormalTok{(}\AttributeTok{width =} \FloatTok{0.9}\NormalTok{)) }\SpecialCharTok{+} 
  
  \CommentTok{\# Etiquetas de Frecuencia}
  \FunctionTok{geom\_text}\NormalTok{(}
    \FunctionTok{aes}\NormalTok{(}\AttributeTok{label =}\NormalTok{ Frecuencia),}
    \AttributeTok{position =} \FunctionTok{position\_dodge}\NormalTok{(}\AttributeTok{width =} \FloatTok{0.9}\NormalTok{),}
    \AttributeTok{hjust =} \SpecialCharTok{{-}}\FloatTok{0.2}\NormalTok{, }\CommentTok{\# Mueve la etiqueta {-}0.2 unidades a lo largo del eje X (separa de la barra)}
    \AttributeTok{size =} \FloatTok{3.5}
\NormalTok{  ) }\SpecialCharTok{+}
  
  \CommentTok{\# Títulos y Ejes}
  \FunctionTok{labs}\NormalTok{(}
    \AttributeTok{title =} \StringTok{"Figura 10. Gráfico de barras (Sí/No) por Preferencia Complementos"}\NormalTok{,}
    \AttributeTok{x =} \StringTok{"Preferencia Complementos"}\NormalTok{,}
    \AttributeTok{y =} \StringTok{"Número de Respuestas"}\NormalTok{,}
    \AttributeTok{fill =} \StringTok{"Pondría"}
\NormalTok{  ) }\SpecialCharTok{+}
  
  \CommentTok{\# Colores}
  \FunctionTok{scale\_fill\_manual}\NormalTok{(}\AttributeTok{values =} \FunctionTok{c}\NormalTok{(}\StringTok{"No"} \OtherTok{=} \StringTok{"\#a6cee3"}\NormalTok{, }\StringTok{"Sí"} \OtherTok{=} \StringTok{"\#1f78b4"}\NormalTok{)) }\SpecialCharTok{+} 
  
  \CommentTok{\# Invertir coordenadas para hacerlo horizontal}
  \FunctionTok{coord\_flip}\NormalTok{() }\SpecialCharTok{+} 
  
  \CommentTok{\# Ajuste del eje para que no se corten las etiquetas (Aumentamos el espacio)}
  \FunctionTok{scale\_y\_continuous}\NormalTok{(}\AttributeTok{expand =} \FunctionTok{expansion}\NormalTok{(}\AttributeTok{mult =} \FunctionTok{c}\NormalTok{(}\DecValTok{0}\NormalTok{, }\FloatTok{0.20}\NormalTok{))) }\SpecialCharTok{+} 
  
  \CommentTok{\# Ajuste de Tema}
  \FunctionTok{theme}\NormalTok{(}\AttributeTok{legend.position =} \StringTok{"bottom"}\NormalTok{)}
\end{Highlighting}
\end{Shaded}

\begin{center}\includegraphics{ICO-analisis_files/figure-latex/P15-barras-1} \end{center}

Se observa que el complemento que más se pondría en la ensalada sería
frutos secos (con 37 respuestas afirmativas), seguido de verduras
cocinadas (con 27 respuestas de afirmativas).

\begin{Shaded}
\begin{Highlighting}[]
\CommentTok{\# Transformar a formato largo, contar frecuencias de SÍ/NO y calcular porcentajes}
\NormalTok{df\_tabla\_temp }\OtherTok{\textless{}{-}}\NormalTok{ datos }\SpecialCharTok{\%\textgreater{}\%}
\NormalTok{  dplyr}\SpecialCharTok{::}\FunctionTok{select}\NormalTok{(}\FunctionTok{all\_of}\NormalTok{(multi\_vars)) }\SpecialCharTok{\%\textgreater{}\%}
\NormalTok{  tidyr}\SpecialCharTok{::}\FunctionTok{pivot\_longer}\NormalTok{(}
    \AttributeTok{cols =} \FunctionTok{everything}\NormalTok{(), }
    \AttributeTok{names\_to =} \StringTok{"Preferenica\_Complementos\_Raw"}\NormalTok{, }
    \AttributeTok{values\_to =} \StringTok{"Respuesta"} 
\NormalTok{  ) }\SpecialCharTok{\%\textgreater{}\%}
\NormalTok{  dplyr}\SpecialCharTok{::}\FunctionTok{count}\NormalTok{(Preferenica\_Complementos\_Raw, Respuesta, }\AttributeTok{name =} \StringTok{"Frecuencia"}\NormalTok{) }\SpecialCharTok{\%\textgreater{}\%}
\NormalTok{  dplyr}\SpecialCharTok{::}\FunctionTok{mutate}\NormalTok{(}
    \AttributeTok{Porcentaje =}\NormalTok{ (Frecuencia }\SpecialCharTok{/}\NormalTok{ n.filas) }\SpecialCharTok{*} \DecValTok{100}
\NormalTok{  ) }\SpecialCharTok{\%\textgreater{}\%}
\NormalTok{  tidyr}\SpecialCharTok{::}\FunctionTok{pivot\_wider}\NormalTok{(}
    \AttributeTok{names\_from =}\NormalTok{ Respuesta, }
    \AttributeTok{values\_from =} \FunctionTok{c}\NormalTok{(Frecuencia, Porcentaje),}
    \AttributeTok{values\_fill =} \DecValTok{0} 
\NormalTok{  ) }\SpecialCharTok{\%\textgreater{}\%}
\NormalTok{  dplyr}\SpecialCharTok{::}\FunctionTok{mutate}\NormalTok{(}
    \AttributeTok{Preferenica\_Complementos =} \FunctionTok{gsub}\NormalTok{(}\StringTok{"f\_Complem."}\NormalTok{, }\StringTok{""}\NormalTok{, Preferenica\_Complementos\_Raw, }\AttributeTok{fixed =} \ConstantTok{TRUE}\NormalTok{),}
    \AttributeTok{Preferenica\_Complementos =}\NormalTok{ tools}\SpecialCharTok{::}\FunctionTok{toTitleCase}\NormalTok{(Preferenica\_Complementos)}
\NormalTok{  ) }\SpecialCharTok{\%\textgreater{}\%}
\NormalTok{  dplyr}\SpecialCharTok{::}\FunctionTok{select}\NormalTok{(}
\NormalTok{    Preferenica\_Complementos,}
    \StringTok{\textasciigrave{}}\AttributeTok{Frecuencia\_Sí}\StringTok{\textasciigrave{}}\NormalTok{, }\StringTok{\textasciigrave{}}\AttributeTok{Porcentaje\_Sí}\StringTok{\textasciigrave{}}\NormalTok{,}
    \StringTok{\textasciigrave{}}\AttributeTok{Frecuencia\_No}\StringTok{\textasciigrave{}}\NormalTok{, }\StringTok{\textasciigrave{}}\AttributeTok{Porcentaje\_No}\StringTok{\textasciigrave{}}
\NormalTok{  )}


\CommentTok{\# Fila de Total}
\NormalTok{df\_total }\OtherTok{\textless{}{-}} \FunctionTok{data.frame}\NormalTok{(}
  \AttributeTok{Preferenica\_Complementos =} \StringTok{"Total"}\NormalTok{,}
\NormalTok{  Frecuencia\_Sí }\OtherTok{=} \FunctionTok{sum}\NormalTok{(df\_tabla\_temp}\SpecialCharTok{$}\NormalTok{Frecuencia\_Sí),}
\NormalTok{  Porcentaje\_Sí }\OtherTok{=} \ConstantTok{NA}\NormalTok{, }\CommentTok{\# No se suma, sería engañoso}
  \AttributeTok{Frecuencia\_No =} \FunctionTok{sum}\NormalTok{(df\_tabla\_temp}\SpecialCharTok{$}\NormalTok{Frecuencia\_No),}
  \AttributeTok{Porcentaje\_No =} \ConstantTok{NA} \CommentTok{\# No se suma, sería engañoso}
\NormalTok{)}

\CommentTok{\# Unir el data frame con la fila total}
\NormalTok{df\_tabla\_final }\OtherTok{\textless{}{-}} \FunctionTok{bind\_rows}\NormalTok{(df\_tabla\_temp, df\_total)}


\CommentTok{\# Generación de la tabla (flextable)}
\NormalTok{ft\_si\_no }\OtherTok{\textless{}{-}}\NormalTok{ flextable}\SpecialCharTok{::}\FunctionTok{flextable}\NormalTok{(df\_tabla\_final) }\SpecialCharTok{\%\textgreater{}\%}
  
  \CommentTok{\# Añadir Título}
\NormalTok{  flextable}\SpecialCharTok{::}\FunctionTok{set\_caption}\NormalTok{(}\AttributeTok{caption =} \StringTok{"Tabla 10. Distribución de Frecuencias (Sí/No) para Preferencia de Complementos"}\NormalTok{) }\SpecialCharTok{\%\textgreater{}\%}
  
  \CommentTok{\# Renombrar cabeceras (se usan para la segunda fila de la cabecera)}
\NormalTok{  flextable}\SpecialCharTok{::}\FunctionTok{set\_header\_labels}\NormalTok{(}
    \AttributeTok{Preferenica\_Complementos =} \StringTok{"Preferencia Complementos"}\NormalTok{,}
\NormalTok{    Frecuencia\_Sí }\OtherTok{=} \StringTok{"Frecuencia (n)"}\NormalTok{,}
\NormalTok{    Porcentaje\_Sí }\OtherTok{=} \StringTok{"Porcentaje"}\NormalTok{,}
    \AttributeTok{Frecuencia\_No =} \StringTok{"Frecuencia (n)"}\NormalTok{,}
    \AttributeTok{Porcentaje\_No =} \StringTok{"Porcentaje"}
\NormalTok{  ) }\SpecialCharTok{\%\textgreater{}\%}
  
  \CommentTok{\# Agrupar las columnas \textquotesingle{}Sí\textquotesingle{} y \textquotesingle{}No\textquotesingle{} en una fila superior}
\NormalTok{  flextable}\SpecialCharTok{::}\FunctionTok{add\_header\_row}\NormalTok{(}
    \AttributeTok{values =} \FunctionTok{c}\NormalTok{(}\StringTok{" "}\NormalTok{, }\StringTok{"SÍ"}\NormalTok{, }\StringTok{"NO"}\NormalTok{),}
    \AttributeTok{colwidths =} \FunctionTok{c}\NormalTok{(}\DecValTok{1}\NormalTok{, }\DecValTok{2}\NormalTok{, }\DecValTok{2}\NormalTok{)}
\NormalTok{  ) }\SpecialCharTok{\%\textgreater{}\%}
  
  \CommentTok{\# Formato y Tema}
\NormalTok{  flextable}\SpecialCharTok{::}\FunctionTok{colformat\_double}\NormalTok{(}\AttributeTok{j =} \FunctionTok{c}\NormalTok{(}\DecValTok{3}\NormalTok{, }\DecValTok{5}\NormalTok{), }\AttributeTok{digits =} \DecValTok{1}\NormalTok{, }\AttributeTok{suffix =} \StringTok{"\%"}\NormalTok{, }\AttributeTok{na\_str =} \StringTok{""}\NormalTok{) }\SpecialCharTok{\%\textgreater{}\%} 
  
\NormalTok{  flextable}\SpecialCharTok{::}\FunctionTok{theme\_booktabs}\NormalTok{() }\SpecialCharTok{\%\textgreater{}\%} 
  
  \CommentTok{\# Alineación}
  \CommentTok{\# Aplicar negrita y bordes a la fila "Total" (la última fila)}
\NormalTok{  flextable}\SpecialCharTok{::}\FunctionTok{bold}\NormalTok{(}\AttributeTok{i =} \FunctionTok{nrow}\NormalTok{(df\_tabla\_final), }\AttributeTok{part =} \StringTok{"body"}\NormalTok{) }\SpecialCharTok{\%\textgreater{}\%} 
\NormalTok{  flextable}\SpecialCharTok{::}\FunctionTok{hline}\NormalTok{(}\AttributeTok{i =} \FunctionTok{nrow}\NormalTok{(df\_tabla\_final) }\SpecialCharTok{{-}} \DecValTok{1}\NormalTok{, }\AttributeTok{border =}\NormalTok{ officer}\SpecialCharTok{::}\FunctionTok{fp\_border}\NormalTok{(}\AttributeTok{width =} \FloatTok{1.5}\NormalTok{, }\AttributeTok{color =} \StringTok{"black"}\NormalTok{)) }\SpecialCharTok{\%\textgreater{}\%}
  
  \CommentTok{\# Alineación}
\NormalTok{  flextable}\SpecialCharTok{::}\FunctionTok{align}\NormalTok{(}\AttributeTok{align =} \StringTok{"center"}\NormalTok{, }\AttributeTok{part =} \StringTok{"all"}\NormalTok{) }\SpecialCharTok{\%\textgreater{}\%}
\NormalTok{  flextable}\SpecialCharTok{::}\FunctionTok{align}\NormalTok{(}\AttributeTok{j =} \DecValTok{1}\NormalTok{, }\AttributeTok{align =} \StringTok{"left"}\NormalTok{, }\AttributeTok{part =} \StringTok{"body"}\NormalTok{) }\SpecialCharTok{\%\textgreater{}\%}
  
  \CommentTok{\# Ajustar}
\NormalTok{  flextable}\SpecialCharTok{::}\FunctionTok{autofit}\NormalTok{()}

\CommentTok{\# Mostrar la tabla}
\NormalTok{ft\_si\_no}
\end{Highlighting}
\end{Shaded}

\global\setlength{\Oldarrayrulewidth}{\arrayrulewidth}

\global\setlength{\Oldtabcolsep}{\tabcolsep}

\setlength{\tabcolsep}{2pt}

\renewcommand*{\arraystretch}{1.5}



\providecommand{\ascline}[3]{\noalign{\global\arrayrulewidth #1}\arrayrulecolor[HTML]{#2}\cline{#3}}

\begin{longtable}[c]{|p{2.14in}|p{1.27in}|p{1.02in}|p{1.27in}|p{1.02in}}

\caption{Tabla\ 10.\ Distribución\ de\ Frecuencias\ (Sí/No)\ para\ Preferencia\ de\ Complementos}\\

\ascline{1.5pt}{666666}{1-5}

\multicolumn{1}{>{\centering}m{\dimexpr 2.14in+0\tabcolsep}}{\textcolor[HTML]{000000}{\fontsize{11}{11}\selectfont{\ }}} & \multicolumn{2}{>{\centering}m{\dimexpr 2.29in+2\tabcolsep}}{\textcolor[HTML]{000000}{\fontsize{11}{11}\selectfont{SÍ}}} & \multicolumn{2}{>{\centering}m{\dimexpr 2.29in+2\tabcolsep}}{\textcolor[HTML]{000000}{\fontsize{11}{11}\selectfont{NO}}} \\





\multicolumn{1}{>{\centering}m{\dimexpr 2.14in+0\tabcolsep}}{\textcolor[HTML]{000000}{\fontsize{11}{11}\selectfont{Preferencia\ Complementos}}} & \multicolumn{1}{>{\centering}m{\dimexpr 1.27in+0\tabcolsep}}{\textcolor[HTML]{000000}{\fontsize{11}{11}\selectfont{Frecuencia\ (n)}}} & \multicolumn{1}{>{\centering}m{\dimexpr 1.02in+0\tabcolsep}}{\textcolor[HTML]{000000}{\fontsize{11}{11}\selectfont{Porcentaje}}} & \multicolumn{1}{>{\centering}m{\dimexpr 1.27in+0\tabcolsep}}{\textcolor[HTML]{000000}{\fontsize{11}{11}\selectfont{Frecuencia\ (n)}}} & \multicolumn{1}{>{\centering}m{\dimexpr 1.02in+0\tabcolsep}}{\textcolor[HTML]{000000}{\fontsize{11}{11}\selectfont{Porcentaje}}} \\

\ascline{1.5pt}{666666}{1-5}\endfirsthead \caption[]{Tabla\ 10.\ Distribución\ de\ Frecuencias\ (Sí/No)\ para\ Preferencia\ de\ Complementos}\\

\ascline{1.5pt}{666666}{1-5}

\multicolumn{1}{>{\centering}m{\dimexpr 2.14in+0\tabcolsep}}{\textcolor[HTML]{000000}{\fontsize{11}{11}\selectfont{\ }}} & \multicolumn{2}{>{\centering}m{\dimexpr 2.29in+2\tabcolsep}}{\textcolor[HTML]{000000}{\fontsize{11}{11}\selectfont{SÍ}}} & \multicolumn{2}{>{\centering}m{\dimexpr 2.29in+2\tabcolsep}}{\textcolor[HTML]{000000}{\fontsize{11}{11}\selectfont{NO}}} \\





\multicolumn{1}{>{\centering}m{\dimexpr 2.14in+0\tabcolsep}}{\textcolor[HTML]{000000}{\fontsize{11}{11}\selectfont{Preferencia\ Complementos}}} & \multicolumn{1}{>{\centering}m{\dimexpr 1.27in+0\tabcolsep}}{\textcolor[HTML]{000000}{\fontsize{11}{11}\selectfont{Frecuencia\ (n)}}} & \multicolumn{1}{>{\centering}m{\dimexpr 1.02in+0\tabcolsep}}{\textcolor[HTML]{000000}{\fontsize{11}{11}\selectfont{Porcentaje}}} & \multicolumn{1}{>{\centering}m{\dimexpr 1.27in+0\tabcolsep}}{\textcolor[HTML]{000000}{\fontsize{11}{11}\selectfont{Frecuencia\ (n)}}} & \multicolumn{1}{>{\centering}m{\dimexpr 1.02in+0\tabcolsep}}{\textcolor[HTML]{000000}{\fontsize{11}{11}\selectfont{Porcentaje}}} \\

\ascline{1.5pt}{666666}{1-5}\endhead



\multicolumn{1}{>{\raggedright}m{\dimexpr 2.14in+0\tabcolsep}}{\textcolor[HTML]{000000}{\fontsize{11}{11}\selectfont{Especias}}} & \multicolumn{1}{>{\centering}m{\dimexpr 1.27in+0\tabcolsep}}{\textcolor[HTML]{000000}{\fontsize{11}{11}\selectfont{6}}} & \multicolumn{1}{>{\centering}m{\dimexpr 1.02in+0\tabcolsep}}{\textcolor[HTML]{000000}{\fontsize{11}{11}\selectfont{9.5\%}}} & \multicolumn{1}{>{\centering}m{\dimexpr 1.27in+0\tabcolsep}}{\textcolor[HTML]{000000}{\fontsize{11}{11}\selectfont{57}}} & \multicolumn{1}{>{\centering}m{\dimexpr 1.02in+0\tabcolsep}}{\textcolor[HTML]{000000}{\fontsize{11}{11}\selectfont{90.5\%}}} \\





\multicolumn{1}{>{\raggedright}m{\dimexpr 2.14in+0\tabcolsep}}{\textcolor[HTML]{000000}{\fontsize{11}{11}\selectfont{Frutas}}} & \multicolumn{1}{>{\centering}m{\dimexpr 1.27in+0\tabcolsep}}{\textcolor[HTML]{000000}{\fontsize{11}{11}\selectfont{25}}} & \multicolumn{1}{>{\centering}m{\dimexpr 1.02in+0\tabcolsep}}{\textcolor[HTML]{000000}{\fontsize{11}{11}\selectfont{39.7\%}}} & \multicolumn{1}{>{\centering}m{\dimexpr 1.27in+0\tabcolsep}}{\textcolor[HTML]{000000}{\fontsize{11}{11}\selectfont{38}}} & \multicolumn{1}{>{\centering}m{\dimexpr 1.02in+0\tabcolsep}}{\textcolor[HTML]{000000}{\fontsize{11}{11}\selectfont{60.3\%}}} \\





\multicolumn{1}{>{\raggedright}m{\dimexpr 2.14in+0\tabcolsep}}{\textcolor[HTML]{000000}{\fontsize{11}{11}\selectfont{Salsas}}} & \multicolumn{1}{>{\centering}m{\dimexpr 1.27in+0\tabcolsep}}{\textcolor[HTML]{000000}{\fontsize{11}{11}\selectfont{23}}} & \multicolumn{1}{>{\centering}m{\dimexpr 1.02in+0\tabcolsep}}{\textcolor[HTML]{000000}{\fontsize{11}{11}\selectfont{36.5\%}}} & \multicolumn{1}{>{\centering}m{\dimexpr 1.27in+0\tabcolsep}}{\textcolor[HTML]{000000}{\fontsize{11}{11}\selectfont{40}}} & \multicolumn{1}{>{\centering}m{\dimexpr 1.02in+0\tabcolsep}}{\textcolor[HTML]{000000}{\fontsize{11}{11}\selectfont{63.5\%}}} \\





\multicolumn{1}{>{\raggedright}m{\dimexpr 2.14in+0\tabcolsep}}{\textcolor[HTML]{000000}{\fontsize{11}{11}\selectfont{Secos}}} & \multicolumn{1}{>{\centering}m{\dimexpr 1.27in+0\tabcolsep}}{\textcolor[HTML]{000000}{\fontsize{11}{11}\selectfont{37}}} & \multicolumn{1}{>{\centering}m{\dimexpr 1.02in+0\tabcolsep}}{\textcolor[HTML]{000000}{\fontsize{11}{11}\selectfont{58.7\%}}} & \multicolumn{1}{>{\centering}m{\dimexpr 1.27in+0\tabcolsep}}{\textcolor[HTML]{000000}{\fontsize{11}{11}\selectfont{26}}} & \multicolumn{1}{>{\centering}m{\dimexpr 1.02in+0\tabcolsep}}{\textcolor[HTML]{000000}{\fontsize{11}{11}\selectfont{41.3\%}}} \\





\multicolumn{1}{>{\raggedright}m{\dimexpr 2.14in+0\tabcolsep}}{\textcolor[HTML]{000000}{\fontsize{11}{11}\selectfont{Verduras}}} & \multicolumn{1}{>{\centering}m{\dimexpr 1.27in+0\tabcolsep}}{\textcolor[HTML]{000000}{\fontsize{11}{11}\selectfont{27}}} & \multicolumn{1}{>{\centering}m{\dimexpr 1.02in+0\tabcolsep}}{\textcolor[HTML]{000000}{\fontsize{11}{11}\selectfont{42.9\%}}} & \multicolumn{1}{>{\centering}m{\dimexpr 1.27in+0\tabcolsep}}{\textcolor[HTML]{000000}{\fontsize{11}{11}\selectfont{36}}} & \multicolumn{1}{>{\centering}m{\dimexpr 1.02in+0\tabcolsep}}{\textcolor[HTML]{000000}{\fontsize{11}{11}\selectfont{57.1\%}}} \\

\ascline{1.5pt}{000000}{1-5}



\multicolumn{1}{>{\raggedright}m{\dimexpr 2.14in+0\tabcolsep}}{\textcolor[HTML]{000000}{\fontsize{11}{11}\selectfont{\textbf{Total}}}} & \multicolumn{1}{>{\centering}m{\dimexpr 1.27in+0\tabcolsep}}{\textcolor[HTML]{000000}{\fontsize{11}{11}\selectfont{\textbf{118}}}} & \multicolumn{1}{>{\centering}m{\dimexpr 1.02in+0\tabcolsep}}{\textcolor[HTML]{000000}{\fontsize{11}{11}\selectfont{\textbf{}}}} & \multicolumn{1}{>{\centering}m{\dimexpr 1.27in+0\tabcolsep}}{\textcolor[HTML]{000000}{\fontsize{11}{11}\selectfont{\textbf{197}}}} & \multicolumn{1}{>{\centering}m{\dimexpr 1.02in+0\tabcolsep}}{\textcolor[HTML]{000000}{\fontsize{11}{11}\selectfont{\textbf{}}}} \\

\ascline{1.5pt}{666666}{1-5}



\end{longtable}



\arrayrulecolor[HTML]{000000}

\global\setlength{\arrayrulewidth}{\Oldarrayrulewidth}

\global\setlength{\tabcolsep}{\Oldtabcolsep}

\renewcommand*{\arraystretch}{1}

\hypertarget{f_preferencia.dispensacion}{%
\paragraph{\texorpdfstring{\texttt{f\_Preferencia.Dispensacion}}{f\_Preferencia.Dispensacion}}\label{f_preferencia.dispensacion}}

Las categorías de \texttt{f\_Preferencia.Dispensacion}hacen referencia a
los 3 formatos de dispensación presentados para la barra de ensaladas.
Se trata de una variable cuyas categorías no guardan una relación de
orden entre sí.

\begin{Shaded}
\begin{Highlighting}[]
\CommentTok{\# Crear el data frame de frecuencias y porcentajes usando dplyr}
\NormalTok{datos\_grafico }\OtherTok{\textless{}{-}}\NormalTok{ datos }\SpecialCharTok{\%\textgreater{}\%}
  \CommentTok{\# Contar la frecuencia de cada nivel de la variable f\_Preferencia.Dispensacion}
  \FunctionTok{count}\NormalTok{(f\_Preferencia.Dispensacion, }\AttributeTok{name =} \StringTok{"Frecuencia"}\NormalTok{) }\SpecialCharTok{\%\textgreater{}\%}
  \CommentTok{\# Calcular el porcentaje y el texto de la etiqueta}
  \FunctionTok{mutate}\NormalTok{(}
    \AttributeTok{Porcentaje =}\NormalTok{ Frecuencia }\SpecialCharTok{/} \FunctionTok{sum}\NormalTok{(Frecuencia) }\SpecialCharTok{*} \DecValTok{100}\NormalTok{,}
    \AttributeTok{Etiquetas =} \FunctionTok{paste0}\NormalTok{(f\_Preferencia.Dispensacion, }\StringTok{"}\SpecialCharTok{\textbackslash{}n}\StringTok{("}\NormalTok{, }\FunctionTok{round}\NormalTok{(Porcentaje, }\DecValTok{1}\NormalTok{), }\StringTok{"\%)"}\NormalTok{) }\CommentTok{\# Etiqueta para el gráfico}
\NormalTok{  ) }\SpecialCharTok{\%\textgreater{}\%}
  \CommentTok{\# Eliminar filas con 0\% si existen, y categorías no deseadas}
  \FunctionTok{filter}\NormalTok{(Porcentaje }\SpecialCharTok{\textgreater{}} \DecValTok{0} \SpecialCharTok{\&}\NormalTok{ f\_Preferencia.Dispensacion }\SpecialCharTok{!=} \StringTok{"Total"}\NormalTok{) }\CommentTok{\# Se filtra \textquotesingle{}Total\textquotesingle{} aunque dplyr no lo añade}

\CommentTok{\# Usamos el vector de Porcentaje para el tamaño de las porciones}
\FunctionTok{pie}\NormalTok{(datos\_grafico}\SpecialCharTok{$}\NormalTok{Porcentaje,}
    \CommentTok{\# Usamos las Etiquetas que combinan nombre y porcentaje}
    \AttributeTok{labels =}\NormalTok{ datos\_grafico}\SpecialCharTok{$}\NormalTok{Etiquetas,}
    \CommentTok{\# Título}
    \AttributeTok{main =} \StringTok{\textquotesingle{}Figura 11. Preferencia del Formato de Dispensación\textquotesingle{}}\NormalTok{,}
    \CommentTok{\# Asignar un color diferente a cada porción}
    \AttributeTok{col =}\NormalTok{ RColorBrewer}\SpecialCharTok{::}\FunctionTok{brewer.pal}\NormalTok{(}\AttributeTok{n =} \FunctionTok{nrow}\NormalTok{(datos\_grafico), }\AttributeTok{name =} \StringTok{"Set3"}\NormalTok{))}
\end{Highlighting}
\end{Shaded}

\begin{center}\includegraphics{ICO-analisis_files/figure-latex/P18-tarta-1} \end{center}

Se observa que el formato de dispensación preferido es que tiene
protección fija y los ingredientes no son accesibles (con el 38.1\%),
seguido del formato en el que los ingredientes esta protegidos por una
mampara de cristal que se abre (con el34.9\&)

\begin{Shaded}
\begin{Highlighting}[]
\CommentTok{\# Preparación de datos}
\NormalTok{df\_tabla\_dispensacion }\OtherTok{\textless{}{-}}\NormalTok{ datos }\SpecialCharTok{\%\textgreater{}\%}
  \CommentTok{\# Contar la frecuencia de cada nivel}
\NormalTok{  dplyr}\SpecialCharTok{::}\FunctionTok{count}\NormalTok{(f\_Preferencia.Dispensacion, }\AttributeTok{name =} \StringTok{"Frecuencia"}\NormalTok{) }\SpecialCharTok{\%\textgreater{}\%}
  \CommentTok{\# Calcular porcentajes}
\NormalTok{  dplyr}\SpecialCharTok{::}\FunctionTok{mutate}\NormalTok{(}
    \AttributeTok{Porcentaje =}\NormalTok{ Frecuencia }\SpecialCharTok{/} \FunctionTok{sum}\NormalTok{(Frecuencia) }\SpecialCharTok{*} \DecValTok{100}
\NormalTok{  ) }\SpecialCharTok{\%\textgreater{}\%}
  \CommentTok{\# Redondear y formatear los porcentajes}
\NormalTok{  dplyr}\SpecialCharTok{::}\FunctionTok{mutate}\NormalTok{(}
    \AttributeTok{Porcentaje =} \FunctionTok{paste0}\NormalTok{(}\FunctionTok{round}\NormalTok{(Porcentaje, }\DecValTok{1}\NormalTok{), }\StringTok{"\%"}\NormalTok{)}
\NormalTok{  )}

\CommentTok{\# Calcular y añadir la fila total}
\NormalTok{df\_total }\OtherTok{\textless{}{-}} \FunctionTok{data.frame}\NormalTok{(}
  \AttributeTok{f\_Preferencia.Dispensacion =} \StringTok{"Total"}\NormalTok{,}
  \AttributeTok{Frecuencia =} \FunctionTok{sum}\NormalTok{(df\_tabla\_dispensacion}\SpecialCharTok{$}\NormalTok{Frecuencia),}
  \AttributeTok{Porcentaje =} \StringTok{"100.0\%"}
\NormalTok{)}

\CommentTok{\# Unir el data frame de frecuencias con la fila total}
\NormalTok{df\_tabla\_final }\OtherTok{\textless{}{-}} \FunctionTok{bind\_rows}\NormalTok{(df\_tabla\_dispensacion, df\_total)}

\CommentTok{\# Generación de la tabla (flextable)}
\NormalTok{ft }\OtherTok{\textless{}{-}}\NormalTok{ flextable}\SpecialCharTok{::}\FunctionTok{flextable}\NormalTok{(df\_tabla\_final) }\SpecialCharTok{\%\textgreater{}\%}

\NormalTok{flextable}\SpecialCharTok{::}\FunctionTok{set\_caption}\NormalTok{(}\AttributeTok{caption =} \StringTok{"Tabla 11. Distribución de frecuencias para Formato de Dispensación Preferido"}\NormalTok{) }\SpecialCharTok{\%\textgreater{}\%}
  
  \CommentTok{\# Renombrar las cabeceras}
\NormalTok{  flextable}\SpecialCharTok{::}\FunctionTok{set\_header\_labels}\NormalTok{(}
    \AttributeTok{f\_Preferencia.Dispensacion =} \StringTok{"Formato de Dispensación"}\NormalTok{,}
    \AttributeTok{Frecuencia =} \StringTok{"Frecuencia (n)"}\NormalTok{,}
    \AttributeTok{Porcentaje =} \StringTok{"Porcentaje"}
\NormalTok{  ) }\SpecialCharTok{\%\textgreater{}\%}
  
  \CommentTok{\# Aplicar negrita y bordes a la fila "Total"}
\NormalTok{  flextable}\SpecialCharTok{::}\FunctionTok{bold}\NormalTok{(}\AttributeTok{i =} \FunctionTok{nrow}\NormalTok{(df\_tabla\_final), }\AttributeTok{part =} \StringTok{"body"}\NormalTok{) }\SpecialCharTok{\%\textgreater{}\%} \CommentTok{\# Negrita a la última fila (Total)}
\NormalTok{  flextable}\SpecialCharTok{::}\FunctionTok{border\_remove}\NormalTok{() }\SpecialCharTok{\%\textgreater{}\%} \CommentTok{\# Quitar bordes predeterminados}
\NormalTok{  flextable}\SpecialCharTok{::}\FunctionTok{theme\_booktabs}\NormalTok{() }\SpecialCharTok{\%\textgreater{}\%} \CommentTok{\# Aplicar un tema con líneas horizontales}
  
  \CommentTok{\# Formato de alineación y cabecera}
\NormalTok{  flextable}\SpecialCharTok{::}\FunctionTok{align}\NormalTok{(}\AttributeTok{j =} \DecValTok{1}\NormalTok{, }\AttributeTok{align =} \StringTok{"left"}\NormalTok{, }\AttributeTok{part =} \StringTok{"body"}\NormalTok{) }\SpecialCharTok{\%\textgreater{}\%}
\NormalTok{  flextable}\SpecialCharTok{::}\FunctionTok{align}\NormalTok{(}\AttributeTok{j =} \DecValTok{2}\SpecialCharTok{:}\DecValTok{3}\NormalTok{, }\AttributeTok{align =} \StringTok{"center"}\NormalTok{, }\AttributeTok{part =} \StringTok{"all"}\NormalTok{) }\SpecialCharTok{\%\textgreater{}\%} \CommentTok{\# Columnas 2 y 3 (Datos) CENTRADAS}
\NormalTok{  flextable}\SpecialCharTok{::}\FunctionTok{align}\NormalTok{(}\AttributeTok{align =} \StringTok{"center"}\NormalTok{, }\AttributeTok{part =} \StringTok{"header"}\NormalTok{) }\SpecialCharTok{\%\textgreater{}\%}        \CommentTok{\# Encabezados CENTRADOS}
  
  \CommentTok{\# Añadir una línea superior a la fila "Total" para separarla}
\NormalTok{  flextable}\SpecialCharTok{::}\FunctionTok{hline}\NormalTok{(}\AttributeTok{i =} \FunctionTok{nrow}\NormalTok{(df\_tabla\_final) }\SpecialCharTok{{-}} \DecValTok{1}\NormalTok{, }\AttributeTok{border =}\NormalTok{ officer}\SpecialCharTok{::}\FunctionTok{fp\_border}\NormalTok{(}\AttributeTok{width =} \FloatTok{1.5}\NormalTok{, }\AttributeTok{color =} \StringTok{"black"}\NormalTok{)) }\SpecialCharTok{\%\textgreater{}\%}
  
  \CommentTok{\# Ajustar el ancho de las columnas}
\NormalTok{  flextable}\SpecialCharTok{::}\FunctionTok{autofit}\NormalTok{()}

\CommentTok{\# Mostrar la tabla}
\NormalTok{ft}
\end{Highlighting}
\end{Shaded}

\global\setlength{\Oldarrayrulewidth}{\arrayrulewidth}

\global\setlength{\Oldtabcolsep}{\tabcolsep}

\setlength{\tabcolsep}{2pt}

\renewcommand*{\arraystretch}{1.5}



\providecommand{\ascline}[3]{\noalign{\global\arrayrulewidth #1}\arrayrulecolor[HTML]{#2}\cline{#3}}

\begin{longtable}[c]{|p{2.03in}|p{1.27in}|p{1.02in}}

\caption{Tabla\ 11.\ Distribución\ de\ frecuencias\ para\ Formato\ de\ Dispensación\ Preferido}\\

\ascline{1.5pt}{666666}{1-3}

\multicolumn{1}{>{\centering}m{\dimexpr 2.03in+0\tabcolsep}}{\textcolor[HTML]{000000}{\fontsize{11}{11}\selectfont{Formato\ de\ Dispensación}}} & \multicolumn{1}{>{\centering}m{\dimexpr 1.27in+0\tabcolsep}}{\textcolor[HTML]{000000}{\fontsize{11}{11}\selectfont{Frecuencia\ (n)}}} & \multicolumn{1}{>{\centering}m{\dimexpr 1.02in+0\tabcolsep}}{\textcolor[HTML]{000000}{\fontsize{11}{11}\selectfont{Porcentaje}}} \\

\ascline{1.5pt}{666666}{1-3}\endfirsthead \caption[]{Tabla\ 11.\ Distribución\ de\ frecuencias\ para\ Formato\ de\ Dispensación\ Preferido}\\

\ascline{1.5pt}{666666}{1-3}

\multicolumn{1}{>{\centering}m{\dimexpr 2.03in+0\tabcolsep}}{\textcolor[HTML]{000000}{\fontsize{11}{11}\selectfont{Formato\ de\ Dispensación}}} & \multicolumn{1}{>{\centering}m{\dimexpr 1.27in+0\tabcolsep}}{\textcolor[HTML]{000000}{\fontsize{11}{11}\selectfont{Frecuencia\ (n)}}} & \multicolumn{1}{>{\centering}m{\dimexpr 1.02in+0\tabcolsep}}{\textcolor[HTML]{000000}{\fontsize{11}{11}\selectfont{Porcentaje}}} \\

\ascline{1.5pt}{666666}{1-3}\endhead



\multicolumn{1}{>{\raggedright}m{\dimexpr 2.03in+0\tabcolsep}}{\textcolor[HTML]{000000}{\fontsize{11}{11}\selectfont{Autoservicio}}} & \multicolumn{1}{>{\centering}m{\dimexpr 1.27in+0\tabcolsep}}{\textcolor[HTML]{000000}{\fontsize{11}{11}\selectfont{22}}} & \multicolumn{1}{>{\centering}m{\dimexpr 1.02in+0\tabcolsep}}{\textcolor[HTML]{000000}{\fontsize{11}{11}\selectfont{34.9\%}}} \\





\multicolumn{1}{>{\raggedright}m{\dimexpr 2.03in+0\tabcolsep}}{\textcolor[HTML]{000000}{\fontsize{11}{11}\selectfont{Protección\ Fija}}} & \multicolumn{1}{>{\centering}m{\dimexpr 1.27in+0\tabcolsep}}{\textcolor[HTML]{000000}{\fontsize{11}{11}\selectfont{24}}} & \multicolumn{1}{>{\centering}m{\dimexpr 1.02in+0\tabcolsep}}{\textcolor[HTML]{000000}{\fontsize{11}{11}\selectfont{38.1\%}}} \\





\multicolumn{1}{>{\raggedright}m{\dimexpr 2.03in+0\tabcolsep}}{\textcolor[HTML]{000000}{\fontsize{11}{11}\selectfont{Personal}}} & \multicolumn{1}{>{\centering}m{\dimexpr 1.27in+0\tabcolsep}}{\textcolor[HTML]{000000}{\fontsize{11}{11}\selectfont{17}}} & \multicolumn{1}{>{\centering}m{\dimexpr 1.02in+0\tabcolsep}}{\textcolor[HTML]{000000}{\fontsize{11}{11}\selectfont{27\%}}} \\

\ascline{1.5pt}{000000}{1-3}



\multicolumn{1}{>{\raggedright}m{\dimexpr 2.03in+0\tabcolsep}}{\textcolor[HTML]{000000}{\fontsize{11}{11}\selectfont{\textbf{Total}}}} & \multicolumn{1}{>{\centering}m{\dimexpr 1.27in+0\tabcolsep}}{\textcolor[HTML]{000000}{\fontsize{11}{11}\selectfont{\textbf{63}}}} & \multicolumn{1}{>{\centering}m{\dimexpr 1.02in+0\tabcolsep}}{\textcolor[HTML]{000000}{\fontsize{11}{11}\selectfont{\textbf{100.0\%}}}} \\

\ascline{1.5pt}{666666}{1-3}



\end{longtable}



\arrayrulecolor[HTML]{000000}

\global\setlength{\arrayrulewidth}{\Oldarrayrulewidth}

\global\setlength{\tabcolsep}{\Oldtabcolsep}

\renewcommand*{\arraystretch}{1}

\hypertarget{f_intencion.uso}{%
\paragraph{\texorpdfstring{\texttt{f\_Intencion.Uso}}{f\_Intencion.Uso}}\label{f_intencion.uso}}

Las categorías de \texttt{f\_Intencion.Uso} hacen referencia a si los
encuestados utilizarían nuestra barra de ensaladas o no (grado de
aceptación). Se trata de una variable cuyas categorías no guardan una
relación de orden entre sí.

\begin{Shaded}
\begin{Highlighting}[]
\CommentTok{\# Crear el data frame de frecuencias y porcentajes usando dplyr}
\NormalTok{datos\_grafico }\OtherTok{\textless{}{-}}\NormalTok{ datos }\SpecialCharTok{\%\textgreater{}\%}
  \CommentTok{\# Contar la frecuencia de cada nivel de la variable f\_Intencion.Uso}
  \FunctionTok{count}\NormalTok{(f\_Intencion.Uso, }\AttributeTok{name =} \StringTok{"Frecuencia"}\NormalTok{) }\SpecialCharTok{\%\textgreater{}\%}
  \CommentTok{\# Calcular el porcentaje y el texto de la etiqueta}
  \FunctionTok{mutate}\NormalTok{(}
    \AttributeTok{Porcentaje =}\NormalTok{ Frecuencia }\SpecialCharTok{/} \FunctionTok{sum}\NormalTok{(Frecuencia) }\SpecialCharTok{*} \DecValTok{100}\NormalTok{,}
    \AttributeTok{Etiquetas =} \FunctionTok{paste0}\NormalTok{(f\_Intencion.Uso, }\StringTok{"}\SpecialCharTok{\textbackslash{}n}\StringTok{("}\NormalTok{, }\FunctionTok{round}\NormalTok{(Porcentaje, }\DecValTok{1}\NormalTok{), }\StringTok{"\%)"}\NormalTok{) }\CommentTok{\# Etiqueta para el gráfico}
\NormalTok{  ) }\SpecialCharTok{\%\textgreater{}\%}
  \CommentTok{\# Eliminar filas con 0\% si existen, y categorías no deseadas}
  \FunctionTok{filter}\NormalTok{(Porcentaje }\SpecialCharTok{\textgreater{}} \DecValTok{0} \SpecialCharTok{\&}\NormalTok{ f\_Intencion.Uso }\SpecialCharTok{!=} \StringTok{"Total"}\NormalTok{) }\CommentTok{\# Se filtra \textquotesingle{}Total\textquotesingle{} aunque dplyr no lo añade}

\CommentTok{\# Usamos el vector de Porcentaje para el tamaño de las porciones}
\FunctionTok{pie}\NormalTok{(datos\_grafico}\SpecialCharTok{$}\NormalTok{Porcentaje,}
    \CommentTok{\# Usamos las Etiquetas que combinan nombre y porcentaje}
    \AttributeTok{labels =}\NormalTok{ datos\_grafico}\SpecialCharTok{$}\NormalTok{Etiquetas,}
    \CommentTok{\# Título}
    \AttributeTok{main =} \StringTok{\textquotesingle{}Figura 12. Grado de Aceptación\textquotesingle{}}\NormalTok{,}
    \CommentTok{\# Asignar un color diferente a cada porción}
    \AttributeTok{col =}\NormalTok{ RColorBrewer}\SpecialCharTok{::}\FunctionTok{brewer.pal}\NormalTok{(}\AttributeTok{n =} \FunctionTok{nrow}\NormalTok{(datos\_grafico), }\AttributeTok{name =} \StringTok{"Set3"}\NormalTok{))}
\end{Highlighting}
\end{Shaded}

\begin{center}\includegraphics{ICO-analisis_files/figure-latex/P19-tarta-1} \end{center}

Podemos concluir que la mayoría de los encuestados aceptarían el
producto y lo usarían, siendo del 73\%.

\begin{Shaded}
\begin{Highlighting}[]
\CommentTok{\# Preparación de datos}
\NormalTok{df\_tabla\_aceptacion }\OtherTok{\textless{}{-}}\NormalTok{ datos }\SpecialCharTok{\%\textgreater{}\%}
  \CommentTok{\# Contar la frecuencia de cada nivel}
\NormalTok{  dplyr}\SpecialCharTok{::}\FunctionTok{count}\NormalTok{(f\_Intencion.Uso, }\AttributeTok{name =} \StringTok{"Frecuencia"}\NormalTok{) }\SpecialCharTok{\%\textgreater{}\%}
  \CommentTok{\# Calcular porcentajes}
\NormalTok{  dplyr}\SpecialCharTok{::}\FunctionTok{mutate}\NormalTok{(}
    \AttributeTok{Porcentaje =}\NormalTok{ Frecuencia }\SpecialCharTok{/} \FunctionTok{sum}\NormalTok{(Frecuencia) }\SpecialCharTok{*} \DecValTok{100}
\NormalTok{  ) }\SpecialCharTok{\%\textgreater{}\%}
  \CommentTok{\# Redondear y formatear los porcentajes}
\NormalTok{  dplyr}\SpecialCharTok{::}\FunctionTok{mutate}\NormalTok{(}
    \AttributeTok{Porcentaje =} \FunctionTok{paste0}\NormalTok{(}\FunctionTok{round}\NormalTok{(Porcentaje, }\DecValTok{1}\NormalTok{), }\StringTok{"\%"}\NormalTok{)}
\NormalTok{  )}

\CommentTok{\# Calcular y añadir la fila total}
\NormalTok{df\_total }\OtherTok{\textless{}{-}} \FunctionTok{data.frame}\NormalTok{(}
  \AttributeTok{f\_Intencion.Uso =} \StringTok{"Total"}\NormalTok{,}
  \AttributeTok{Frecuencia =} \FunctionTok{sum}\NormalTok{(df\_tabla\_aceptacion}\SpecialCharTok{$}\NormalTok{Frecuencia),}
  \AttributeTok{Porcentaje =} \StringTok{"100.0\%"}
\NormalTok{)}

\CommentTok{\# Unir el data frame de frecuencias con la fila total}
\NormalTok{df\_tabla\_final }\OtherTok{\textless{}{-}} \FunctionTok{bind\_rows}\NormalTok{(df\_tabla\_aceptacion, df\_total)}

\CommentTok{\# Generación de la tabla (flextable)}
\NormalTok{ft }\OtherTok{\textless{}{-}}\NormalTok{ flextable}\SpecialCharTok{::}\FunctionTok{flextable}\NormalTok{(df\_tabla\_final) }\SpecialCharTok{\%\textgreater{}\%}

\NormalTok{flextable}\SpecialCharTok{::}\FunctionTok{set\_caption}\NormalTok{(}\AttributeTok{caption =} \StringTok{"Tabla 12. Distribución de frecuencias para el Grado de Aceptación"}\NormalTok{) }\SpecialCharTok{\%\textgreater{}\%}
  
  \CommentTok{\# Renombrar las cabeceras}
\NormalTok{  flextable}\SpecialCharTok{::}\FunctionTok{set\_header\_labels}\NormalTok{(}
    \AttributeTok{f\_Intencion.Uso =} \StringTok{"Grado de Aceptación"}\NormalTok{,}
    \AttributeTok{Frecuencia =} \StringTok{"Frecuencia (n)"}\NormalTok{,}
    \AttributeTok{Porcentaje =} \StringTok{"Porcentaje"}
\NormalTok{  ) }\SpecialCharTok{\%\textgreater{}\%}
  
  \CommentTok{\# Aplicar negrita y bordes a la fila "Total"}
\NormalTok{  flextable}\SpecialCharTok{::}\FunctionTok{bold}\NormalTok{(}\AttributeTok{i =} \FunctionTok{nrow}\NormalTok{(df\_tabla\_final), }\AttributeTok{part =} \StringTok{"body"}\NormalTok{) }\SpecialCharTok{\%\textgreater{}\%} \CommentTok{\# Negrita a la última fila (Total)}
\NormalTok{  flextable}\SpecialCharTok{::}\FunctionTok{border\_remove}\NormalTok{() }\SpecialCharTok{\%\textgreater{}\%} \CommentTok{\# Quitar bordes predeterminados}
\NormalTok{  flextable}\SpecialCharTok{::}\FunctionTok{theme\_booktabs}\NormalTok{() }\SpecialCharTok{\%\textgreater{}\%} \CommentTok{\# Aplicar un tema con líneas horizontales}
  
  \CommentTok{\# Formato de alineación y cabecera}
\NormalTok{  flextable}\SpecialCharTok{::}\FunctionTok{align}\NormalTok{(}\AttributeTok{j =} \DecValTok{1}\NormalTok{, }\AttributeTok{align =} \StringTok{"left"}\NormalTok{, }\AttributeTok{part =} \StringTok{"body"}\NormalTok{) }\SpecialCharTok{\%\textgreater{}\%}
\NormalTok{  flextable}\SpecialCharTok{::}\FunctionTok{align}\NormalTok{(}\AttributeTok{j =} \DecValTok{2}\SpecialCharTok{:}\DecValTok{3}\NormalTok{, }\AttributeTok{align =} \StringTok{"center"}\NormalTok{, }\AttributeTok{part =} \StringTok{"all"}\NormalTok{) }\SpecialCharTok{\%\textgreater{}\%} \CommentTok{\# Columnas 2 y 3 (Datos) CENTRADAS}
\NormalTok{  flextable}\SpecialCharTok{::}\FunctionTok{align}\NormalTok{(}\AttributeTok{align =} \StringTok{"center"}\NormalTok{, }\AttributeTok{part =} \StringTok{"header"}\NormalTok{) }\SpecialCharTok{\%\textgreater{}\%}        \CommentTok{\# Encabezados CENTRADOS}
  
  \CommentTok{\# Añadir una línea superior a la fila "Total" para separarla}
\NormalTok{  flextable}\SpecialCharTok{::}\FunctionTok{hline}\NormalTok{(}\AttributeTok{i =} \FunctionTok{nrow}\NormalTok{(df\_tabla\_final) }\SpecialCharTok{{-}} \DecValTok{1}\NormalTok{, }\AttributeTok{border =}\NormalTok{ officer}\SpecialCharTok{::}\FunctionTok{fp\_border}\NormalTok{(}\AttributeTok{width =} \FloatTok{1.5}\NormalTok{, }\AttributeTok{color =} \StringTok{"black"}\NormalTok{)) }\SpecialCharTok{\%\textgreater{}\%}
  
  \CommentTok{\# Ajustar el ancho de las columnas}
\NormalTok{  flextable}\SpecialCharTok{::}\FunctionTok{autofit}\NormalTok{()}

\CommentTok{\# Mostrar la tabla}
\NormalTok{ft}
\end{Highlighting}
\end{Shaded}

\global\setlength{\Oldarrayrulewidth}{\arrayrulewidth}

\global\setlength{\Oldtabcolsep}{\tabcolsep}

\setlength{\tabcolsep}{2pt}

\renewcommand*{\arraystretch}{1.5}



\providecommand{\ascline}[3]{\noalign{\global\arrayrulewidth #1}\arrayrulecolor[HTML]{#2}\cline{#3}}

\begin{longtable}[c]{|p{1.72in}|p{1.27in}|p{1.02in}}

\caption{Tabla\ 12.\ Distribución\ de\ frecuencias\ para\ el\ Grado\ de\ Aceptación}\\

\ascline{1.5pt}{666666}{1-3}

\multicolumn{1}{>{\centering}m{\dimexpr 1.72in+0\tabcolsep}}{\textcolor[HTML]{000000}{\fontsize{11}{11}\selectfont{Grado\ de\ Aceptación}}} & \multicolumn{1}{>{\centering}m{\dimexpr 1.27in+0\tabcolsep}}{\textcolor[HTML]{000000}{\fontsize{11}{11}\selectfont{Frecuencia\ (n)}}} & \multicolumn{1}{>{\centering}m{\dimexpr 1.02in+0\tabcolsep}}{\textcolor[HTML]{000000}{\fontsize{11}{11}\selectfont{Porcentaje}}} \\

\ascline{1.5pt}{666666}{1-3}\endfirsthead \caption[]{Tabla\ 12.\ Distribución\ de\ frecuencias\ para\ el\ Grado\ de\ Aceptación}\\

\ascline{1.5pt}{666666}{1-3}

\multicolumn{1}{>{\centering}m{\dimexpr 1.72in+0\tabcolsep}}{\textcolor[HTML]{000000}{\fontsize{11}{11}\selectfont{Grado\ de\ Aceptación}}} & \multicolumn{1}{>{\centering}m{\dimexpr 1.27in+0\tabcolsep}}{\textcolor[HTML]{000000}{\fontsize{11}{11}\selectfont{Frecuencia\ (n)}}} & \multicolumn{1}{>{\centering}m{\dimexpr 1.02in+0\tabcolsep}}{\textcolor[HTML]{000000}{\fontsize{11}{11}\selectfont{Porcentaje}}} \\

\ascline{1.5pt}{666666}{1-3}\endhead



\multicolumn{1}{>{\raggedright}m{\dimexpr 1.72in+0\tabcolsep}}{\textcolor[HTML]{000000}{\fontsize{11}{11}\selectfont{Sí}}} & \multicolumn{1}{>{\centering}m{\dimexpr 1.27in+0\tabcolsep}}{\textcolor[HTML]{000000}{\fontsize{11}{11}\selectfont{46}}} & \multicolumn{1}{>{\centering}m{\dimexpr 1.02in+0\tabcolsep}}{\textcolor[HTML]{000000}{\fontsize{11}{11}\selectfont{73\%}}} \\





\multicolumn{1}{>{\raggedright}m{\dimexpr 1.72in+0\tabcolsep}}{\textcolor[HTML]{000000}{\fontsize{11}{11}\selectfont{No}}} & \multicolumn{1}{>{\centering}m{\dimexpr 1.27in+0\tabcolsep}}{\textcolor[HTML]{000000}{\fontsize{11}{11}\selectfont{17}}} & \multicolumn{1}{>{\centering}m{\dimexpr 1.02in+0\tabcolsep}}{\textcolor[HTML]{000000}{\fontsize{11}{11}\selectfont{27\%}}} \\

\ascline{1.5pt}{000000}{1-3}



\multicolumn{1}{>{\raggedright}m{\dimexpr 1.72in+0\tabcolsep}}{\textcolor[HTML]{000000}{\fontsize{11}{11}\selectfont{\textbf{Total}}}} & \multicolumn{1}{>{\centering}m{\dimexpr 1.27in+0\tabcolsep}}{\textcolor[HTML]{000000}{\fontsize{11}{11}\selectfont{\textbf{63}}}} & \multicolumn{1}{>{\centering}m{\dimexpr 1.02in+0\tabcolsep}}{\textcolor[HTML]{000000}{\fontsize{11}{11}\selectfont{\textbf{100.0\%}}}} \\

\ascline{1.5pt}{666666}{1-3}



\end{longtable}



\arrayrulecolor[HTML]{000000}

\global\setlength{\arrayrulewidth}{\Oldarrayrulewidth}

\global\setlength{\tabcolsep}{\Oldtabcolsep}

\renewcommand*{\arraystretch}{1}

\hypertarget{f_genero}{%
\paragraph{\texorpdfstring{\texttt{f\_Genero}}{f\_Genero}}\label{f_genero}}

Las categorías de \texttt{f\_Genero} diferencia el género del
encuestado. Se trata de una variable cuyas categorías no guardan una
relación de orden entre sí.

\begin{Shaded}
\begin{Highlighting}[]
\CommentTok{\# Crear el data frame de frecuencias y porcentajes usando dplyr}
\NormalTok{datos\_grafico }\OtherTok{\textless{}{-}}\NormalTok{ datos }\SpecialCharTok{\%\textgreater{}\%}
  \CommentTok{\# Contar la frecuencia de cada nivel de la variable f\_Genero}
  \FunctionTok{count}\NormalTok{(f\_Genero, }\AttributeTok{name =} \StringTok{"Frecuencia"}\NormalTok{) }\SpecialCharTok{\%\textgreater{}\%}
  \CommentTok{\# Calcular el porcentaje y el texto de la etiqueta}
  \FunctionTok{mutate}\NormalTok{(}
    \AttributeTok{Porcentaje =}\NormalTok{ Frecuencia }\SpecialCharTok{/} \FunctionTok{sum}\NormalTok{(Frecuencia) }\SpecialCharTok{*} \DecValTok{100}\NormalTok{,}
    \AttributeTok{Etiquetas =} \FunctionTok{paste0}\NormalTok{(f\_Genero, }\StringTok{"}\SpecialCharTok{\textbackslash{}n}\StringTok{("}\NormalTok{, }\FunctionTok{round}\NormalTok{(Porcentaje, }\DecValTok{1}\NormalTok{), }\StringTok{"\%)"}\NormalTok{) }\CommentTok{\# Etiqueta para el gráfico}
\NormalTok{  ) }\SpecialCharTok{\%\textgreater{}\%}
  \CommentTok{\# Eliminar filas con 0\% si existen, y categorías no deseadas}
  \FunctionTok{filter}\NormalTok{(Porcentaje }\SpecialCharTok{\textgreater{}} \DecValTok{0} \SpecialCharTok{\&}\NormalTok{ f\_Genero }\SpecialCharTok{!=} \StringTok{"Total"}\NormalTok{) }\CommentTok{\# Se filtra \textquotesingle{}Total\textquotesingle{} aunque dplyr no lo añade}

\CommentTok{\# Usamos el vector de Porcentaje para el tamaño de las porciones}
\FunctionTok{pie}\NormalTok{(datos\_grafico}\SpecialCharTok{$}\NormalTok{Porcentaje,}
    \CommentTok{\# Usamos las Etiquetas que combinan nombre y porcentaje}
    \AttributeTok{labels =}\NormalTok{ datos\_grafico}\SpecialCharTok{$}\NormalTok{Etiquetas,}
    \CommentTok{\# Título}
    \AttributeTok{main =} \StringTok{\textquotesingle{}Figura 6. Género\textquotesingle{}}\NormalTok{,}
    \CommentTok{\# Asignar un color diferente a cada porción}
    \AttributeTok{col =}\NormalTok{ RColorBrewer}\SpecialCharTok{::}\FunctionTok{brewer.pal}\NormalTok{(}\AttributeTok{n =} \FunctionTok{nrow}\NormalTok{(datos\_grafico), }\AttributeTok{name =} \StringTok{"Set3"}\NormalTok{))}
\end{Highlighting}
\end{Shaded}

\begin{center}\includegraphics{ICO-analisis_files/figure-latex/P21-tarta-1} \end{center}

Se puede observar que la mayoría de los encuestados son hombres con un
50.8\%, mientras que las mujeres representan el 46\% de la respuestas.

\begin{Shaded}
\begin{Highlighting}[]
\CommentTok{\# Preparación de datos}
\NormalTok{df\_tabla\_genero }\OtherTok{\textless{}{-}}\NormalTok{ datos }\SpecialCharTok{\%\textgreater{}\%}
  \CommentTok{\# Contar la frecuencia de cada nivel}
\NormalTok{  dplyr}\SpecialCharTok{::}\FunctionTok{count}\NormalTok{(f\_Genero, }\AttributeTok{name =} \StringTok{"Frecuencia"}\NormalTok{) }\SpecialCharTok{\%\textgreater{}\%}
  \CommentTok{\# Calcular porcentajes}
\NormalTok{  dplyr}\SpecialCharTok{::}\FunctionTok{mutate}\NormalTok{(}
    \AttributeTok{Porcentaje =}\NormalTok{ Frecuencia }\SpecialCharTok{/} \FunctionTok{sum}\NormalTok{(Frecuencia) }\SpecialCharTok{*} \DecValTok{100}
\NormalTok{  ) }\SpecialCharTok{\%\textgreater{}\%}
  \CommentTok{\# Redondear y formatear los porcentajes}
\NormalTok{  dplyr}\SpecialCharTok{::}\FunctionTok{mutate}\NormalTok{(}
    \AttributeTok{Porcentaje =} \FunctionTok{paste0}\NormalTok{(}\FunctionTok{round}\NormalTok{(Porcentaje, }\DecValTok{1}\NormalTok{), }\StringTok{"\%"}\NormalTok{)}
\NormalTok{  )}

\CommentTok{\# Calcular y añadir la fila total}
\NormalTok{df\_total }\OtherTok{\textless{}{-}} \FunctionTok{data.frame}\NormalTok{(}
  \AttributeTok{f\_Genero =} \StringTok{"Total"}\NormalTok{,}
  \AttributeTok{Frecuencia =} \FunctionTok{sum}\NormalTok{(df\_tabla\_genero}\SpecialCharTok{$}\NormalTok{Frecuencia),}
  \AttributeTok{Porcentaje =} \StringTok{"100.0\%"}
\NormalTok{)}

\CommentTok{\# Unir el data frame de frecuencias con la fila total}
\NormalTok{df\_tabla\_final }\OtherTok{\textless{}{-}} \FunctionTok{bind\_rows}\NormalTok{(df\_tabla\_genero, df\_total)}

\CommentTok{\# Generación de la tabla (flextable)}
\NormalTok{ft }\OtherTok{\textless{}{-}}\NormalTok{ flextable}\SpecialCharTok{::}\FunctionTok{flextable}\NormalTok{(df\_tabla\_final) }\SpecialCharTok{\%\textgreater{}\%}

\NormalTok{flextable}\SpecialCharTok{::}\FunctionTok{set\_caption}\NormalTok{(}\AttributeTok{caption =} \StringTok{"Tabla 2. Distribución de frecuencias por Género"}\NormalTok{) }\SpecialCharTok{\%\textgreater{}\%}
  
  \CommentTok{\# Renombrar las cabeceras}
\NormalTok{  flextable}\SpecialCharTok{::}\FunctionTok{set\_header\_labels}\NormalTok{(}
    \AttributeTok{f\_Genero =} \StringTok{"Género"}\NormalTok{,}
    \AttributeTok{Frecuencia =} \StringTok{"Frecuencia (n)"}\NormalTok{,}
    \AttributeTok{Porcentaje =} \StringTok{"Porcentaje"}
\NormalTok{  ) }\SpecialCharTok{\%\textgreater{}\%}
  
  \CommentTok{\# Aplicar negrita y bordes a la fila "Total"}
\NormalTok{  flextable}\SpecialCharTok{::}\FunctionTok{bold}\NormalTok{(}\AttributeTok{i =} \FunctionTok{nrow}\NormalTok{(df\_tabla\_final), }\AttributeTok{part =} \StringTok{"body"}\NormalTok{) }\SpecialCharTok{\%\textgreater{}\%} \CommentTok{\# Negrita a la última fila (Total)}
\NormalTok{  flextable}\SpecialCharTok{::}\FunctionTok{border\_remove}\NormalTok{() }\SpecialCharTok{\%\textgreater{}\%} \CommentTok{\# Quitar bordes predeterminados}
\NormalTok{  flextable}\SpecialCharTok{::}\FunctionTok{theme\_booktabs}\NormalTok{() }\SpecialCharTok{\%\textgreater{}\%} \CommentTok{\# Aplicar un tema con líneas horizontales}
  
  \CommentTok{\# Formato de alineación y cabecera}
\NormalTok{  flextable}\SpecialCharTok{::}\FunctionTok{align}\NormalTok{(}\AttributeTok{j =} \DecValTok{1}\NormalTok{, }\AttributeTok{align =} \StringTok{"left"}\NormalTok{, }\AttributeTok{part =} \StringTok{"body"}\NormalTok{) }\SpecialCharTok{\%\textgreater{}\%}
\NormalTok{  flextable}\SpecialCharTok{::}\FunctionTok{align}\NormalTok{(}\AttributeTok{j =} \DecValTok{2}\SpecialCharTok{:}\DecValTok{3}\NormalTok{, }\AttributeTok{align =} \StringTok{"center"}\NormalTok{, }\AttributeTok{part =} \StringTok{"all"}\NormalTok{) }\SpecialCharTok{\%\textgreater{}\%} \CommentTok{\# Columnas 2 y 3 (Datos) CENTRADAS}
\NormalTok{  flextable}\SpecialCharTok{::}\FunctionTok{align}\NormalTok{(}\AttributeTok{align =} \StringTok{"center"}\NormalTok{, }\AttributeTok{part =} \StringTok{"header"}\NormalTok{) }\SpecialCharTok{\%\textgreater{}\%}        \CommentTok{\# Encabezados CENTRADOS}
  
  \CommentTok{\# Añadir una línea superior a la fila "Total" para separarla}
\NormalTok{  flextable}\SpecialCharTok{::}\FunctionTok{hline}\NormalTok{(}\AttributeTok{i =} \FunctionTok{nrow}\NormalTok{(df\_tabla\_final) }\SpecialCharTok{{-}} \DecValTok{1}\NormalTok{, }\AttributeTok{border =}\NormalTok{ officer}\SpecialCharTok{::}\FunctionTok{fp\_border}\NormalTok{(}\AttributeTok{width =} \FloatTok{1.5}\NormalTok{, }\AttributeTok{color =} \StringTok{"black"}\NormalTok{)) }\SpecialCharTok{\%\textgreater{}\%}
  
  \CommentTok{\# Ajustar el ancho de las columnas}
\NormalTok{  flextable}\SpecialCharTok{::}\FunctionTok{autofit}\NormalTok{()}

\CommentTok{\# Mostrar la tabla}
\NormalTok{ft}
\end{Highlighting}
\end{Shaded}

\global\setlength{\Oldarrayrulewidth}{\arrayrulewidth}

\global\setlength{\Oldtabcolsep}{\tabcolsep}

\setlength{\tabcolsep}{2pt}

\renewcommand*{\arraystretch}{1.5}



\providecommand{\ascline}[3]{\noalign{\global\arrayrulewidth #1}\arrayrulecolor[HTML]{#2}\cline{#3}}

\begin{longtable}[c]{|p{0.83in}|p{1.27in}|p{1.02in}}

\caption{Tabla\ 2.\ Distribución\ de\ frecuencias\ por\ Género}\\

\ascline{1.5pt}{666666}{1-3}

\multicolumn{1}{>{\centering}m{\dimexpr 0.83in+0\tabcolsep}}{\textcolor[HTML]{000000}{\fontsize{11}{11}\selectfont{Género}}} & \multicolumn{1}{>{\centering}m{\dimexpr 1.27in+0\tabcolsep}}{\textcolor[HTML]{000000}{\fontsize{11}{11}\selectfont{Frecuencia\ (n)}}} & \multicolumn{1}{>{\centering}m{\dimexpr 1.02in+0\tabcolsep}}{\textcolor[HTML]{000000}{\fontsize{11}{11}\selectfont{Porcentaje}}} \\

\ascline{1.5pt}{666666}{1-3}\endfirsthead \caption[]{Tabla\ 2.\ Distribución\ de\ frecuencias\ por\ Género}\\

\ascline{1.5pt}{666666}{1-3}

\multicolumn{1}{>{\centering}m{\dimexpr 0.83in+0\tabcolsep}}{\textcolor[HTML]{000000}{\fontsize{11}{11}\selectfont{Género}}} & \multicolumn{1}{>{\centering}m{\dimexpr 1.27in+0\tabcolsep}}{\textcolor[HTML]{000000}{\fontsize{11}{11}\selectfont{Frecuencia\ (n)}}} & \multicolumn{1}{>{\centering}m{\dimexpr 1.02in+0\tabcolsep}}{\textcolor[HTML]{000000}{\fontsize{11}{11}\selectfont{Porcentaje}}} \\

\ascline{1.5pt}{666666}{1-3}\endhead



\multicolumn{1}{>{\raggedright}m{\dimexpr 0.83in+0\tabcolsep}}{\textcolor[HTML]{000000}{\fontsize{11}{11}\selectfont{Mujer}}} & \multicolumn{1}{>{\centering}m{\dimexpr 1.27in+0\tabcolsep}}{\textcolor[HTML]{000000}{\fontsize{11}{11}\selectfont{29}}} & \multicolumn{1}{>{\centering}m{\dimexpr 1.02in+0\tabcolsep}}{\textcolor[HTML]{000000}{\fontsize{11}{11}\selectfont{46\%}}} \\





\multicolumn{1}{>{\raggedright}m{\dimexpr 0.83in+0\tabcolsep}}{\textcolor[HTML]{000000}{\fontsize{11}{11}\selectfont{Hombre}}} & \multicolumn{1}{>{\centering}m{\dimexpr 1.27in+0\tabcolsep}}{\textcolor[HTML]{000000}{\fontsize{11}{11}\selectfont{32}}} & \multicolumn{1}{>{\centering}m{\dimexpr 1.02in+0\tabcolsep}}{\textcolor[HTML]{000000}{\fontsize{11}{11}\selectfont{50.8\%}}} \\





\multicolumn{1}{>{\raggedright}m{\dimexpr 0.83in+0\tabcolsep}}{\textcolor[HTML]{000000}{\fontsize{11}{11}\selectfont{NS/NC}}} & \multicolumn{1}{>{\centering}m{\dimexpr 1.27in+0\tabcolsep}}{\textcolor[HTML]{000000}{\fontsize{11}{11}\selectfont{2}}} & \multicolumn{1}{>{\centering}m{\dimexpr 1.02in+0\tabcolsep}}{\textcolor[HTML]{000000}{\fontsize{11}{11}\selectfont{3.2\%}}} \\

\ascline{1.5pt}{000000}{1-3}



\multicolumn{1}{>{\raggedright}m{\dimexpr 0.83in+0\tabcolsep}}{\textcolor[HTML]{000000}{\fontsize{11}{11}\selectfont{\textbf{Total}}}} & \multicolumn{1}{>{\centering}m{\dimexpr 1.27in+0\tabcolsep}}{\textcolor[HTML]{000000}{\fontsize{11}{11}\selectfont{\textbf{63}}}} & \multicolumn{1}{>{\centering}m{\dimexpr 1.02in+0\tabcolsep}}{\textcolor[HTML]{000000}{\fontsize{11}{11}\selectfont{\textbf{100.0\%}}}} \\

\ascline{1.5pt}{666666}{1-3}



\end{longtable}



\arrayrulecolor[HTML]{000000}

\global\setlength{\arrayrulewidth}{\Oldarrayrulewidth}

\global\setlength{\tabcolsep}{\Oldtabcolsep}

\renewcommand*{\arraystretch}{1}

\hypertarget{f_situacion.laboral}{%
\paragraph{\texorpdfstring{\texttt{f\_Situacion.Laboral}}{f\_Situacion.Laboral}}\label{f_situacion.laboral}}

Las categorías de \texttt{f\_Situacion.Laboral} indican la situación
laboral del encuestado. Se trata de una variable cuyas categorías no
guardan una relación de orden entre sí.

\begin{Shaded}
\begin{Highlighting}[]
\CommentTok{\# Crear el data frame de frecuencias y porcentajes usando dplyr}
\NormalTok{datos\_grafico }\OtherTok{\textless{}{-}}\NormalTok{ datos }\SpecialCharTok{\%\textgreater{}\%}
  \CommentTok{\# Contar la frecuencia de cada nivel de la variable f\_Situacion.Laboral}
  \FunctionTok{count}\NormalTok{(f\_Situacion.Laboral, }\AttributeTok{name =} \StringTok{"Frecuencia"}\NormalTok{) }\SpecialCharTok{\%\textgreater{}\%}
  \CommentTok{\# Calcular el porcentaje y el texto de la etiqueta}
  \FunctionTok{mutate}\NormalTok{(}
    \AttributeTok{Porcentaje =}\NormalTok{ Frecuencia }\SpecialCharTok{/} \FunctionTok{sum}\NormalTok{(Frecuencia) }\SpecialCharTok{*} \DecValTok{100}\NormalTok{,}
    \AttributeTok{Etiquetas =} \FunctionTok{paste0}\NormalTok{(f\_Situacion.Laboral, }\StringTok{"}\SpecialCharTok{\textbackslash{}n}\StringTok{("}\NormalTok{, }\FunctionTok{round}\NormalTok{(Porcentaje, }\DecValTok{1}\NormalTok{), }\StringTok{"\%)"}\NormalTok{) }\CommentTok{\# Etiqueta para el gráfico}
\NormalTok{  ) }\SpecialCharTok{\%\textgreater{}\%}
  \CommentTok{\# Eliminar filas con 0\% si existen, y categorías no deseadas}
  \FunctionTok{filter}\NormalTok{(Porcentaje }\SpecialCharTok{\textgreater{}} \DecValTok{0} \SpecialCharTok{\&}\NormalTok{ f\_Situacion.Laboral }\SpecialCharTok{!=} \StringTok{"Total"}\NormalTok{) }\CommentTok{\# Se filtra \textquotesingle{}Total\textquotesingle{} aunque dplyr no lo añade}

\CommentTok{\# Usamos el vector de Porcentaje para el tamaño de las porciones}
\FunctionTok{pie}\NormalTok{(datos\_grafico}\SpecialCharTok{$}\NormalTok{Porcentaje,}
    \CommentTok{\# Usamos las Etiquetas que combinan nombre y porcentaje}
    \AttributeTok{labels =}\NormalTok{ datos\_grafico}\SpecialCharTok{$}\NormalTok{Etiquetas,}
    \CommentTok{\# Título}
    \AttributeTok{main =} \StringTok{\textquotesingle{}Figura 8. Situación Laboral\textquotesingle{}}\NormalTok{,}
    \CommentTok{\# Asignar un color diferente a cada porción}
    \AttributeTok{col =}\NormalTok{ RColorBrewer}\SpecialCharTok{::}\FunctionTok{brewer.pal}\NormalTok{(}\AttributeTok{n =} \FunctionTok{nrow}\NormalTok{(datos\_grafico), }\AttributeTok{name =} \StringTok{"Set3"}\NormalTok{))}
\end{Highlighting}
\end{Shaded}

\begin{center}\includegraphics{ICO-analisis_files/figure-latex/P23-tarta-1} \end{center}

Se observa que la mayoría de encuestados son estudiantes(con un 61,9\%),
seguidos de trabajadores por cuenta ajena (con un 25,4\%).

\begin{Shaded}
\begin{Highlighting}[]
\CommentTok{\# Preparación de datos}
\NormalTok{df\_tabla\_laboral }\OtherTok{\textless{}{-}}\NormalTok{ datos }\SpecialCharTok{\%\textgreater{}\%}
  \CommentTok{\# Contar la frecuencia de cada nivel}
\NormalTok{  dplyr}\SpecialCharTok{::}\FunctionTok{count}\NormalTok{(f\_Situacion.Laboral, }\AttributeTok{name =} \StringTok{"Frecuencia"}\NormalTok{) }\SpecialCharTok{\%\textgreater{}\%}
  \CommentTok{\# Calcular porcentajes}
\NormalTok{  dplyr}\SpecialCharTok{::}\FunctionTok{mutate}\NormalTok{(}
    \AttributeTok{Porcentaje =}\NormalTok{ Frecuencia }\SpecialCharTok{/} \FunctionTok{sum}\NormalTok{(Frecuencia) }\SpecialCharTok{*} \DecValTok{100}
\NormalTok{  ) }\SpecialCharTok{\%\textgreater{}\%}
  \CommentTok{\# Redondear y formatear los porcentajes}
\NormalTok{  dplyr}\SpecialCharTok{::}\FunctionTok{mutate}\NormalTok{(}
    \AttributeTok{Porcentaje =} \FunctionTok{paste0}\NormalTok{(}\FunctionTok{round}\NormalTok{(Porcentaje, }\DecValTok{1}\NormalTok{), }\StringTok{"\%"}\NormalTok{)}
\NormalTok{  )}

\CommentTok{\# Calcular y añadir la fila total}
\NormalTok{df\_total }\OtherTok{\textless{}{-}} \FunctionTok{data.frame}\NormalTok{(}
  \AttributeTok{f\_Situacion.Laboral =} \StringTok{"Total"}\NormalTok{,}
  \AttributeTok{Frecuencia =} \FunctionTok{sum}\NormalTok{(df\_tabla\_laboral}\SpecialCharTok{$}\NormalTok{Frecuencia),}
  \AttributeTok{Porcentaje =} \StringTok{"100.0\%"}
\NormalTok{)}

\CommentTok{\# Unir el data frame de frecuencias con la fila total}
\NormalTok{df\_tabla\_final }\OtherTok{\textless{}{-}} \FunctionTok{bind\_rows}\NormalTok{(df\_tabla\_laboral, df\_total)}

\CommentTok{\# Generación de la tabla (flextable)}
\NormalTok{ft }\OtherTok{\textless{}{-}}\NormalTok{ flextable}\SpecialCharTok{::}\FunctionTok{flextable}\NormalTok{(df\_tabla\_final) }\SpecialCharTok{\%\textgreater{}\%}

\NormalTok{flextable}\SpecialCharTok{::}\FunctionTok{set\_caption}\NormalTok{(}\AttributeTok{caption =} \StringTok{"Tabla 4. Distribución de frecuencias para Situación Laboral"}\NormalTok{) }\SpecialCharTok{\%\textgreater{}\%}
  
  \CommentTok{\# Renombrar las cabeceras}
\NormalTok{  flextable}\SpecialCharTok{::}\FunctionTok{set\_header\_labels}\NormalTok{(}
    \AttributeTok{f\_Situacion.Laboral =} \StringTok{"Situación Laboral"}\NormalTok{,}
    \AttributeTok{Frecuencia =} \StringTok{"Frecuencia (n)"}\NormalTok{,}
    \AttributeTok{Porcentaje =} \StringTok{"Porcentaje"}
\NormalTok{  ) }\SpecialCharTok{\%\textgreater{}\%}
  
  \CommentTok{\# Aplicar negrita y bordes a la fila "Total"}
\NormalTok{  flextable}\SpecialCharTok{::}\FunctionTok{bold}\NormalTok{(}\AttributeTok{i =} \FunctionTok{nrow}\NormalTok{(df\_tabla\_final), }\AttributeTok{part =} \StringTok{"body"}\NormalTok{) }\SpecialCharTok{\%\textgreater{}\%} \CommentTok{\# Negrita a la última fila (Total)}
\NormalTok{  flextable}\SpecialCharTok{::}\FunctionTok{border\_remove}\NormalTok{() }\SpecialCharTok{\%\textgreater{}\%} \CommentTok{\# Quitar bordes predeterminados}
\NormalTok{  flextable}\SpecialCharTok{::}\FunctionTok{theme\_booktabs}\NormalTok{() }\SpecialCharTok{\%\textgreater{}\%} \CommentTok{\# Aplicar un tema con líneas horizontales}
  
  \CommentTok{\# Formato de alineación y cabecera}
\NormalTok{  flextable}\SpecialCharTok{::}\FunctionTok{align}\NormalTok{(}\AttributeTok{j =} \DecValTok{1}\NormalTok{, }\AttributeTok{align =} \StringTok{"left"}\NormalTok{, }\AttributeTok{part =} \StringTok{"body"}\NormalTok{) }\SpecialCharTok{\%\textgreater{}\%}
\NormalTok{  flextable}\SpecialCharTok{::}\FunctionTok{align}\NormalTok{(}\AttributeTok{j =} \DecValTok{2}\SpecialCharTok{:}\DecValTok{3}\NormalTok{, }\AttributeTok{align =} \StringTok{"center"}\NormalTok{, }\AttributeTok{part =} \StringTok{"all"}\NormalTok{) }\SpecialCharTok{\%\textgreater{}\%} \CommentTok{\# Columnas 2 y 3 (Datos) CENTRADAS}
\NormalTok{  flextable}\SpecialCharTok{::}\FunctionTok{align}\NormalTok{(}\AttributeTok{align =} \StringTok{"center"}\NormalTok{, }\AttributeTok{part =} \StringTok{"header"}\NormalTok{) }\SpecialCharTok{\%\textgreater{}\%}        \CommentTok{\# Encabezados CENTRADOS}
  
  \CommentTok{\# Añadir una línea superior a la fila "Total" para separarla}
\NormalTok{  flextable}\SpecialCharTok{::}\FunctionTok{hline}\NormalTok{(}\AttributeTok{i =} \FunctionTok{nrow}\NormalTok{(df\_tabla\_final) }\SpecialCharTok{{-}} \DecValTok{1}\NormalTok{, }\AttributeTok{border =}\NormalTok{ officer}\SpecialCharTok{::}\FunctionTok{fp\_border}\NormalTok{(}\AttributeTok{width =} \FloatTok{1.5}\NormalTok{, }\AttributeTok{color =} \StringTok{"black"}\NormalTok{)) }\SpecialCharTok{\%\textgreater{}\%}
  
  \CommentTok{\# Ajustar el ancho de las columnas}
\NormalTok{  flextable}\SpecialCharTok{::}\FunctionTok{autofit}\NormalTok{()}

\CommentTok{\# Mostrar la tabla}
\NormalTok{ft}
\end{Highlighting}
\end{Shaded}

\global\setlength{\Oldarrayrulewidth}{\arrayrulewidth}

\global\setlength{\Oldtabcolsep}{\tabcolsep}

\setlength{\tabcolsep}{2pt}

\renewcommand*{\arraystretch}{1.5}



\providecommand{\ascline}[3]{\noalign{\global\arrayrulewidth #1}\arrayrulecolor[HTML]{#2}\cline{#3}}

\begin{longtable}[c]{|p{1.47in}|p{1.27in}|p{1.02in}}

\caption{Tabla\ 4.\ Distribución\ de\ frecuencias\ para\ Situación\ Laboral}\\

\ascline{1.5pt}{666666}{1-3}

\multicolumn{1}{>{\centering}m{\dimexpr 1.47in+0\tabcolsep}}{\textcolor[HTML]{000000}{\fontsize{11}{11}\selectfont{Situación\ Laboral}}} & \multicolumn{1}{>{\centering}m{\dimexpr 1.27in+0\tabcolsep}}{\textcolor[HTML]{000000}{\fontsize{11}{11}\selectfont{Frecuencia\ (n)}}} & \multicolumn{1}{>{\centering}m{\dimexpr 1.02in+0\tabcolsep}}{\textcolor[HTML]{000000}{\fontsize{11}{11}\selectfont{Porcentaje}}} \\

\ascline{1.5pt}{666666}{1-3}\endfirsthead \caption[]{Tabla\ 4.\ Distribución\ de\ frecuencias\ para\ Situación\ Laboral}\\

\ascline{1.5pt}{666666}{1-3}

\multicolumn{1}{>{\centering}m{\dimexpr 1.47in+0\tabcolsep}}{\textcolor[HTML]{000000}{\fontsize{11}{11}\selectfont{Situación\ Laboral}}} & \multicolumn{1}{>{\centering}m{\dimexpr 1.27in+0\tabcolsep}}{\textcolor[HTML]{000000}{\fontsize{11}{11}\selectfont{Frecuencia\ (n)}}} & \multicolumn{1}{>{\centering}m{\dimexpr 1.02in+0\tabcolsep}}{\textcolor[HTML]{000000}{\fontsize{11}{11}\selectfont{Porcentaje}}} \\

\ascline{1.5pt}{666666}{1-3}\endhead



\multicolumn{1}{>{\raggedright}m{\dimexpr 1.47in+0\tabcolsep}}{\textcolor[HTML]{000000}{\fontsize{11}{11}\selectfont{Estudiante}}} & \multicolumn{1}{>{\centering}m{\dimexpr 1.27in+0\tabcolsep}}{\textcolor[HTML]{000000}{\fontsize{11}{11}\selectfont{39}}} & \multicolumn{1}{>{\centering}m{\dimexpr 1.02in+0\tabcolsep}}{\textcolor[HTML]{000000}{\fontsize{11}{11}\selectfont{61.9\%}}} \\





\multicolumn{1}{>{\raggedright}m{\dimexpr 1.47in+0\tabcolsep}}{\textcolor[HTML]{000000}{\fontsize{11}{11}\selectfont{Cuenta\ Ajena}}} & \multicolumn{1}{>{\centering}m{\dimexpr 1.27in+0\tabcolsep}}{\textcolor[HTML]{000000}{\fontsize{11}{11}\selectfont{16}}} & \multicolumn{1}{>{\centering}m{\dimexpr 1.02in+0\tabcolsep}}{\textcolor[HTML]{000000}{\fontsize{11}{11}\selectfont{25.4\%}}} \\





\multicolumn{1}{>{\raggedright}m{\dimexpr 1.47in+0\tabcolsep}}{\textcolor[HTML]{000000}{\fontsize{11}{11}\selectfont{Autónomo}}} & \multicolumn{1}{>{\centering}m{\dimexpr 1.27in+0\tabcolsep}}{\textcolor[HTML]{000000}{\fontsize{11}{11}\selectfont{3}}} & \multicolumn{1}{>{\centering}m{\dimexpr 1.02in+0\tabcolsep}}{\textcolor[HTML]{000000}{\fontsize{11}{11}\selectfont{4.8\%}}} \\





\multicolumn{1}{>{\raggedright}m{\dimexpr 1.47in+0\tabcolsep}}{\textcolor[HTML]{000000}{\fontsize{11}{11}\selectfont{Desempleado}}} & \multicolumn{1}{>{\centering}m{\dimexpr 1.27in+0\tabcolsep}}{\textcolor[HTML]{000000}{\fontsize{11}{11}\selectfont{2}}} & \multicolumn{1}{>{\centering}m{\dimexpr 1.02in+0\tabcolsep}}{\textcolor[HTML]{000000}{\fontsize{11}{11}\selectfont{3.2\%}}} \\





\multicolumn{1}{>{\raggedright}m{\dimexpr 1.47in+0\tabcolsep}}{\textcolor[HTML]{000000}{\fontsize{11}{11}\selectfont{Jubilado}}} & \multicolumn{1}{>{\centering}m{\dimexpr 1.27in+0\tabcolsep}}{\textcolor[HTML]{000000}{\fontsize{11}{11}\selectfont{3}}} & \multicolumn{1}{>{\centering}m{\dimexpr 1.02in+0\tabcolsep}}{\textcolor[HTML]{000000}{\fontsize{11}{11}\selectfont{4.8\%}}} \\

\ascline{1.5pt}{000000}{1-3}



\multicolumn{1}{>{\raggedright}m{\dimexpr 1.47in+0\tabcolsep}}{\textcolor[HTML]{000000}{\fontsize{11}{11}\selectfont{\textbf{Total}}}} & \multicolumn{1}{>{\centering}m{\dimexpr 1.27in+0\tabcolsep}}{\textcolor[HTML]{000000}{\fontsize{11}{11}\selectfont{\textbf{63}}}} & \multicolumn{1}{>{\centering}m{\dimexpr 1.02in+0\tabcolsep}}{\textcolor[HTML]{000000}{\fontsize{11}{11}\selectfont{\textbf{100.0\%}}}} \\

\ascline{1.5pt}{666666}{1-3}



\end{longtable}



\arrayrulecolor[HTML]{000000}

\global\setlength{\arrayrulewidth}{\Oldarrayrulewidth}

\global\setlength{\tabcolsep}{\Oldtabcolsep}

\renewcommand*{\arraystretch}{1}

\hypertarget{caso-de-variables-con-categoruxedas-ordenadas}{%
\subsubsection{Caso de variables con categorías
ordenadas}\label{caso-de-variables-con-categoruxedas-ordenadas}}

\hypertarget{f_frecuencia}{%
\paragraph{\texorpdfstring{\texttt{f\_Frecuencia}}{f\_Frecuencia}}\label{f_frecuencia}}

Las categorías de \texttt{f\_Frecuencia} miden la frecuencia con la que
los encuestados comer fuera de casa. La variable tiene 5 categorías, que
se ordenan desde menos tiempo a más tiempo: Nunca, Rara/Ocasionalmente,
1-2 veces por semana, 3-4 veces por semana y más de 5 veces por semana.
Este orden se fijó mediante el parámetro levels de la instrucción
factor().

\begin{Shaded}
\begin{Highlighting}[]
\NormalTok{df\_frecuencia\_tiempo }\OtherTok{\textless{}{-}}\NormalTok{ datos }\SpecialCharTok{\%\textgreater{}\%}
  \CommentTok{\# Contar la frecuencia de cada nivel (respeta el orden del factor f\_Frecuencia)}
\NormalTok{  dplyr}\SpecialCharTok{::}\FunctionTok{count}\NormalTok{(f\_Frecuencia, }\AttributeTok{name =} \StringTok{"Frecuencia"}\NormalTok{) }\SpecialCharTok{\%\textgreater{}\%}
  
  \CommentTok{\# Calcular porcentajes y porcentaje acumulado}
\NormalTok{  dplyr}\SpecialCharTok{::}\FunctionTok{mutate}\NormalTok{(}
    \AttributeTok{Porcentaje =}\NormalTok{ Frecuencia }\SpecialCharTok{/} \FunctionTok{sum}\NormalTok{(Frecuencia) }\SpecialCharTok{*} \DecValTok{100}\NormalTok{,}
    \AttributeTok{Porcentaje\_Acumulado =} \FunctionTok{cumsum}\NormalTok{(Porcentaje)}
\NormalTok{  ) }\SpecialCharTok{\%\textgreater{}\%}
  
\NormalTok{  dplyr}\SpecialCharTok{::}\FunctionTok{mutate}\NormalTok{(}
    \AttributeTok{Porcentaje =} \FunctionTok{round}\NormalTok{(Porcentaje, }\DecValTok{1}\NormalTok{), }\CommentTok{\# Redondeado a 1 decimal}
    \AttributeTok{Porcentaje\_Acumulado =} \FunctionTok{round}\NormalTok{(Porcentaje\_Acumulado, }\DecValTok{1}\NormalTok{)}
\NormalTok{  )}


\CommentTok{\# Generación del gráfico de barras}
\FunctionTok{ggplot}\NormalTok{(df\_frecuencia\_tiempo, }\FunctionTok{aes}\NormalTok{(}\AttributeTok{x =}\NormalTok{ f\_Frecuencia, }\AttributeTok{y =}\NormalTok{ Frecuencia)) }\SpecialCharTok{+}
  \CommentTok{\# Barras}
  \FunctionTok{geom\_bar}\NormalTok{(}\AttributeTok{stat =} \StringTok{"identity"}\NormalTok{, }\AttributeTok{fill =} \StringTok{"\#1f78b4"}\NormalTok{) }\SpecialCharTok{+}
  
  \CommentTok{\# Etiquetas de Porcentaje sobre las barras}
  \FunctionTok{geom\_text}\NormalTok{(}\FunctionTok{aes}\NormalTok{(}\AttributeTok{label =} \FunctionTok{paste0}\NormalTok{(Porcentaje, }\StringTok{"\%"}\NormalTok{)), }
            \AttributeTok{vjust =} \SpecialCharTok{{-}}\FloatTok{0.5}\NormalTok{, }\CommentTok{\# Ajuste vertical para que quede sobre la barra}
            \AttributeTok{size =} \DecValTok{4}\NormalTok{) }\SpecialCharTok{+}
  
  \CommentTok{\# Títulos y Ejes}
  \FunctionTok{labs}\NormalTok{(}
    \AttributeTok{title =} \StringTok{\textquotesingle{}Figura 15. Frecuencia de Comer Fuera de Casa\textquotesingle{}}\NormalTok{,}
    \AttributeTok{x =} \StringTok{"Frecuencia"}\NormalTok{,}
    \AttributeTok{y =} \StringTok{"Numero de respuestas"}
\NormalTok{  ) }\SpecialCharTok{+}
  
  \CommentTok{\# Corrección del eje Y para evitar el recorte de las etiquetas}
  \FunctionTok{scale\_y\_continuous}\NormalTok{(}\AttributeTok{expand =} \FunctionTok{expansion}\NormalTok{(}\AttributeTok{mult =} \FunctionTok{c}\NormalTok{(}\DecValTok{0}\NormalTok{, }\FloatTok{0.15}\NormalTok{)))}
\end{Highlighting}
\end{Shaded}

\begin{center}\includegraphics{ICO-analisis_files/figure-latex/P2-barra-1} \end{center}

Se observa que la mayoría de los encuestados salen a comer fuera de casa
entre 1 y 2 veces por semana (con un 34.9\%). También se observa que el
gráfico tiene una ligera asimetría a la derecha.

\begin{Shaded}
\begin{Highlighting}[]
\CommentTok{\# Preparación de datos}
\NormalTok{df\_tabla\_freq }\OtherTok{\textless{}{-}}\NormalTok{ datos }\SpecialCharTok{\%\textgreater{}\%}
  \CommentTok{\# Contar la frecuencia de cada nivel}
\NormalTok{  dplyr}\SpecialCharTok{::}\FunctionTok{count}\NormalTok{(f\_Frecuencia, }\AttributeTok{name =} \StringTok{"Frecuencia"}\NormalTok{) }\SpecialCharTok{\%\textgreater{}\%}
  \CommentTok{\# Calcular porcentajes}
\NormalTok{  dplyr}\SpecialCharTok{::}\FunctionTok{mutate}\NormalTok{(}
    \AttributeTok{Porcentaje =}\NormalTok{ Frecuencia }\SpecialCharTok{/} \FunctionTok{sum}\NormalTok{(Frecuencia) }\SpecialCharTok{*} \DecValTok{100}
\NormalTok{  ) }\SpecialCharTok{\%\textgreater{}\%}
  \CommentTok{\# Redondear y formatear los porcentajes}
\NormalTok{  dplyr}\SpecialCharTok{::}\FunctionTok{mutate}\NormalTok{(}
    \AttributeTok{Porcentaje =} \FunctionTok{paste0}\NormalTok{(}\FunctionTok{round}\NormalTok{(Porcentaje, }\DecValTok{1}\NormalTok{), }\StringTok{"\%"}\NormalTok{)}
\NormalTok{  )}

\CommentTok{\# Calcular y añadir la fila total}
\NormalTok{df\_total }\OtherTok{\textless{}{-}} \FunctionTok{data.frame}\NormalTok{(}
  \AttributeTok{f\_Frecuencia =} \StringTok{"Total"}\NormalTok{,}
  \AttributeTok{Frecuencia =} \FunctionTok{sum}\NormalTok{(df\_tabla\_freq}\SpecialCharTok{$}\NormalTok{Frecuencia),}
  \AttributeTok{Porcentaje =} \StringTok{"100.0\%"}
\NormalTok{)}

\CommentTok{\# Unir el data frame de frecuencias con la fila total}
\NormalTok{df\_tabla\_final }\OtherTok{\textless{}{-}} \FunctionTok{bind\_rows}\NormalTok{(df\_tabla\_freq, df\_total)}

\CommentTok{\# Generación de la tabla (flextable)}
\NormalTok{ft }\OtherTok{\textless{}{-}}\NormalTok{ flextable}\SpecialCharTok{::}\FunctionTok{flextable}\NormalTok{(df\_tabla\_final) }\SpecialCharTok{\%\textgreater{}\%}

\NormalTok{flextable}\SpecialCharTok{::}\FunctionTok{set\_caption}\NormalTok{(}\AttributeTok{caption =} \StringTok{"Tabla 15. Distribución de frecuencias para Frecuencia de Comer Fuera de Casa"}\NormalTok{) }\SpecialCharTok{\%\textgreater{}\%}
  
  \CommentTok{\# Renombrar las cabeceras}
\NormalTok{  flextable}\SpecialCharTok{::}\FunctionTok{set\_header\_labels}\NormalTok{(}
    \AttributeTok{f\_Frecuencia =} \StringTok{"Frecuencia de comer fuera de casa"}\NormalTok{,}
    \AttributeTok{Frecuencia =} \StringTok{"Frecuencia (n)"}\NormalTok{,}
    \AttributeTok{Porcentaje =} \StringTok{"Porcentaje"}
\NormalTok{  ) }\SpecialCharTok{\%\textgreater{}\%}
  
  \CommentTok{\# Aplicar negrita y bordes a la fila "Total"}
\NormalTok{  flextable}\SpecialCharTok{::}\FunctionTok{bold}\NormalTok{(}\AttributeTok{i =} \FunctionTok{nrow}\NormalTok{(df\_tabla\_final), }\AttributeTok{part =} \StringTok{"body"}\NormalTok{) }\SpecialCharTok{\%\textgreater{}\%} \CommentTok{\# Negrita a la última fila (Total)}
\NormalTok{  flextable}\SpecialCharTok{::}\FunctionTok{border\_remove}\NormalTok{() }\SpecialCharTok{\%\textgreater{}\%} \CommentTok{\# Quitar bordes predeterminados}
\NormalTok{  flextable}\SpecialCharTok{::}\FunctionTok{theme\_booktabs}\NormalTok{() }\SpecialCharTok{\%\textgreater{}\%} \CommentTok{\# Aplicar un tema con líneas horizontales}
  
  \CommentTok{\# Formato de alineación y cabecera}
\NormalTok{  flextable}\SpecialCharTok{::}\FunctionTok{align}\NormalTok{(}\AttributeTok{j =} \DecValTok{1}\NormalTok{, }\AttributeTok{align =} \StringTok{"left"}\NormalTok{, }\AttributeTok{part =} \StringTok{"body"}\NormalTok{) }\SpecialCharTok{\%\textgreater{}\%}
\NormalTok{  flextable}\SpecialCharTok{::}\FunctionTok{align}\NormalTok{(}\AttributeTok{j =} \DecValTok{2}\SpecialCharTok{:}\DecValTok{3}\NormalTok{, }\AttributeTok{align =} \StringTok{"center"}\NormalTok{, }\AttributeTok{part =} \StringTok{"all"}\NormalTok{) }\SpecialCharTok{\%\textgreater{}\%} \CommentTok{\# Columnas 2 y 3 (Datos) CENTRADAS}
\NormalTok{  flextable}\SpecialCharTok{::}\FunctionTok{align}\NormalTok{(}\AttributeTok{align =} \StringTok{"center"}\NormalTok{, }\AttributeTok{part =} \StringTok{"header"}\NormalTok{) }\SpecialCharTok{\%\textgreater{}\%}        \CommentTok{\# Encabezados CENTRADOS}
  
  \CommentTok{\# Añadir una línea superior a la fila "Total" para separarla}
\NormalTok{  flextable}\SpecialCharTok{::}\FunctionTok{hline}\NormalTok{(}\AttributeTok{i =} \FunctionTok{nrow}\NormalTok{(df\_tabla\_final) }\SpecialCharTok{{-}} \DecValTok{1}\NormalTok{, }\AttributeTok{border =}\NormalTok{ officer}\SpecialCharTok{::}\FunctionTok{fp\_border}\NormalTok{(}\AttributeTok{width =} \FloatTok{1.5}\NormalTok{, }\AttributeTok{color =} \StringTok{"black"}\NormalTok{)) }\SpecialCharTok{\%\textgreater{}\%}
  
  \CommentTok{\# Ajustar el ancho de las columnas}
\NormalTok{  flextable}\SpecialCharTok{::}\FunctionTok{autofit}\NormalTok{()}

\CommentTok{\# Mostrar la tabla}
\NormalTok{ft}
\end{Highlighting}
\end{Shaded}

\global\setlength{\Oldarrayrulewidth}{\arrayrulewidth}

\global\setlength{\Oldtabcolsep}{\tabcolsep}

\setlength{\tabcolsep}{2pt}

\renewcommand*{\arraystretch}{1.5}



\providecommand{\ascline}[3]{\noalign{\global\arrayrulewidth #1}\arrayrulecolor[HTML]{#2}\cline{#3}}

\begin{longtable}[c]{|p{2.69in}|p{1.27in}|p{1.02in}}

\caption{Tabla\ 15.\ Distribución\ de\ frecuencias\ para\ Frecuencia\ de\ Comer\ Fuera\ de\ Casa}\\

\ascline{1.5pt}{666666}{1-3}

\multicolumn{1}{>{\centering}m{\dimexpr 2.69in+0\tabcolsep}}{\textcolor[HTML]{000000}{\fontsize{11}{11}\selectfont{Frecuencia\ de\ comer\ fuera\ de\ casa}}} & \multicolumn{1}{>{\centering}m{\dimexpr 1.27in+0\tabcolsep}}{\textcolor[HTML]{000000}{\fontsize{11}{11}\selectfont{Frecuencia\ (n)}}} & \multicolumn{1}{>{\centering}m{\dimexpr 1.02in+0\tabcolsep}}{\textcolor[HTML]{000000}{\fontsize{11}{11}\selectfont{Porcentaje}}} \\

\ascline{1.5pt}{666666}{1-3}\endfirsthead \caption[]{Tabla\ 15.\ Distribución\ de\ frecuencias\ para\ Frecuencia\ de\ Comer\ Fuera\ de\ Casa}\\

\ascline{1.5pt}{666666}{1-3}

\multicolumn{1}{>{\centering}m{\dimexpr 2.69in+0\tabcolsep}}{\textcolor[HTML]{000000}{\fontsize{11}{11}\selectfont{Frecuencia\ de\ comer\ fuera\ de\ casa}}} & \multicolumn{1}{>{\centering}m{\dimexpr 1.27in+0\tabcolsep}}{\textcolor[HTML]{000000}{\fontsize{11}{11}\selectfont{Frecuencia\ (n)}}} & \multicolumn{1}{>{\centering}m{\dimexpr 1.02in+0\tabcolsep}}{\textcolor[HTML]{000000}{\fontsize{11}{11}\selectfont{Porcentaje}}} \\

\ascline{1.5pt}{666666}{1-3}\endhead



\multicolumn{1}{>{\raggedright}m{\dimexpr 2.69in+0\tabcolsep}}{\textcolor[HTML]{000000}{\fontsize{11}{11}\selectfont{Nunca/Rara}}} & \multicolumn{1}{>{\centering}m{\dimexpr 1.27in+0\tabcolsep}}{\textcolor[HTML]{000000}{\fontsize{11}{11}\selectfont{5}}} & \multicolumn{1}{>{\centering}m{\dimexpr 1.02in+0\tabcolsep}}{\textcolor[HTML]{000000}{\fontsize{11}{11}\selectfont{7.9\%}}} \\





\multicolumn{1}{>{\raggedright}m{\dimexpr 2.69in+0\tabcolsep}}{\textcolor[HTML]{000000}{\fontsize{11}{11}\selectfont{Ocasional}}} & \multicolumn{1}{>{\centering}m{\dimexpr 1.27in+0\tabcolsep}}{\textcolor[HTML]{000000}{\fontsize{11}{11}\selectfont{18}}} & \multicolumn{1}{>{\centering}m{\dimexpr 1.02in+0\tabcolsep}}{\textcolor[HTML]{000000}{\fontsize{11}{11}\selectfont{28.6\%}}} \\





\multicolumn{1}{>{\raggedright}m{\dimexpr 2.69in+0\tabcolsep}}{\textcolor[HTML]{000000}{\fontsize{11}{11}\selectfont{1-2/sem}}} & \multicolumn{1}{>{\centering}m{\dimexpr 1.27in+0\tabcolsep}}{\textcolor[HTML]{000000}{\fontsize{11}{11}\selectfont{22}}} & \multicolumn{1}{>{\centering}m{\dimexpr 1.02in+0\tabcolsep}}{\textcolor[HTML]{000000}{\fontsize{11}{11}\selectfont{34.9\%}}} \\





\multicolumn{1}{>{\raggedright}m{\dimexpr 2.69in+0\tabcolsep}}{\textcolor[HTML]{000000}{\fontsize{11}{11}\selectfont{3-4/sem}}} & \multicolumn{1}{>{\centering}m{\dimexpr 1.27in+0\tabcolsep}}{\textcolor[HTML]{000000}{\fontsize{11}{11}\selectfont{15}}} & \multicolumn{1}{>{\centering}m{\dimexpr 1.02in+0\tabcolsep}}{\textcolor[HTML]{000000}{\fontsize{11}{11}\selectfont{23.8\%}}} \\





\multicolumn{1}{>{\raggedright}m{\dimexpr 2.69in+0\tabcolsep}}{\textcolor[HTML]{000000}{\fontsize{11}{11}\selectfont{5+/sem}}} & \multicolumn{1}{>{\centering}m{\dimexpr 1.27in+0\tabcolsep}}{\textcolor[HTML]{000000}{\fontsize{11}{11}\selectfont{3}}} & \multicolumn{1}{>{\centering}m{\dimexpr 1.02in+0\tabcolsep}}{\textcolor[HTML]{000000}{\fontsize{11}{11}\selectfont{4.8\%}}} \\

\ascline{1.5pt}{000000}{1-3}



\multicolumn{1}{>{\raggedright}m{\dimexpr 2.69in+0\tabcolsep}}{\textcolor[HTML]{000000}{\fontsize{11}{11}\selectfont{\textbf{Total}}}} & \multicolumn{1}{>{\centering}m{\dimexpr 1.27in+0\tabcolsep}}{\textcolor[HTML]{000000}{\fontsize{11}{11}\selectfont{\textbf{63}}}} & \multicolumn{1}{>{\centering}m{\dimexpr 1.02in+0\tabcolsep}}{\textcolor[HTML]{000000}{\fontsize{11}{11}\selectfont{\textbf{100.0\%}}}} \\

\ascline{1.5pt}{666666}{1-3}



\end{longtable}



\arrayrulecolor[HTML]{000000}

\global\setlength{\arrayrulewidth}{\Oldarrayrulewidth}

\global\setlength{\tabcolsep}{\Oldtabcolsep}

\renewcommand*{\arraystretch}{1}

\hypertarget{f_frecuncia.comida.preparada}{%
\paragraph{\texorpdfstring{\texttt{f\_Frecuncia.Comida.Preparada}}{f\_Frecuncia.Comida.Preparada}}\label{f_frecuncia.comida.preparada}}

Las categorías de \texttt{f\_Frecuencia.Comida.Preparada} miden la
frecuencia con la que los encuestados compran comida preparada en un
supermercado. La variable tiene 5 categorías, que se ordenan desde menos
tiempo a más tiempo: Nunca, Rara/Ocasionalmente, 1-2 veces por semana,
3-4 veces por semana y más de 5 veces por semana. Este orden se fijó
mediante el parámetro levels de la instrucción factor().

\begin{Shaded}
\begin{Highlighting}[]
\NormalTok{df\_frecuencia\_tiempo }\OtherTok{\textless{}{-}}\NormalTok{ datos }\SpecialCharTok{\%\textgreater{}\%}
  \CommentTok{\# Contar la frecuencia de cada nivel (respeta el orden del factor f\_Frecuencia.Comida.Preparada)}
\NormalTok{  dplyr}\SpecialCharTok{::}\FunctionTok{count}\NormalTok{(f\_Frecuencia.Comida.Preparada, }\AttributeTok{name =} \StringTok{"Frecuencia"}\NormalTok{) }\SpecialCharTok{\%\textgreater{}\%}
  
  \CommentTok{\# Calcular porcentajes y porcentaje acumulado}
\NormalTok{  dplyr}\SpecialCharTok{::}\FunctionTok{mutate}\NormalTok{(}
    \AttributeTok{Porcentaje =}\NormalTok{ Frecuencia }\SpecialCharTok{/} \FunctionTok{sum}\NormalTok{(Frecuencia) }\SpecialCharTok{*} \DecValTok{100}\NormalTok{,}
    \AttributeTok{Porcentaje\_Acumulado =} \FunctionTok{cumsum}\NormalTok{(Porcentaje)}
\NormalTok{  ) }\SpecialCharTok{\%\textgreater{}\%}
  
\NormalTok{  dplyr}\SpecialCharTok{::}\FunctionTok{mutate}\NormalTok{(}
    \AttributeTok{Porcentaje =} \FunctionTok{round}\NormalTok{(Porcentaje, }\DecValTok{1}\NormalTok{), }\CommentTok{\# Redondeado a 1 decimal}
    \AttributeTok{Porcentaje\_Acumulado =} \FunctionTok{round}\NormalTok{(Porcentaje\_Acumulado, }\DecValTok{1}\NormalTok{)}
\NormalTok{  )}


\CommentTok{\# Generación del gráfico de barras}
\FunctionTok{ggplot}\NormalTok{(df\_frecuencia\_tiempo, }\FunctionTok{aes}\NormalTok{(}\AttributeTok{x =}\NormalTok{ f\_Frecuencia.Comida.Preparada, }\AttributeTok{y =}\NormalTok{ Frecuencia)) }\SpecialCharTok{+}
  \CommentTok{\# Barras}
  \FunctionTok{geom\_bar}\NormalTok{(}\AttributeTok{stat =} \StringTok{"identity"}\NormalTok{, }\AttributeTok{fill =} \StringTok{"\#1f78b4"}\NormalTok{) }\SpecialCharTok{+}
  
  \CommentTok{\# Etiquetas de Porcentaje sobre las barras}
  \FunctionTok{geom\_text}\NormalTok{(}\FunctionTok{aes}\NormalTok{(}\AttributeTok{label =} \FunctionTok{paste0}\NormalTok{(Porcentaje, }\StringTok{"\%"}\NormalTok{)), }
            \AttributeTok{vjust =} \SpecialCharTok{{-}}\FloatTok{0.5}\NormalTok{, }\CommentTok{\# Ajuste vertical para que quede sobre la barra}
            \AttributeTok{size =} \DecValTok{4}\NormalTok{) }\SpecialCharTok{+}
  
  \CommentTok{\# Títulos y Ejes}
  \FunctionTok{labs}\NormalTok{(}
    \AttributeTok{title =} \StringTok{\textquotesingle{}Figura 16. Frecuencia de Comprar Comida Preparada en Supermercados\textquotesingle{}}\NormalTok{,}
    \AttributeTok{x =} \StringTok{"Frecuencia de Comprar Comida Preparada en Supermercados"}\NormalTok{,}
    \AttributeTok{y =} \StringTok{"Numero de respuestas"}
\NormalTok{  ) }\SpecialCharTok{+}
  
  \CommentTok{\# Corrección del eje Y para evitar el recorte de las etiquetas}
  \FunctionTok{scale\_y\_continuous}\NormalTok{(}\AttributeTok{expand =} \FunctionTok{expansion}\NormalTok{(}\AttributeTok{mult =} \FunctionTok{c}\NormalTok{(}\DecValTok{0}\NormalTok{, }\FloatTok{0.15}\NormalTok{)))}
\end{Highlighting}
\end{Shaded}

\begin{center}\includegraphics{ICO-analisis_files/figure-latex/P5-barra-1} \end{center}

Se observa que la mayoría de encuestados compran ocasionalmante comida
preparada en supermercados (con un 36.5\%), seguido de que los que
directamente no compran o rara vez lo hacen (con un 34.9\%).

\begin{Shaded}
\begin{Highlighting}[]
\CommentTok{\# Preparación de datos}
\NormalTok{df\_tabla\_freq\_prep }\OtherTok{\textless{}{-}}\NormalTok{ datos }\SpecialCharTok{\%\textgreater{}\%}
  \CommentTok{\# Contar la frecuencia de cada nivel}
\NormalTok{  dplyr}\SpecialCharTok{::}\FunctionTok{count}\NormalTok{(f\_Frecuencia.Comida.Preparada, }\AttributeTok{name =} \StringTok{"Frecuencia"}\NormalTok{) }\SpecialCharTok{\%\textgreater{}\%}
  \CommentTok{\# Calcular porcentajes}
\NormalTok{  dplyr}\SpecialCharTok{::}\FunctionTok{mutate}\NormalTok{(}
    \AttributeTok{Porcentaje =}\NormalTok{ Frecuencia }\SpecialCharTok{/} \FunctionTok{sum}\NormalTok{(Frecuencia) }\SpecialCharTok{*} \DecValTok{100}
\NormalTok{  ) }\SpecialCharTok{\%\textgreater{}\%}
  \CommentTok{\# Redondear y formatear los porcentajes}
\NormalTok{  dplyr}\SpecialCharTok{::}\FunctionTok{mutate}\NormalTok{(}
    \AttributeTok{Porcentaje =} \FunctionTok{paste0}\NormalTok{(}\FunctionTok{round}\NormalTok{(Porcentaje, }\DecValTok{1}\NormalTok{), }\StringTok{"\%"}\NormalTok{)}
\NormalTok{  )}

\CommentTok{\# Calcular y añadir la fila total}
\NormalTok{df\_total }\OtherTok{\textless{}{-}} \FunctionTok{data.frame}\NormalTok{(}
  \AttributeTok{f\_Frecuencia.Comida.Preparada =} \StringTok{"Total"}\NormalTok{,}
  \AttributeTok{Frecuencia =} \FunctionTok{sum}\NormalTok{(df\_tabla\_freq\_prep}\SpecialCharTok{$}\NormalTok{Frecuencia),}
  \AttributeTok{Porcentaje =} \StringTok{"100.0\%"}
\NormalTok{)}

\CommentTok{\# Unir el data frame de frecuencias con la fila total}
\NormalTok{df\_tabla\_final }\OtherTok{\textless{}{-}} \FunctionTok{bind\_rows}\NormalTok{(df\_tabla\_freq\_prep, df\_total)}

\CommentTok{\# Generación de la tabla (flextable)}
\NormalTok{ft }\OtherTok{\textless{}{-}}\NormalTok{ flextable}\SpecialCharTok{::}\FunctionTok{flextable}\NormalTok{(df\_tabla\_final) }\SpecialCharTok{\%\textgreater{}\%}

\NormalTok{flextable}\SpecialCharTok{::}\FunctionTok{set\_caption}\NormalTok{(}\AttributeTok{caption =} \StringTok{"Tabla 16. Distribución de frecuencias para Frecuencia de Comprar Comida Preparada en un Supermercado"}\NormalTok{) }\SpecialCharTok{\%\textgreater{}\%}
  
  \CommentTok{\# Renombrar las cabeceras}
\NormalTok{  flextable}\SpecialCharTok{::}\FunctionTok{set\_header\_labels}\NormalTok{(}
    \AttributeTok{f\_Frecuencia.Comida.Preparada =} \StringTok{"Frecuencia de comprar comida preparada en un supermercado"}\NormalTok{,}
    \AttributeTok{Frecuencia =} \StringTok{"Frecuencia (n)"}\NormalTok{,}
    \AttributeTok{Porcentaje =} \StringTok{"Porcentaje"}
\NormalTok{  ) }\SpecialCharTok{\%\textgreater{}\%}
  
  \CommentTok{\# Aplicar negrita y bordes a la fila "Total"}
\NormalTok{  flextable}\SpecialCharTok{::}\FunctionTok{bold}\NormalTok{(}\AttributeTok{i =} \FunctionTok{nrow}\NormalTok{(df\_tabla\_final), }\AttributeTok{part =} \StringTok{"body"}\NormalTok{) }\SpecialCharTok{\%\textgreater{}\%} \CommentTok{\# Negrita a la última fila (Total)}
\NormalTok{  flextable}\SpecialCharTok{::}\FunctionTok{border\_remove}\NormalTok{() }\SpecialCharTok{\%\textgreater{}\%} \CommentTok{\# Quitar bordes predeterminados}
\NormalTok{  flextable}\SpecialCharTok{::}\FunctionTok{theme\_booktabs}\NormalTok{() }\SpecialCharTok{\%\textgreater{}\%} \CommentTok{\# Aplicar un tema con líneas horizontales}
  
  \CommentTok{\# Formato de alineación y cabecera}
\NormalTok{  flextable}\SpecialCharTok{::}\FunctionTok{align}\NormalTok{(}\AttributeTok{j =} \DecValTok{1}\NormalTok{, }\AttributeTok{align =} \StringTok{"left"}\NormalTok{, }\AttributeTok{part =} \StringTok{"body"}\NormalTok{) }\SpecialCharTok{\%\textgreater{}\%}
\NormalTok{  flextable}\SpecialCharTok{::}\FunctionTok{align}\NormalTok{(}\AttributeTok{j =} \DecValTok{2}\SpecialCharTok{:}\DecValTok{3}\NormalTok{, }\AttributeTok{align =} \StringTok{"center"}\NormalTok{, }\AttributeTok{part =} \StringTok{"all"}\NormalTok{) }\SpecialCharTok{\%\textgreater{}\%} \CommentTok{\# Columnas 2 y 3 (Datos) CENTRADAS}
\NormalTok{  flextable}\SpecialCharTok{::}\FunctionTok{align}\NormalTok{(}\AttributeTok{align =} \StringTok{"center"}\NormalTok{, }\AttributeTok{part =} \StringTok{"header"}\NormalTok{) }\SpecialCharTok{\%\textgreater{}\%}        \CommentTok{\# Encabezados CENTRADOS}
  
  \CommentTok{\# Añadir una línea superior a la fila "Total" para separarla}
\NormalTok{  flextable}\SpecialCharTok{::}\FunctionTok{hline}\NormalTok{(}\AttributeTok{i =} \FunctionTok{nrow}\NormalTok{(df\_tabla\_final) }\SpecialCharTok{{-}} \DecValTok{1}\NormalTok{, }\AttributeTok{border =}\NormalTok{ officer}\SpecialCharTok{::}\FunctionTok{fp\_border}\NormalTok{(}\AttributeTok{width =} \FloatTok{1.5}\NormalTok{, }\AttributeTok{color =} \StringTok{"black"}\NormalTok{)) }\SpecialCharTok{\%\textgreater{}\%}
  
  \CommentTok{\# Ajustar el ancho de las columnas}
\NormalTok{  flextable}\SpecialCharTok{::}\FunctionTok{autofit}\NormalTok{()}

\CommentTok{\# Mostrar la tabla}
\NormalTok{ft}
\end{Highlighting}
\end{Shaded}

\global\setlength{\Oldarrayrulewidth}{\arrayrulewidth}

\global\setlength{\Oldtabcolsep}{\tabcolsep}

\setlength{\tabcolsep}{2pt}

\renewcommand*{\arraystretch}{1.5}



\providecommand{\ascline}[3]{\noalign{\global\arrayrulewidth #1}\arrayrulecolor[HTML]{#2}\cline{#3}}

\begin{longtable}[c]{|p{4.57in}|p{1.27in}|p{1.02in}}

\caption{Tabla\ 16.\ Distribución\ de\ frecuencias\ para\ Frecuencia\ de\ Comprar\ Comida\ Preparada\ en\ un\ Supermercado}\\

\ascline{1.5pt}{666666}{1-3}

\multicolumn{1}{>{\centering}m{\dimexpr 4.57in+0\tabcolsep}}{\textcolor[HTML]{000000}{\fontsize{11}{11}\selectfont{Frecuencia\ de\ comprar\ comida\ preparada\ en\ un\ supermercado}}} & \multicolumn{1}{>{\centering}m{\dimexpr 1.27in+0\tabcolsep}}{\textcolor[HTML]{000000}{\fontsize{11}{11}\selectfont{Frecuencia\ (n)}}} & \multicolumn{1}{>{\centering}m{\dimexpr 1.02in+0\tabcolsep}}{\textcolor[HTML]{000000}{\fontsize{11}{11}\selectfont{Porcentaje}}} \\

\ascline{1.5pt}{666666}{1-3}\endfirsthead \caption[]{Tabla\ 16.\ Distribución\ de\ frecuencias\ para\ Frecuencia\ de\ Comprar\ Comida\ Preparada\ en\ un\ Supermercado}\\

\ascline{1.5pt}{666666}{1-3}

\multicolumn{1}{>{\centering}m{\dimexpr 4.57in+0\tabcolsep}}{\textcolor[HTML]{000000}{\fontsize{11}{11}\selectfont{Frecuencia\ de\ comprar\ comida\ preparada\ en\ un\ supermercado}}} & \multicolumn{1}{>{\centering}m{\dimexpr 1.27in+0\tabcolsep}}{\textcolor[HTML]{000000}{\fontsize{11}{11}\selectfont{Frecuencia\ (n)}}} & \multicolumn{1}{>{\centering}m{\dimexpr 1.02in+0\tabcolsep}}{\textcolor[HTML]{000000}{\fontsize{11}{11}\selectfont{Porcentaje}}} \\

\ascline{1.5pt}{666666}{1-3}\endhead



\multicolumn{1}{>{\raggedright}m{\dimexpr 4.57in+0\tabcolsep}}{\textcolor[HTML]{000000}{\fontsize{11}{11}\selectfont{Nunca/Rara}}} & \multicolumn{1}{>{\centering}m{\dimexpr 1.27in+0\tabcolsep}}{\textcolor[HTML]{000000}{\fontsize{11}{11}\selectfont{22}}} & \multicolumn{1}{>{\centering}m{\dimexpr 1.02in+0\tabcolsep}}{\textcolor[HTML]{000000}{\fontsize{11}{11}\selectfont{34.9\%}}} \\





\multicolumn{1}{>{\raggedright}m{\dimexpr 4.57in+0\tabcolsep}}{\textcolor[HTML]{000000}{\fontsize{11}{11}\selectfont{Ocasional}}} & \multicolumn{1}{>{\centering}m{\dimexpr 1.27in+0\tabcolsep}}{\textcolor[HTML]{000000}{\fontsize{11}{11}\selectfont{23}}} & \multicolumn{1}{>{\centering}m{\dimexpr 1.02in+0\tabcolsep}}{\textcolor[HTML]{000000}{\fontsize{11}{11}\selectfont{36.5\%}}} \\





\multicolumn{1}{>{\raggedright}m{\dimexpr 4.57in+0\tabcolsep}}{\textcolor[HTML]{000000}{\fontsize{11}{11}\selectfont{1-2/sem}}} & \multicolumn{1}{>{\centering}m{\dimexpr 1.27in+0\tabcolsep}}{\textcolor[HTML]{000000}{\fontsize{11}{11}\selectfont{11}}} & \multicolumn{1}{>{\centering}m{\dimexpr 1.02in+0\tabcolsep}}{\textcolor[HTML]{000000}{\fontsize{11}{11}\selectfont{17.5\%}}} \\





\multicolumn{1}{>{\raggedright}m{\dimexpr 4.57in+0\tabcolsep}}{\textcolor[HTML]{000000}{\fontsize{11}{11}\selectfont{3-4/sem}}} & \multicolumn{1}{>{\centering}m{\dimexpr 1.27in+0\tabcolsep}}{\textcolor[HTML]{000000}{\fontsize{11}{11}\selectfont{6}}} & \multicolumn{1}{>{\centering}m{\dimexpr 1.02in+0\tabcolsep}}{\textcolor[HTML]{000000}{\fontsize{11}{11}\selectfont{9.5\%}}} \\





\multicolumn{1}{>{\raggedright}m{\dimexpr 4.57in+0\tabcolsep}}{\textcolor[HTML]{000000}{\fontsize{11}{11}\selectfont{5+/sem}}} & \multicolumn{1}{>{\centering}m{\dimexpr 1.27in+0\tabcolsep}}{\textcolor[HTML]{000000}{\fontsize{11}{11}\selectfont{1}}} & \multicolumn{1}{>{\centering}m{\dimexpr 1.02in+0\tabcolsep}}{\textcolor[HTML]{000000}{\fontsize{11}{11}\selectfont{1.6\%}}} \\

\ascline{1.5pt}{000000}{1-3}



\multicolumn{1}{>{\raggedright}m{\dimexpr 4.57in+0\tabcolsep}}{\textcolor[HTML]{000000}{\fontsize{11}{11}\selectfont{\textbf{Total}}}} & \multicolumn{1}{>{\centering}m{\dimexpr 1.27in+0\tabcolsep}}{\textcolor[HTML]{000000}{\fontsize{11}{11}\selectfont{\textbf{63}}}} & \multicolumn{1}{>{\centering}m{\dimexpr 1.02in+0\tabcolsep}}{\textcolor[HTML]{000000}{\fontsize{11}{11}\selectfont{\textbf{100.0\%}}}} \\

\ascline{1.5pt}{666666}{1-3}



\end{longtable}



\arrayrulecolor[HTML]{000000}

\global\setlength{\arrayrulewidth}{\Oldarrayrulewidth}

\global\setlength{\tabcolsep}{\Oldtabcolsep}

\renewcommand*{\arraystretch}{1}

\#\#\#\#\texttt{f\_Disposicion.Pago} Las categorías de
\texttt{f\_Disposicion.Pago} miden la disposición a pagar de los
encuestados por las ensaladas de la barra. La variable tiene 5
categorías, que se ordenan desde menos dinero a más dinero: Menos de 4€,
Entre 4 y 5,49€, Entre 5,50 y 7€, Entre 7 y 9,50€ y más de 9,50€. Este
orden se fijó mediante el parámetro levels de la instrucción factor().

\begin{Shaded}
\begin{Highlighting}[]
\NormalTok{df\_frecuencia\_tiempo }\OtherTok{\textless{}{-}}\NormalTok{ datos }\SpecialCharTok{\%\textgreater{}\%}
  \CommentTok{\# Contar la frecuencia de cada nivel (respeta el orden del factor f\_Disposicion.Pago)}
\NormalTok{  dplyr}\SpecialCharTok{::}\FunctionTok{count}\NormalTok{(f\_Disposicion.Pago, }\AttributeTok{name =} \StringTok{"Frecuencia"}\NormalTok{) }\SpecialCharTok{\%\textgreater{}\%}
  
  \CommentTok{\# Calcular porcentajes y porcentaje acumulado}
\NormalTok{  dplyr}\SpecialCharTok{::}\FunctionTok{mutate}\NormalTok{(}
    \AttributeTok{Porcentaje =}\NormalTok{ Frecuencia }\SpecialCharTok{/} \FunctionTok{sum}\NormalTok{(Frecuencia) }\SpecialCharTok{*} \DecValTok{100}\NormalTok{,}
    \AttributeTok{Porcentaje\_Acumulado =} \FunctionTok{cumsum}\NormalTok{(Porcentaje)}
\NormalTok{  ) }\SpecialCharTok{\%\textgreater{}\%}
  
\NormalTok{  dplyr}\SpecialCharTok{::}\FunctionTok{mutate}\NormalTok{(}
    \AttributeTok{Porcentaje =} \FunctionTok{round}\NormalTok{(Porcentaje, }\DecValTok{1}\NormalTok{), }\CommentTok{\# Redondeado a 1 decimal}
    \AttributeTok{Porcentaje\_Acumulado =} \FunctionTok{round}\NormalTok{(Porcentaje\_Acumulado, }\DecValTok{1}\NormalTok{)}
\NormalTok{  )}


\CommentTok{\# Generación del gráfico de barras}
\FunctionTok{ggplot}\NormalTok{(df\_frecuencia\_tiempo, }\FunctionTok{aes}\NormalTok{(}\AttributeTok{x =}\NormalTok{ f\_Disposicion.Pago, }\AttributeTok{y =}\NormalTok{ Frecuencia)) }\SpecialCharTok{+}
  \CommentTok{\# Barras}
  \FunctionTok{geom\_bar}\NormalTok{(}\AttributeTok{stat =} \StringTok{"identity"}\NormalTok{, }\AttributeTok{fill =} \StringTok{"\#1f78b4"}\NormalTok{) }\SpecialCharTok{+}
  
  \CommentTok{\# Etiquetas de Porcentaje sobre las barras}
  \FunctionTok{geom\_text}\NormalTok{(}\FunctionTok{aes}\NormalTok{(}\AttributeTok{label =} \FunctionTok{paste0}\NormalTok{(Porcentaje, }\StringTok{"\%"}\NormalTok{)), }
            \AttributeTok{vjust =} \SpecialCharTok{{-}}\FloatTok{0.5}\NormalTok{, }\CommentTok{\# Ajuste vertical para que quede sobre la barra}
            \AttributeTok{size =} \DecValTok{4}\NormalTok{) }\SpecialCharTok{+}
  
  \CommentTok{\# Títulos y Ejes}
  \FunctionTok{labs}\NormalTok{(}
    \AttributeTok{title =} \StringTok{\textquotesingle{}Figura 17. Disposición a Pagar\textquotesingle{}}\NormalTok{,}
    \AttributeTok{x =} \StringTok{"Disposición a Pagar"}\NormalTok{,}
    \AttributeTok{y =} \StringTok{"Numero de respuestas"}
\NormalTok{  ) }\SpecialCharTok{+}
  
  \CommentTok{\# Corrección del eje Y para evitar el recorte de las etiquetas}
  \FunctionTok{scale\_y\_continuous}\NormalTok{(}\AttributeTok{expand =} \FunctionTok{expansion}\NormalTok{(}\AttributeTok{mult =} \FunctionTok{c}\NormalTok{(}\DecValTok{0}\NormalTok{, }\FloatTok{0.15}\NormalTok{)))}
\end{Highlighting}
\end{Shaded}

\begin{center}\includegraphics{ICO-analisis_files/figure-latex/P16-barra-1} \end{center}

Se observa que la mayoría de los encuestados estan dispuestos a pagar
entre 4 y 5,50€ (con un 41.3\%), seguido de los que estan dispuestos a
pagar entre 5,50 y 7€ (con un 28.6\%).

\begin{Shaded}
\begin{Highlighting}[]
\CommentTok{\# Preparación de datos}
\NormalTok{df\_tabla\_disp }\OtherTok{\textless{}{-}}\NormalTok{ datos }\SpecialCharTok{\%\textgreater{}\%}
  \CommentTok{\# Contar la frecuencia de cada nivel}
\NormalTok{  dplyr}\SpecialCharTok{::}\FunctionTok{count}\NormalTok{(f\_Disposicion.Pago, }\AttributeTok{name =} \StringTok{"Frecuencia"}\NormalTok{) }\SpecialCharTok{\%\textgreater{}\%}
  \CommentTok{\# Calcular porcentajes}
\NormalTok{  dplyr}\SpecialCharTok{::}\FunctionTok{mutate}\NormalTok{(}
    \AttributeTok{Porcentaje =}\NormalTok{ Frecuencia }\SpecialCharTok{/} \FunctionTok{sum}\NormalTok{(Frecuencia) }\SpecialCharTok{*} \DecValTok{100}
\NormalTok{  ) }\SpecialCharTok{\%\textgreater{}\%}
  \CommentTok{\# Redondear y formatear los porcentajes}
\NormalTok{  dplyr}\SpecialCharTok{::}\FunctionTok{mutate}\NormalTok{(}
    \AttributeTok{Porcentaje =} \FunctionTok{paste0}\NormalTok{(}\FunctionTok{round}\NormalTok{(Porcentaje, }\DecValTok{1}\NormalTok{), }\StringTok{"\%"}\NormalTok{)}
\NormalTok{  )}

\CommentTok{\# Calcular y añadir la fila total}
\NormalTok{df\_total }\OtherTok{\textless{}{-}} \FunctionTok{data.frame}\NormalTok{(}
  \AttributeTok{f\_Disposicion.Pago =} \StringTok{"Total"}\NormalTok{,}
  \AttributeTok{Frecuencia =} \FunctionTok{sum}\NormalTok{(df\_tabla\_disp}\SpecialCharTok{$}\NormalTok{Frecuencia),}
  \AttributeTok{Porcentaje =} \StringTok{"100.0\%"}
\NormalTok{)}

\CommentTok{\# Unir el data frame de frecuencias con la fila total}
\NormalTok{df\_tabla\_final }\OtherTok{\textless{}{-}} \FunctionTok{bind\_rows}\NormalTok{(df\_tabla\_disp, df\_total)}

\CommentTok{\# Generación de la tabla (flextable)}
\NormalTok{ft }\OtherTok{\textless{}{-}}\NormalTok{ flextable}\SpecialCharTok{::}\FunctionTok{flextable}\NormalTok{(df\_tabla\_final) }\SpecialCharTok{\%\textgreater{}\%}

\NormalTok{flextable}\SpecialCharTok{::}\FunctionTok{set\_caption}\NormalTok{(}\AttributeTok{caption =} \StringTok{"Tabla 17. Distribución de frecuencias para la Disposición a Pagar"}\NormalTok{) }\SpecialCharTok{\%\textgreater{}\%}
  
  \CommentTok{\# Renombrar las cabeceras}
\NormalTok{  flextable}\SpecialCharTok{::}\FunctionTok{set\_header\_labels}\NormalTok{(}
    \AttributeTok{f\_Disposicion.Pago =} \StringTok{"Disposición a Pagar"}\NormalTok{,}
    \AttributeTok{Frecuencia =} \StringTok{"Frecuencia (n)"}\NormalTok{,}
    \AttributeTok{Porcentaje =} \StringTok{"Porcentaje"}
\NormalTok{  ) }\SpecialCharTok{\%\textgreater{}\%}
  
  \CommentTok{\# Aplicar negrita y bordes a la fila "Total"}
\NormalTok{  flextable}\SpecialCharTok{::}\FunctionTok{bold}\NormalTok{(}\AttributeTok{i =} \FunctionTok{nrow}\NormalTok{(df\_tabla\_final), }\AttributeTok{part =} \StringTok{"body"}\NormalTok{) }\SpecialCharTok{\%\textgreater{}\%} \CommentTok{\# Negrita a la última fila (Total)}
\NormalTok{  flextable}\SpecialCharTok{::}\FunctionTok{border\_remove}\NormalTok{() }\SpecialCharTok{\%\textgreater{}\%} \CommentTok{\# Quitar bordes predeterminados}
\NormalTok{  flextable}\SpecialCharTok{::}\FunctionTok{theme\_booktabs}\NormalTok{() }\SpecialCharTok{\%\textgreater{}\%} \CommentTok{\# Aplicar un tema con líneas horizontales}
  
  \CommentTok{\# Formato de alineación y cabecera}
\NormalTok{  flextable}\SpecialCharTok{::}\FunctionTok{align}\NormalTok{(}\AttributeTok{j =} \DecValTok{1}\NormalTok{, }\AttributeTok{align =} \StringTok{"left"}\NormalTok{, }\AttributeTok{part =} \StringTok{"body"}\NormalTok{) }\SpecialCharTok{\%\textgreater{}\%}
\NormalTok{  flextable}\SpecialCharTok{::}\FunctionTok{align}\NormalTok{(}\AttributeTok{j =} \DecValTok{2}\SpecialCharTok{:}\DecValTok{3}\NormalTok{, }\AttributeTok{align =} \StringTok{"center"}\NormalTok{, }\AttributeTok{part =} \StringTok{"all"}\NormalTok{) }\SpecialCharTok{\%\textgreater{}\%} \CommentTok{\# Columnas 2 y 3 (Datos) CENTRADAS}
\NormalTok{  flextable}\SpecialCharTok{::}\FunctionTok{align}\NormalTok{(}\AttributeTok{align =} \StringTok{"center"}\NormalTok{, }\AttributeTok{part =} \StringTok{"header"}\NormalTok{) }\SpecialCharTok{\%\textgreater{}\%}        \CommentTok{\# Encabezados CENTRADOS}
  
  \CommentTok{\# Añadir una línea superior a la fila "Total" para separarla}
\NormalTok{  flextable}\SpecialCharTok{::}\FunctionTok{hline}\NormalTok{(}\AttributeTok{i =} \FunctionTok{nrow}\NormalTok{(df\_tabla\_final) }\SpecialCharTok{{-}} \DecValTok{1}\NormalTok{, }\AttributeTok{border =}\NormalTok{ officer}\SpecialCharTok{::}\FunctionTok{fp\_border}\NormalTok{(}\AttributeTok{width =} \FloatTok{1.5}\NormalTok{, }\AttributeTok{color =} \StringTok{"black"}\NormalTok{)) }\SpecialCharTok{\%\textgreater{}\%}
  
  \CommentTok{\# Ajustar el ancho de las columnas}
\NormalTok{  flextable}\SpecialCharTok{::}\FunctionTok{autofit}\NormalTok{()}

\CommentTok{\# Mostrar la tabla}
\NormalTok{ft}
\end{Highlighting}
\end{Shaded}

\global\setlength{\Oldarrayrulewidth}{\arrayrulewidth}

\global\setlength{\Oldtabcolsep}{\tabcolsep}

\setlength{\tabcolsep}{2pt}

\renewcommand*{\arraystretch}{1.5}



\providecommand{\ascline}[3]{\noalign{\global\arrayrulewidth #1}\arrayrulecolor[HTML]{#2}\cline{#3}}

\begin{longtable}[c]{|p{1.64in}|p{1.27in}|p{1.02in}}

\caption{Tabla\ 17.\ Distribución\ de\ frecuencias\ para\ la\ Disposición\ a\ Pagar}\\

\ascline{1.5pt}{666666}{1-3}

\multicolumn{1}{>{\centering}m{\dimexpr 1.64in+0\tabcolsep}}{\textcolor[HTML]{000000}{\fontsize{11}{11}\selectfont{Disposición\ a\ Pagar}}} & \multicolumn{1}{>{\centering}m{\dimexpr 1.27in+0\tabcolsep}}{\textcolor[HTML]{000000}{\fontsize{11}{11}\selectfont{Frecuencia\ (n)}}} & \multicolumn{1}{>{\centering}m{\dimexpr 1.02in+0\tabcolsep}}{\textcolor[HTML]{000000}{\fontsize{11}{11}\selectfont{Porcentaje}}} \\

\ascline{1.5pt}{666666}{1-3}\endfirsthead \caption[]{Tabla\ 17.\ Distribución\ de\ frecuencias\ para\ la\ Disposición\ a\ Pagar}\\

\ascline{1.5pt}{666666}{1-3}

\multicolumn{1}{>{\centering}m{\dimexpr 1.64in+0\tabcolsep}}{\textcolor[HTML]{000000}{\fontsize{11}{11}\selectfont{Disposición\ a\ Pagar}}} & \multicolumn{1}{>{\centering}m{\dimexpr 1.27in+0\tabcolsep}}{\textcolor[HTML]{000000}{\fontsize{11}{11}\selectfont{Frecuencia\ (n)}}} & \multicolumn{1}{>{\centering}m{\dimexpr 1.02in+0\tabcolsep}}{\textcolor[HTML]{000000}{\fontsize{11}{11}\selectfont{Porcentaje}}} \\

\ascline{1.5pt}{666666}{1-3}\endhead



\multicolumn{1}{>{\raggedright}m{\dimexpr 1.64in+0\tabcolsep}}{\textcolor[HTML]{000000}{\fontsize{11}{11}\selectfont{<4€}}} & \multicolumn{1}{>{\centering}m{\dimexpr 1.27in+0\tabcolsep}}{\textcolor[HTML]{000000}{\fontsize{11}{11}\selectfont{15}}} & \multicolumn{1}{>{\centering}m{\dimexpr 1.02in+0\tabcolsep}}{\textcolor[HTML]{000000}{\fontsize{11}{11}\selectfont{23.8\%}}} \\





\multicolumn{1}{>{\raggedright}m{\dimexpr 1.64in+0\tabcolsep}}{\textcolor[HTML]{000000}{\fontsize{11}{11}\selectfont{4-5.5€}}} & \multicolumn{1}{>{\centering}m{\dimexpr 1.27in+0\tabcolsep}}{\textcolor[HTML]{000000}{\fontsize{11}{11}\selectfont{26}}} & \multicolumn{1}{>{\centering}m{\dimexpr 1.02in+0\tabcolsep}}{\textcolor[HTML]{000000}{\fontsize{11}{11}\selectfont{41.3\%}}} \\





\multicolumn{1}{>{\raggedright}m{\dimexpr 1.64in+0\tabcolsep}}{\textcolor[HTML]{000000}{\fontsize{11}{11}\selectfont{5.5-7€}}} & \multicolumn{1}{>{\centering}m{\dimexpr 1.27in+0\tabcolsep}}{\textcolor[HTML]{000000}{\fontsize{11}{11}\selectfont{18}}} & \multicolumn{1}{>{\centering}m{\dimexpr 1.02in+0\tabcolsep}}{\textcolor[HTML]{000000}{\fontsize{11}{11}\selectfont{28.6\%}}} \\





\multicolumn{1}{>{\raggedright}m{\dimexpr 1.64in+0\tabcolsep}}{\textcolor[HTML]{000000}{\fontsize{11}{11}\selectfont{7-9.5€}}} & \multicolumn{1}{>{\centering}m{\dimexpr 1.27in+0\tabcolsep}}{\textcolor[HTML]{000000}{\fontsize{11}{11}\selectfont{3}}} & \multicolumn{1}{>{\centering}m{\dimexpr 1.02in+0\tabcolsep}}{\textcolor[HTML]{000000}{\fontsize{11}{11}\selectfont{4.8\%}}} \\





\multicolumn{1}{>{\raggedright}m{\dimexpr 1.64in+0\tabcolsep}}{\textcolor[HTML]{000000}{\fontsize{11}{11}\selectfont{>9.5€}}} & \multicolumn{1}{>{\centering}m{\dimexpr 1.27in+0\tabcolsep}}{\textcolor[HTML]{000000}{\fontsize{11}{11}\selectfont{1}}} & \multicolumn{1}{>{\centering}m{\dimexpr 1.02in+0\tabcolsep}}{\textcolor[HTML]{000000}{\fontsize{11}{11}\selectfont{1.6\%}}} \\

\ascline{1.5pt}{000000}{1-3}



\multicolumn{1}{>{\raggedright}m{\dimexpr 1.64in+0\tabcolsep}}{\textcolor[HTML]{000000}{\fontsize{11}{11}\selectfont{\textbf{Total}}}} & \multicolumn{1}{>{\centering}m{\dimexpr 1.27in+0\tabcolsep}}{\textcolor[HTML]{000000}{\fontsize{11}{11}\selectfont{\textbf{63}}}} & \multicolumn{1}{>{\centering}m{\dimexpr 1.02in+0\tabcolsep}}{\textcolor[HTML]{000000}{\fontsize{11}{11}\selectfont{\textbf{100.0\%}}}} \\

\ascline{1.5pt}{666666}{1-3}



\end{longtable}



\arrayrulecolor[HTML]{000000}

\global\setlength{\arrayrulewidth}{\Oldarrayrulewidth}

\global\setlength{\tabcolsep}{\Oldtabcolsep}

\renewcommand*{\arraystretch}{1}

\hypertarget{f.edad}{%
\paragraph{\texorpdfstring{\texttt{f.Edad}}{f.Edad}}\label{f.edad}}

Las categorías de \texttt{f\_Edad} miden la edad de los encuestados. La
variable tiene 5 categorías, que se ordenan desde menor a edad a más
edad: Menos de 18, Entre 18-24 años, Entre 25-44 años, Entre 45-64 años
y más de 65 años. Este orden se fijó mediante el parámetro levels de la
instrucción factor().

\begin{Shaded}
\begin{Highlighting}[]
\NormalTok{df\_frecuencia\_tiempo }\OtherTok{\textless{}{-}}\NormalTok{ datos }\SpecialCharTok{\%\textgreater{}\%}
  \CommentTok{\# Contar la frecuencia de cada nivel (respeta el orden del factor f\_Edad)}
\NormalTok{  dplyr}\SpecialCharTok{::}\FunctionTok{count}\NormalTok{(f\_Edad, }\AttributeTok{name =} \StringTok{"Frecuencia"}\NormalTok{) }\SpecialCharTok{\%\textgreater{}\%}
  
  \CommentTok{\# Calcular porcentajes y porcentaje acumulado}
\NormalTok{  dplyr}\SpecialCharTok{::}\FunctionTok{mutate}\NormalTok{(}
    \AttributeTok{Porcentaje =}\NormalTok{ Frecuencia }\SpecialCharTok{/} \FunctionTok{sum}\NormalTok{(Frecuencia) }\SpecialCharTok{*} \DecValTok{100}\NormalTok{,}
    \AttributeTok{Porcentaje\_Acumulado =} \FunctionTok{cumsum}\NormalTok{(Porcentaje)}
\NormalTok{  ) }\SpecialCharTok{\%\textgreater{}\%}
  
\NormalTok{  dplyr}\SpecialCharTok{::}\FunctionTok{mutate}\NormalTok{(}
    \AttributeTok{Porcentaje =} \FunctionTok{round}\NormalTok{(Porcentaje, }\DecValTok{1}\NormalTok{), }\CommentTok{\# Redondeado a 1 decimal}
    \AttributeTok{Porcentaje\_Acumulado =} \FunctionTok{round}\NormalTok{(Porcentaje\_Acumulado, }\DecValTok{1}\NormalTok{)}
\NormalTok{  )}


\CommentTok{\# Generación del gráfico de barras}
\FunctionTok{ggplot}\NormalTok{(df\_frecuencia\_tiempo, }\FunctionTok{aes}\NormalTok{(}\AttributeTok{x =}\NormalTok{ f\_Edad, }\AttributeTok{y =}\NormalTok{ Frecuencia)) }\SpecialCharTok{+}
  \CommentTok{\# Barras}
  \FunctionTok{geom\_bar}\NormalTok{(}\AttributeTok{stat =} \StringTok{"identity"}\NormalTok{, }\AttributeTok{fill =} \StringTok{"\#1f78b4"}\NormalTok{) }\SpecialCharTok{+}
  
  \CommentTok{\# Etiquetas de Porcentaje sobre las barras}
  \FunctionTok{geom\_text}\NormalTok{(}\FunctionTok{aes}\NormalTok{(}\AttributeTok{label =} \FunctionTok{paste0}\NormalTok{(Porcentaje, }\StringTok{"\%"}\NormalTok{)), }
            \AttributeTok{vjust =} \SpecialCharTok{{-}}\FloatTok{0.5}\NormalTok{, }\CommentTok{\# Ajuste vertical para que quede sobre la barra}
            \AttributeTok{size =} \DecValTok{4}\NormalTok{) }\SpecialCharTok{+}
  
  \CommentTok{\# Títulos y Ejes}
  \FunctionTok{labs}\NormalTok{(}
    \AttributeTok{title =} \StringTok{\textquotesingle{}Figura 5. Edad\textquotesingle{}}\NormalTok{,}
    \AttributeTok{x =} \StringTok{"Edad"}\NormalTok{,}
    \AttributeTok{y =} \StringTok{"Numero de respuestas"}
\NormalTok{  ) }\SpecialCharTok{+}
  
  \CommentTok{\# Corrección del eje Y para evitar el recorte de las etiquetas}
  \FunctionTok{scale\_y\_continuous}\NormalTok{(}\AttributeTok{expand =} \FunctionTok{expansion}\NormalTok{(}\AttributeTok{mult =} \FunctionTok{c}\NormalTok{(}\DecValTok{0}\NormalTok{, }\FloatTok{0.15}\NormalTok{)))}
\end{Highlighting}
\end{Shaded}

\begin{center}\includegraphics{ICO-analisis_files/figure-latex/P20-barra-1} \end{center}

Se observa que la edad de la mayoría de los encuestados se encuentra
entre 18 y 24 años (con un 63.5\%), seguido del rango de edades entre 45
y 64 años (con un 19\%).

\begin{Shaded}
\begin{Highlighting}[]
\CommentTok{\# Preparación de datos}
\NormalTok{df\_tabla\_edad }\OtherTok{\textless{}{-}}\NormalTok{ datos }\SpecialCharTok{\%\textgreater{}\%}
  \CommentTok{\# Contar la frecuencia de cada nivel}
\NormalTok{  dplyr}\SpecialCharTok{::}\FunctionTok{count}\NormalTok{(f\_Edad, }\AttributeTok{name =} \StringTok{"Frecuencia"}\NormalTok{) }\SpecialCharTok{\%\textgreater{}\%}
  \CommentTok{\# Calcular porcentajes}
\NormalTok{  dplyr}\SpecialCharTok{::}\FunctionTok{mutate}\NormalTok{(}
    \AttributeTok{Porcentaje =}\NormalTok{ Frecuencia }\SpecialCharTok{/} \FunctionTok{sum}\NormalTok{(Frecuencia) }\SpecialCharTok{*} \DecValTok{100}
\NormalTok{  ) }\SpecialCharTok{\%\textgreater{}\%}
  \CommentTok{\# Redondear y formatear los porcentajes}
\NormalTok{  dplyr}\SpecialCharTok{::}\FunctionTok{mutate}\NormalTok{(}
    \AttributeTok{Porcentaje =} \FunctionTok{paste0}\NormalTok{(}\FunctionTok{round}\NormalTok{(Porcentaje, }\DecValTok{1}\NormalTok{), }\StringTok{"\%"}\NormalTok{)}
\NormalTok{  )}

\CommentTok{\# Calcular y añadir la fila total}
\NormalTok{df\_total }\OtherTok{\textless{}{-}} \FunctionTok{data.frame}\NormalTok{(}
  \AttributeTok{f\_Edad =} \StringTok{"Total"}\NormalTok{,}
  \AttributeTok{Frecuencia =} \FunctionTok{sum}\NormalTok{(df\_tabla\_edad}\SpecialCharTok{$}\NormalTok{Frecuencia),}
  \AttributeTok{Porcentaje =} \StringTok{"100.0\%"}
\NormalTok{)}

\CommentTok{\# Unir el data frame de frecuencias con la fila total}
\NormalTok{df\_tabla\_final }\OtherTok{\textless{}{-}} \FunctionTok{bind\_rows}\NormalTok{(df\_tabla\_edad, df\_total)}

\CommentTok{\# Generación de la tabla (flextable)}
\NormalTok{ft }\OtherTok{\textless{}{-}}\NormalTok{ flextable}\SpecialCharTok{::}\FunctionTok{flextable}\NormalTok{(df\_tabla\_final) }\SpecialCharTok{\%\textgreater{}\%}

\NormalTok{flextable}\SpecialCharTok{::}\FunctionTok{set\_caption}\NormalTok{(}\AttributeTok{caption =} \StringTok{"Tabla 1. Distribución de frecuencias por Edad"}\NormalTok{) }\SpecialCharTok{\%\textgreater{}\%}
  
  \CommentTok{\# Renombrar las cabeceras}
\NormalTok{  flextable}\SpecialCharTok{::}\FunctionTok{set\_header\_labels}\NormalTok{(}
    \AttributeTok{f\_Edad =} \StringTok{"Edad"}\NormalTok{,}
    \AttributeTok{Frecuencia =} \StringTok{"Frecuencia (n)"}\NormalTok{,}
    \AttributeTok{Porcentaje =} \StringTok{"Porcentaje"}
\NormalTok{  ) }\SpecialCharTok{\%\textgreater{}\%}
  
  \CommentTok{\# Aplicar negrita y bordes a la fila "Total"}
\NormalTok{  flextable}\SpecialCharTok{::}\FunctionTok{bold}\NormalTok{(}\AttributeTok{i =} \FunctionTok{nrow}\NormalTok{(df\_tabla\_final), }\AttributeTok{part =} \StringTok{"body"}\NormalTok{) }\SpecialCharTok{\%\textgreater{}\%} \CommentTok{\# Negrita a la última fila (Total)}
\NormalTok{  flextable}\SpecialCharTok{::}\FunctionTok{border\_remove}\NormalTok{() }\SpecialCharTok{\%\textgreater{}\%} \CommentTok{\# Quitar bordes predeterminados}
\NormalTok{  flextable}\SpecialCharTok{::}\FunctionTok{theme\_booktabs}\NormalTok{() }\SpecialCharTok{\%\textgreater{}\%} \CommentTok{\# Aplicar un tema con líneas horizontales}
  
  \CommentTok{\# Formato de alineación y cabecera}
\NormalTok{  flextable}\SpecialCharTok{::}\FunctionTok{align}\NormalTok{(}\AttributeTok{j =} \DecValTok{1}\NormalTok{, }\AttributeTok{align =} \StringTok{"left"}\NormalTok{, }\AttributeTok{part =} \StringTok{"body"}\NormalTok{) }\SpecialCharTok{\%\textgreater{}\%}
\NormalTok{  flextable}\SpecialCharTok{::}\FunctionTok{align}\NormalTok{(}\AttributeTok{j =} \DecValTok{2}\SpecialCharTok{:}\DecValTok{3}\NormalTok{, }\AttributeTok{align =} \StringTok{"center"}\NormalTok{, }\AttributeTok{part =} \StringTok{"all"}\NormalTok{) }\SpecialCharTok{\%\textgreater{}\%} \CommentTok{\# Columnas 2 y 3 (Datos) CENTRADAS}
\NormalTok{  flextable}\SpecialCharTok{::}\FunctionTok{align}\NormalTok{(}\AttributeTok{align =} \StringTok{"center"}\NormalTok{, }\AttributeTok{part =} \StringTok{"header"}\NormalTok{) }\SpecialCharTok{\%\textgreater{}\%}        \CommentTok{\# Encabezados CENTRADOS}
  
  \CommentTok{\# Añadir una línea superior a la fila "Total" para separarla}
\NormalTok{  flextable}\SpecialCharTok{::}\FunctionTok{hline}\NormalTok{(}\AttributeTok{i =} \FunctionTok{nrow}\NormalTok{(df\_tabla\_final) }\SpecialCharTok{{-}} \DecValTok{1}\NormalTok{, }\AttributeTok{border =}\NormalTok{ officer}\SpecialCharTok{::}\FunctionTok{fp\_border}\NormalTok{(}\AttributeTok{width =} \FloatTok{1.5}\NormalTok{, }\AttributeTok{color =} \StringTok{"black"}\NormalTok{)) }\SpecialCharTok{\%\textgreater{}\%}
  
  \CommentTok{\# Ajustar el ancho de las columnas}
\NormalTok{  flextable}\SpecialCharTok{::}\FunctionTok{autofit}\NormalTok{()}

\CommentTok{\# Mostrar la tabla}
\NormalTok{ft}
\end{Highlighting}
\end{Shaded}

\global\setlength{\Oldarrayrulewidth}{\arrayrulewidth}

\global\setlength{\Oldtabcolsep}{\tabcolsep}

\setlength{\tabcolsep}{2pt}

\renewcommand*{\arraystretch}{1.5}



\providecommand{\ascline}[3]{\noalign{\global\arrayrulewidth #1}\arrayrulecolor[HTML]{#2}\cline{#3}}

\begin{longtable}[c]{|p{1.14in}|p{1.27in}|p{1.02in}}

\caption{Tabla\ 1.\ Distribución\ de\ frecuencias\ por\ Edad}\\

\ascline{1.5pt}{666666}{1-3}

\multicolumn{1}{>{\centering}m{\dimexpr 1.14in+0\tabcolsep}}{\textcolor[HTML]{000000}{\fontsize{11}{11}\selectfont{Edad}}} & \multicolumn{1}{>{\centering}m{\dimexpr 1.27in+0\tabcolsep}}{\textcolor[HTML]{000000}{\fontsize{11}{11}\selectfont{Frecuencia\ (n)}}} & \multicolumn{1}{>{\centering}m{\dimexpr 1.02in+0\tabcolsep}}{\textcolor[HTML]{000000}{\fontsize{11}{11}\selectfont{Porcentaje}}} \\

\ascline{1.5pt}{666666}{1-3}\endfirsthead \caption[]{Tabla\ 1.\ Distribución\ de\ frecuencias\ por\ Edad}\\

\ascline{1.5pt}{666666}{1-3}

\multicolumn{1}{>{\centering}m{\dimexpr 1.14in+0\tabcolsep}}{\textcolor[HTML]{000000}{\fontsize{11}{11}\selectfont{Edad}}} & \multicolumn{1}{>{\centering}m{\dimexpr 1.27in+0\tabcolsep}}{\textcolor[HTML]{000000}{\fontsize{11}{11}\selectfont{Frecuencia\ (n)}}} & \multicolumn{1}{>{\centering}m{\dimexpr 1.02in+0\tabcolsep}}{\textcolor[HTML]{000000}{\fontsize{11}{11}\selectfont{Porcentaje}}} \\

\ascline{1.5pt}{666666}{1-3}\endhead



\multicolumn{1}{>{\raggedright}m{\dimexpr 1.14in+0\tabcolsep}}{\textcolor[HTML]{000000}{\fontsize{11}{11}\selectfont{Menor\ de\ 18}}} & \multicolumn{1}{>{\centering}m{\dimexpr 1.27in+0\tabcolsep}}{\textcolor[HTML]{000000}{\fontsize{11}{11}\selectfont{1}}} & \multicolumn{1}{>{\centering}m{\dimexpr 1.02in+0\tabcolsep}}{\textcolor[HTML]{000000}{\fontsize{11}{11}\selectfont{1.6\%}}} \\





\multicolumn{1}{>{\raggedright}m{\dimexpr 1.14in+0\tabcolsep}}{\textcolor[HTML]{000000}{\fontsize{11}{11}\selectfont{18\ -\ 24}}} & \multicolumn{1}{>{\centering}m{\dimexpr 1.27in+0\tabcolsep}}{\textcolor[HTML]{000000}{\fontsize{11}{11}\selectfont{40}}} & \multicolumn{1}{>{\centering}m{\dimexpr 1.02in+0\tabcolsep}}{\textcolor[HTML]{000000}{\fontsize{11}{11}\selectfont{63.5\%}}} \\





\multicolumn{1}{>{\raggedright}m{\dimexpr 1.14in+0\tabcolsep}}{\textcolor[HTML]{000000}{\fontsize{11}{11}\selectfont{25\ -\ 44}}} & \multicolumn{1}{>{\centering}m{\dimexpr 1.27in+0\tabcolsep}}{\textcolor[HTML]{000000}{\fontsize{11}{11}\selectfont{7}}} & \multicolumn{1}{>{\centering}m{\dimexpr 1.02in+0\tabcolsep}}{\textcolor[HTML]{000000}{\fontsize{11}{11}\selectfont{11.1\%}}} \\





\multicolumn{1}{>{\raggedright}m{\dimexpr 1.14in+0\tabcolsep}}{\textcolor[HTML]{000000}{\fontsize{11}{11}\selectfont{45\ -\ 64}}} & \multicolumn{1}{>{\centering}m{\dimexpr 1.27in+0\tabcolsep}}{\textcolor[HTML]{000000}{\fontsize{11}{11}\selectfont{12}}} & \multicolumn{1}{>{\centering}m{\dimexpr 1.02in+0\tabcolsep}}{\textcolor[HTML]{000000}{\fontsize{11}{11}\selectfont{19\%}}} \\





\multicolumn{1}{>{\raggedright}m{\dimexpr 1.14in+0\tabcolsep}}{\textcolor[HTML]{000000}{\fontsize{11}{11}\selectfont{65+}}} & \multicolumn{1}{>{\centering}m{\dimexpr 1.27in+0\tabcolsep}}{\textcolor[HTML]{000000}{\fontsize{11}{11}\selectfont{3}}} & \multicolumn{1}{>{\centering}m{\dimexpr 1.02in+0\tabcolsep}}{\textcolor[HTML]{000000}{\fontsize{11}{11}\selectfont{4.8\%}}} \\

\ascline{1.5pt}{000000}{1-3}



\multicolumn{1}{>{\raggedright}m{\dimexpr 1.14in+0\tabcolsep}}{\textcolor[HTML]{000000}{\fontsize{11}{11}\selectfont{\textbf{Total}}}} & \multicolumn{1}{>{\centering}m{\dimexpr 1.27in+0\tabcolsep}}{\textcolor[HTML]{000000}{\fontsize{11}{11}\selectfont{\textbf{63}}}} & \multicolumn{1}{>{\centering}m{\dimexpr 1.02in+0\tabcolsep}}{\textcolor[HTML]{000000}{\fontsize{11}{11}\selectfont{\textbf{100.0\%}}}} \\

\ascline{1.5pt}{666666}{1-3}



\end{longtable}



\arrayrulecolor[HTML]{000000}

\global\setlength{\arrayrulewidth}{\Oldarrayrulewidth}

\global\setlength{\tabcolsep}{\Oldtabcolsep}

\renewcommand*{\arraystretch}{1}

\hypertarget{f_nivel.educativo}{%
\paragraph{\texorpdfstring{\texttt{f\_Nivel.Educativo}}{f\_Nivel.Educativo}}\label{f_nivel.educativo}}

Las categorías de \texttt{f\_Nivel.Educativo} miden el nivel educativo
máximo alcanzado de los encuestados. La variable tiene 4 categorías, que
se ordenan desde menos nivel a más nivel: ESO, FP medio/Bachillerato, FP
superior y Universitario. Este orden se fijó mediante el parámetro
levels de la instrucción factor().

\begin{Shaded}
\begin{Highlighting}[]
\NormalTok{df\_frecuencia\_tiempo }\OtherTok{\textless{}{-}}\NormalTok{ datos }\SpecialCharTok{\%\textgreater{}\%}
  \CommentTok{\# Contar la frecuencia de cada nivel (respeta el orden del factor f\_Nivel.Educativo)}
\NormalTok{  dplyr}\SpecialCharTok{::}\FunctionTok{count}\NormalTok{(f\_Nivel.Educativo, }\AttributeTok{name =} \StringTok{"Frecuencia"}\NormalTok{) }\SpecialCharTok{\%\textgreater{}\%}
  
  \CommentTok{\# Calcular porcentajes y porcentaje acumulado}
\NormalTok{  dplyr}\SpecialCharTok{::}\FunctionTok{mutate}\NormalTok{(}
    \AttributeTok{Porcentaje =}\NormalTok{ Frecuencia }\SpecialCharTok{/} \FunctionTok{sum}\NormalTok{(Frecuencia) }\SpecialCharTok{*} \DecValTok{100}\NormalTok{,}
    \AttributeTok{Porcentaje\_Acumulado =} \FunctionTok{cumsum}\NormalTok{(Porcentaje)}
\NormalTok{  ) }\SpecialCharTok{\%\textgreater{}\%}
  
\NormalTok{  dplyr}\SpecialCharTok{::}\FunctionTok{mutate}\NormalTok{(}
    \AttributeTok{Porcentaje =} \FunctionTok{round}\NormalTok{(Porcentaje, }\DecValTok{1}\NormalTok{), }\CommentTok{\# Redondeado a 1 decimal}
    \AttributeTok{Porcentaje\_Acumulado =} \FunctionTok{round}\NormalTok{(Porcentaje\_Acumulado, }\DecValTok{1}\NormalTok{)}
\NormalTok{  )}


\CommentTok{\# Generación del gráfico de barras}
\FunctionTok{ggplot}\NormalTok{(df\_frecuencia\_tiempo, }\FunctionTok{aes}\NormalTok{(}\AttributeTok{x =}\NormalTok{ f\_Nivel.Educativo, }\AttributeTok{y =}\NormalTok{ Frecuencia)) }\SpecialCharTok{+}
  \CommentTok{\# Barras}
  \FunctionTok{geom\_bar}\NormalTok{(}\AttributeTok{stat =} \StringTok{"identity"}\NormalTok{, }\AttributeTok{fill =} \StringTok{"\#1f78b4"}\NormalTok{) }\SpecialCharTok{+}
  
  \CommentTok{\# Etiquetas de Porcentaje sobre las barras}
  \FunctionTok{geom\_text}\NormalTok{(}\FunctionTok{aes}\NormalTok{(}\AttributeTok{label =} \FunctionTok{paste0}\NormalTok{(Porcentaje, }\StringTok{"\%"}\NormalTok{)), }
            \AttributeTok{vjust =} \SpecialCharTok{{-}}\FloatTok{0.5}\NormalTok{, }\CommentTok{\# Ajuste vertical para que quede sobre la barra}
            \AttributeTok{size =} \DecValTok{4}\NormalTok{) }\SpecialCharTok{+}
  
  \CommentTok{\# Títulos y Ejes}
  \FunctionTok{labs}\NormalTok{(}
    \AttributeTok{title =} \StringTok{\textquotesingle{}Figura 7. Nivel Educativo Máximo Alcanzado\textquotesingle{}}\NormalTok{,}
    \AttributeTok{x =} \StringTok{"Nivel Educativo"}\NormalTok{,}
    \AttributeTok{y =} \StringTok{"Numero de respuestas"}
\NormalTok{  ) }\SpecialCharTok{+}
  
  \CommentTok{\# Corrección del eje Y para evitar el recorte de las etiquetas}
  \FunctionTok{scale\_y\_continuous}\NormalTok{(}\AttributeTok{expand =} \FunctionTok{expansion}\NormalTok{(}\AttributeTok{mult =} \FunctionTok{c}\NormalTok{(}\DecValTok{0}\NormalTok{, }\FloatTok{0.15}\NormalTok{)))}
\end{Highlighting}
\end{Shaded}

\begin{center}\includegraphics{ICO-analisis_files/figure-latex/P22-barra-1} \end{center}

Se observa que el nivel educativo alcanzado por la mayoría de
encuestados es el universitario (con un 61.9\%), seguido de encuestados
con nivel máximo alcanzado de bachillerato o FP media (con un 20.6\%).

\begin{Shaded}
\begin{Highlighting}[]
\CommentTok{\# Preparación de datos}
\NormalTok{df\_tabla\_edu }\OtherTok{\textless{}{-}}\NormalTok{ datos }\SpecialCharTok{\%\textgreater{}\%}
  \CommentTok{\# Contar la frecuencia de cada nivel}
\NormalTok{  dplyr}\SpecialCharTok{::}\FunctionTok{count}\NormalTok{(f\_Nivel.Educativo, }\AttributeTok{name =} \StringTok{"Frecuencia"}\NormalTok{) }\SpecialCharTok{\%\textgreater{}\%}
  \CommentTok{\# Calcular porcentajes}
\NormalTok{  dplyr}\SpecialCharTok{::}\FunctionTok{mutate}\NormalTok{(}
    \AttributeTok{Porcentaje =}\NormalTok{ Frecuencia }\SpecialCharTok{/} \FunctionTok{sum}\NormalTok{(Frecuencia) }\SpecialCharTok{*} \DecValTok{100}
\NormalTok{  ) }\SpecialCharTok{\%\textgreater{}\%}
  \CommentTok{\# Redondear y formatear los porcentajes}
\NormalTok{  dplyr}\SpecialCharTok{::}\FunctionTok{mutate}\NormalTok{(}
    \AttributeTok{Porcentaje =} \FunctionTok{paste0}\NormalTok{(}\FunctionTok{round}\NormalTok{(Porcentaje, }\DecValTok{1}\NormalTok{), }\StringTok{"\%"}\NormalTok{)}
\NormalTok{  )}

\CommentTok{\# Calcular y añadir la fila total}
\NormalTok{df\_total }\OtherTok{\textless{}{-}} \FunctionTok{data.frame}\NormalTok{(}
  \AttributeTok{f\_Nivel.Educativo =} \StringTok{"Total"}\NormalTok{,}
  \AttributeTok{Frecuencia =} \FunctionTok{sum}\NormalTok{(df\_tabla\_edu}\SpecialCharTok{$}\NormalTok{Frecuencia),}
  \AttributeTok{Porcentaje =} \StringTok{"100.0\%"}
\NormalTok{)}

\CommentTok{\# Unir el data frame de frecuencias con la fila total}
\NormalTok{df\_tabla\_final }\OtherTok{\textless{}{-}} \FunctionTok{bind\_rows}\NormalTok{(df\_tabla\_edu, df\_total)}

\CommentTok{\# Generación de la tabla (flextable)}
\NormalTok{ft }\OtherTok{\textless{}{-}}\NormalTok{ flextable}\SpecialCharTok{::}\FunctionTok{flextable}\NormalTok{(df\_tabla\_final) }\SpecialCharTok{\%\textgreater{}\%}

\NormalTok{flextable}\SpecialCharTok{::}\FunctionTok{set\_caption}\NormalTok{(}\AttributeTok{caption =} \StringTok{"Tabla 3. Distribución de frecuencias para Nivel Educativo Máximo Alcanzado"}\NormalTok{) }\SpecialCharTok{\%\textgreater{}\%}
  
  \CommentTok{\# Renombrar las cabeceras}
\NormalTok{  flextable}\SpecialCharTok{::}\FunctionTok{set\_header\_labels}\NormalTok{(}
    \AttributeTok{f\_Nivel.Educativo =} \StringTok{"Nivel Educativo Máximo Alcanzado"}\NormalTok{,}
    \AttributeTok{Frecuencia =} \StringTok{"Frecuencia (n)"}\NormalTok{,}
    \AttributeTok{Porcentaje =} \StringTok{"Porcentaje"}
\NormalTok{  ) }\SpecialCharTok{\%\textgreater{}\%}
  
  \CommentTok{\# Aplicar negrita y bordes a la fila "Total"}
\NormalTok{  flextable}\SpecialCharTok{::}\FunctionTok{bold}\NormalTok{(}\AttributeTok{i =} \FunctionTok{nrow}\NormalTok{(df\_tabla\_final), }\AttributeTok{part =} \StringTok{"body"}\NormalTok{) }\SpecialCharTok{\%\textgreater{}\%} \CommentTok{\# Negrita a la última fila (Total)}
\NormalTok{  flextable}\SpecialCharTok{::}\FunctionTok{border\_remove}\NormalTok{() }\SpecialCharTok{\%\textgreater{}\%} \CommentTok{\# Quitar bordes predeterminados}
\NormalTok{  flextable}\SpecialCharTok{::}\FunctionTok{theme\_booktabs}\NormalTok{() }\SpecialCharTok{\%\textgreater{}\%} \CommentTok{\# Aplicar un tema con líneas horizontales}
  
  \CommentTok{\# Formato de alineación y cabecera}
\NormalTok{  flextable}\SpecialCharTok{::}\FunctionTok{align}\NormalTok{(}\AttributeTok{j =} \DecValTok{1}\NormalTok{, }\AttributeTok{align =} \StringTok{"left"}\NormalTok{, }\AttributeTok{part =} \StringTok{"body"}\NormalTok{) }\SpecialCharTok{\%\textgreater{}\%}
\NormalTok{  flextable}\SpecialCharTok{::}\FunctionTok{align}\NormalTok{(}\AttributeTok{j =} \DecValTok{2}\SpecialCharTok{:}\DecValTok{3}\NormalTok{, }\AttributeTok{align =} \StringTok{"center"}\NormalTok{, }\AttributeTok{part =} \StringTok{"all"}\NormalTok{) }\SpecialCharTok{\%\textgreater{}\%} \CommentTok{\# Columnas 2 y 3 (Datos) CENTRADAS}
\NormalTok{  flextable}\SpecialCharTok{::}\FunctionTok{align}\NormalTok{(}\AttributeTok{align =} \StringTok{"center"}\NormalTok{, }\AttributeTok{part =} \StringTok{"header"}\NormalTok{) }\SpecialCharTok{\%\textgreater{}\%}        \CommentTok{\# Encabezados CENTRADOS}
  
  \CommentTok{\# Añadir una línea superior a la fila "Total" para separarla}
\NormalTok{  flextable}\SpecialCharTok{::}\FunctionTok{hline}\NormalTok{(}\AttributeTok{i =} \FunctionTok{nrow}\NormalTok{(df\_tabla\_final) }\SpecialCharTok{{-}} \DecValTok{1}\NormalTok{, }\AttributeTok{border =}\NormalTok{ officer}\SpecialCharTok{::}\FunctionTok{fp\_border}\NormalTok{(}\AttributeTok{width =} \FloatTok{1.5}\NormalTok{, }\AttributeTok{color =} \StringTok{"black"}\NormalTok{)) }\SpecialCharTok{\%\textgreater{}\%}
  
  \CommentTok{\# Ajustar el ancho de las columnas}
\NormalTok{  flextable}\SpecialCharTok{::}\FunctionTok{autofit}\NormalTok{()}

\CommentTok{\# Mostrar la tabla}
\NormalTok{ft}
\end{Highlighting}
\end{Shaded}

\global\setlength{\Oldarrayrulewidth}{\arrayrulewidth}

\global\setlength{\Oldtabcolsep}{\tabcolsep}

\setlength{\tabcolsep}{2pt}

\renewcommand*{\arraystretch}{1.5}



\providecommand{\ascline}[3]{\noalign{\global\arrayrulewidth #1}\arrayrulecolor[HTML]{#2}\cline{#3}}

\begin{longtable}[c]{|p{2.67in}|p{1.27in}|p{1.02in}}

\caption{Tabla\ 3.\ Distribución\ de\ frecuencias\ para\ Nivel\ Educativo\ Máximo\ Alcanzado}\\

\ascline{1.5pt}{666666}{1-3}

\multicolumn{1}{>{\centering}m{\dimexpr 2.67in+0\tabcolsep}}{\textcolor[HTML]{000000}{\fontsize{11}{11}\selectfont{Nivel\ Educativo\ Máximo\ Alcanzado}}} & \multicolumn{1}{>{\centering}m{\dimexpr 1.27in+0\tabcolsep}}{\textcolor[HTML]{000000}{\fontsize{11}{11}\selectfont{Frecuencia\ (n)}}} & \multicolumn{1}{>{\centering}m{\dimexpr 1.02in+0\tabcolsep}}{\textcolor[HTML]{000000}{\fontsize{11}{11}\selectfont{Porcentaje}}} \\

\ascline{1.5pt}{666666}{1-3}\endfirsthead \caption[]{Tabla\ 3.\ Distribución\ de\ frecuencias\ para\ Nivel\ Educativo\ Máximo\ Alcanzado}\\

\ascline{1.5pt}{666666}{1-3}

\multicolumn{1}{>{\centering}m{\dimexpr 2.67in+0\tabcolsep}}{\textcolor[HTML]{000000}{\fontsize{11}{11}\selectfont{Nivel\ Educativo\ Máximo\ Alcanzado}}} & \multicolumn{1}{>{\centering}m{\dimexpr 1.27in+0\tabcolsep}}{\textcolor[HTML]{000000}{\fontsize{11}{11}\selectfont{Frecuencia\ (n)}}} & \multicolumn{1}{>{\centering}m{\dimexpr 1.02in+0\tabcolsep}}{\textcolor[HTML]{000000}{\fontsize{11}{11}\selectfont{Porcentaje}}} \\

\ascline{1.5pt}{666666}{1-3}\endhead



\multicolumn{1}{>{\raggedright}m{\dimexpr 2.67in+0\tabcolsep}}{\textcolor[HTML]{000000}{\fontsize{11}{11}\selectfont{ESO}}} & \multicolumn{1}{>{\centering}m{\dimexpr 1.27in+0\tabcolsep}}{\textcolor[HTML]{000000}{\fontsize{11}{11}\selectfont{6}}} & \multicolumn{1}{>{\centering}m{\dimexpr 1.02in+0\tabcolsep}}{\textcolor[HTML]{000000}{\fontsize{11}{11}\selectfont{9.5\%}}} \\





\multicolumn{1}{>{\raggedright}m{\dimexpr 2.67in+0\tabcolsep}}{\textcolor[HTML]{000000}{\fontsize{11}{11}\selectfont{Bachiller/FPm}}} & \multicolumn{1}{>{\centering}m{\dimexpr 1.27in+0\tabcolsep}}{\textcolor[HTML]{000000}{\fontsize{11}{11}\selectfont{13}}} & \multicolumn{1}{>{\centering}m{\dimexpr 1.02in+0\tabcolsep}}{\textcolor[HTML]{000000}{\fontsize{11}{11}\selectfont{20.6\%}}} \\





\multicolumn{1}{>{\raggedright}m{\dimexpr 2.67in+0\tabcolsep}}{\textcolor[HTML]{000000}{\fontsize{11}{11}\selectfont{FPs}}} & \multicolumn{1}{>{\centering}m{\dimexpr 1.27in+0\tabcolsep}}{\textcolor[HTML]{000000}{\fontsize{11}{11}\selectfont{5}}} & \multicolumn{1}{>{\centering}m{\dimexpr 1.02in+0\tabcolsep}}{\textcolor[HTML]{000000}{\fontsize{11}{11}\selectfont{7.9\%}}} \\





\multicolumn{1}{>{\raggedright}m{\dimexpr 2.67in+0\tabcolsep}}{\textcolor[HTML]{000000}{\fontsize{11}{11}\selectfont{Univ}}} & \multicolumn{1}{>{\centering}m{\dimexpr 1.27in+0\tabcolsep}}{\textcolor[HTML]{000000}{\fontsize{11}{11}\selectfont{39}}} & \multicolumn{1}{>{\centering}m{\dimexpr 1.02in+0\tabcolsep}}{\textcolor[HTML]{000000}{\fontsize{11}{11}\selectfont{61.9\%}}} \\

\ascline{1.5pt}{000000}{1-3}



\multicolumn{1}{>{\raggedright}m{\dimexpr 2.67in+0\tabcolsep}}{\textcolor[HTML]{000000}{\fontsize{11}{11}\selectfont{\textbf{Total}}}} & \multicolumn{1}{>{\centering}m{\dimexpr 1.27in+0\tabcolsep}}{\textcolor[HTML]{000000}{\fontsize{11}{11}\selectfont{\textbf{63}}}} & \multicolumn{1}{>{\centering}m{\dimexpr 1.02in+0\tabcolsep}}{\textcolor[HTML]{000000}{\fontsize{11}{11}\selectfont{\textbf{100.0\%}}}} \\

\ascline{1.5pt}{666666}{1-3}



\end{longtable}



\arrayrulecolor[HTML]{000000}

\global\setlength{\arrayrulewidth}{\Oldarrayrulewidth}

\global\setlength{\tabcolsep}{\Oldtabcolsep}

\renewcommand*{\arraystretch}{1}

\hypertarget{variables-nuxfamericas}{%
\subsection{Variables númericas}\label{variables-nuxfamericas}}

\hypertarget{figuras-de-las-variables-de-escala-likert}{%
\subsubsection{Figuras de las variables de escala
Likert}\label{figuras-de-las-variables-de-escala-likert}}

La representación de las variables con escala Likert resulta más visual
con un diagrama de barras acumulado:

\begin{Shaded}
\begin{Highlighting}[]
\CommentTok{\# 1. Definir los vectores de variables y etiquetas para mayor claridad}

\CommentTok{\# Variables de escala Likert (factores)}
\NormalTok{likert\_factors }\OtherTok{\textless{}{-}} \FunctionTok{c}\NormalTok{(}
  \StringTok{"f\_Importancia.Saludable"}\NormalTok{, }
  \StringTok{"f\_Imp.Sostenibilidad"}\NormalTok{, }\StringTok{"f\_Imp.Origen.Local"}\NormalTok{, }\StringTok{"f\_Imp.Variedad"}\NormalTok{, }
  \StringTok{"f\_Imp.Sabor"}\NormalTok{, }\StringTok{"f\_Imp.Precio"}\NormalTok{, }\StringTok{"f\_Imp.Higiene"}\NormalTok{, }\StringTok{"f\_Imp.Rapidez"}\NormalTok{, }
  \StringTok{"f\_Aptitud.Social"}
\NormalTok{)}

\CommentTok{\# Etiquetas descriptivas correspondientes para el gráfico (Eje Y)}
\NormalTok{likert\_labels }\OtherTok{\textless{}{-}} \FunctionTok{c}\NormalTok{(}
  \StringTok{\textquotesingle{}Me importa comer saludable\textquotesingle{}}\NormalTok{,}
  \StringTok{\textquotesingle{}Doy importancia a la sostenibilidad\textquotesingle{}}\NormalTok{,}
  \StringTok{\textquotesingle{}Doy importancia a productos de origen local\textquotesingle{}}\NormalTok{,}
  \StringTok{\textquotesingle{}Doy importancia a la variedad de ingredientes\textquotesingle{}}\NormalTok{,}
  \StringTok{\textquotesingle{}Doy importancia al sabor\textquotesingle{}}\NormalTok{,}
  \StringTok{\textquotesingle{}Doy importancia al precio\textquotesingle{}}\NormalTok{,}
  \StringTok{\textquotesingle{}Doy importancia a la higiene\textquotesingle{}}\NormalTok{,}
  \StringTok{\textquotesingle{}Doy importancia a la rapidez\textquotesingle{}}\NormalTok{,}
  \StringTok{\textquotesingle{}Considero las ensaladas una comida social\textquotesingle{}}
\NormalTok{)}


\CommentTok{\# 2. SELECCIONAR DATOS (Sustitución de variables)}
\NormalTok{datos.likert }\OtherTok{\textless{}{-}}\NormalTok{ datos[likert\_factors]}

\CommentTok{\# 3. CAMBIAR EL NOMBRE DE LAS COLUMNAS para que aparezcan las afirmaciones evaluadas}
\FunctionTok{names}\NormalTok{(datos.likert) }\OtherTok{\textless{}{-}}\NormalTok{ likert\_labels}

\CommentTok{\# 4. GENERAR LA FIGURA}
\CommentTok{\# Se mantiene \textquotesingle{}likert.bar.plot\textquotesingle{} y \textquotesingle{}centered = FALSE\textquotesingle{} para replicar el formato del profesor}
\NormalTok{fig.likert }\OtherTok{\textless{}{-}}\NormalTok{ likert}\SpecialCharTok{::}\FunctionTok{likert.bar.plot}\NormalTok{(}
\NormalTok{  likert}\SpecialCharTok{::}\FunctionTok{likert}\NormalTok{(datos.likert),}
  
  \CommentTok{\# Orden de los ítems (se mantiene el orden de las etiquetas)}
  \AttributeTok{group.order =}\NormalTok{ likert\_labels,      }
  
  \CommentTok{\# Configuración para hacer un gráfico de barras apiladas simple (NO divergente)}
  \AttributeTok{centered =} \ConstantTok{FALSE}\NormalTok{,}
  \AttributeTok{plot.percent.low =} \ConstantTok{FALSE}\NormalTok{,}
  \AttributeTok{plot.percent.high =} \ConstantTok{FALSE}\NormalTok{,}
  \AttributeTok{plot.percent.neutral =} \ConstantTok{FALSE}\NormalTok{,}
  
  \CommentTok{\# Leyenda que explica las respuestas}
  \AttributeTok{legend.position =} \StringTok{\textquotesingle{}right\textquotesingle{}}\NormalTok{,}
  \AttributeTok{legend =} \StringTok{\textquotesingle{}Respuesta (1{-}5)\textquotesingle{}}
\NormalTok{) }\SpecialCharTok{+}
  \CommentTok{\# Título de la figura (Actualizado)}
  \FunctionTok{labs}\NormalTok{(}
    \AttributeTok{title =} \StringTok{\textquotesingle{}Figura 20. Valoración de Factores de Importancia (Escala 1{-}5)\textquotesingle{}}\NormalTok{,}
    \AttributeTok{y =} \StringTok{\textquotesingle{}Porcentaje (\%)\textquotesingle{}}\NormalTok{,}
    \AttributeTok{x =} \ConstantTok{NULL}
\NormalTok{  )}

\CommentTok{\# Imprimimos la figura}
\NormalTok{fig.likert}
\end{Highlighting}
\end{Shaded}

\begin{center}\includegraphics{ICO-analisis_files/figure-latex/P7-P12-P17-likert-1} \end{center}

Según la valoración de los factores de importancia en una escala Likert,
existe una clara jerarquía de prioridades: los aspectos más cruciales
son la Higiene, el Sabor y el Precio, ya que la inmensa mayoría de los
encuestados los considera ``muy importantes'' (categorías 4 y 5); les
siguen en importancia la Variedad de Ingredientes, el Origen Local y el
valor de Comer Saludable. Por otro lado, la Sostenibilidad es el factor
que recibe la mayor proporción de valoraciones bajas en comparación con
los demás, y la Rapidez también muestra una importancia variable.
Finalmente, una conclusión de opinión destacada es que la mayoría de los
encuestados no percibe las ensaladas como una comida social.

\hypertarget{variables-cualitativas}{%
\subsection{Variables cualitativas}\label{variables-cualitativas}}

La variable \texttt{Mala.Experiencia.Descripcion} se ha registrado
mediante una pregunta abierta. El motivo de que fuera abierta, sin que
se proporcionara un conjunto cerrado de respuestas, es que se quiere
conocer con un poco más de detalle algunas experiencias negativas
previas para relacionarla con el grado de aceptación. Por tanto, se
trata de una variable cualitativa, y por este motivo se planteó en forma
de pregunta abierta.

Este tipo de variables no puede ser tratada estadísticamente. Su
análisis tiene que llevarse a cabo de forma manual. Lo que sí vamos a
hacer es seleccionar las respuestas válidas mediante R para después
proceder a su análisis cualitativo

\begin{Shaded}
\begin{Highlighting}[]
\CommentTok{\# generamos una tabla con las filas que contienen alguna razón para no usar el escáner}
\NormalTok{datos.cualitativos }\OtherTok{\textless{}{-}} \FunctionTok{drop\_na}\NormalTok{(datos, }\StringTok{\textquotesingle{}Mala.Experiencia.Descripcion\textquotesingle{}}\NormalTok{)}

\CommentTok{\# imprimimos únicamente esa columna}
\FunctionTok{flextable}\NormalTok{(}
  \CommentTok{\# convertimos a un dataframe}
  \FunctionTok{as.data.frame}\NormalTok{(datos.cualitativos}\SpecialCharTok{$}\NormalTok{Mala.Experiencia.Descripcion)) }\SpecialCharTok{\%\textgreater{}\%} 
  \CommentTok{\# establecemos el título de la tabla}
  \FunctionTok{set\_caption}\NormalTok{(}\StringTok{\textquotesingle{}Tabla 20. Malas experiencias con comida preparada o autoservicio.\textquotesingle{}}\NormalTok{) }\SpecialCharTok{\%\textgreater{}\%}
  \CommentTok{\# ajustamos la tabla para que pueda leerse mejor en la página}
  \FunctionTok{autofit}\NormalTok{()}
\end{Highlighting}
\end{Shaded}

\global\setlength{\Oldarrayrulewidth}{\arrayrulewidth}

\global\setlength{\Oldtabcolsep}{\tabcolsep}

\setlength{\tabcolsep}{2pt}

\renewcommand*{\arraystretch}{1.5}



\providecommand{\ascline}[3]{\noalign{\global\arrayrulewidth #1}\arrayrulecolor[HTML]{#2}\cline{#3}}

\begin{longtable}[c]{|p{22.99in}}

\caption{Tabla\ 20.\ Malas\ experiencias\ con\ comida\ preparada\ o\ autoservicio.}\\

\ascline{1.5pt}{666666}{1-1}

\multicolumn{1}{>{\raggedright}m{\dimexpr 22.99in+0\tabcolsep}}{\textcolor[HTML]{000000}{\fontsize{11}{11}\selectfont{datos.cualitativos\$Mala.Experiencia.Descripcion}}} \\

\ascline{1.5pt}{666666}{1-1}\endfirsthead \caption[]{Tabla\ 20.\ Malas\ experiencias\ con\ comida\ preparada\ o\ autoservicio.}\\

\ascline{1.5pt}{666666}{1-1}

\multicolumn{1}{>{\raggedright}m{\dimexpr 22.99in+0\tabcolsep}}{\textcolor[HTML]{000000}{\fontsize{11}{11}\selectfont{datos.cualitativos\$Mala.Experiencia.Descripcion}}} \\

\ascline{1.5pt}{666666}{1-1}\endhead



\multicolumn{1}{>{\raggedright}m{\dimexpr 22.99in+0\tabcolsep}}{\textcolor[HTML]{000000}{\fontsize{11}{11}\selectfont{No\ he\ tenido\ experiencias\ previas}}} \\





\multicolumn{1}{>{\raggedright}m{\dimexpr 22.99in+0\tabcolsep}}{\textcolor[HTML]{000000}{\fontsize{11}{11}\selectfont{No\ he\ tenido\ ninguna\ mala\ experiencia\ que\ recuerde.\ No\ suelo\ usarlo.}}} \\





\multicolumn{1}{>{\raggedright}m{\dimexpr 22.99in+0\tabcolsep}}{\textcolor[HTML]{000000}{\fontsize{11}{11}\selectfont{comida\ grasienta}}} \\





\multicolumn{1}{>{\raggedright}m{\dimexpr 22.99in+0\tabcolsep}}{\textcolor[HTML]{000000}{\fontsize{11}{11}\selectfont{Alguna\ vez\ no\ tenía\ muy\ buena\ pinta,\ posiblemente\ había\ algo\ en\ mal\ estado}}} \\





\multicolumn{1}{>{\raggedright}m{\dimexpr 22.99in+0\tabcolsep}}{\textcolor[HTML]{000000}{\fontsize{11}{11}\selectfont{Falta\ de\ higiene}}} \\





\multicolumn{1}{>{\raggedright}m{\dimexpr 22.99in+0\tabcolsep}}{\textcolor[HTML]{000000}{\fontsize{11}{11}\selectfont{No}}} \\





\multicolumn{1}{>{\raggedright}m{\dimexpr 22.99in+0\tabcolsep}}{\textcolor[HTML]{000000}{\fontsize{11}{11}\selectfont{Ninguna}}} \\





\multicolumn{1}{>{\raggedright}m{\dimexpr 22.99in+0\tabcolsep}}{\textcolor[HTML]{000000}{\fontsize{11}{11}\selectfont{No}}} \\





\multicolumn{1}{>{\raggedright}m{\dimexpr 22.99in+0\tabcolsep}}{\textcolor[HTML]{000000}{\fontsize{11}{11}\selectfont{No}}} \\





\multicolumn{1}{>{\raggedright}m{\dimexpr 22.99in+0\tabcolsep}}{\textcolor[HTML]{000000}{\fontsize{11}{11}\selectfont{no}}} \\





\multicolumn{1}{>{\raggedright}m{\dimexpr 22.99in+0\tabcolsep}}{\textcolor[HTML]{000000}{\fontsize{11}{11}\selectfont{Me\ ha\ aparecido\ un\ trozo\ de\ madera\ con\ pintura\ en\ la\ comida}}} \\





\multicolumn{1}{>{\raggedright}m{\dimexpr 22.99in+0\tabcolsep}}{\textcolor[HTML]{000000}{\fontsize{11}{11}\selectfont{Se\ me\ olvidaron\ coger\ cubiertos\ y\ luego\ no\ tenía\ como\ comer,\ así\ que\ me\ tocó\ comer\ de\ malas\ maneras.}}} \\





\multicolumn{1}{>{\raggedright}m{\dimexpr 22.99in+0\tabcolsep}}{\textcolor[HTML]{000000}{\fontsize{11}{11}\selectfont{Ocasionalmente\ el\ menú\ de\ hamburguesas\ del\ trinquet,\ tanto\ las\ patatas\ como\ la\ hamburguesa\ saben\ a\ cartón,\ como\ si\ estuviese\ comiendo\ pared\ de\ casa\ americana,\ pero\ como\ en\ relación\ abusiva,\ tarde\ o\ temprano\ vuelvo\ porque\ está\ vez\ cambiaran.\ También\ nunca\ hay\ pan\ grande,\ lo\ cual\ es\ una\ blasfemia\ contra\ la\ moral\ pública.}}} \\





\multicolumn{1}{>{\raggedright}m{\dimexpr 22.99in+0\tabcolsep}}{\textcolor[HTML]{000000}{\fontsize{11}{11}\selectfont{no}}} \\





\multicolumn{1}{>{\raggedright}m{\dimexpr 22.99in+0\tabcolsep}}{\textcolor[HTML]{000000}{\fontsize{11}{11}\selectfont{Comida\ mal\ preparada\ o\ congelada.}}} \\





\multicolumn{1}{>{\raggedright}m{\dimexpr 22.99in+0\tabcolsep}}{\textcolor[HTML]{000000}{\fontsize{11}{11}\selectfont{No\ he\ tenido\ ninguna\ mala\ experiencia.}}} \\





\multicolumn{1}{>{\raggedright}m{\dimexpr 22.99in+0\tabcolsep}}{\textcolor[HTML]{000000}{\fontsize{11}{11}\selectfont{No}}} \\





\multicolumn{1}{>{\raggedright}m{\dimexpr 22.99in+0\tabcolsep}}{\textcolor[HTML]{000000}{\fontsize{11}{11}\selectfont{Un\ gusano}}} \\





\multicolumn{1}{>{\raggedright}m{\dimexpr 22.99in+0\tabcolsep}}{\textcolor[HTML]{000000}{\fontsize{11}{11}\selectfont{Si,\ que\ me\ cuesta\ mucho\ decidir\ que\ coger}}} \\





\multicolumn{1}{>{\raggedright}m{\dimexpr 22.99in+0\tabcolsep}}{\textcolor[HTML]{000000}{\fontsize{11}{11}\selectfont{No}}} \\





\multicolumn{1}{>{\raggedright}m{\dimexpr 22.99in+0\tabcolsep}}{\textcolor[HTML]{000000}{\fontsize{11}{11}\selectfont{Algun\ ingrediente\ de\ hojas\ verdes\ sin\ cocinar\ poco\ fresco\ o\ algo\ deteriorado\ por\ ser\ un\ vegetal\ delicado\ para\ conservar}}} \\





\multicolumn{1}{>{\raggedright}m{\dimexpr 22.99in+0\tabcolsep}}{\textcolor[HTML]{000000}{\fontsize{11}{11}\selectfont{Falta\ de\ frescura}}} \\





\multicolumn{1}{>{\raggedright}m{\dimexpr 22.99in+0\tabcolsep}}{\textcolor[HTML]{000000}{\fontsize{11}{11}\selectfont{Control\ sanitario\ /\ Reposición\ /\ Dosificación\ /\ bacterizacion\ por\ contaminacion\ de\ clientela,\ mendigos,\ viejos,\ niños}}} \\





\multicolumn{1}{>{\raggedright}m{\dimexpr 22.99in+0\tabcolsep}}{\textcolor[HTML]{000000}{\fontsize{11}{11}\selectfont{Los\ ingredientes\ no\ parecían\ estar\ muy\ frescos,\ o\ había\ muy\ poca\ variedad\ de\ ingredientes.}}} \\





\multicolumn{1}{>{\raggedright}m{\dimexpr 22.99in+0\tabcolsep}}{\textcolor[HTML]{000000}{\fontsize{11}{11}\selectfont{Qué\ esté\ la\ comida\ fría\ o\ por\ lo\ menos\ se\ note\ que\ lleva\ mucho\ tiempo\ hecha.}}} \\





\multicolumn{1}{>{\raggedright}m{\dimexpr 22.99in+0\tabcolsep}}{\textcolor[HTML]{000000}{\fontsize{11}{11}\selectfont{No}}} \\





\multicolumn{1}{>{\raggedright}m{\dimexpr 22.99in+0\tabcolsep}}{\textcolor[HTML]{000000}{\fontsize{11}{11}\selectfont{no}}} \\





\multicolumn{1}{>{\raggedright}m{\dimexpr 22.99in+0\tabcolsep}}{\textcolor[HTML]{000000}{\fontsize{11}{11}\selectfont{si}}} \\





\multicolumn{1}{>{\raggedright}m{\dimexpr 22.99in+0\tabcolsep}}{\textcolor[HTML]{000000}{\fontsize{11}{11}\selectfont{Buffet\ de\ hoteles\ con\ poca\ variedad.\ La\ calidad\ de\ la\ comida\ baja.}}} \\





\multicolumn{1}{>{\raggedright}m{\dimexpr 22.99in+0\tabcolsep}}{\textcolor[HTML]{000000}{\fontsize{11}{11}\selectfont{Productos\ en\ mal\ estado}}} \\





\multicolumn{1}{>{\raggedright}m{\dimexpr 22.99in+0\tabcolsep}}{\textcolor[HTML]{000000}{\fontsize{11}{11}\selectfont{No\ he\ tenido,\ pero\ prefiero\ hacerme\ las\ ensaladas\ en\ cada}}} \\





\multicolumn{1}{>{\raggedright}m{\dimexpr 22.99in+0\tabcolsep}}{\textcolor[HTML]{000000}{\fontsize{11}{11}\selectfont{Es\ cara\ y\ no\ sabe\ a\ nada\ en\ específico,\ pura\ quimica}}} \\





\multicolumn{1}{>{\raggedright}m{\dimexpr 22.99in+0\tabcolsep}}{\textcolor[HTML]{000000}{\fontsize{11}{11}\selectfont{Una\ vez\ me\ intoxique\ con\ la\ comida\ preparada\ de\ Mercadona.\ A\ partir\ de\ ese\ momento\ me\ lo\ pienso\ bastante\ para\ volver\ a\ comprar\ los\ platos\ preaprados}}} \\





\multicolumn{1}{>{\raggedright}m{\dimexpr 22.99in+0\tabcolsep}}{\textcolor[HTML]{000000}{\fontsize{11}{11}\selectfont{No\ recuerdo}}} \\





\multicolumn{1}{>{\raggedright}m{\dimexpr 22.99in+0\tabcolsep}}{\textcolor[HTML]{000000}{\fontsize{11}{11}\selectfont{Ninguna}}} \\





\multicolumn{1}{>{\raggedright}m{\dimexpr 22.99in+0\tabcolsep}}{\textcolor[HTML]{000000}{\fontsize{11}{11}\selectfont{En\ las\ barras\ de\ ensalada\ ponen\ mucha\ lechuga\ y\ poco\ condumio}}} \\





\multicolumn{1}{>{\raggedright}m{\dimexpr 22.99in+0\tabcolsep}}{\textcolor[HTML]{000000}{\fontsize{11}{11}\selectfont{No\ he\ tenido}}} \\





\multicolumn{1}{>{\raggedright}m{\dimexpr 22.99in+0\tabcolsep}}{\textcolor[HTML]{000000}{\fontsize{11}{11}\selectfont{Nunca\ he\ tenido\ malas\ experiencias}}} \\





\multicolumn{1}{>{\raggedright}m{\dimexpr 22.99in+0\tabcolsep}}{\textcolor[HTML]{000000}{\fontsize{11}{11}\selectfont{La\ verdad\ es\ que\ mientras\ los\ productos\ estén\ frescos,\ no.}}} \\





\multicolumn{1}{>{\raggedright}m{\dimexpr 22.99in+0\tabcolsep}}{\textcolor[HTML]{000000}{\fontsize{11}{11}\selectfont{No}}} \\





\multicolumn{1}{>{\raggedright}m{\dimexpr 22.99in+0\tabcolsep}}{\textcolor[HTML]{000000}{\fontsize{11}{11}\selectfont{No\ he\ tenido}}} \\





\multicolumn{1}{>{\raggedright}m{\dimexpr 22.99in+0\tabcolsep}}{\textcolor[HTML]{000000}{\fontsize{11}{11}\selectfont{No}}} \\





\multicolumn{1}{>{\raggedright}m{\dimexpr 22.99in+0\tabcolsep}}{\textcolor[HTML]{000000}{\fontsize{11}{11}\selectfont{No}}} \\





\multicolumn{1}{>{\raggedright}m{\dimexpr 22.99in+0\tabcolsep}}{\textcolor[HTML]{000000}{\fontsize{11}{11}\selectfont{No}}} \\





\multicolumn{1}{>{\raggedright}m{\dimexpr 22.99in+0\tabcolsep}}{\textcolor[HTML]{000000}{\fontsize{11}{11}\selectfont{Ninguna}}} \\





\multicolumn{1}{>{\raggedright}m{\dimexpr 22.99in+0\tabcolsep}}{\textcolor[HTML]{000000}{\fontsize{11}{11}\selectfont{Me\ preocupa\ la\ falta\ de\ higiene\ en\ la\ manipulación\ de\ los\ alimentos}}} \\





\multicolumn{1}{>{\raggedright}m{\dimexpr 22.99in+0\tabcolsep}}{\textcolor[HTML]{000000}{\fontsize{11}{11}\selectfont{Princi\ son\ saludables,\ no\ puedo\ cocinar\ o\ calentar\ los\ platos\ preparados\ }}\textcolor[HTML]{000000}{\fontsize{11}{11}\selectfont{\linebreak }}} \\





\multicolumn{1}{>{\raggedright}m{\dimexpr 22.99in+0\tabcolsep}}{\textcolor[HTML]{000000}{\fontsize{11}{11}\selectfont{No}}} \\





\multicolumn{1}{>{\raggedright}m{\dimexpr 22.99in+0\tabcolsep}}{\textcolor[HTML]{000000}{\fontsize{11}{11}\selectfont{No\ he\ tenido\ ninguna\ mala\ experiencia}}} \\





\multicolumn{1}{>{\raggedright}m{\dimexpr 22.99in+0\tabcolsep}}{\textcolor[HTML]{000000}{\fontsize{11}{11}\selectfont{No\ he\ tenido\ ninguna\ mala\ experiencia}}} \\





\multicolumn{1}{>{\raggedright}m{\dimexpr 22.99in+0\tabcolsep}}{\textcolor[HTML]{000000}{\fontsize{11}{11}\selectfont{no\ podría\ describir\ ninguna.\ ya\ que\ los\ bares,\ cafeterías\ y\ supermercados\ en\ los\ que\ suelo\ comer\ fuera\ de\ casa\ no\ tienen\ implementado\ esta\ función\ de\ autoservicio.}}} \\





\multicolumn{1}{>{\raggedright}m{\dimexpr 22.99in+0\tabcolsep}}{\textcolor[HTML]{000000}{\fontsize{11}{11}\selectfont{No\ he\ tenido\ ninguna}}} \\





\multicolumn{1}{>{\raggedright}m{\dimexpr 22.99in+0\tabcolsep}}{\textcolor[HTML]{000000}{\fontsize{11}{11}\selectfont{Falta\ de\ higiene}}} \\





\multicolumn{1}{>{\raggedright}m{\dimexpr 22.99in+0\tabcolsep}}{\textcolor[HTML]{000000}{\fontsize{11}{11}\selectfont{Problemas\ con\ la\ higiene}}} \\





\multicolumn{1}{>{\raggedright}m{\dimexpr 22.99in+0\tabcolsep}}{\textcolor[HTML]{000000}{\fontsize{11}{11}\selectfont{El\ Sibo}}} \\





\multicolumn{1}{>{\raggedright}m{\dimexpr 22.99in+0\tabcolsep}}{\textcolor[HTML]{000000}{\fontsize{11}{11}\selectfont{No\ comer\ en\ Pompeya}}} \\





\multicolumn{1}{>{\raggedright}m{\dimexpr 22.99in+0\tabcolsep}}{\textcolor[HTML]{000000}{\fontsize{11}{11}\selectfont{No\ he\ tenido}}} \\





\multicolumn{1}{>{\raggedright}m{\dimexpr 22.99in+0\tabcolsep}}{\textcolor[HTML]{000000}{\fontsize{11}{11}\selectfont{Comida\ con\ mala\ pinta}}} \\





\multicolumn{1}{>{\raggedright}m{\dimexpr 22.99in+0\tabcolsep}}{\textcolor[HTML]{000000}{\fontsize{11}{11}\selectfont{no}}} \\





\multicolumn{1}{>{\raggedright}m{\dimexpr 22.99in+0\tabcolsep}}{\textcolor[HTML]{000000}{\fontsize{11}{11}\selectfont{no\ he\ tenido}}} \\





\multicolumn{1}{>{\raggedright}m{\dimexpr 22.99in+0\tabcolsep}}{\textcolor[HTML]{000000}{\fontsize{11}{11}\selectfont{no}}} \\





\multicolumn{1}{>{\raggedright}m{\dimexpr 22.99in+0\tabcolsep}}{\textcolor[HTML]{000000}{\fontsize{11}{11}\selectfont{No\ he\ tenido.}}} \\





\multicolumn{1}{>{\raggedright}m{\dimexpr 22.99in+0\tabcolsep}}{\textcolor[HTML]{000000}{\fontsize{11}{11}\selectfont{No}}} \\

\ascline{1.5pt}{666666}{1-1}



\end{longtable}



\arrayrulecolor[HTML]{000000}

\global\setlength{\arrayrulewidth}{\Oldarrayrulewidth}

\global\setlength{\tabcolsep}{\Oldtabcolsep}

\renewcommand*{\arraystretch}{1}

Hemos optado por agrupar las respuestas por categorías para poder
analizarlas en mayor profundidad. Nos hemos decantado por 7 categorías:
1. No han tenido malas experiencias. 2. Relacionadas con la higiene,
salud y contaminación. 3. Relacionadas con la frescura y calidad. 4.
Relacionadas con el sabor y la preparación. 5. Relacionadas con el
servicio y la logística detrás de este. 6. Relacionadas con la variedad
de ingredientes y la cantidad de porción. 7. Otros (no queda claro donde
categorizarlas).

\begin{Shaded}
\begin{Highlighting}[]
\CommentTok{\# Simulamos la columna de datos (asumimos que esta columna no se cargó como factor)}
\NormalTok{respuestas\_raw }\OtherTok{\textless{}{-}} \FunctionTok{c}\NormalTok{(}
  \StringTok{"No he tenido experiencias previas"}\NormalTok{, }\StringTok{"No he tenido ninguna mala experiencia que recuerde. No suelo usarlo."}\NormalTok{, }\StringTok{"comida grasienta"}\NormalTok{, }\StringTok{"Alguna vez no tenía muy buena pinta, posiblemente había algo en mal estado"}\NormalTok{, }\StringTok{"Falta de higiene"}\NormalTok{, }\StringTok{"No"}\NormalTok{, }\StringTok{"Ninguna"}\NormalTok{, }\StringTok{"No"}\NormalTok{, }\StringTok{"No"}\NormalTok{, }\StringTok{"no"}\NormalTok{, }
  \StringTok{"Me ha aparecido un trozo de madera con pintura en la comida"}\NormalTok{, }\StringTok{"Se me olvidaron coger cubiertos y luego no tenía como comer, así que me tocó comer de malas maneras."}\NormalTok{, }\StringTok{"Ocasionalmente el menú de hamburguesas del trinquet, tanto las patatas como la hamburguesa saben a cartón, como si estuviese comiendo pared de casa americana, pero como en relación abusiva, tarde o temprano vuelvo porque está vez cambiaran. También nunca hay pan grande, lo cual es una blasfemia contra la moral pública."}\NormalTok{, }\StringTok{"no"}\NormalTok{, }\StringTok{"Comida mal preparada o congelada."}\NormalTok{, }\StringTok{"No he tenido ninguna mala experiencia."}\NormalTok{, }\StringTok{"No"}\NormalTok{, }\StringTok{"Un gusano"}\NormalTok{, }\StringTok{"Si, que me cuesta mucho decidir que coger"}\NormalTok{, }\StringTok{"No"}\NormalTok{, }
  \StringTok{"Algun ingrediente de hojas verdes sin cocinar poco fresco o algo deteriorado por ser un vegetal delicado para conservar"}\NormalTok{, }\StringTok{"Falta de frescura"}\NormalTok{, }\StringTok{"Control sanitario / Reposición / Dosificación / bacterizacion por contaminacion de clientela, mendigos, viejos, niños"}\NormalTok{, }\StringTok{"Los ingredientes no parecían estar muy frescos, o había muy poca variedad de ingredientes."}\NormalTok{, }\StringTok{"Qué esté la comida fría o por lo menos se note que lleva mucho tiempo hecha."}\NormalTok{, }\StringTok{"No"}\NormalTok{, }\StringTok{"no"}\NormalTok{, }\StringTok{"si"}\NormalTok{, }\StringTok{"Buffet de hoteles con poca variedad. La calidad de la comida baja."}\NormalTok{, }\StringTok{"Productos en mal estado"}\NormalTok{, }
  \StringTok{"No he tenido, pero prefiero hacerme las ensaladas en cada"}\NormalTok{, }\StringTok{"Es cara y no sabe a nada en específico, pura quimica"}\NormalTok{, }\StringTok{"Una vez me intoxique con la comida preparada de Mercadona. A partir de ese momento me lo pienso bastante para volver a comprar los platos preaprados"}\NormalTok{, }\StringTok{"No recuerdo"}\NormalTok{, }\StringTok{"Ninguna"}\NormalTok{, }\StringTok{"En las barras de ensalada ponen mucha lechuga y poco condumio"}\NormalTok{, }\StringTok{"No he tenido"}\NormalTok{, }\StringTok{"Nunca he tenido malas experiencias"}\NormalTok{, }\StringTok{"La verdad es que mientras los productos estén frescos, no."}\NormalTok{, }\StringTok{"No"}\NormalTok{, }
  \StringTok{"No he tenido"}\NormalTok{, }\StringTok{"No"}\NormalTok{, }\StringTok{"No"}\NormalTok{,}\StringTok{"No"}\NormalTok{, }\StringTok{"Ninguna"}\NormalTok{, }\StringTok{"Me preocupa la falta de higiene en la manipulación de los alimentos"}\NormalTok{, }\StringTok{"Princi son saludables, no puedo cocinar o calentar los platos preparados"}\NormalTok{, }\StringTok{"No"}\NormalTok{, }\StringTok{"No he tenido ninguna mala experiencia"}\NormalTok{, }\StringTok{"No he tenido ninguna mala experiencia"}\NormalTok{, }\StringTok{"no podría describir ninguna. ya que los bares, cafeterías y supermercados en los que suelo comer fuera de casa no tienen implementado esta función de autoservicio."}\NormalTok{, }\StringTok{"No he tenido ninguna"}\NormalTok{, }\StringTok{"Falta de higiene"}\NormalTok{, }\StringTok{"Problemas con la higiene"}\NormalTok{, }\StringTok{"El Sibo"}\NormalTok{, }
  \StringTok{"No comer en Pompeya"}\NormalTok{, }\StringTok{"No he tenido"}\NormalTok{, }\StringTok{"Comida con mala pinta"}\NormalTok{, }\StringTok{"no"}\NormalTok{, }\StringTok{"no he tenido"}\NormalTok{, }\StringTok{"no"}\NormalTok{, }\StringTok{"No he tenido."}\NormalTok{, }\StringTok{"No"}
\NormalTok{)}
\CommentTok{\# Creamos un dataframe temporal para la categorización}
\NormalTok{df\_cualitativo }\OtherTok{\textless{}{-}}\NormalTok{ tibble}\SpecialCharTok{::}\FunctionTok{tibble}\NormalTok{(}\AttributeTok{respuesta\_raw =}\NormalTok{ respuestas\_raw)}

\CommentTok{\# Categorización usando case\_when (patrones de texto)}
\NormalTok{df\_categorizado }\OtherTok{\textless{}{-}}\NormalTok{ df\_cualitativo }\SpecialCharTok{\%\textgreater{}\%}
\NormalTok{  dplyr}\SpecialCharTok{::}\FunctionTok{mutate}\NormalTok{(}
    \AttributeTok{Categoria =}\NormalTok{ dplyr}\SpecialCharTok{::}\FunctionTok{case\_when}\NormalTok{(}
      \CommentTok{\# Higiene, Salud y Contaminación (Máxima Prioridad)}
\NormalTok{      stringr}\SpecialCharTok{::}\FunctionTok{str\_detect}\NormalTok{(respuesta\_raw, stringr}\SpecialCharTok{::}\FunctionTok{regex}\NormalTok{(}\StringTok{"gusano|intoxique|madera|higiene|sanitario|contaminacion|Sibo|saludables"}\NormalTok{, }\AttributeTok{ignore\_case =} \ConstantTok{TRUE}\NormalTok{)) }\SpecialCharTok{\textasciitilde{}} \StringTok{"Higiene / Contaminación / Salud"}\NormalTok{,}
      
      \CommentTok{\# Frescura y Calidad}
\NormalTok{      stringr}\SpecialCharTok{::}\FunctionTok{str\_detect}\NormalTok{(respuesta\_raw, stringr}\SpecialCharTok{::}\FunctionTok{regex}\NormalTok{(}\StringTok{"frescura|fresco|deteriorado|mal estado|mala pinta|fría|baja"}\NormalTok{, }\AttributeTok{ignore\_case =} \ConstantTok{TRUE}\NormalTok{)) }\SpecialCharTok{\textasciitilde{}} \StringTok{"Calidad / Frescura"}\NormalTok{,}
      
      \CommentTok{\# Sabor y Preparación}
\NormalTok{      stringr}\SpecialCharTok{::}\FunctionTok{str\_detect}\NormalTok{(respuesta\_raw, stringr}\SpecialCharTok{::}\FunctionTok{regex}\NormalTok{(}\StringTok{"grasienta|mal preparada|congelada|cartón|química|saben|caras"}\NormalTok{, }\AttributeTok{ignore\_case =} \ConstantTok{TRUE}\NormalTok{)) }\SpecialCharTok{\textasciitilde{}} \StringTok{"Sabor / Preparación"}\NormalTok{,}
      
      \CommentTok{\# Variedad y Porción}
\NormalTok{      stringr}\SpecialCharTok{::}\FunctionTok{str\_detect}\NormalTok{(respuesta\_raw, stringr}\SpecialCharTok{::}\FunctionTok{regex}\NormalTok{(}\StringTok{"poca variedad|mucha lechuga|poco condumio|buffet de hoteles"}\NormalTok{, }\AttributeTok{ignore\_case =} \ConstantTok{TRUE}\NormalTok{)) }\SpecialCharTok{\textasciitilde{}} \StringTok{"Variedad / Porción"}\NormalTok{,}
      
      \CommentTok{\# Logística / Servicio}
\NormalTok{      stringr}\SpecialCharTok{::}\FunctionTok{str\_detect}\NormalTok{(respuesta\_raw, stringr}\SpecialCharTok{::}\FunctionTok{regex}\NormalTok{(}\StringTok{"cubiertos|decidir|calentar|problema|lento"}\NormalTok{, }\AttributeTok{ignore\_case =} \ConstantTok{TRUE}\NormalTok{)) }\SpecialCharTok{\textasciitilde{}} \StringTok{"Logística / Servicio"}\NormalTok{,}
      
      \CommentTok{\# No Mala Experiencia}
\NormalTok{      stringr}\SpecialCharTok{::}\FunctionTok{str\_detect}\NormalTok{(respuesta\_raw, stringr}\SpecialCharTok{::}\FunctionTok{regex}\NormalTok{(}\StringTok{"no|ninguna|nunca|recuerdo|previas|prefiero|suelo comer fuera|Pompeya"}\NormalTok{, }\AttributeTok{ignore\_case =} \ConstantTok{TRUE}\NormalTok{)) }\SpecialCharTok{\textasciitilde{}} \StringTok{"No Mala Experiencia"}\NormalTok{,}
      
      \CommentTok{\# Por defecto}
      \ConstantTok{TRUE} \SpecialCharTok{\textasciitilde{}} \StringTok{"Otros (No Categorizado)"}
\NormalTok{    )}
\NormalTok{  )}

\CommentTok{\# Crear tabla de frecuencias}
\NormalTok{df\_resumen }\OtherTok{\textless{}{-}}\NormalTok{ df\_categorizado }\SpecialCharTok{\%\textgreater{}\%}
\NormalTok{  dplyr}\SpecialCharTok{::}\FunctionTok{count}\NormalTok{(Categoria, }\AttributeTok{name =} \StringTok{"Frecuencia"}\NormalTok{) }\SpecialCharTok{\%\textgreater{}\%}
\NormalTok{  dplyr}\SpecialCharTok{::}\FunctionTok{mutate}\NormalTok{(}
    \AttributeTok{Porcentaje =}\NormalTok{ (Frecuencia }\SpecialCharTok{/} \FunctionTok{sum}\NormalTok{(Frecuencia)) }\SpecialCharTok{*} \DecValTok{100}
\NormalTok{  ) }\SpecialCharTok{\%\textgreater{}\%}
\NormalTok{  dplyr}\SpecialCharTok{::}\FunctionTok{arrange}\NormalTok{(}\FunctionTok{desc}\NormalTok{(Frecuencia)) }\CommentTok{\# Ordenar por frecuencia}


\CommentTok{\# Añadir Fila Total y Formato (como en la guía)}
\NormalTok{df\_total }\OtherTok{\textless{}{-}} \FunctionTok{data.frame}\NormalTok{(}
  \AttributeTok{Categoria =} \StringTok{"TOTAL"}\NormalTok{,}
  \AttributeTok{Frecuencia =} \FunctionTok{sum}\NormalTok{(df\_resumen}\SpecialCharTok{$}\NormalTok{Frecuencia),}
  \AttributeTok{Porcentaje =} \FloatTok{100.0}
\NormalTok{)}
\NormalTok{df\_resumen\_final }\OtherTok{\textless{}{-}} \FunctionTok{bind\_rows}\NormalTok{(df\_resumen, df\_total)}

\CommentTok{\# Generación de la tabla (flextable)}
\NormalTok{ft\_resumen }\OtherTok{\textless{}{-}}\NormalTok{ flextable}\SpecialCharTok{::}\FunctionTok{flextable}\NormalTok{(df\_resumen\_final) }\SpecialCharTok{\%\textgreater{}\%}
\NormalTok{  flextable}\SpecialCharTok{::}\FunctionTok{set\_caption}\NormalTok{(}\AttributeTok{caption =} \StringTok{"Tabla 21. Agrupación de Motivos de Mala Experiencia (Respuestas Abiertas)"}\NormalTok{) }\SpecialCharTok{\%\textgreater{}\%}
\NormalTok{  flextable}\SpecialCharTok{::}\FunctionTok{set\_header\_labels}\NormalTok{(}
    \AttributeTok{Categoria =} \StringTok{"Categoría de Fallo"}\NormalTok{,}
    \AttributeTok{Frecuencia =} \StringTok{"Frecuencia (n)"}\NormalTok{,}
    \AttributeTok{Porcentaje =} \StringTok{"Porcentaje (\%)"}
\NormalTok{  ) }\SpecialCharTok{\%\textgreater{}\%}
\NormalTok{  flextable}\SpecialCharTok{::}\FunctionTok{colformat\_double}\NormalTok{(}\AttributeTok{j =} \DecValTok{3}\NormalTok{, }\AttributeTok{digits =} \DecValTok{1}\NormalTok{, }\AttributeTok{suffix =} \StringTok{"\%"}\NormalTok{) }\SpecialCharTok{\%\textgreater{}\%}
\NormalTok{  flextable}\SpecialCharTok{::}\FunctionTok{theme\_booktabs}\NormalTok{() }\SpecialCharTok{\%\textgreater{}\%}
\NormalTok{  flextable}\SpecialCharTok{::}\FunctionTok{bold}\NormalTok{(}\AttributeTok{i =} \FunctionTok{nrow}\NormalTok{(df\_resumen\_final), }\AttributeTok{part =} \StringTok{"body"}\NormalTok{) }\SpecialCharTok{\%\textgreater{}\%}
\NormalTok{  flextable}\SpecialCharTok{::}\FunctionTok{hline}\NormalTok{(}\AttributeTok{i =} \FunctionTok{nrow}\NormalTok{(df\_resumen\_final) }\SpecialCharTok{{-}} \DecValTok{1}\NormalTok{, }\AttributeTok{border =}\NormalTok{ officer}\SpecialCharTok{::}\FunctionTok{fp\_border}\NormalTok{(}\AttributeTok{width =} \FloatTok{1.5}\NormalTok{, }\AttributeTok{color =} \StringTok{"black"}\NormalTok{)) }\SpecialCharTok{\%\textgreater{}\%}
\NormalTok{  flextable}\SpecialCharTok{::}\FunctionTok{align}\NormalTok{(}\AttributeTok{align =} \StringTok{"center"}\NormalTok{, }\AttributeTok{part =} \StringTok{"all"}\NormalTok{) }\SpecialCharTok{\%\textgreater{}\%}
\NormalTok{  flextable}\SpecialCharTok{::}\FunctionTok{align}\NormalTok{(}\AttributeTok{j =} \DecValTok{1}\NormalTok{, }\AttributeTok{align =} \StringTok{"left"}\NormalTok{, }\AttributeTok{part =} \StringTok{"body"}\NormalTok{) }\SpecialCharTok{\%\textgreater{}\%}
\NormalTok{  flextable}\SpecialCharTok{::}\FunctionTok{autofit}\NormalTok{()}

\CommentTok{\# Mostrar la tabla}
\NormalTok{ft\_resumen}
\end{Highlighting}
\end{Shaded}

\global\setlength{\Oldarrayrulewidth}{\arrayrulewidth}

\global\setlength{\Oldtabcolsep}{\tabcolsep}

\setlength{\tabcolsep}{2pt}

\renewcommand*{\arraystretch}{1.5}



\providecommand{\ascline}[3]{\noalign{\global\arrayrulewidth #1}\arrayrulecolor[HTML]{#2}\cline{#3}}

\begin{longtable}[c]{|p{2.47in}|p{1.27in}|p{1.30in}}

\caption{Tabla\ 21.\ Agrupación\ de\ Motivos\ de\ Mala\ Experiencia\ (Respuestas\ Abiertas)}\\

\ascline{1.5pt}{666666}{1-3}

\multicolumn{1}{>{\centering}m{\dimexpr 2.47in+0\tabcolsep}}{\textcolor[HTML]{000000}{\fontsize{11}{11}\selectfont{Categoría\ de\ Fallo}}} & \multicolumn{1}{>{\centering}m{\dimexpr 1.27in+0\tabcolsep}}{\textcolor[HTML]{000000}{\fontsize{11}{11}\selectfont{Frecuencia\ (n)}}} & \multicolumn{1}{>{\centering}m{\dimexpr 1.3in+0\tabcolsep}}{\textcolor[HTML]{000000}{\fontsize{11}{11}\selectfont{Porcentaje\ (\%)}}} \\

\ascline{1.5pt}{666666}{1-3}\endfirsthead \caption[]{Tabla\ 21.\ Agrupación\ de\ Motivos\ de\ Mala\ Experiencia\ (Respuestas\ Abiertas)}\\

\ascline{1.5pt}{666666}{1-3}

\multicolumn{1}{>{\centering}m{\dimexpr 2.47in+0\tabcolsep}}{\textcolor[HTML]{000000}{\fontsize{11}{11}\selectfont{Categoría\ de\ Fallo}}} & \multicolumn{1}{>{\centering}m{\dimexpr 1.27in+0\tabcolsep}}{\textcolor[HTML]{000000}{\fontsize{11}{11}\selectfont{Frecuencia\ (n)}}} & \multicolumn{1}{>{\centering}m{\dimexpr 1.3in+0\tabcolsep}}{\textcolor[HTML]{000000}{\fontsize{11}{11}\selectfont{Porcentaje\ (\%)}}} \\

\ascline{1.5pt}{666666}{1-3}\endhead



\multicolumn{1}{>{\raggedright}m{\dimexpr 2.47in+0\tabcolsep}}{\textcolor[HTML]{000000}{\fontsize{11}{11}\selectfont{No\ Mala\ Experiencia}}} & \multicolumn{1}{>{\centering}m{\dimexpr 1.27in+0\tabcolsep}}{\textcolor[HTML]{000000}{\fontsize{11}{11}\selectfont{37}}} & \multicolumn{1}{>{\centering}m{\dimexpr 1.3in+0\tabcolsep}}{\textcolor[HTML]{000000}{\fontsize{11}{11}\selectfont{58.7\%}}} \\





\multicolumn{1}{>{\raggedright}m{\dimexpr 2.47in+0\tabcolsep}}{\textcolor[HTML]{000000}{\fontsize{11}{11}\selectfont{Higiene\ /\ Contaminación\ /\ Salud}}} & \multicolumn{1}{>{\centering}m{\dimexpr 1.27in+0\tabcolsep}}{\textcolor[HTML]{000000}{\fontsize{11}{11}\selectfont{10}}} & \multicolumn{1}{>{\centering}m{\dimexpr 1.3in+0\tabcolsep}}{\textcolor[HTML]{000000}{\fontsize{11}{11}\selectfont{15.9\%}}} \\





\multicolumn{1}{>{\raggedright}m{\dimexpr 2.47in+0\tabcolsep}}{\textcolor[HTML]{000000}{\fontsize{11}{11}\selectfont{Calidad\ /\ Frescura}}} & \multicolumn{1}{>{\centering}m{\dimexpr 1.27in+0\tabcolsep}}{\textcolor[HTML]{000000}{\fontsize{11}{11}\selectfont{9}}} & \multicolumn{1}{>{\centering}m{\dimexpr 1.3in+0\tabcolsep}}{\textcolor[HTML]{000000}{\fontsize{11}{11}\selectfont{14.3\%}}} \\





\multicolumn{1}{>{\raggedright}m{\dimexpr 2.47in+0\tabcolsep}}{\textcolor[HTML]{000000}{\fontsize{11}{11}\selectfont{Sabor\ /\ Preparación}}} & \multicolumn{1}{>{\centering}m{\dimexpr 1.27in+0\tabcolsep}}{\textcolor[HTML]{000000}{\fontsize{11}{11}\selectfont{3}}} & \multicolumn{1}{>{\centering}m{\dimexpr 1.3in+0\tabcolsep}}{\textcolor[HTML]{000000}{\fontsize{11}{11}\selectfont{4.8\%}}} \\





\multicolumn{1}{>{\raggedright}m{\dimexpr 2.47in+0\tabcolsep}}{\textcolor[HTML]{000000}{\fontsize{11}{11}\selectfont{Logística\ /\ Servicio}}} & \multicolumn{1}{>{\centering}m{\dimexpr 1.27in+0\tabcolsep}}{\textcolor[HTML]{000000}{\fontsize{11}{11}\selectfont{2}}} & \multicolumn{1}{>{\centering}m{\dimexpr 1.3in+0\tabcolsep}}{\textcolor[HTML]{000000}{\fontsize{11}{11}\selectfont{3.2\%}}} \\





\multicolumn{1}{>{\raggedright}m{\dimexpr 2.47in+0\tabcolsep}}{\textcolor[HTML]{000000}{\fontsize{11}{11}\selectfont{Otros\ (No\ Categorizado)}}} & \multicolumn{1}{>{\centering}m{\dimexpr 1.27in+0\tabcolsep}}{\textcolor[HTML]{000000}{\fontsize{11}{11}\selectfont{1}}} & \multicolumn{1}{>{\centering}m{\dimexpr 1.3in+0\tabcolsep}}{\textcolor[HTML]{000000}{\fontsize{11}{11}\selectfont{1.6\%}}} \\





\multicolumn{1}{>{\raggedright}m{\dimexpr 2.47in+0\tabcolsep}}{\textcolor[HTML]{000000}{\fontsize{11}{11}\selectfont{Variedad\ /\ Porción}}} & \multicolumn{1}{>{\centering}m{\dimexpr 1.27in+0\tabcolsep}}{\textcolor[HTML]{000000}{\fontsize{11}{11}\selectfont{1}}} & \multicolumn{1}{>{\centering}m{\dimexpr 1.3in+0\tabcolsep}}{\textcolor[HTML]{000000}{\fontsize{11}{11}\selectfont{1.6\%}}} \\

\ascline{1.5pt}{000000}{1-3}



\multicolumn{1}{>{\raggedright}m{\dimexpr 2.47in+0\tabcolsep}}{\textcolor[HTML]{000000}{\fontsize{11}{11}\selectfont{\textbf{TOTAL}}}} & \multicolumn{1}{>{\centering}m{\dimexpr 1.27in+0\tabcolsep}}{\textcolor[HTML]{000000}{\fontsize{11}{11}\selectfont{\textbf{63}}}} & \multicolumn{1}{>{\centering}m{\dimexpr 1.3in+0\tabcolsep}}{\textcolor[HTML]{000000}{\fontsize{11}{11}\selectfont{\textbf{100.0\%}}}} \\

\ascline{1.5pt}{666666}{1-3}



\end{longtable}



\arrayrulecolor[HTML]{000000}

\global\setlength{\arrayrulewidth}{\Oldarrayrulewidth}

\global\setlength{\tabcolsep}{\Oldtabcolsep}

\renewcommand*{\arraystretch}{1}

\end{document}
